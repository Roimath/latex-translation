\documentclass{article}
\usepackage[utf8]{inputenc}
\usepackage{graphicx}
\usepackage{fancyhdr}
\usepackage{amsmath}
\numberwithin{equation}{section}
\usepackage{amssymb}
\usepackage{amsthm}
\usepackage{pgf-pie}
\usepackage{fontspec}
\usepackage{polyglossia}
\usepackage{chemformula}
\usetikzlibrary{calc}
\usetikzlibrary{arrows.meta, positioning}
\usepackage{subfigure}
\usepackage{booktabs}
\usepackage{array}
\usepackage{epsfig, graphicx}
\usepackage{color}
\usepackage{amsfonts,bm,color}
\usepackage{float}
\usepackage{latexsym}
\usepackage[svgnames]{xcolor} % already in your preamble
\definecolor{darkgreen}{RGB}{0,100,0} % dark, visible green for equations
\definecolor{darkpurple}{RGB}{90, 0, 120} % deep, visible purple for equations
\definecolor{gold}{RGB}{212,175,55} % rich metallic gold
\usepackage{colortbl}
\usepackage{wrapfig}
\usepackage{multirow}
\usepackage{amsfonts}
\usepackage{geometry}
\usepackage{enumitem} % For bullet points
\renewcommand\labelitemi{$\bullet$}
\usepackage{pgfplots} % For graph plotting
\usepackage{tikz}
\usepackage{bidi}
\usepgfplotslibrary{fillbetween}
\usepackage{geometry}
\usepackage{pdfpages}
\usepackage{chancery}
\usepackage{tcolorbox}
\tcbuselibrary{skins, breakable, shadows}

% Exercise counter reset every section
\newcounter{exercise}[section]
\renewcommand{\theexercise}{\thesection.\arabic{exercise}}

% Command to start an exercise
\newcommand{\exercise}[1][]{%
  \refstepcounter{exercise}%
  \subsubsection*{תרגיל \theexercise\; #1}%
  \label{ex:\theexercise}%
}

% Solution counter
\newcounter{solution}[section]
\renewcommand{\thesolution}{\thesection.\arabic{solution}}

\newcommand{\solution}{%
  \refstepcounter{solution}%
  \subsubsection*{תרגיל \thesolution}%
  \label{sol:\thesolution}%
}

% Example counter reset every section
\newcounter{example}[section]
\renewcommand{\theexample}{\thesection.\arabic{example}}

% Command to start an example
\newcommand{\example}[1][]{%
  \refstepcounter{example}%
  \subsubsection*{דוגמה \theexample\; #1}%
  \label{exm:\theexample}%
}

% Command for the "explanation" part of an example
\newcommand{\explanation}{%
  \subsubsection*{הסבר לדוגמה \theexample}%
}

\tcbset{
  rule/.style={
    enhanced,
    colframe=red!90!black,   % lighter frame (less dark than 70)
    colback=red!3,           % very light background
    coltitle=red!20!black,   % softer title color
    fonttitle=\bfseries,
    boxrule=0.8pt,
    arc=4pt,
    left=6pt,right=6pt,top=6pt,bottom=6pt
  }
}

\newtcolorbox{ruleofthumb}[1][]{rule,title={כלל אצבע},#1}


% === GOLDEN BOX STYLE ===
\tcbset{
  gold/.style={
    enhanced,
    colframe=orange!80!black,   % deep gold border
    colback=yellow!8,           % soft golden background
    coltitle=orange!30!black,   % optional title color
    fonttitle=\bfseries\Large,  % larger, bold title font
    fontupper=\Large,           % larger text inside the box
    boxrule=0.8pt,
    arc=4pt,
    left=6pt,right=6pt,top=6pt,bottom=6pt
  }
}

% === Define the new environment ===
\newtcolorbox{goldenbox}[1][]{gold,title={״חוק החמש״},#1}



% Hebrew letters as labels
\makeatletter
\newcommand{\hebrewlabel}[1]{%
  \ifcase#1 א%
  \or ב%
  \or ג%
  \or ד%
  \or ה%
  \or ו%
  \or ז%
  \or ח%
  \or ט%
  \or י%
  \or יא%
  \or יב%
  \or יג%
  \or יד%
  \or טו%
  \or טז%
  \or יז%
  \or יח%
  \or יט%
  \or כ%
  \else \@ctrerr\fi}

\makeatother

% New list type for Hebrew letters
\newlist{hebrewenum}{enumerate}{1}
\setlist[hebrewenum]{label=\hebrewlabel{#1}., ref=\theexercise\hebrewlabel{#1}, leftmargin=2em}

% Custom centered exercise environment
\newenvironment{exline}{%
  \refstepcounter{exercise}%
  \par\vspace{1em} % vertical space before
  \noindent
  \makebox[0pt][r]{\textbf{תרגיל \theexercise}\quad}% bold "תרגיל <num>" on the side
  \begin{minipage}{0.9\linewidth}\centering
}{%
  \end{minipage}%
  \par\vspace{1em} % vertical space after
}

% Define a remark environment numbered within sections
\newtheorem{remark}{הערה}[section]

% in preamble
\newtheorem{insight}{תובנה}[section]

% === QUESTION COUNTER ===
\newcounter{question}[subsection]
\renewcommand{\thequestion}{\arabic{question}}

% === QUESTION COMMAND ===
\newcommand{\question}[1][]{%
  \refstepcounter{question}%
  \subsubsection*{שאלה \thequestion}%
  \ifx&#1&%
  \else
    % Store the current question number for later use in \answer{label}
    \expandafter\xdef\csname qnum@#1\endcsname{\thequestion}%
  \fi
}

% === ANSWER COMMAND ===
\newcommand{\answer}[1]{%
  \subsubsection*{פתרון לשאלה~\csname qnum@#1\endcsname}%
  \label{ans:#1}%
}




\geometry{
  top=3cm,
  bottom=5cm, % ← enlarge this value for more bottom space
  left=2.5cm,
  right=2.5cm
}
 % Adjust to create space for header
\setlength{\headheight}{3cm} % Adjust header height accordingly
\pgfplotsset{compat=1.16}

% Set Hebrew as the main language
\setdefaultlanguage{hebrew}
\setotherlanguages{english}

% Specify the font for Hebrew
\newfontfamily\hebrewfont[Script=Hebrew]{David CLM}

% Add header for the logo on all pages
\pagestyle{fancy}
\fancyhf{}

\fancyhead[L]{\raisebox{-1cm}{\includegraphics[width=3.5cm]{מתמטי-קל לבן מוקטן 20 אחוז.jpeg}}} % Adjust logo position

\fancyhead[R]{\leftmark} % Position section header

% === Number paragraphs and subparagraphs ===
\setcounter{secnumdepth}{5} % enable numbering down to paragraph level
\setcounter{tocdepth}{5}    % include them in TOC

\makeatletter
\renewcommand\theparagraph{\thesubsubsection.\arabic{paragraph}}
\renewcommand\thesubparagraph{\theparagraph.\arabic{subparagraph}}

% Make paragraphs appear like proper small titles (bold + newline and spacing)
\renewcommand\paragraph{\@startsection{paragraph}{4}{\z@}%
   {1.5ex \@plus 1ex \@minus .2ex}% space before
   {0.8ex}% space after (newline)
   {\normalfont\normalsize\bfseries}} % style

\renewcommand\subparagraph{\@startsection{subparagraph}{5}{\z@}%
   {1ex \@plus 1ex \@minus .2ex}% space before
   {0.6ex}% space after
   {\normalfont\normalsize\bfseries}} % style
\makeatother




\begin{document}

\cleardoublepage
\pagestyle{fancy}
\fancyfoot[C]{\thepage}
\pagenumbering{arabic}

\includepdf[pages={1-4}]{ode_cover.pdf}
    
\title{
\LARGE משוואות דיפרנציאליות רגילות למהנדסים וסטודנטים למדעים מדויקים\\[6pt]
\large כל חומר הלימוד האקדמאי בליווי תרגילים עם פתרונות מלאים ובחינות לדוגמה\\[4pt]
\LARGE שנת תשפ״ו (2025)
}

\author{\Large ד׳׳ר רועי בר-און} % Replace with the actual author's name
\date{}
\maketitle

\pagenumbering{gobble}
\newpage

% Add nice title to TOC
\addtocontents{toc}{\protect\begin{center}\Large \protect\end{center}\par}
\tableofcontents

% Start numbering after TOC
\newpage
\pagenumbering{arabic}

\section{מבוא ומוטיבציה}

משוואות דיפרנציאליות רגילות אינן מופיעות רק בספרי לימוד מתמטיים – הן הלב הפועם של כל תחום מדעי והנדסי.  
באמצעותן אנו מתארים כיצד גדלים משתנים עם הזמן, כיצד הם מגיבים לגירויים חיצוניים, ואיך מערכת שלמה יכולה להתפתח מדינמיקה פשוטה יחסית. הלא אנו כבר יודעים מלימודי התיכון (למרות שכבר רמזו לנו על כך בחטיבת הביניים ואף לפני..), כי נגזרת מסמלת את קצב השינוי של הפונקציה. 

נפתח במודל הבסיסי ביותר: \textbf{דינמיקת אוכלוסייה}.  
כאשר אוכלוסייה גדלה ללא מגבלות חיצוניות, מקובל להניח שהקצב שבו היא גדלה פרופורציונלי לגודל הקיים:
\[
y' = k y,
\]
כאשר $k$ הוא מקדם הגדילה ו - $y$ הוא גודל האוכלוסיה.
זהו \emph{מודל הגידול המעריכי}, הפשוט אך גם הבעייתי, משום שהוא צופה גידול אינסופי.  
כדי לתקן זאת, מציעים את \emph{המודל הלוגיסטי}, שבו נוספה מגבלה פנימית – כושר הנשיאה $M$ של הסביבה:
\[
y' = k y\Bigl(1-\tfrac{y}{M}\Bigr).
\]

מעולם הביולוגיה נעבור לפיזיקה יומיומית. דמיינו כוס קפה חמה על השולחן: הקצב שבו טמפרטורת הקפה $y(t)$ יורדת תלוי בהפרש בינה לבין טמפרטורת הסביבה $T_{\text{סביבה}}$.  
כך מתקבל \textbf{חוק הקירור של ניוטון}:
\[
y' = -k\big(y - T_{\text{סביבה}}\big).
\]

גם בהנדסת חשמל המשוואות הללו שולטות. במעגל RC פשוט, המתאר קבל ונגד המחוברים למקור מתח, הקשר בין מתח הקבל $v(x)$ למתח המקור $E(x)$ נתון על ידי:
\[
\tau\, v'(x) + v(x) = E(x), \qquad \tau = RC.
\]
כאן רואים בבירור כיצד נגזרת בזמן קובעת את תגובת המערכת לאות משתנה.

ולבסוף, דוגמה מתחום ההנדסה הכימית: \textbf{מכל ערבוב}.  
נניח מכל בנפח קבוע $V$, שבתוכו נמהל נוזל בריכוז $y(t)$ של חומר מסוים.  
אל המכל נכנס זרם בעל ספיקה $F_{\text{in}}$ וריכוז $C_{\text{in}}$, ויוצא זרם בעל אותה ספיקה $F_{\text{out}}=F_{\text{in}}$ (כדי לשמור על נפח קבוע).  

מאזן החומר על המכל נותן:
\[
\frac{dy}{dt} = \frac{F_{\text{in}}}{V}\,C_{\text{in}}
              - \frac{F_{\text{out}}}{V}\,y(t).
\]

או באופן קומפקטי יותר:
\[
y' = \frac{F}{V}\big(C_{\text{in}} - y\big),
\]
כאשר $F=F_{\text{in}}=F_{\text{out}}$.

דוגמאות אלו מראות כי בכל פעם שאנו עוסקים בשינוי בזמן – בין אם זה גידול אוכלוסיות, התקררות נוזלים, תגובה חשמלית או ערבוב חומרים – משוואות דיפרנציאליות רגילות מספקות את הכלי המתמטי המרכזי לתיאור, ניבוי והבנה של התהליך.

ספר זה מיועד לסטודנטים בשנתם הראשונה והשנייה בתחומי המדעים המדויקים וההנדסה, והוא מבקש להקנות בסיס מתמטי מוצק במשוואות דיפרנציאליות רגילות. המיקוד הוא בשילוב שבין תיאוריה פורמלית וכתיבה מתמטית מוקפדת לבין דגש על הבנה אינטואיטיבית ויכולת יישומית. לאורך הפרקים משולבים נוסחאות ותוצאות חשובות, תרגילים עם \textbf{פתרונות מלאים}, וכן גרפים ואיורים שנועדו להמחיש את עקרונות הפתרון וליצור בהירות מחשבתית. לצד היסודות המתמטיים, הספר מציג דוגמאות רבות מעולמות הפיזיקה, הכימיה, הביולוגיה וההנדסה, תחומים בהם הסטודנט עתיד להיתקל בהמשך לימודיו ובעבודתו המקצועית. מטרת הדוגמאות היא להראות כיצד בעיות ממשיות – מתנועת גוף תחת השפעת כוחות, דרך דינמיקת תגובות כימיות, ועד תיאור של מעגלים חשמליים או מערכות ביולוגיות – יכולות להיבחן ולבוא על פתרונן באמצעות בניית מודל מתמטי בצורת משוואה דיפרנציאלית רגילה. כך יוכל הלומד לפתח לא רק שליטה טכנית בשיטות פתרון, אלא גם תפיסה רחבה יותר של הקשר שבין המתמטיקה התיאורטית לבין עולמות היישום, ולצאת לדרכו האקדמית והמקצועית כשהוא מצויד בכלים מעשיים לניתוח ותכנון מערכות בתחומו.

נדגיש כי ארגז הכלים שתרכשו כאן, כמו גם הכלים שתפגשו בכל קורס למשוואות דיפרנציאליות רגילות,  
מוגבל ביכולתו לפתור בעיות ריאליסטיות ומתקדמות יותר — אך בו־בזמן עוצמתי בצורה בלתי רגילה.  
עם ההבנה הפיזיקלית הנכונה של הבעיה, תראו כי ברוב המקרים המודל המתמטי יתואר לפחות באמצעות משוואה דיפרנציאלית \textbf{חלקית},  
אשר טכניקות הפתרון שלה חורגות מעבר להיקף ספר זה.  
אולם בעזרת הנחות מדויקות ותפיסה עמוקה של טבע הבעיה, נוכל לצמצם אותה — במקרה ה״פחות טוב״ —  
למשוואה דיפרנציאלית \textbf{רגילה} אשר מצריכה פתרון נומרי, ובמקרה ה״טוב״ יותר —  
לפתרון אנליטי סגור, באמצעות הכלים שנרכוש כאן.


\begin{comment}
נדגיש כי ארגז הכלים שתרכשו כאן, כמו גם הכלים שתפגשו בכל קורס למשוואות דיפרנציאליות רגילות,  
מוגבל ביכולתו לפתור בעיות ריאליסטיות ומתקדמות יותר — אך בו־בזמן עוצמתי בצורה בלתי רגילה.  
עם ההבנה הפיזיקלית הנכונה של הבעיה, תראו כי ברוב המקרים המודל המתמטי יתואר לפחות באמצעות משוואה דיפרנציאלית \textbf{חלקית},  
אשר טכניקות הפתרון שלה חורגות מעבר להיקף ספר זה.  
אולם בעזרת הנחות מדויקות ותפיסה עמוקה של טבע הבעיה, נוכל לצמצם אותה — במקרה ה״פחות טוב״ —  
למשוואה דיפרנציאלית רגילה אשר מצריכה פתרון נומרי, ובמקרה ה״טוב״ יותר —  
לפתרון אנליטי סגור, באמצעות הכלים שרכשנו כאן.

כדי לחתום את רעיון המוטיבציה לענף היפהפה הזה במתמטיקה,  
נציג את \textbf{``סיסמת הזהב''} לצליחת כל קורס מד״ר — מה שאני מכנה: \textbf{``דרך המלך''}.  

חמישה צעדים שכל סטודנט חייב לשאול בכל בעיה:  
1. מהו סדר המשוואה?  
2. האם היא לינארית?  
3. האם היא הומוגנית?  
4. האם יש לה מקדמים קבועים?  
5. האם היא מנורמלת?  

כאשר סטודנט מסוגל לענות בביטחון על חמש השאלות הללו — הוא \textbf{על דרך המלך}.  
בנק השיטות שתלמדו כאן יאפשר לכם להתמודד עם כל משוואה דיפרנציאלית רגילה,  
בעלת תנאים מספיקים, אשר ניתן למצוא לה פתרון אנליטי.
\end{comment}

\begin{goldenbox}
כדי לחתום את רעיון המוטיבציה לענף היפהפה הזה במתמטיקה,  
נציג את \textbf{״הסוד״} לצליחת כל קורס מד״ר — מה שאני מכנה: \textbf{׳׳חוק החמש׳׳}.  

חמש שאלות שכל סטודנט חייב לשאול את עצמו בכל בעיה:  
1. מהו סדר המשוואה?  
2. האם היא לינארית?  
3. האם היא הומוגנית?  
4. האם יש לה מקדמים קבועים?  
5. האם היא מנורמלת?  

כאשר אתם הסטודנטים מסוגלים לענות בביטחון על חמש השאלות הללו,  
יחד עם הבנק השיטות שתרכשו כאן, תוכלו להתמודד עם \textbf{כל} משוואה דיפרנציאלית רגילה,  
בעלת תנאים מספיקים, אשר ניתן למצוא לה פתרון אנליטי.
\end{goldenbox}


\newpage

\section{הגדרת המשוואה הדיפרנציאלית הרגילה ומינוחים}\label{sec:what-is-ode}
משוואה דיפרנציאלית רגילה (\LR{Ordinary differential equation - Ode}) היא קשר בין פונקציה נעלמת (המשתנה התלוי של הבעיה) $y=y(x)$ לבין הנגזרות שלה, ומשתנה בלתי תלוי כלשהו (למשל $x$):
\begin{equation}
F\big(x,\,y,\,y',\,y'',\ldots, y^{(n)}\big)=0.
\end{equation}
\textbf{סדר} המשוואה הוא הדרגה הגבוהה ביותר של נגזרת המופיעה בה.

% שקופית: הצורה הכללית של מד״ר מסדר ראשון
הצורה הכללית של מד״ר מסדר \textbf{ראשון} היא
\begin{equation}
y' = f(x,y)
\end{equation}

\noindent עם תנאי התחלה מהצורה:
\begin{equation}
y(x_{0}) = y_{0}
\end{equation}

\noindent פונקציה \( y(x) \) המוגדרת על \( x \in I \in \mathbb{R}
 \) (כאשר \( I \) הוא קטע/אינטרוול)
היא \textbf{פתרון} של \( y' = f(x,y) \) אם
\begin{equation}
y'(x) = f\big(x,\,y(x)\big)
\end{equation}
בתחום ההגדרה \( I\in \mathbb{R} \) לכל \( x \in I \).

באופן כללי, משוואה דיפרנציאלית רגילה מסדר $n$ תיתן פתרון כללי הכולל $n$ קבועים חופשיים. הדבר נובע מכך שכל אינטגרציה מוסיפה קבוע לא ידוע, וכאשר פותרים משוואה מסדר $n$, מבצעים למעשה $n$ אינטגרציות. כדי לעבור מפתרון כללי לפתרון \emph{פרטי/מסוים}, יש צורך ב־$n$ תנאי התחלה (או תנאי שפה) בלתי תלויים:
\[
y(x_0) = y_0, \quad y'(x_0) = y_1, \quad y''(x_0) = y_2, \ \ldots, \ y^{(n-1)}(x_0) = y_{n-1}.
\]

רק באמצעות נתונים אלו ניתן לקבוע את כל הקבועים ולקבל פתרון ייחודי.

\vspace{0.5cm}

\textbf{הגדרת הלינאריות, הומוגניות ונרמול משוואה}

% \vspace{0.5cm}

משוואה היא \textbf{לינארית} אם היא מהצורה
\begin{equation}
a_n(x)\,y^{(n)}+\cdots+a_1(x)\,y'+a_0(x)\,y = g(x),
\end{equation}
כאשר $a_i(x)$ ו-$g(x)$ נתונות.

הצורה הכללית של מד׳׳ר לינארית מסדר ראשון היא
\begin{equation}
a(x)\,y' + b(x)\,y = c(x)
\end{equation}
כאשר $a,b,c$ רציפות על $I$, והיא \textbf{הומוגנית} אם $c(x)\equiv 0$.

במשוואה לינארית $y$ \textbf{וכל} נגזרותיה נמצאות בפעולות לינאריות בלבד.

\vspace{0.5cm}

\textbf{נרמול מד׳׳ר}

הצורה הכללית של מד׳׳ר לינארית מנורמלת מסדר 1 היא
\begin{equation}
y' + p(x)\,y = q(x)
\end{equation}
כאשר $p,q$ רציפות על $I$.
במד׳׳ר מנורמלת, המקדם של הנגזרת בעלת הסדר הגבוה ביותר שווה ל-1.

\newpage

\subsection{סיווג מד״רים}

\textbf{נבצע סיווג מלא של הדוגמאות הבאות על פי: סדר, לינאריות, נרמול, הומוגניות.}

\example{}
$y' + y = 8$

\explanation
סדר: ראשון. לינארית: כן. מנורמלת: כן. הומוגנית: לא (אגף ימין $\neq 0$).

\example{}
$y' + y^{2} = 9$

\explanation
סדר: ראשון. לינארית: לא ($y^2$). מנורמלת: כן. הומוגנית: לא.

\example{}
$y' + \dfrac{1}{y} = 0$

\explanation
סדר: ראשון. לינארית: לא (נוכחות $1/y$). מנורמלת: כן. הומוגנית: כן (אגף ימין אפס).

\example{}
$x y' + y = 0$

\explanation
סדר: ראשון. לינארית: כן. מנורמלת: לא (צריך לחלק ב-$x$). הומוגנית: כן (אגף ימין אפס).

\example{}
$y''' + y' + y = \cos x$

\explanation
סדר: שלישי. לינארית: כן. מנורמלת: כן (מקדם הנגזרת הגבוהה ביותר = 1). הומוגנית: לא ($\cos x$).

\newpage
\underline{תרגילים}

\textbf{סווגו בעצמכם את המד״רים הבאים על פי: סדר, לינאריות, נרמול, הומוגניות.}

\exercise{}
$4y'' + y + 19 = 0$

\exercise{}
$y'' + \sin y = 11$

\exercise{}
$y\,y' = 1$

\exercise{}
$e^x y' + y = 0$

\exercise{}
$y'' + (y')^2 = 0$

%%%CUT%%%

\newpage

\underline{פתרונות}

\solution
$4y'' + y + 19 = 0$  
סדר: שני. לינארית: כן. מנורמלת: \textbf{לא} (מקדם $y''$ הוא $4$). הומוגנית: \textbf{לא} (קיים איבר חופשי $+19$).

\solution
$y'' + \sin y = 11$  
סדר: שני. לינארית: \textbf{לא} ($\sin y$). מנורמלת: כן (מקדם $y''$ הוא $1$). הומוגנית: \textbf{לא} (אגף ימין $\neq 0$).

\solution
$y\,y' = 1$  
סדר: ראשון. לינארית: \textbf{לא} (מכפלה $yy'$). מנורמלת: \textbf{לא} (אין צורה $y' + \ldots$). הומוגנית: \textbf{לא} (אגף ימין $\neq 0$).

\solution
$e^x y' + y = 0$  
סדר: ראשון. לינארית: כן. מנורמלת: \textbf{לא} (מקדם הנגזרת הוא $e^x$). הומוגנית: כן (אגף ימין $=0$).

\solution
$y'' + (y')^2 = 0$  
סדר: שני. לינארית: \textbf{לא} (קיים איבר $(y')^2$). מנורמלת: כן (מקדם $y''$ הוא $1$). הומוגנית: כן (אגף ימין $=0$).

\newpage

\subsection{אימות ושלילת פתרונות של מד״רים}

\begin{example}
\begin{hebrewenum}
  \item[א.] ודאו כי $y = e^{\sin(x)}$ הוא פתרון של המד״ר:
  \[
  y' - \cos(x)\,y = 0.
  \]

  \item[ב.] ודאו כי $y = e^x$ \textbf{איננו} פתרון של אותה משוואה.
\end{hebrewenum}
\end{example}

\explanation
\begin{hebrewenum}
  \item[א.] נחשב את הנגזרת של $y = e^{\sin(x)}$:
  \[
  y' = \cos(x)\,e^{\sin(x)}.
  \]
  נציב במשוואה:
  \[
  y' - \cos(x)\,y = \cos(x)e^{\sin(x)} - \cos(x)e^{\sin(x)} = 0,
  \]
  ולכן $y = e^{\sin(x)}$ הוא פתרון על $\mathbb{R}$.

  \item[ב.] נציב את $y = e^x$ ואת נגזרתו $y' = e^x$:
  \[
  y' - \cos(x)\,y = e^x - \cos(x)\,e^x = e^x(1-\cos x).
  \]
  ביטוי זה מתאפס רק עבור $x=2\pi k, \; k\in\mathbb{Z}$.  
  מאחר שהפתרון חייב להתקיים על \emph{קטע} ולא רק בנקודות בדידות, 
  $y=e^x$ \textbf{איננו} פתרון אמיתי. לפניכם גרף הפתרון:
\end{hebrewenum}

\begin{figure}[H]
    \centering
    \includegraphics[width=0.7\textwidth]{esinx.png}
    \caption{\footnotesize גרף הפתרון $y = e^{\sin(x)}$ .}
    \label{fig:exp_sin}
\end{figure}

\begin{example}
ודאו כי $y=\tfrac{1}{x}$ הוא פתרון של המד״ר:
\[
x y' + y = 0.
\]
\end{example}

\explanation
נחשב:
\[
y(x) = \frac{1}{x}, \qquad y'(x) = -\frac{1}{x^2}.
\]
מדובר במד"ר לינארית מסדר ראשון.  נציב את הפונקציה ונגזרתה במד"ר:
\[
x\cdot\Big(-\frac{1}{x^2}\Big) + \frac{1}{x} = -\frac{1}{x} + \frac{1}{x} = 0.
\]
כלומר הפתרון נכון לכל $x\neq 0$. לכן $y(x)=1/x$ הוא פתרון, אך יש להגדירו על שני קטעים נפרדים:
\[
y(x)=
\begin{cases}
\dfrac{1}{x}, & x>0,\\[6pt] \vspace{6pt} \\[-6pt]
\dfrac{1}{x}, & x<0.
\end{cases}
\]

\begin{remark}
תחום הגדרה יכול להיות מורכב מכמה אינטרוולים.
\end{remark}


\newpage
\section{משוואות מסדר ראשון}

\subsection{שיטת גורם אינטגרציה (ג״א)}
נציג אלגוריתם לפתרון כללי של  מד"ר לינארית מסדר ראשון. נדגיש כאן שנקבל נוסחה \textbf{סגורה} אשר תיתן לנו מענה לכל מד׳׳ר לינארית סדר 1 (הומוגנית או לא הומוגנית) שקיימת.

נרצה לפתח את הנוסחה עבור הצורה המנורמלת:
\begin{equation}\label{if}
y'(x)+p(x)\,y(x)=q(x).
\end{equation}

היינו מאוד רוצים בשלב זה ׳׳פשוט׳׳ לבצע אינטגרל על שני האגפים ולקבל את הפונקציה הנעלמת שלנו (כפי שלא מעט סטודנטים נוטים להציע בשיעור הראשון בקורס המד׳׳ר), אך הדבר לא אפשרי שכן ׳׳נתקע׳׳ עם איבר שאנו לא יודעים לחשב, הלוא הוא $\int p(x)y(x)dx$, שכן אנו לא יודעים את $y(x)$. למעשה, זה מה שאנחנו מחפשים. כיוון ש-$y$ נמצאת באגף שמאל, לצד הנגזרת שלה, הדבר יכול לעורר מוטביציה כלשהי שיכולה לבוא לידי ביטוי בנגזרת של מכפלה. אם רק היינו מצליחים לחשוב איך ׳׳לתקן׳׳ את המקדמים של $y$, הנגזרת שלה, או אפילו של שתיהן, היינו יכולים לקבל (תחת אילוץ מסוים כמובן), ׳׳נגזרת של מכפלה׳׳ אשר מכילה את $y$. במצב כזה, אינטגרל היה ׳׳מסיים את הסיפור׳׳, והיינו מחלצים את הפונקציה הנעלמת שלנו בבעיה, $y$. לשם כך, אפשר לחשוב על
הכפלת משוואה (\ref{if}) באיזושהי פונקציה (ששונה מאפס כמובן) $\mu(x)\neq 0$ , כדי לקבל:
\begin{equation}
\mu(x)y' + \textcolor{blue}{\mu(x)p(x)y} = \mu(x)q(x).
\end{equation}

נרצה שאגף שמאל כאמור יתאים לאיזושהי נגזרת של מכפלה.
נבחר $\mu$ כך שיתקיים:
\[
\textcolor{blue}{\mu(x)p(x)y}=\mu'(x)y\Rightarrow\boxed{\mu'(x)=p(x)\mu(x)}.
\]
נראה כאילו לא עשינו יותר מידי שכן עדיין אנחנו מתמודדים עם מד׳׳ר (ואנחנו עדיין לא יודעים לפתור מד׳׳ר בשלב זה), אך ׳׳במקרה׳׳ נוכל לזהות את התבנית הזו עוד מימי התיכון, והיא:
\[
\mu'(x)=p(x)\mu(x)\;\;\;\Rightarrow \frac{\mu'(x)}{\mu(x)}=p(x) .
\]
אגף שמאל הוא בדיוק הנגזרת המורכבת של לוגריתם בבסיס $e$ ולכן
\[
\frac{\mu'(x)}{\mu(x)}=p(x)=\left(\ln{\mu(x)}\right)' \;\;\;\Rightarrow\;\;\; \boxed{\mu(x)=e^{\int p(x)\,dx}}.
\]
זהו התנאי על גורם האינטגרציה $\mu(x)$ שיבטיח לנו נגזרת של מכפלה בצד שמאל של המד׳׳ר שלנו.
נבצע את ההצבה ונקבל:
\[
(\mu y)' = \mu q(x).
\]

אינטגרציה נותנת:
\[
\mu(x)\,y(x)=\int \mu(x)\,q(x)\,dx + C,
\]
כאשר
$C$ הוא קבוע אינטגרציה.
מכאן הפתרון הכללי:
\begin{equation}
\boxed{y(x)=\frac{1}{\mu(x)}\left(\int \mu(x)\,q(x)\,dx + C\right)},
\end{equation}
כאשר 
\begin{equation}
\mu(x)=e^{\int p(x)\,dx}
\end{equation}

\begin{comment}
\textcolor{red}{
 כלל אצבע  – אם $|\mu(x)|$ ג''א, אז גם $\mu(x)$ (וניתן לעבוד איתו).
בשימוש בנוסחה הסגורה לג׳׳א, חייב לנרמל את המד׳׳ר לפני שניגשים ליישום האלגוריתם!
} נוכיח זאת כעת.
\end{comment}

\begin{ruleofthumb}
אם $|\mu(x)|$ ג״א, אז גם $\mu(x)$ (וניתן לעבוד איתו).  
בשימוש בנוסחה הסגורה לג״א, חייב לנרמל את המד״ר לפני שניגשים ליישום האלגוריתם! נוכיח זאת כעת.
\end{ruleofthumb}


\begin{proof}
נניח כי $\mu(x)$ הוא גורם אינטגרציה עבור המד״ר
\[
y'(x)+p(x)\,y(x)=q(x).
\]
כלומר מתקיים
\[
(\mu(x)y(x))' = \mu(x)q(x).
\]

כעת נניח כי $c\neq 0\in \mathbb{R}$ הוא קבוע כלשהו.  
נכפול את $\mu(x)$ בקבוע זה ונגדיר $\tilde{\mu}(x)=c\,\mu(x)$.  
אז:
\[
(\tilde{\mu}(x)y(x))' = (c\,\mu(x)y(x))' = c(\mu(x)y(x))' = c\,\mu(x)q(x) = \tilde{\mu}(x)q(x).
\]

כלומר גם $\tilde{\mu}(x)$ מקיים את תנאי גורם האינטגרציה.  
בפרט, אם $\mu(x)$ הוא גורם אינטגרציה אז גם $-\mu(x)$ הוא גורם אינטגרציה, ולכן הטענה נכונה גם עבור $|\mu(x)|$ (שמתקבל מהכפלה ב־$\pm 1$ בהתאם לסימן של $\mu$).
\end{proof}



\example{}\label{sinh}
מצאו פתרון פרטי למד״ר הבא עם תנאי ההתחלה:
\[
y' + y = e^x, \qquad y(0)=0.
\]
\label{ex:integrationfactor}


באמצעות שיטת גורם אינטגרציה – פעם אחת ע״י שימוש בנוסחה הסגורה,
ופעם אחת ע״י פיתוח מלא.

% --- Approach 1: Closed formula ---

\explanation

\textbf{פתרון באמצעות הנוסחה הסגורה}

נרצה ב\textbf{כל} משוואה שנרצה לפתור, לשאול את עצמנו \textbf{ארבע}
שאלות עיקריות והן: מה סדר המד׳׳ר? האם היא לינארית? האם היא הומוגנית? האם היא מנורמלת? זו בדיקת שפיות חשובה מאוד שתעזור לנו לסנן שיטות לא מתאימות לפתרון הבעיה שלנו. זה יהפוך קריטי הרבה יותר בשלבים מתקדמים יותר בהם ׳׳בנק׳׳ השיטות שיהיה ברשותנו יהיה גדול דיו. אך, הרגלים טובים מאמצים כמה שיותר מוקדם.
\textbf{ניתוח המשוואה:}

\begin{itemize}
  \item[] סדר: ראשון (מופיעה רק נגזרת $y'$).
  \item[] לינארית: כן (איברי $y$ ו-$y'$ מופיעים בקומבינציה לינארית).
  \item[] מנורמלת: כן (מקדם הנגזרת $y'$ הוא 1).
  \item[] הומוגנית: לא (אגף ימין $q(x)=e^x \neq 0$).
  \item[] נתונים: $p(x)=1$, \ $q(x)=e^x$.
\end{itemize}

נחשב את גורם האינטגרציה:
\[
\mu(x) = e^{\int 1 \, dx} = e^x.
\]

לכן לפי הנוסחה הכללית:
\[
y(x) = \frac{1}{\mu(x)} \left( \int \mu(x)\,q(x)\,dx + C \right).
\]

נציב:
\[
y(x) = \frac{1}{e^x}\left( \int e^x \cdot e^x\, dx + C\right) 
= \frac{1}{e^x}\left( \int e^{2x}\, dx + C \right).
\]

נחשב אינטגרל:
\[
\int e^{2x}\,dx = \tfrac{1}{2}e^{2x}.
\]

ולכן:
\[
y(x)=\tfrac{1}{2}e^x+\frac{C}{e^x}.
\]

נשתמש בתנאי ההתחלה $y(0)=0$:
\[
0 = \tfrac{1}{2}\cdot 1 + C \;\;\;\Rightarrow\;\;\; C=-\tfrac{1}{2}.
\]

\[
\boxed{\,y(x)=\tfrac{1}{2}(e^x-e^{-x})=\sinh(x),\qquad x\in\mathbb{R}.\,}
\]
רובכם פוגשים לראשונה את הפונקציה המיוחדת $\sinh$. נראה כאן סקיצה שלה:

\begin{figure}[H]
    \centering
    \includegraphics[width=0.7\textwidth]{sinhx.png} % export this from MATLAB
    \caption{\footnotesize גרף הפתרון $y = \sinh(x)$.}
    \label{fig:sinhx}
\end{figure}


% --- Approach 2: Full development ---

\textbf{פתרון באמצעות פיתוח מלא}

נחשב את גורם האינטגרציה:
\[
\mu(x) = e^{\int 1 \, dx} = e^x.
\]

נכפיל את המשוואה ב-$\mu(x)$:
\[
e^x y' + e^x y = e^x \cdot e^x \quad \Rightarrow \quad (e^x y)' = e^{2x}.
\]

נבצע אינטגרציה:
\[
\int (e^x y)'\, dx = \int e^{2x}\, dx 
\quad \Rightarrow \quad e^x y = \tfrac{1}{2}e^{2x} + C.
\]

נחלק ב-$e^x$:
\[
y(x) = \tfrac{1}{2}e^x + \frac{C}{e^x}.
\]

נציב תנאי התחלה $y(0)=0$:
\[
0 = \tfrac{1}{2} \cdot 1 + C \;\;\;\Rightarrow\;\;\; C=-\tfrac{1}{2}.
\]

ולכן גם כאן:
\[
\boxed{\,y(x)=\tfrac{1}{2}(e^x-e^{-x})=\sinh(x),\qquad x\in\mathbb{R}.\,}
\]

\example{}
כדור ששוקל $100$ גרם נופל ממנוחה, בנוכחות אוויר המפעיל על הכדור כוח התנגדות הפרופורציונלי למהירות הנפילה עם מקדם התנגדות $k=0.2$.  
התאוצה המופעלת ע״י כוח הכבידה היא $g=10 \tfrac{\text{מ׳}}{\text{ש׳}^2}$.  

\begin{enumerate}[label=\alph*.]
  \item מצאו את המשוואה הדיפרנציאלית המתארת את מהירות הכדור והציגו אותה בצורה $v' = f(t,v)$, כאשר $v$ היא מהירות הכדור ו- $t$ הוא הזמן.
  סווגו את סוג המד״ר.
  \item מצאו פתרון כללי באמצעות \textbf{שיטת גורם אינטגרציה}.
  \item חשבו את מהירות הכדור בזמן $t=2$ שניות.
  \item חשבו את הגובה ההתחלתי של הכדור.
\end{enumerate}

\explanation

\textbf{סעיף א׳} 

נשען על החוק השני של ניוטון: $F = m a$, כאשר $F$ הוא הכוח נטו הפועל על הגוף, $m$ היא מסת הגוף ו- $a$ היא תאוצת הגוף. נזכר בעובדה שקצב שינוי המהירות של גוף, שווה לתאוצה שלו. בנוסף, אם ניקח מערכת ייחוס כלפי מטה, נוכל לומר שקיים כוח חיובי שמוריד את הכוח מטה (כוח הכובד) $F_{g}$ וכוח שמתנגד לצניחת הכדור, הלוא הוא התנגדות האוויר $F_{r}$. מכאן שמאזן הכוחות יתן:
\[
m v' = F_g - F_r = m g - k v.
\]
נציב $m=0.1,\ g=10,\ k=0.2$ ונקבל:
\[
0.1 v' = 1 - 0.2 v \quad \Rightarrow \quad v' = 10 - 2v.
\]

\textbf{סיווג:}
\begin{itemize}
  \item[] סדר: ראשון (מופיעה $v'$ בלבד).
  \item[] לינארית: כן (צורה $a(x)v'+b(x)v=c(x)$).
  \item[] מקדמים: קבועים ($a=1$, $b=2$, $c=10$).
  \item[] הומוגניות: לא ($c(x)\neq 0$).
  \item[] מנורמלת: כן (מקדם $v'$ שווה 1).
\end{itemize}

נכתוב את המד׳׳ר בצורתה הסטנדרטית:
\[
v' + 2v = 10.
\]

\textbf{סעיף ב׳} 

הצורה היא:
\[
v' + p v = q, \quad p=2,\ q=10.
\]

נחשב את גורם האינטגרציה:
\[
\mu(t) = e^{\int p \, dt} = e^{2t}.
\]

נכפיל את המשוואה ב־$\mu(t)$:
\[
e^{2t} v' + 2 e^{2t} v = 10 e^{2t}.
\]

האיבר השמאלי הוא נגזרת של מכפלה:
\[
(e^{2t} v)' = 10 e^{2t}.
\]

אינטגרציה:
\[
e^{2t} v = \int 10 e^{2t}\, dt + C = 5 e^{2t} + C.
\]

נחלק ב־$e^{2t}$:
\[
v(t) = 5 + C e^{-2t}.
\]

\textbf{סעיף ג׳ – מציאת קבוע באמצעות תנאי התחלה}  

ידוע $v(0)=0$ (הכדור התחיל ממנוחה):
\[
0 = 5 + C \cdot 1 \quad \Rightarrow \quad C=-5.
\]

ולכן הפתרון הפרטי הוא:
\[
\boxed{\, v(t) = 5 - 5 e^{-2t}, \qquad t \geq 0 \,}
\]

\[
v(2) = 5 - 5 e^{-4} \approx 5 - 5 \cdot 0.0183 = \boxed{4.91 \ \tfrac{\text{מ׳}}{\text{ש׳}}}.
\]

\textbf{סעיף ד׳}

המיקום מתקבל מאינטגרציה של המהירות:
\[
x(t) = \int v(t)\,dt = \int (5 - 5 e^{-2t})\, dt = 5t + \tfrac{5}{2} e^{-2t} + C.
\]

נציב תנאי ש־$x(2)=0$ (פגיעה בקרקע):
\[
0 = 5\cdot 2 + \tfrac{5}{2} e^{-4} + C \quad \Rightarrow \quad C = -10 - \tfrac{5}{2} e^{-4}.
\]

ולכן:
\[
x(t) = 5t + \tfrac{5}{2} e^{-2t} - 10 - \tfrac{5}{2} e^{-4}.
\]

בפרט:
\[
x(0) = 0 + \tfrac{5}{2}\cdot 1 - 10 - \tfrac{5}{2} e^{-4} \approx -7.5.
\]

כיוון שכיוון התנועה נבחר כלפי מטה, נקבל שהגובה ההתחלתי הוא:
\[
\boxed{\, h \approx \ \text{מטרים} \,7.5 }
\]

\example{}

נבחן מעגל RC מחובר למקור מתח קבוע $E_0$, כאשר $R$ מבטא את ההתנגדות ו־$C$ יכולת הקיבול.  
המתח על הקבל מסומן $v(t)$ ומתפתח לפי המשוואה הדיפרנציאלית:
\[
v'(t)+\frac{1}{\tau}v(t)=\frac{1}{\tau}E_0, 
\qquad \tau=RC.
\]
קבלו פתרון פרטי לבעיה אם ידוע שהמתח ההתחלתי על הקבל הוא $v_{0}$.

\explanation
זוהי משוואה לינארית מסדר ראשון. הפתרון יתקבל בעזרת גורם אינטגרציה.

נחשב גורם אינטגרציה:
\[
\mu(t)=e^{\int \tfrac{1}{\tau}\,dt}=e^{t/\tau}.
\]

נכפיל את המשוואה בג׳׳א:
\[
e^{t/\tau} v'(t) + \frac{1}{\tau} e^{t/\tau} v(t) = \frac{1}{\tau} E_0 e^{t/\tau}.
\]

אגף שמאל הוא נגזרת של מכפלה:
\[
\big(e^{t/\tau}v(t)\big)' = \frac{1}{\tau} E_0 e^{t/\tau}.
\]

נבצע אינטגרציה:
\[
e^{t/\tau}v(t) = E_0 e^{t/\tau} + C.
\]

ולכן הפתרון הכללי הוא:
\[
v(t) = E_0 + C e^{-t/\tau},\qquad t\geq0.
\]

נציב תנאי התחלה $v(0)=v_0$:
\[
v_0 = E_0 + C \quad \Rightarrow \quad C=v_0-E_0.
\]

לכן הפתרון הפרטי יהיה:
\[
\boxed{\, v(t) = E_0 + (v_0-E_0)e^{-t/\tau} ,\qquad t\geq0}
\]

\textbf{פירוש פיזיקלי:}  
הקבל נטען לערך יציב $E_0$ של מתח המקור, כאשר קצב ההתקרבות נקבע ע"י קבוע הזמן $\tau=RC$. נראה כעת את גרף הפתרון עבור פרמטרים ספציפיים:  

\begin{figure}[H]
    \centering
    \includegraphics[width=0.7\textwidth]{rc.png}
    \caption{הגרף של הפתרון $v(t) = 1 - e^{-t/2}$ במעגל RC עבור 
    $E_0=1\,\text{V}$, $R=1\,\Omega$, $C=2\,\text{F}$ ותנאי התחלה $v(0)=0$. 
    הקו האדום המקווקו מייצג את מצב השיווי־משקל $v(t)=E_0=1$.}
    \label{fig:rc_circuit}
\end{figure}


\example{}

נבחן בקבוק יין שמוכנס למקרר, כאשר טמפרטורת הסביבה במקרר קבועה $T_s$.  
הטמפרטורה של היין $T(t)$ מתפתחת לפי חוק הקירור של ניוטון:
\[
T'(t)+kT(t)=kT_s,
\]
כאשר $k>0$ הוא מקדם הקירור.

קבלו פתרון פרטי לבעיה אם ידוע שטמפרטורת היין בתחילה היא $T(0)=T_0$.

\explanation
זוהי משוואה לינארית מסדר ראשון. הפתרון יתקבל בעזרת גורם אינטגרציה.

נחשב גורם אינטגרציה:
\[
\mu(t)=e^{\int k\,dt}=e^{kt}.
\]

נכפיל את המשוואה בג׳׳א:
\[
e^{kt} T'(t) + k e^{kt} T(t) = k T_s e^{kt}.
\]

אגף שמאל הוא נגזרת של מכפלה:
\[
\big(e^{kt}T(t)\big)' = k T_s e^{kt}.
\]

נבצע אינטגרציה:
\[
e^{kt}T(t) = T_s e^{kt} + C.
\]

ולכן הפתרון הכללי הוא:
\[
T(t) = T_s + C e^{-kt},\qquad t\geq0.
\]

נציב תנאי התחלה $T(0)=T_0$:
\[
T_0 = T_s + C \quad \Rightarrow \quad C=T_0-T_s.
\]

לכן הפתרון הפרטי יהיה:
\[
\boxed{\, T(t) = T_s + (T_0-T_s)e^{-kt} ,\qquad t\geq0}
\]

\textbf{פירוש פיזיקלי:}  
טמפרטורת היין מתקרבת באופן אקספוננציאלי לערך היציב $T_s$ של טמפרטורת המקרר, כאשר קצב ההתקרבות נקבע ע"י מקדם הקירור $k$. נראה כעת את גרף הפתרון עבור פרמטרים ספציפיים:  

\begin{figure}[H]
    \centering
    \includegraphics[width=0.7\textwidth]{cooling.png}
    \caption{הגרף של הפתרון $T(t) = 10 + 20 e^{-0.2t}$ עבור 
    $T_s=10^\circ\text{C}$, $T_0=30^\circ\text{C}$, $k=0.2\,\text{ש}^{-1}$. 
    הקו האדום המקווקו מייצג את מצב השיווי־משקל $T(t)=T_s=10^\circ\text{C}$.}
    \label{fig:cooling_law}
\end{figure}

%%%CUT%%%

\newpage
\underline{תרגילים}

\exercise{}
מיכל מלא מים בנפח $V=\text{ליטרים} \,100$, אליו נכנסים מים בריכוז מלח קבוע $c_{in}=0.2 \tfrac{\text{גרם}}{\text{ליטר}}$ בקצב זרימה $Q=5 \tfrac{\text{ליטר}}{\text{דקה}}$. המים מעורבבים היטב, והזרימה החוצה היא באותו קצב $Q$. בתחילת התהליך אין כלל מלח במיכל.
\begin{enumerate}[label=\alph*.]
  \item כתבו את המד״ר המתארת את כמות המלח $y(t)$ במיכל.
  \item סווגו את סוג המד״ר.
  \item מצאו פתרון כללי למשוואה.
  \item מצאו את כמות המלח במיכל אחרי $t=30$ דקות.
\end{enumerate}

\exercise{}
חלקיק נע על ציר $x$ בכיוון חיובי תחת הכוחות: כוח קבוע $F_0=10$ , והתנגדות לינארית לתנועת החלקיק $-k v$ עם $k=3$, כאשר $v$ היא מהירות החלקיק ו-$k$ הוא מקדם ההתנגדות לתנועת החלקיק. הניחו כי כל הפרמטרים הנתונים מכילים את היחידות הנכונות (mks) ואין שום צורך בהמרה.
מסת החלקיק היא $m=1$, גם כן ביחידות המתאימות.
\begin{enumerate}[label=\alph*.]
  \item קבלו את המשוואה $v'(t)=f(t,v)$ למהירות החלקיק.
  \item סווגו את סוג המד״ר.
  \item מצאו פתרון כללי והשתמשו בתנאי התחלה $v(0)=0$ כדי לקבל פתרון פרטי.
  \item חשבו את הגבול $\lim_{t\to\infty} v(t)$ והסבירו את המשמעות הפיזיקלית.
\end{enumerate}

\exercise{}
קבל עם קיבול $C=1F$ ונגד עם התנגדות $R=2\,\Omega$ מחוברים בטור למקור מתח קבוע $E=10V$. 
המטען על הקבל $q(t)$ מקיים את המשוואה הכללית:
\[
R\,q'(t) + \tfrac{1}{C}\,q(t) = E.
\]


\begin{enumerate}[label=\alph*.]
  \item סווגו את סוג המד״ר.
  \item מצאו פתרון כללי.
  \item קבעו את הפתרון $q(t)$ עבור תנאי התחלה $q(0)=0$.
\end{enumerate}


\exercise{}
בקבוק יין מתקרר במקרר. טמפרטורת המקרר היא $T_s=4^\circ C$. בתחילה ($t=0$) הטמפרטורה היא $T(0)=20^\circ C$. מקדם הקירור הוא $k=0.1$. הניחו כי מקדם הקירור תואם לזמן ביחידות של שעות.
\begin{enumerate}[label=\alph*.]
  \item כתבו את המד״ר המתארת את $T(t)$ בעזרת חוק הקירור של ניוטון (החוק נתון בפרק המבוא).
  \item סווגו את המד״ר.
  \item מצאו פתרון כללי.
  \item חשבו את טמפרטורת היין לאחר שעה.
\end{enumerate}

\exercise{}
פתרו את המד״ר:
\[
y'(x) + \frac{2}{x} y(x) = \sin(x), \quad x > 0.
\]
\begin{enumerate}[label=\alph*.]
  \item סווגו את המשוואה.
  \item מצאו את הפתרון הכללי.
  \item מצאו את הפתרון הפרטי עבור $y(1)=0$.
  \item שרטטו את הפתרון באמצעות תוכנת מחשב 
  (
  MATLAB
  ,
  python
  ,
  Maple
  ,
  Mathematica
  ,
  או באמצעות כל תוכנה מתאימה).
\end{enumerate}

\exercise{}
פתרו את המד״ר:
\[
y'(x) - \tan(x)\,y(x) = \cos^2(x), \quad -\tfrac{\pi}{2} < x < \tfrac{\pi}{2}.
\]
\begin{enumerate}[label=\alph*.]
  \item סווגו את המשוואה.
  \item מצאו פתרון כללי.
  \item מצאו פתרון פרטי עבור $y(0)=2$.
  \item שרטטו את הפתרון באמצעות תוכנת מחשב 
  (
  MATLAB
  ,
  python
  ,
  Maple
  ,
  Mathematica
  ,
  או באמצעות כל תוכנה מתאימה).
\end{enumerate}

\exercise{}
פתרו את המד״ר:
\[
y'(x) + \big(1+x\big)y(x) = e^{-x^2/2}.
\]
\begin{enumerate}[label=\alph*.]
  \item סווגו את המשוואה.
  \item מצאו פתרון כללי בעזרת גורם אינטגרציה.
  \item מצאו פתרון פרטי עבור $y(0)=1$.
  \item שרטטו את הפתרון באמצעות תוכנת מחשב 
  (
  MATLAB
  ,
  python
  ,
  Maple
  ,
  Mathematica
  ,
  או באמצעות כל תוכנה מתאימה).
\end{enumerate}

\exercise{}
פתרו את המד״ר:
\[
y'(x) + \alpha y(x) = e^{-\alpha x}, \quad \alpha \in \mathbb{R}.
\]
\begin{enumerate}[label=\alph*.]
  \item סווגו את המשוואה.
  \item מצאו פתרון כללי בעזרת גורם אינטגרציה.
  \item מצאו פתרון פרטי עבור $y(0)=1$.
  \item נתחו את הפתרון עבור ערכים שונים של $\alpha$ (למשל $\alpha=0$, $\alpha>0$, $\alpha<0$). 
  כיצד משפיע הסימן של $\alpha$ על התנהגות הפתרון כאשר $x \to \infty$?
\end{enumerate}

\exercise{}

כוח משתנה $F$ אותו מפעילה הרוח (עם כיוון התנועה) פרופורציונלי לזמן התנועה עם המקדם  
$k = 0.5 \; \tfrac{N}{sec}$.  
מסת הגוף היא $100 \, gr$. מקדם התנגדות התווך בו נע הגוף הוא  
$\mu = 0.2 \; \tfrac{N \cdot sec}{m}$.  

\begin{enumerate}[label=(\alph*)]
\item הציגו את המשוואה הדיפרנציאלית, המתארת את תנועת הגוף (מהירותו).
\item בהנחה שהמהירות ההתחלתית היא $v(0)=0 \; \tfrac{m}{sec}$, מצאו את מהירות הגוף ואת העתקו $6$ שניות מתחילת התנועה.  
\end{enumerate}

\exercise{}\label{ex:newtoncooling}


נבחן את \textbf{חוק הקירור של ניוטון עם קרינה}:
\[
\frac{dT}{dt} = -k\,(T - T_a) - \sigma \,(T^4 - T_a^4),
\]
כאשר $T(t)$ היא הטמפרטורה של הגוף, $T_a$ היא טמפרטורת הסביבה, $k>0$ הוא קבוע הקירור (בהסעה), ו־$\sigma>0$ מייצג את עוצמת איבוד החום ע״י קרינה.

\begin{enumerate}[label=\alph*.] 

  \item נניח כי הטמפרטורה $T(t)$ קרובה ל־$T_a$. הראו את הקירוב באמצעות טור טיילור
  והסבירו כיצד קירוב זה מפשט את האגף הימני של המד׳׳ר.

  \item כתבו את המשוואה המתקבלת אחרי הקירוב. 

  \item פתרו את המשוואה המתקבלת ומצאו ביטוי מפורש ל־$T(t)$ התלוי בקבועים $k,\sigma,T_a$ ובתנאי ההתחלה $T(0)=T_0$.  
  פרטו את שלבי הפתרון.

  \item נתחו את ההתנהגות של $T(t)$ כאשר $t\to\infty$. מה מצופה שיקרה מבחינה פיזיקלית?
\end{enumerate}

\newpage

\underline{פתרונות}

\solution{}
\textbf{סעיף א׳}

נניח כי $y(t)$ היא כמות המלח (בגרמים) במיכל בזמן $t$ (בדקות).  
קצב שינוי המלח במיכל מתקבל לפי:
\[
\frac{dy}{dt} = \text{קצב כניסה} - \text{קצב יציאה}.
\]

\begin{itemize}
  \item[] קצב כניסה: $Q \cdot c_{in} = 5 \cdot 0.2 = 1 \ \tfrac{\text{גרם}}{\text{דקה}}$.
  \item[] קצב יציאה: הריכוז במיכל הוא $\tfrac{y(t)}{V}$, ולכן כמות המלח היוצאת היא $Q \cdot \tfrac{y(t)}{V} = 5 \cdot \tfrac{y(t)}{100} = \tfrac{1}{20}y(t)$.
\end{itemize}

לכן נקבל:
\[
y'(t) = 1 - \tfrac{1}{20} y(t).
\]

\textbf{סעיף ב׳ – סיווג}

\begin{itemize}
  \item[] סדר: ראשון.
  \item[] לינאריות: כן, צורה $y'+p(x)y=q(x)$.
  \item[] הומוגניות: לא (כי $q(x)=1 \neq 0$).
  \item[] מנורמלת: כן (מקדם $y'$ שווה 1).
\end{itemize}

\textbf{סעיף ג׳ – פתרון כללי}

נכתוב:
\[
y' + \tfrac{1}{20} y = 1.
\]

גורם אינטגרציה:
\[
\mu(t) = e^{\int \tfrac{1}{20}\,dt} = e^{t/20}.
\]

נכפיל:
\[
e^{t/20} y' + \tfrac{1}{20} e^{t/20} y = e^{t/20}.
\]

האגף השמאלי הוא נגזרת מכפלה:
\[
\big(e^{t/20} y\big)' = e^{t/20}.
\]

אינטגרציה:
\[
e^{t/20} y = \int e^{t/20}\,dt + C = 20 e^{t/20} + C.
\]

ולכן:
\[
y(t) = 20 + C e^{-t/20}.
\]

תנאי התחלה $y(0)=0$:
\[
0 = 20 + C \quad \Rightarrow \quad C=-20.
\]

נמצא פתרון פרטי:
\[
\boxed{\, y(t) = 20 - 20 e^{-t/20}, \quad t \geq 0 \,}
\]

\textbf{סעיף ד׳ – חישוב עבור $t=30$}

\[
y(30) = 20 - 20 e^{-30/20} = 20 - 20 e^{-1.5}.
\]

נחשב בקירוב:
\[
e^{-1.5} \approx 0.2231 \quad \Rightarrow \quad y(30) \approx 20 - 20 \cdot 0.2231 = \boxed{15.54}.
\]
גרף הפתרון נראה כך:
\begin{figure}[H]
    \centering
    \includegraphics[width=0.7\textwidth]{if_one.png}
    \caption{תהליך ערבוב מלח במיכל: הפתרון $y(t) = 20 - 20e^{-t/20}$ והערך עבור $t=30$ דקות.}
    \label{fig:mixing_tank}
\end{figure}


\solution{}
\textbf{סעיף א׳}

נשתמש בחוק השני של ניוטון: $F = m a = m v'$.  
הכוחות הפועלים:
\[
F_0 - k v = m v'.
\]

נציב את הנתונים $F_0=10$, $k=3$, $m=1$:
\[
10 - 3v = v' \quad \Rightarrow \quad v'(t) = 10 - 3v(t).
\]

\textbf{סעיף ב׳ – סיווג}

\begin{itemize}
  \item[] סדר: ראשון (רק $v'$).
  \item[] לינאריות: כן (צורה $v' + p v = q$).
  \item[] הומוגניות: לא ($q=10 \neq 0$).
  \item[] מנורמלת: כן (מקדם $v'$ שווה 1).
\end{itemize}

\textbf{סעיף ג׳ – פתרון כללי ופרטי}

נכתוב בצורתה הסטנדרטית:
\[
v' + 3v = 10.
\]

גורם אינטגרציה:
\[
\mu(t) = e^{\int 3\,dt} = e^{3t}.
\]

נכפיל:
\[
e^{3t} v' + 2 e^{3t} v = 10 e^{3t}.
\]

האגף השמאלי הוא נגזרת מכפלה:
\[
\big(e^{3t} v\big)' = 10 e^{3t}.
\]

נבצע אינטגרציה:
\[
e^{3t} v = \int 10 e^{3t}\,dt + C = \frac{10}{3} e^{3t} + C.
\]

ולכן:
\[
v(t) = \frac{10}{3} + C e^{-3t}.
\]

תנאי התחלה $v(0)=0$:
\[
0 = \frac{10}{3} + C \quad \Rightarrow \quad C=-\frac{10}{3}.
\]

ולכן:
\[
\boxed{\, v(t) = \frac{10}{3} - \frac{10}{3} e^{-3t}, \quad t \geq 0 \,}
\]

\textbf{סעיף ד׳ – גבול כאשר $t\to\infty$}

\[
\lim_{t\to\infty} v(t) = \frac{10}{3} - \frac{10}{3}\cdot 0 = \frac{10}{3}.
\]

\[
\boxed{\, \lim_{t\to\infty} v(t) = \frac{10}{3} \,}
\]

\textbf{פירוש פיזיקלי:}  
החלקיק מגיע למהירות סופית $v=3\frac{1}{3}\frac{m}{s}$, כתוצאה מאיזון בין הכוח הקבוע $F_0$ לבין כוח ההתנגדות $-kv$.

גרף הפתרון יראה כך:
\begin{figure}[H]
    \centering
    \includegraphics[width=0.7\textwidth]{if_two.png}
    \caption{תנועת חלקיק תחת כוח קבוע והתנגדות לינארית: הפתרון $v(t) = \tfrac{10}{3} - \tfrac{10}{3}e^{-3t}$ והגבול $\lim_{t \to \infty} v(t) = \tfrac{10}{3}$.}
    \label{fig:particle_velocity}
\end{figure}

\solution{}
\textbf{סעיף א׳ – סיווג}

נכתוב את המשוואה עם הנתונים $R=2,\ C=1,\ E=10$:
\[
2q'(t) + q(t) = 10 \quad \Rightarrow \quad q'(t) + \tfrac{1}{2}q(t) = 5.
\]

\begin{itemize}
  \item[] סדר: ראשון (רק $q'$ מופיע).
  \item[] לינארית: כן (צורה $q' + p(x)q = g(x)$).
  \item[] הומוגנית: לא (כי $g(x)=5 \neq 0$).
  \item[] מנורמלת: כן (מקדם $q'$ שווה 1).
\end{itemize}

\textbf{סעיף ב׳ – פתרון כללי}

נפתור:
\[
q' + \tfrac{1}{2} q = 5.
\]

גורם אינטגרציה:
\[
\mu(t) = e^{\int \tfrac{1}{2}\,dt} = e^{t/2}.
\]

נכפיל:
\[
e^{t/2} q' + \tfrac{1}{2} e^{t/2} q = 5 e^{t/2}.
\]

האגף השמאלי הוא נגזרת של מכפלה:
\[
\big(e^{t/2} q\big)' = 5 e^{t/2}.
\]

אינטגרציה:
\[
e^{t/2} q = \int 5 e^{t/2}\,dt + C = 10 e^{t/2} + C.
\]

ולכן:
\[
q(t) = 10 + C e^{-t/2}.
\]

\textbf{סעיף ג׳ – פתרון פרטי עם תנאי התחלה}

נתון $q(0)=0$:
\[
0 = 10 + C \quad \Rightarrow \quad C=-10.
\]

ולכן:
\[
\boxed{\, q(t) = 10 - 10 e^{-t/2}, \quad t \geq 0 \,}
\]
גרף הפתרון נראה כך:
\begin{figure}[H]
    \centering
    \includegraphics[width=0.7\textwidth]{if_three.png}
    \caption{מעגל RC
    : $q(t) = 10 - 10e^{-t/2}$ והגבול $\lim_{t \to \infty} q(t) = 10$.}
    \label{fig:rc_circuit}
\end{figure}

\solution{}
\textbf{סעיף א׳}

חוק הקירור של ניוטון קובע:
\[
\frac{dT}{dt} = -k \big(T - T_s\big).
\]

במקרה הנתון $T_s=4$, $k=0.1$:
\[
\boxed{T'(t) = -0.1\,(T(t)-4)}.
\]

\textbf{סעיף ב׳ – סיווג}

\begin{itemize}
  \item[] סדר: ראשון (רק $T'$ מופיע).
  \item[] לינארית: כן (צורה $T'+p(x)T=q(x)$).
  \item[] הומוגניות: לא ($q(x)\neq 0$).
  \item[] מנורמלת: כן (מקדם $T'$ שווה 1).
\end{itemize}

\textbf{סעיף ג׳ – פתרון כללי}

נכתוב:
\[
T' + 0.1\,T = 0.4.
\]

גורם אינטגרציה:
\[
\mu(t) = e^{\int 0.1\,dt} = e^{0.1t}.
\]

נכפיל:
\[
e^{0.1t} T' + 0.1 e^{0.1t} T = 0.4 e^{0.1t}.
\]

האגף השמאלי הוא נגזרת מכפלה:
\[
\big(e^{0.1t} T\big)' = 0.4 e^{0.1t}.
\]

אינטגרציה:
\[
e^{0.1t} T = \int 0.4 e^{0.1t}\,dt + C = 4 e^{0.1t} + C.
\]

ולכן:
\[
T(t) = 4 + C e^{-0.1t}.
\]

תנאי התחלה $T(0)=20$:
\[
20 = 4 + C \quad \Rightarrow \quad C = 16.
\]

ולכן הפתרון הפרטי:
\[
\boxed{\, T(t) = 4 + 16 e^{-0.1t}, \quad t \geq 0 \,}
\]

\textbf{סעיף ד׳ – טמפרטורה לאחר שעה}

נציב $t=1$:
\[
T(1) = 4 + 16 e^{-0.1 \cdot 1} = 4 + 16 e^{-0.1}.
\]

מאחר ש־$e^{-0.1} \approx 0.9048$:
\[
T(1) \approx 4 + 16 \cdot 0.9048 = 4 + 14.48 = \boxed{18.48^\circ C}.
\]

\textbf{פירוש פיזיקלי:}  
לאחר שעה הטמפרטורה ירדה מ־$20^\circ C$ ל־$18.5^\circ C$ בערך, והיא תמשיך להתקרב בהדרגה לטמפרטורת המקרר $4^\circ C$. גרף הפתרון:
\begin{figure}[H]
    \centering
    \includegraphics[width=0.7\textwidth]{if_four.png}
    \caption{חוק הקירור של ניוטון: הפתרון $T(t) = 4 + 16e^{-0.1t}$ והגבול $\lim_{t \to \infty} T(t) = 4$.}
    \label{fig:wine_cooling}
\end{figure}

\solution{}
\textbf{סעיף א׳ – סיווג}

המשוואה היא:
\[
y'(x) + \frac{2}{x} y(x) = \sin(x), \quad x>0.
\]

\begin{itemize}
  \item[] סדר: ראשון.
  \item[] לינאריות: כן, צורה $y' + p(x)y = g(x)$.
  \item[] הומוגנית: לא ($g(x)=\sin(x)\neq 0$).
  \item[] מנורמלת: כן (מקדם $y'$ שווה 1).
\end{itemize}

\textbf{סעיף ב׳ – פתרון כללי}

נזהה:
\[
p(x) = \tfrac{2}{x}, \quad g(x)=\sin(x).
\]

גורם אינטגרציה:
\[
\mu(x) = e^{\int \tfrac{2}{x}\,dx} = e^{2\ln x} = x^2.
\]

נכפיל:
\[
x^2 y' + 2x y = x^2 \sin(x).
\]

האיבר השמאלי הוא נגזרת מכפלה:
\[
(x^2 y)' = x^2 \sin(x).
\]

אינטגרציה:
\[
x^2 y = \int x^2 \sin(x)\,dx + C.
\]

נחשב את האינטגרל בעזרת אינטגרציה בחלקים:

- נציב $u = x^2,\ dv=\sin(x)\,dx \quad\Rightarrow\quad du=2x\,dx,\ v=-\cos(x)$.

\[
\int x^2 \sin(x)\,dx = -x^2 \cos(x) + \int 2x \cos(x)\,dx.
\]

שוב אינטגרציה בחלקים לאיבר $\int 2x \cos(x)\,dx$:

- נציב $u=2x,\ dv=\cos(x)\,dx \quad\Rightarrow\quad du=2\,dx,\ v=\sin(x)$.

\[
\int 2x \cos(x)\,dx = 2x \sin(x) - \int 2\sin(x)\,dx = 2x \sin(x) + 2\cos(x).
\]

נציב חזרה:
\[
\int x^2 \sin(x)\,dx = -x^2 \cos(x) + 2x \sin(x) + 2\cos(x).
\]

ולכן:
\[
x^2 y = -x^2 \cos(x) + 2x \sin(x) + 2\cos(x) + C.
\]

נחלק ב־$x^2$:
\[
y(x) = -\cos(x) + \frac{2}{x}\sin(x) + \frac{2}{x^2}\cos(x) + \frac{C}{x^2}.
\]
נשים לב כי קיימת בעיית תחום הגדרה כאשר $x=0$, אך מהנתון $x>0$.

\textbf{סעיף ג׳ – פתרון פרטי עם תנאי התחלה}

תנאי: $y(1)=0$.

נציב:
\[
0 = -\cos(1) + 2\sin(1) + 2\cos(1) + C.
\]

כלומר:
\[
0 = \cos(1) + 2\sin(1) + C \quad \Rightarrow \quad C = -\cos(1) - 2\sin(1).
\]

ולכן:
\[
\boxed{\, y(x) = -\cos(x) + \frac{2}{x}\sin(x) + \frac{2}{x^2}\cos(x) + \frac{-\cos(1) - 2\sin(1)}{x^2}, \quad x>0 \,}
\]

\textbf{סעיף ד׳ – סרטוט}

\begin{figure}[H]
    \centering
    \includegraphics[width=0.7\textwidth]{if_five.png}
    \caption{פתרון המד״ר $y'(x)+\tfrac{2}{x}y(x)=\sin(x)$ עם תנאי התחלה $y(1)=0$.}
    \label{fig:ode_ifactor_ex1}
\end{figure}

%%%CUT%%%

\solution{}
\textbf{סעיף א׳ – סיווג}

המשוואה:
\[
y'(x) - \tan(x)\,y(x) = \cos^2(x), \quad -\tfrac{\pi}{2} < x < \tfrac{\pi}{2}.
\]

\begin{itemize}
  \item[] סדר: ראשון.
  \item[] לינארית: כן (צורה $y' + p(x)y = g(x)$).
  \item[] הומוגנית: לא ($g(x)=\cos^2(x)\neq 0$).
  \item[] מנורמלת: כן (מקדם $y'$ שווה 1).
\end{itemize}

\textbf{סעיף ב׳ – פתרון כללי}

נכתוב את המשוואה בצורה הסטנדרטית:
\[
y'(x) + p(x)y(x) = g(x), \quad p(x) = -\tan(x), \ g(x)=\cos^2(x).
\]

גורם אינטגרציה:
\[
\mu(x) = e^{\int -\tan(x)\,dx} = e^{\ln(\cos(x))} = \cos(x).
\]

נכפיל את המד׳׳ר בג׳׳א:
\[
\cos(x) y' - \sin(x) y = \cos^3(x).
\]

האיבר השמאלי הוא נגזרת מכפלה:
\[
(\cos(x) y(x))' = \cos^3(x).
\]

אינטגרציה:
\[
\cos(x) y(x) = \int \cos^3(x)\,dx + C.
\]

נפרק את $\cos^3(x)$:
\[
\cos^3(x) = \cos(x)(1-\sin^2(x)).
\]

נשתמש בהצבה $u=\sin(x), \ du=\cos(x)\,dx$:
\[
\int \cos^3(x)\,dx = \int (1-u^2)\,du = u - \tfrac{u^3}{3} = \sin(x) - \tfrac{\sin^3(x)}{3}.
\]

ולכן:
\[
\cos(x)y(x) = \sin(x) - \tfrac{\sin^3(x)}{3} + C.
\]

נחלק ב־$\cos(x)$:
\[
y(x) = \frac{\sin(x) - \tfrac{1}{3}\sin^3(x) + C}{\cos(x)}, \quad -\tfrac{\pi}{2} < x < \tfrac{\pi}{2}.
\]

\textbf{סעיף ג׳ – פתרון פרטי עם תנאי התחלה}

נתון $y(0)=2$.  

נחשב:
\[
y(0) = \frac{\sin(0) - \tfrac{1}{3}\sin^3(0) + C}{\cos(0)} = \frac{0 - 0 + C}{1} = C.
\]

ולכן $C=2$.  

לכן הפתרון הפרטי:
\[
\boxed{\, y(x) = \frac{\sin(x) - \tfrac{1}{3}\sin^3(x) + 2}{\cos(x)}, \quad -\tfrac{\pi}{2}<x<\tfrac{\pi}{2} \,}
\]

\textbf{סעיף ד׳ – שרטוט פתרון}

\begin{figure}[H]
    \centering
    \includegraphics[width=0.7\textwidth]{if_six.png}
    \caption{פתרון המד״ר $y'(x) - \tan(x)\,y(x) = \cos^2(x)$ עם תנאי התחלה $y(0)=2$.}
    \label{fig:ode_ifactor_ex2}
\end{figure}

\solution{}
\textbf{סעיף א׳ – סיווג}

המשוואה:
\[
y'(x) + (1+x)y(x) = e^{-x^2/2}.
\]

\begin{itemize}
  \item[] סדר: ראשון.
  \item[] לינאריות: כן (צורה $y' + p(x)y = g(x)$).
  \item[] הומוגנית: לא ($g(x)=e^{-x^2/2}\neq 0$).
  \item[] מנורמלת: כן (מקדם $y'$ שווה 1).
\end{itemize}

\textbf{סעיף ב׳ – פתרון כללי}

נזהה את האיברים השונים:
\[
p(x) = 1+x, \quad g(x) = e^{-x^2/2}.
\]

גורם אינטגרציה:
\[
\mu(x) = e^{\int (1+x)\,dx} = e^{x+\frac{x^2}{2}}.
\]

נכפיל את המשוואה:
\[
e^{x+\frac{x^2}{2}} y'(x) + (1+x)e^{x+\frac{x^2}{2}} y(x) = e^{x+\frac{x^2}{2}} e^{-x^2/2}.
\]

האיבר השמאלי הוא נגזרת מכפלה:
\[
\big(e^{x+\frac{x^2}{2}} y(x)\big)' = e^{x}.
\]

אינטגרציה:
\[
e^{x+\frac{x^2}{2}} y(x) = \int e^x\,dx + C = e^x + C.
\]

ולכן:
\[
y(x) = \frac{e^x + C}{e^{x+\frac{x^2}{2}}}.
\]

נפשט:
\[
y(x) = e^{-x^2/2} + C e^{-x-\frac{x^2}{2}}.
\]

\textbf{סעיף ג׳ – פתרון פרטי עם תנאי התחלה}

נתון $y(0)=1$:  
\[
1 = e^{0} + C e^{0} = 1 + C.
\]

מכאן $C=0$.  

ולכן הפתרון הפרטי:
\[
\boxed{\, y(x) = e^{-x^2/2}, \quad x \in \mathbb{R} \,}
\]

\textbf{סעיף ד׳ – שרטוט הפתרון}

\begin{figure}[H]
    \centering
    \includegraphics[width=0.7\textwidth]{if_seven.png}
    \caption{פתרון המד״ר $y'(x) + (1+x)y(x) = e^{-x^2/2}$ עם תנאי התחלה $y(0)=1$.}
    \label{fig:ode_ifactor_ex3}
\end{figure}

\solution{}
\textbf{סעיף א׳ – סיווג}

המשוואה:
\[
y'(x) + \alpha y(x) = e^{-\alpha x}, \quad \alpha \in \mathbb{R}.
\]

\begin{itemize}
  \item[] סדר: ראשון.
  \item[] לינארית: כן (צורה $y' + p(x)y = g(x)$).
  \item[] הומוגנית: לא ($g(x)=e^{-\alpha x}\neq 0$).
  \item[] מנורמלת: כן (מקדם $y'$ שווה 1).
\end{itemize}

\textbf{סעיף ב׳ – פתרון כללי}

גורם אינטגרציה:
\[
\mu(x) = e^{\int \alpha dx} = e^{\alpha x}.
\]

נכפיל:
\[
e^{\alpha x} y'(x) + \alpha e^{\alpha x} y(x) = e^{\alpha x} e^{-\alpha x} = 1.
\]

אגף שמאל הוא נגזרת מכפלה:
\[
\big(e^{\alpha x} y(x)\big)' = 1.
\]

אינטגרציה:
\[
e^{\alpha x} y(x) = x + C.
\]

ולכן:
\[
y(x) = e^{-\alpha x}(x+C).
\]

\textbf{סעיף ג׳ – פתרון פרטי עם תנאי התחלה}

נתון $y(0)=1$:
\[
1 = e^{0}(0+C) = C.
\]

כלומר $C=1$, ולכן:
\[
\boxed{\, y(x) = e^{-\alpha x}(x+1), \quad x \in \mathbb{R} \,}
\]

\textbf{סעיף ד׳ – ניתוח לפי ערכי $\alpha$}

1. מקרה בו $\alpha=0$:  
   המשוואה המקורית הופכת ל־
   \[
   y'(x) = 1 \quad \Rightarrow \quad y(x)=x+1,
   \]
   שהוא גבול נכון של הנוסחה הכללית כאשר $\alpha \to 0$.

2. מקרה בו $\alpha>0$:  
   הפתרון הוא
   \[
   y(x) = (x+1)e^{-\alpha x}.
   \]
   כאשר $x \to \infty$, המכפלה הולכת ל־0 (כי $e^{-\alpha x}$ גובר על $x+1$).  
   \[
   \lim_{x\to\infty} y(x) = 0.
   \]
   כלומר הפתרון מתאפס.

3. מקרה בו $\alpha<0$:  
   נסמן $\alpha=-\beta$ עם $\beta>0$:
   \[
   y(x) = (x+1)e^{\beta x}.
   \]
   במקרה זה, כאשר $x \to \infty$ הפתרון גדל ללא חסם:
   \[
   \lim_{x\to\infty} y(x) = \infty.
   \]

\textbf{סיכום}:  
- $\alpha=0$: פתרון לינארי $y=x+1$.  
- $\alpha>0$: הפתרון מתאפס אסימפטוטית.  
- $\alpha<0$: הפתרון מתבדר לאינסוף.


\solution{}

\begin{enumerate}[label=(\alph*)]

\item
\textbf{שלב 1 – הצגת הכוחות:}  

נזכור כי הקשר בין מיקום, מהירות ותאוצה של הגוף הוא:
\[
x'' = \frac{d^2x}{dt^2} = \frac{dv}{dt} = a(t).
\]

נשתמש גם בחוק השני של ניוטון:  
\[
\sum F = ma.
\]

על הגוף פועלים שני כוחות:
\[
F(t) = kt, \qquad F_{\text{גרר}} = -\mu v.
\]

נוכל לרשום מאזן הכוחות הזה כמד׳׳ר:
\[
kt - \mu v = ma =  m v'.
\]

\textbf{שלב 2 – קבלת המשוואה הדיפרנציאלית:}  

נחלק ב־$m$, נסדר ונקבל:
\[
v'(t) + \frac{\mu}{m} v(t) = \frac{k}{m}t.
\]

זוהי \underline{משוואה לינארית מסדר ראשון}. 

\item

\textbf{שלב 3 – מציאת גורם אינטגרציה:}  

המשוואה היא מהצורה:
\[
v'(t) + p(t) v(t) = q(t), \qquad p(t) = \tfrac{\mu}{m}, \quad q(t) = \tfrac{k}{m}t.
\]

גורם האינטגרציה:
\[
\mu(t) = e^{\int p(t) dt} = e^{\frac{\mu}{m} t}.
\]

\textbf{שלב 4 – פתרון כללי:}  

נשתמש בנוסחה הסגורה:
\[
v(t) = \frac{1}{\mu(t)} \int \mu(t) \cdot q(t) \, dt + \frac{C}{\mu(t)}.
\]

נציב:
\[
v(t) = \frac{1}{e^{\tfrac{\mu}{m} t}} \int e^{\tfrac{\mu}{m} t} \cdot \tfrac{k}{m}t \, dt + \frac{C}{e^{\tfrac{\mu}{m} t}}.
\]

נחשב את האינטגרל בחלקים:

\[
I = \frac{k}{m} \int t \, e^{\tfrac{\mu}{m}t}\, dt.
\]

נבחר:
\[
u = t \quad \Rightarrow \quad du = dt, 
\qquad
dv = e^{\tfrac{\mu}{m}t} dt \quad \Rightarrow \quad v = \frac{m}{\mu} e^{\tfrac{\mu}{m}t}.
\]

לפי אינטגרציה בחלקים:
\[
I = \frac{k}{m}\Big( u v - \int v\, du \Big).
\]

נציב:
\[
I = \frac{k}{m}\left[ t \cdot \frac{m}{\mu} e^{\tfrac{\mu}{m}t} - \int \frac{m}{\mu} e^{\tfrac{\mu}{m}t} dt \right].
\]

נחשב את האינטגרל:
\[
\int \frac{m}{\mu} e^{\tfrac{\mu}{m}t} dt 
= \frac{m}{\mu} \cdot \frac{m}{\mu} e^{\tfrac{\mu}{m}t} 
= \frac{m^2}{\mu^2} e^{\tfrac{\mu}{m}t}.
\]

ולכן:
\[
I = \frac{k}{\mu} t e^{\tfrac{\mu}{m}t} - \frac{km}{\mu^2} e^{\tfrac{\mu}{m}t}.
\]

נחזיר אל הפתרון הכללי:
\[
v(t) = \frac{1}{e^{\tfrac{\mu}{m}t}} \left( I + C \right),
\]

ולכן:
\[
v(t) = \frac{1}{e^{\tfrac{\mu}{m}t}} 
\left( \frac{k}{\mu} t e^{\tfrac{\mu}{m}t} - \frac{km}{\mu^2} e^{\tfrac{\mu}{m}t} + C \right).
\]

נפשט:
\[
v(t) = \frac{k}{\mu}\left( t - \frac{m}{\mu} \right) + C e^{-\tfrac{\mu}{m}t}.
\]


\textbf{שלב 5 – פתרון פרטי עם תנאי התחלה:}  

מהנתון $v(0)=0$ נקבל:
\[
0 = \frac{k}{\mu}\Big(0 - \tfrac{m}{\mu}\Big) + C \;\;\Rightarrow\;\; C = \frac{km}{\mu^2}.
\]

ולכן:
\[
v(t) = \frac{k}{\mu}\Big(t - \tfrac{m}{\mu}\Big) + \frac{km}{\mu^2} e^{-\tfrac{\mu}{m}t}.
\]

נציב ערכים: $k=0.5, \; m=0.1, \; \mu=0.2$.  
\[
v(t) = 2.5t - 1.25 + 1.25 e^{-2t}.
\]

\textbf{שלב 6 – חישוב מהירות והעתק לאחר 6 שניות:}  

\[
v(6) = 2.5 \cdot 6 - 1.25 + 1.25e^{-12} \approx 13.75 \;\;\tfrac{m}{sec}.
\]

כעת נחשב העתק ע״י אינטגרציה:  
\[
x(t) = \int v(t) dt = 1.25t^2 - 1.25t - 0.625e^{-2t} + D.
\]

מאחר ש־$x(0)=0 \;\Rightarrow\; D=0.625$, נקבל:
\[
x(t) = 1.25t^2 - 1.25t - 0.625e^{-2t} + 0.625.
\]

ולכן:
\[
x(6) \approx 38.12 \;\; m.
\]
\end{enumerate}

\solution{}

\begin{enumerate}[label=\alph*.]

\item \textbf{קירוב באמצעות טור טיילור.}  

נפתח את $T^4$ סביב $T=T_a$:  
\[
T^4 = T_a^4 + 4T_a^3 (T-T_a) + \mathcal{O}((T-T_a)^2).
\]

ולכן:
\[
T^4 - T_a^4 \;\approx\; 4T_a^3 (T-T_a).
\]

נחזיר זאת למשוואה המקורית:
\[
\frac{dT}{dt} \;\approx\; -k(T-T_a) - \sigma \cdot 4T_a^3 (T-T_a).
\]

\item \textbf{המשוואה לאחר הקירוב.}  

קיבלנו:
\[
\frac{dT}{dt} = -\big(k+4\sigma T_a^3\big)(T-T_a).
\]

זו משוואה לינארית מסדר ראשון עם מקדמים קבועים.  
ניתן לפתור אותה באמצעות ג׳׳א.

\item

\textbf{פתרון המשוואה.}  

מדובר במשוואה דיפרנציאלית מסדר ראשון, לינארית, לא הומוגנית.  

נסמן \(\lambda = k+4\sigma T_a^3\). נקבל:
\[
T'(t) + \lambda T(t) = \lambda T_a.
\]

\textbf{שלב א׳ – חישוב גורם אינטגרציה}  
\[
\mu(t) = e^{\int \lambda dt} = e^{\lambda t}.
\]

\textbf{שלב ב׳ – הכפלה בגורם}  
\[
e^{\lambda t}T'(t) + \lambda e^{\lambda t}T(t) = \lambda T_a e^{\lambda t}.
\]

\textbf{שלב ג׳ – זיהוי נגזרת מלאה}  
\[
\frac{d}{dt}\!\Big(e^{\lambda t}T(t)\Big) = \lambda T_a e^{\lambda t}.
\]

\textbf{שלב ד׳ – אינטגרציה}  
\[
e^{\lambda t}T(t) = T_a e^{\lambda t} + C.
\]

\textbf{שלב ה׳ – חלוקה בגורם והצבת תנאי התחלה}  
\[
T(t) = T_a + Ce^{-\lambda t}, \qquad T(0)=T_0 \;\Rightarrow\; C=T_0-T_a.
\]
לכן הפתרון הכללי יהיה
\[
\boxed{\, T(t) = T_a + (T_0-T_a)e^{-\lambda t} \,,\qquad t\geq0}
\]
תחום ההגדרה נובע מכך שאנחנו מחשבים את תחילת מעבר החום ב- $t=0$.


\item \textbf{התנהגות אסימפטוטית.}  

כאשר $t \to \infty$ מתקבל:
\[
\lim_{t \to \infty} T(t) = T_a.
\]

\textbf{פירוש פיזיקלי:} הגוף מתקרר בהדרגה עד שהוא מגיע בדיוק לטמפרטורת הסביבה.  
קצב ההתקררות מושפע גם מהמנגנון הלינארי ($k$) וגם מהמנגנון הלא־לינארי שקירבנו ($\sigma$).
\end{enumerate}
נראה את הפתרון בצורה גרפית:
\begin{figure}[H]
    \centering
    \includegraphics[width=0.7\textwidth]{radiation.png}
    \caption{עקומות הפתרון 
    $T(t) = T_a + (T_0 - T_a)e^{-\lambda t}$ 
    עבור ערכי פרמטר שונים של $\lambda$, 
    יחד עם הפתרון הסינגולרי $T(t)\equiv T_a$ המוצג בקו מקווקו בצבע תכלת. 
    ניתן לראות כי לכל ערך של $\lambda$ הפתרון מתכנס אסימפטוטית אל $T_a$.}
    \label{fig:radiation}
\end{figure}
\noindent
כפי שניתן לראות באיור~\ref{fig:radiation}, ככל שערכו של $\lambda$ גדול יותר, 
הפתרון מתכנס מהר יותר לפתרון האסימפטוטי $T(t)\equiv T_a$. 
במילים אחרות, הגדלת $\lambda$ גורמת לכך שהעקומה ``נשברת'' מוקדם יותר אל הטמפרטורה הסביבתית. 
מבחינה פיזיקלית, ערך גדול של $\lambda$ מייצג שילוב חזק יותר של מנגנוני מעבר החום:
הולכת חום בהסעה (דרך הקבוע $k$) ואיבוד חום בקרינה (דרך $\sigma$). 
לכן עבור ערכים גדולים יותר של $\lambda$, קצב הקירור גבוה יותר והגוף מגיע לטמפרטורת הסביבה מהר יותר.

%%%CUT%%%

\newpage
\subsection{שיטת וריאציית הפרמטר}

נעסוק כעת בשיטה כללית למציאת פתרון פרטי למשוואה לינארית מסדר ראשון בעזרת \textbf{וריאציית הפרמטר}.  

נסתכל על הצורה המנורמלת:
\begin{equation}
y'(x) + p(x) \, y(x) = q(x), \qquad x \in I.
\end{equation}

ניתן לפרק את המד׳׳ר לשני חלקים : חלק הומוגני וחלק לא הומוגני.

נתחיל בבחינת המשוואה ההומוגנית המתאימה:
\[
y'(x) + p(x)\, y(x) = 0.
\]

פתרונה הוא:
\begin{equation}
y_H(x) = d_1 e^{-\int p(x)\,dx}, \qquad x \in I,
\end{equation}
כאשר $d_1$ הוא קבוע ממשי.

עבור המשוואה הלא הומוגנית:
\[
y'(x) + p(x)\,y(x) = q(x),
\]
נרצה למצוא פתרון פרטי $y_p(x)$.  

נניח כי ניתן לרשום פתרון פרטי בצורה דומה לפתרון ההומוגני, אלא שבמקום הקבוע $d_1$ נציב פונקציה $C(x)$:
\begin{equation}
y_p(x) = C(x)\,y_H(x) = C(x)e^{-\int p(x)\,dx}.
\end{equation}

ניתן לראות שעשינו וריאציה (שינוי בעברית) לקבוע  (פרמטר) שהיה בפתרון ההומוגני, לפונקציה בפתרון הפרטי. מכאן השם של השיטה.
כיצד מגלים את $C(x)$? 

נגזור:
\[
y_p'(x) = C'(x)e^{-\int p(x)\,dx} - p(x)C(x)e^{-\int p(x)\,dx}.
\]

נציב במשוואה המקורית $y' + p(x)y = q(x)$, ונקבל:
\[
C'(x)e^{-\int p(x)\,dx}- p(x)C(x)e^{-\int p(x)\,dx}+p(x)C(x)e^{-\int p(x)\,dx} = q(x).
\]

ולכן:
\[
C'(x) = q(x)\,e^{\int p(x)\,dx}.
\]

ולבסוף, נבצע אינטגרציה:
\begin{equation}
\boxed{C(x) = \int q(x)\,e^{\int p(x)\,dx}\,dx}.
\end{equation}

על מנת למצוא את הפתרון הכללי למשוואה, נחבר את הפתרון ההומוגני והפרטי:
\begin{equation}
\boxed{y(x) = y_H(x) + y_p(x) 
= d_1 e^{-\int p(x)\,dx} + \left(\int q(x)e^{\int p(x)\,dx}\,dx\right)\,e^{-\int p(x)\,dx}}.
\end{equation}
חיבור זה מבטא את \textbf{עקרון הסופרפוזיציה}, אותו נוכיח עכשיו.

\begin{proof}
נבחן את המשוואה הלינארית מהצורה:
\[
y'(x) + p(x)y(x) = q(x).
\]

נסמן את האופרטור הלינארי
\[
L[y] := y'(x) + p(x)y(x).
\]

האופרטור $L$ הוא לינארי, כלומר:
\[
L[\alpha y_1 + \beta y_2] = \alpha L[y_1] + \beta L[y_2], \qquad \forall \alpha,\beta \in \mathbb{R}.
\]

\textbf{שלב 1 – פתרון הומוגני.}  
אם $y_H$ הוא פתרון הומוגני, מתקיים:
\[
L[y_H] = y_H'(x) + p(x)y_H(x) = 0.
\]

\textbf{שלב 2 – פתרון פרטי.}  
אם $y_p$ הוא פתרון פרטי, מתקיים:
\[
L[y_p] = y_p'(x) + p(x)y_p(x) = q(x).
\]

\textbf{שלב 3 – חיבור הפתרונות.}  
נבחן את $y(x) = y_H(x) + y_p(x)$.  
לפי לינאריות $L$:
\[
L[y] = L[y_H + y_p] = L[y_H] + L[y_p].
\]

אבל $L[y_H]=0$ ו־$L[y_p]=q(x)$, ולכן:
\[
L[y] = 0 + q(x) = q(x).
\]

\textbf{מסקנה:}  
אם $y_H$ פתרון הומוגני ו־$y_p$ פתרון פרטי, אז $y = y_H + y_p$ הוא פתרון שלם למשוואה.  
כלומר \textbf{עקרון הסופרפוזיציה} נובע ישירות מלינאריות האופרטור $L$.
\end{proof}

\begin{remark}
בוריאציית הפרמטר חייב לנרמל לפני שניגשים ליישום האלגוריתם!
\end{remark}
נציג כאן בקצרה דוגמה אחת בלבד, שכן יוקדש תת-פרק שלם לשיטת וריאציית הפרמטר עבור מקרה כללי של מד׳׳ר מסדר $n$. 
\begin{example}

פתרו את המד׳׳ר הבאה באמצעות שיטת וריאציית הפרמטר:
\[
y' + y = e^x, \qquad y(0)=0.
\]

\end{example}

\explanation{}
שימו לב כי זו המשוואה מדוגמה (\ref{sinh}) אשר פתרנו בשיטת ג׳׳א. אנו מצפים כמובן לקבל פתרון זהה.

\textbf{החלק ההומוגני:}
\[
y_H(x) = d_1 e^{-x}.
\]

\textbf{החלק הפרטי:}  
נניח $y_p(x) = C(x)e^{-x}$.  
אז:
\[
C'(x)e^{-x} = e^x \quad \Rightarrow \quad C'(x) = e^{2x}.
\]
ולכן:
\[
C(x) = \int e^{2x}dx = \frac{e^{2x}}{2}.
\]
נציב:
\[
y_p(x) = \frac{e^{2x}}{2} e^{-x} = \frac{e^x}{2}.
\]

\textbf{הפתרון הכללי:}
\[
y(x) = d_1 e^{-x} + \frac{e^x}{2}.
\]

נציב תנאי התחלה $y(0)=0$:
\[
0 = d_1 \cdot 1 + \tfrac{1}{2} \quad \Rightarrow \quad d_1 = -\tfrac{1}{2}.
\]

ולכן:
\[
\boxed{\, y(x) = \frac{1}{2}\big(e^x - e^{-x}\big) = \sinh(x), \qquad x \in \mathbb{R} \,}
\]

\newpage
\subsection{שיטת החלפת תפקידי $x$ ו-$y$}

נבחן משוואה דיפרנציאלית שבה קשה לטפל בצורתה הרגילה:
\begin{equation}
y'(x) = F(x,y).
\end{equation}

במקרים מסוימים, נוח יותר להפוך את המשתנים $x$ ו-$y$ כך שנכתוב את המשוואה בצורתה ההפוכה:
\[
\textcolor{blue}{y'}=\frac{dy}{dx}=\frac{1}{\frac{dx}{dy}} = \textcolor{blue}{\frac{1}{x'(y)} = F(x,y)}.
\]

כעת אנו רואים כי $x$ נחשב כמשתנה התלוי ו-$y$ כמשתנה הבלתי תלוי.  
השיטה שימושית במיוחד כאשר נקבל משוואה מהצורה:
\begin{equation}
y' = \frac{a}{g(y) - bx}, (a,b\in\mathbb{R})
\end{equation}
שבה קשה לבודד את $y$. אולם אם נהפוך את התפקידים, נקבל:
\[
x' = \frac{1}{a}g(y) - \frac{b}{a}x,
\]
שזוהי \textbf{משוואה לינארית} ב-$x(y)$ שמתאימה לצורה הכללית:
\[
x' + \frac{b}{a}x = \frac{1}{a}g(y),
\]
שעבורה יש לנו נוסחה בשיטת ג׳׳א או בשיטת וריאצית הפרמטר. נוסחה סגורה לפתרון מג׳׳א:
\begin{equation}
\boxed{x(y)=\frac{1}{\mu(y)}\left(\int \mu(y)\frac{1}{a}\,g(x)\,dy + C\right)},
\end{equation}
כאשר ג׳׳א שווה ל:
\begin{equation}
\boxed{\mu(y)=e^{\int \frac{b}{a}}=e^{\frac{b}{a}y}}.
\end{equation}

\begin{remark}
\textcolor{red}{השיטה של החלפת המשתנים $x \leftrightarrow y$ ׳׳כשרה׳׳ אם ורק אם הפונקציה $y(x)$ 
\textbf{הפיכה מקומית}, כלומר מונוטונית בתחום בו אנו פותרים. במילים אחרות, חייב להתקיים $y'(x) \neq 0$ בתחום זה, כדי שנוכל להגדיר את $x(y)$.}
\end{remark}

\begin{ruleofthumb}
אם ניתן לחלץ בצורה סבירה את $y(x)$ נעשה זאת. 
צורה סבירה אומרת כי מדובר בכמה פעולות אריתמטיות בודדות הכוללות בדרך כלל 
חיבור/חיסור, כפל/חילוק, חזקה/שורש, פונקציה הופכית מעולם הפונקציות האלמנטריות.
\end{ruleofthumb}



\example{}
קבלו פתרון פרטי לבעיה הבאה:
\[
y' = \frac{1}{e^y - x}, \qquad y(x_0) = y_0,\,\,\,\,\,\,\,\, x_{0},y_{0}\in\mathbb{R}
\]
\explanation
זוהי מד׳׳ר לא לינארית \textbf{בכיוון $y$}, אך אם נחליף תפקידים, נקבל:
\[
x' = e^y - x.
\]

זוהי משוואה לינארית בצורתה:
\[
x'(y) + x(y) = e^y.
\]

שימו לב שזו המשוואה מדוגמה  (\ref{sinh}), רק עם תפקידים הפוכים של המשתנה התלוי והבלתי תלוי. 
ולכן למעשה, נוכל להשתמש בפתרון הכללי אותו כבר מצאנו:
\[
x(y) = \frac{1}{2} e^y + C e^{-y}.
\]

כעת נפעיל את תנאי ההתחלה שלנו כדי לקבל פתרון פרטי.
נתון: 

$y(x_0)=y_0 \;\;\Rightarrow\;\; x(y_0)=x_0$. \textbf{הערה על תנאי ההתחלה:} 
תנאי ההתחלה נתון בצורה $y(x_0)=y_0$. 
כאשר אנו מחליפים בין $x$ ל־$y$, נרצה לרשום אותו כ־$x(y_0)=x_0$. 
הצדקה לכך נשענת על משפט הפונקציה ההפוכה: 
אם $y'(x_0)\neq 0$, אזי $y(x)$ הפיכה מקומית סביב $x_0$ 
ולכן $x(y)$ מוגדרת שם היטב. 
בנוסף, משפט קיום ויחידות (אותו נלמד בפרקים הבאים) מבטיח שעקומת הפתרון העוברת דרך 
$(x_0,y_0)$ היא יחידה, כך שהתנאי הראשוני תקף באותה מידה בין אם 
נרשום אותו כ־$y(x_0)=y_0$ או כ־$x(y_0)=x_0$. ניתן גם לחשוב על כך שהנקודה $(x_{0},y_{0})$ חייבת לשבת על העקום שפותר את המד׳׳ר.


נציב:
\[
x_0 = \frac{1}{2} e^{y_0} + C e^{-y_0}.
\]

ולכן:
\[
C = \Big(x_0 - \tfrac{1}{2} e^{y_0}\Big)e^{y_0}.
\]
הפתרון הכללי מתקבל בצורה הסתומה (בכיוון $y$), אך בצורה מפורשת בכיוון $x$:
\[
\boxed{\, x(y) = \frac{1}{2} e^y + \Big(x_0 - \tfrac{1}{2} e^{y_0}\Big) e^{-(y-y_0)} \,, \qquad e^y\neq x, \qquad y\in\mathbb{R}}.
\]

\textbf{האם ניתן לחזור ל־$y(x)$?}  
במבט ראשון הפתרון מופיע כ־$x(y)$, אך כאן יש דרך סבירה לחזור ל־$y(x)$.  
נכפיל את שני האגפים ב־$e^y$ ונקבל:
\[
x e^y = \tfrac{1}{2} e^{2y} + \Big(x_0 - \tfrac{1}{2} e^{y_0}\Big) e^{y_0}.
\]

נסמן $t = e^y > 0$, ואז המשוואה הופכת לריבועית ב־$t$:
\[
\tfrac{1}{2} t^2 - x t + \Big(x_0 - \tfrac{1}{2} e^{y_0}\Big) e^{y_0} = 0.
\]

פתרון ריבועי עבור $t$ הוא:
\[
t = e^y = \frac{x \pm \sqrt{x^2 - 2\Big(x_0 - \tfrac{1}{2} e^{y_0}\Big)e^{y_0}}}{1}.
\]

לבסוף נקבל:
\[
\boxed{y(x) = \ln\!\left( x \pm \sqrt{x^2 - 2\Big(x_0 - \tfrac{1}{2} e^{y_0}\Big)e^{y_0}} \,\right), \qquad x\neq e^y, \qquad x\in\mathbb{R}}.
\]

יש לבחור את הסימן $\pm$ לפי תנאי ההתחלה כדי להבטיח שהפתרון עובר דרך $(x_0,y_0)$.  

\textbf{מסקנה פדגוגית:}  
במקרה זה אפשר אמנם לחזור לצורה מפורשת $y(x)$ באמצעות פתרון משוואה ריבועית, אך ברוב המקרים שמופיעים בשאלות סטנדרטיות הדבר אינו מתאפשר בצורה פשוטה. לכן מקובל להשאיר את הפתרון בצורת $x(y)$ אלא אם כן מתקבלת צורה נוחה במיוחד להפיכה חזרה.

\newpage
\underline{תרגילים}

\exercise{}
פתרו את הבעיה:
\[
y' = \frac{2}{y - x}, \qquad y(0)=1.
\]

\exercise{}
קבלו פתרון כללי למד׳׳ר הבאה:
\[
y' = \frac{3}{e^y - 2x}.
\]

\exercise{}
פתרו את הבעיה:
\[
y' = \frac{1}{\cos(y) - x}, \qquad y(0)=0.
\]

%%%CUT%%%

\newpage
\underline{פתרונות}
\solution{}
בצורתה הנתונה קשה לפתור את המשוואה עבור $y(x)$. נבצע החלפת משתנים, כלומר נתייחס ל־$x$ כפונקציה של $y$. מתקבל:
\[
x'(y) = \frac{1}{2}(y - x).
\]
זוהי משוואה לינארית מהצורה
\[
x'(y) + \tfrac{1}{2}x(y) = \tfrac{1}{2}y.
\]

כדי לפתור אותה נחשב גורם אינטגרציה:
\[
\mu(y) = e^{\int \tfrac{1}{2}\,dy} = e^{y/2}.
\]
נכפיל את המשוואה ב־$\mu(y)$:
\[
e^{y/2} x'(y) + \tfrac{1}{2} e^{y/2} x(y) = \tfrac{1}{2} y e^{y/2}.
\]
האיבר השמאלי הוא נגזרת של $e^{y/2}x(y)$ ולכן
\[
\big(e^{y/2}x(y)\big)' = \tfrac{1}{2} y e^{y/2}.
\]
נבצע אינטגרציה, ובעזרת אינטגרציה בחלקים נקבל
\[
\int \tfrac{1}{2} y e^{y/2}\,dy = (y-2)e^{y/2}.
\]
מכאן:
\[
e^{y/2} x(y) = (y-2)e^{y/2} + C,
\]
ולכן
\[
x(y) = y-2 + C e^{-y/2}.
\]

כעת נפנה לתנאי ההתחלה. במקור נתון $y(0)=1$. כדי להשתמש בו, עלינו להניח ש־$y(x)$ הפיכה מקומית סביב $x=0$ (כלומר $y'(0)\neq 0$). תחת הנחה זו נרשום $x(y_0)=x_0$, כלומר $x(1)=0$. הצבה נותנת:
\[
0 = 1-2 + C e^{-1/2} \quad \Rightarrow \quad C = e^{1/2}.
\]
בסופו של דבר קיבלנו:
\[
\boxed{\, x(y) = y - 2 + e^{(1-y)/2} \,}.
\]

\textbf{דיון על תחום ההגדרה:}  
המשוואה המקורית $y' = \tfrac{2}{y-x}$ אינה מוגדרת כאשר $y=x$.  
נשאל האם הפתרון $x(y)$ עשוי לחצות את הישר $y=x$.  
לשם כך נציב $x(y)=y$:
\[
y - 2 + e^{(1-y)/2} = y \;\;\;\Rightarrow\;\;\; e^{(1-y)/2} = 2.
\]
מכאן נקבל
\[
y = 1 - 2\ln 2, \qquad x=y.
\]
אכן, מתקבלת נקודת חיתוך יחידה:
\[
(x,y) = (1-2\ln 2,\,1-2\ln 2).
\]

בנקודה זו המכנה $y-x$ מתאפס ולכן המשוואה הדיפרנציאלית עצמה אינה מוגדרת, 
ומכאן שזוהי נקודת גבול לתחום ההגדרה. 
מעבר לכך, עקומת הפתרון נשארת תמיד בצד אחד של הישר $y=x$ — מעליו או מתחתיו — 
בהתאם למיקום התנאי ההתחלתי. 

במקרה שלנו, התנאי ההתחלתי $(0,1)$ נמצא באזור $y>x$, 
ולכן ענף הפתרון הרלוונטי נשאר בתחום זה ואינו חוצה את $y=x$. 
לכן הפתרון הפרטי הוא:
\[
\boxed{\, x(y) = y - 2 + e^{(1-y)/2}, \qquad y\in\mathbb{R} ,\qquad y > 1-2\ln 2 \,}.
\]


נציג את גרף הפתרון:
\begin{figure}[H]
    \centering
    \includegraphics[width=0.7\textwidth]{one.png}
    \caption{הפתרון $x(y) = y - 2 + e^{(1-y)/2}$ מוצג יחד עם הישר $x=y$. 
    הענף המתאים לתנאי ההתחלה $(0,1)$ מודגש בכחול, 
    ואילו הענף הלא־מתאים מוצג באפור. 
    נקודת התנאי ההתחלתי $(0,1)$ מסומנת באדום, 
    ונקודת החיתוך הייחודית $(1-2\ln 2,\,1-2\ln 2)$ מסומנת בירוק.}
    \label{fig:xofy_branches}
\end{figure}



כעת נשאל האם ניתן להפוך חזרה ל־$y(x)$. במקרה זה הפתרון מתקבל כמשוואה $x=f(y)$ שכוללת גם $y$ וגם $e^{-y/2}$, ולכן הפיכת המשתנים חזרה דורשת פתרון טרנסצנדנטי. מאחר שאין כאן פתרון פשוט בצורת $y(x)$, נשאיר את התשובה כ־$x(y)$. לפי כלל האצבע, זהו מקרה שבו \textbf{נשארים עם הפתרון הסתום}.



\solution{}
נבצע החלפת משתנים ונקבל:
\[
x'(y) = \tfrac{1}{3} e^y - \tfrac{2}{3}x(y).
\]
המשוואה לינארית:
\[
x'(y) + \tfrac{2}{3}x(y) = \tfrac{1}{3}e^y.
\]

נחשב גורם אינטגרציה:
\[
\mu(y) = e^{\int \tfrac{2}{3}\,dy} = e^{2y/3}.
\]
נכפיל ונקבל:
\[
(e^{2y/3}x(y))' = \tfrac{1}{3} e^{5y/3}.
\]
אינטגרציה נותנת:
\[
e^{2y/3}x(y) = \tfrac{1}{5} e^{5y/3} + C,
\]
ולכן:
\[
x(y) = \tfrac{1}{5} e^y + C e^{-2y/3}.
\]

נשתמש בתנאי ההתחלה. נתון $y(0)=0$, כלומר $x(0)=0$. כדי לתרגם ל־$x(y)$ נניח שוב הפיכות מקומית של $y(x)$ (כלומר $y'(0)\neq 0$). נציב $x(0)=0$:
\[
0 = \tfrac{1}{5} + C \quad \Rightarrow \quad C=-\tfrac{1}{5}.
\]

לכן:
\[
\boxed{\, x(y) = \tfrac{1}{5}\big(e^y - e^{-2y/3}\big). \,}
\]

\textbf{דיון על תחום ההגדרה:}  
המשוואה המקורית אינה מוגדרת כאשר $e^y - 2x = 0$, כלומר על העקום:
\[
x = \tfrac{1}{2} e^y.
\]
הפתרון שלנו חייב להישאר כולו בצד אחד של עקום זה.  
נבדוק את נקודת ההתחלה $(0,0)$: עבור $y=0$ נקבל ש $\frac{1}{2}e^{y}$ נמצא בנקודה $(\tfrac{1}{2},0)$, ואילו הפתרון נותן $x(0)=0 < \tfrac{1}{2}$.  

כלומר, הפתרון הפרטי שנמצא תקף תמיד בתחום
\[
x(y) < \tfrac{1}{2}e^y,
\]
בהתאם לתנאי ההתחלה. לכן הפתרון הפרטי המתאים הוא:
\[
\boxed{\, x(y) = \tfrac{1}{5}\big(e^y - e^{-2y/3}\big), \qquad y\in\mathbb{R} ,\qquad \tfrac{1}{2}e^y > x. \,}
\]

כאן גם כן הפתרון מופיע בצורה סתומה $x(y)$. חזרה ל־$y(x)$ אינה אפשרית באופן סביר, שכן המשוואה מערבת $e^y$ ו־$e^{-2y/3}$. לכן, לפי כלל האצבע, נשאר עם הצגה של $x(y)$.

נציג את גרף הפתרון:
\begin{figure}[H]
    \centering
    \includegraphics[width=0.7\textwidth]{two.png}
    \caption{הגרף של הפתרון $x(y) = \tfrac{1}{5}\left(e^y - e^{-2y/3}\right)$ 
    מוצג בכחול, יחד עם עקום הסינגולריות $x=\tfrac{1}{2}e^y$ בקו שחור מקווקו. 
    נקודת התנאי ההתחלתי $(0,0)$ מסומנת באדום. הפתרון נשאר כולו מתחת לעקום הסינגולריות (בתחום המסורטט לפחות) בהתאם לתנאי ההתחלה.}
    \label{fig:solution_vs_singularity}
\end{figure}
\begin{remark}
    שימו לב כי רזולוציות אלו של בדיקה לא טריוויאלית של איפוס המכנה (לא על ידי מספר ספציפי, אלא ע׳׳י קשר פונקציונלי של ממש כמו במקרה זה) לא נדרשת ברוב המוחלט של המקרים מסטודנטים. הדבר מובא כאן למען חשיבה ׳׳מחוץ לקופסה׳׳ אשר תשרת אתכם בהקשר לפתרונות סינגולריים ואסימפטוטיקה.
\end{remark}

\solution{}
נבצע החלפת משתנים ונקבל:
\[
x'(y) = \cos(y) - x(y).
\]
המשוואה לינארית מהצורה:
\[
x'(y) + x(y) = \cos(y).
\]

גורם אינטגרציה הוא:
\[
\mu(y) = e^{\int 1\,dy} = e^y.
\]
לאחר הכפלה נקבל:
\[
(e^y x(y))' = e^y \cos(y).
\]
נבצע אינטגרציה:
\[
\int e^y \cos(y)\,dy = \tfrac{1}{2} e^y(\sin(y)+\cos(y)).
\]
ולכן:
\[
e^y x(y) = \tfrac{1}{2} e^y(\sin(y)+\cos(y)) + C,
\]
ומכאן:
\[
x(y) = \tfrac{1}{2}(\sin(y)+\cos(y)) + C e^{-y}.
\]

כעת נפעיל את תנאי ההתחלה. נתון $y(0)=0$, ולכן $x(0)=0$ (שוב בהנחת הפיכות מקומית של $y(x)$ סביב $x=0$). הצבה נותנת:
\[
0 = \tfrac{1}{2}(0+1) + C \quad \Rightarrow \quad C=-\tfrac{1}{2}.
\]

הפתרון הפרטי יהיה:
\[
\boxed{\, x(y) = \tfrac{1}{2}(\sin(y)+\cos(y)) - \tfrac{1}{2} e^{-y}. \,}
\]

כאן הפתרון מערב שילוב של סינוס, קוסינוס ואיבר מעריכי, כך שאין דרך פשוטה לבודד את $y(x)$. לפיכך נשארים עם הצגת $x(y)$. לפי כלל האצבע, זהו מקרה שבו \textbf{הפתרון נשאר בצורתו הסגורה $x(y)$}.

\textbf{דיון על תחום ההגדרה:}  
במשוואה המקורית המכנה הוא $\cos(y)-x$, ולכן המשוואה אינה מוגדרת על העקום
\[
x = \cos(y).
\]
נבדוק את נקודת ההתחלה $(0,0)$: שם מתקבל שעקום הסינגולריות הוא $(1,0)$, 
בעוד הפתרון נותן $x(0)=0 < 1$.  
מכאן שהפתרון מקיים $x<\cos(y)$. הפתרון הפרטי אם כך מוגדר בצורה הבאה:
\[
\boxed{\, x(y) = \tfrac{1}{2}(\sin(y)+\cos(y)) - \tfrac{1}{2} e^{-y}, \qquad y\in\mathbb{R} ,\qquad \cos(y) > x. \,}
\]
גרף הפתרון נראה כך:
\begin{figure}[H]
    \centering
    \includegraphics[width=0.7\textwidth]{three.png}
    \caption{גרף הפתרון $x(y)=\tfrac{1}{2}(\sin(y)+\cos(y))-\tfrac{1}{2}e^{-y}$ 
    יחד עם העקום הסינגולרי $x=\cos(y)$. הנקודה הירוקה מקווה את תנאי ההתחלה של הבעיה}
    \label{fig:cos_case}
\end{figure}
שימו לב כי הפונקציות נחצות כמה פעמים. בעצם בכל הנקודות האלו, המד׳׳ר לא מוגדרת! למעשה, פתרון מוגדר היטב יהיה כזה שיכיל את תנאי ההתחלה, אך לא יחצה את העקום הסינגולרי.

%%%CUT%%%

\newpage
\subsection{משוואות פרידות}

משוואה \textbf{פרידה} מסדר 1 עונה לצורה הכללית הבאה:
\begin{equation}
y' = f(x)\cdot g(y).
\end{equation}

נציג כעת את האלגוריתם לפתרון וקבלת נוסחה סגורה באמצעות מספר שלבים:
\begin{enumerate}
  \item נחפש את השורשים של $g(y)$ (כלומר נפתור את $g(y)=0$). 
  אם $y_0$ שורש של $g$, כלומר $g(y_0)=0$, אז $y \equiv y_0$ הוא פתרון סינגולרי/נוסף/קבוע (חשוד לפחות). שימו לב כי פתרון זה פשוט מאפס את שני אגפי המד׳׳ר, ועל כן הוא פתרון, בין אם נוכיח בהמשך כי הוא סינגולרי או לא.
  באופן כללי:  
\begin{itemize}
  \item[] פתרון \textbf{ סינגולרי} הוא פתרון שמתקבל כחשוד כאשר בודקים את השורשים של $g(y)=0$.
  \item[] פתרון \textbf{סינגולרי אמיתי} הוא פתרון חשוד כזה שאינו נובע מביטוי הפתרון הכללי (כלומר אינו מתקבל בצורה טבעית משלב 3 ע׳׳י הצבה של קבוע מסוים).
\end{itemize}
  \item נגדיר פונקציות:
  \[
  G(y) = \int \frac{1}{g(y)}\,dy, 
  \qquad 
  F(x) = \int f(x)\,dx.
  \]
  \item הפתרון הכללי מתקבל בצורה סתומה (לא כולל את הפתרונות הסינגולריים):
  \begin{equation}
  G(y) = F(x) + C.
  \end{equation}
\end{enumerate}
הפתרון כאן נחשב ׳׳סתום׳׳ שכן על פניו, הפונקציה הנעלמת שלנו $y$ שוכנת באגף שמאל ובשלב זה אנו עוד לא יודעים האם ניתן לחלץ אותה בצורה מפורשת או לא. 
\vspace{0.5cm}

\textbf{מקרה פרטי: משוואות אוטונומיות} 

מקרה פרטי חשוב כאשר $f(x)\equiv 1$.  
במצב זה המשוואה נקראת \textbf{משוואה אוטונומית}, והיא נראית כך:
\begin{equation}
y' = g(y).
\end{equation}

האלגוריתם שהוצג קודם הופך קל יותר בהתאם:  
\begin{enumerate}
  \item ראשית, נבדוק את השורשים של $g(y)$, כלומר נפתור את $g(y)=0$. כל שורש כזה מגדיר פתרון קבוע $y(x)\equiv y_0$. זהו פתרון סינגולרי (ולעיתים גם אמיתי).
  \item נגדיר פונקציה:
  \[
  G(y) = \int \frac{1}{g(y)}\,dy.
  \]
  \item הפתרון הכללי מתקבל מהשוואה:
  \[
  G(y) = x + C.
  \]
\end{enumerate}

\noindent כלומר, עבור משוואה אוטונומית פתרון סתום מתקבל בצורה \textbf{סגורה}:
\begin{equation}
\boxed{
\int \frac{1}{g(y)}\,dy = x + C}.
\end{equation}

\begin{insight}
    שימו לב כי תחום ההגדרה של כל מד׳׳ר אוטונומית הוא אוטומטית $x\in\mathbb{R}$.
\end{insight}

\example{}
קבלו פתרון כללי למשוואה:
\[
y' = -\frac{(x^2+1)(y^2-1)}{xy}.
\]

\explanation
נבחין שניתן לכתוב:
\[
y' = \textcolor{blue}{-\frac{x^2+1}{x}}\cdot \textcolor{red}{\frac{y^2-1}{y}} = \textcolor{blue}{f(x)}\cdot \textcolor{red}{g(y)}.
\]

\textbf{שלב א׳ – פתרונות סינגולריים:}  
נבדוק מתי $g(y)=0$:
\[
\frac{y^2-1}{y}=0 \quad \Rightarrow \quad y=\pm 1.
\]
אם כן, $y\equiv \pm 1$ הם פתרונות סינגולריים (חשודים בשלב זה).

\textbf{שלב ב׳ – אינטגרציה:}  
נחשב:
\[
\int \frac{1}{g(y)}\,dy = \int \frac{y}{y^2-1}\,dy,
\qquad
\int f(x)\,dx = \int -\frac{x^2+1}{x}\,dx.
\]

באמצעות פירוק שברים:
\[
\frac{y}{y^2-1} = \frac{1}{2}\left(\frac{1}{y-1}+\frac{1}{y+1}\right).
\]

ולכן:
\[
G(y) = \tfrac{1}{2}\ln|y-1| + \tfrac{1}{2}\ln|y+1| = \tfrac{1}{2}\ln|y^2-1|+C_{1}.
\]

באופן דומה:
\[
F(x) = -\int \Big(x+\tfrac{1}{x}\Big)\,dx = -\tfrac{1}{2}x^2 - \ln|x|+C_{2}.
\]

\textbf{שלב ג׳ – פתרון כללי:}  
הפתרון הכללי (בצורה סתומה) מתקבל:
\[
\tfrac{1}{2}\ln|y^2-1|+C_{1} = -\tfrac{1}{2}x^2 - \ln|x|+C_{2} + C.
\]

את הקבועים $C_{1},C_{2},C$ נוכל לאחד לקבוע יחיד:
\[
\tfrac{1}{2}\ln|y^2-1| = -\tfrac{1}{2}x^2 - \ln|x| + C^*.
\]

\textbf{שלב ד׳ – בידוד:}  
ניתן לחשוב כי פה בעצם יש לעצור (כי למעשה פתרנו את הבעיה). אך, עם כמה פעולות אריתמטיות (יחסית פשוטות), ובזהירות רבה, נוכל לקבל את הפתרון המבוקש ב׳׳כיוון׳׳ המיוחל ($y(x)$).
נבודד את הביטוי עם $y$. נכפיל את המשוואה ב-2:
\[
\ln|y^2-1| = -x^2 - 2\ln|x| + C^{**}.
\]
נשים לב כי $C^{**}=2C^{*}$. נוכל אף אם נרצה לקרוא לקבוע זה שוב $C$. הוכיחו לעצמכם שזה לא באמת משנה ובסוף בכל מקרה, תקבלו את אותו הפתרון.
כעת נעביר לאקספוננט:
\[
|y^2-1| = e^{-x^2}\cdot e^{-2\ln|x|}\cdot e^{C^{**}}.
\]

נשים לב ש־$e^{-2\ln|x|} = \tfrac{1}{x^2}$. לכן:
\[
|y^2-1| = \tfrac{1}{x^2} e^{-x^2}\cdot e^{C^{**}}.
\]

נסמן $C_1 = \pm e^{C^{**}} \neq 0$:
\[
y^2-1 = \tfrac{C_1 e^{-x^2}}{x^2}.
\]

ולכן:
\[
y^2 = \tfrac{C_1 e^{-x^2}}{x^2} + 1.
\]

\textbf{שלב ה׳ – פתרון מפורש:}  
ניקח שורש:
\[
\boxed{\, y(x) = \pm \sqrt{\tfrac{C_1 e^{-x^2}}{x^2}+1}, \qquad C_1 \neq 0,\; x \neq 0. \,}
\]

\textbf{שלב ו׳ – תחום ההגדרה והפתרונות הסינגולריים:}  
יש להוסיף את הפתרונות הסינגולריים שנמצאו קודם: $y=\pm 1$, וכן להדגיש את תחום ההגדרה $x\neq 0$.  
בנוסף, אם מתקיים ערך מסוים של $C_1$ עבורו הביטוי מתאפס או נהיה שלילי, הפתרון המתקבל אינו ממשי ולכן נדרש להגביל את תחום $x$. כלומר, הפתרון המלא הוא:
\[
\boxed{\, y(x) = \pm \sqrt{\tfrac{C_1 e^{-x^2}}{x^2}+1}, \qquad y\equiv\pm1, \qquad C_1 \neq 0,\; x \neq 0. \,}
\]
ניתן לרשום את הפתרון ׳׳בבת אחת׳׳ עם ׳׳שחרור׳׳ האילוץ על $C_1$ ולהבין כי הפתרונות החשודים להיות סינגולריים הם לא באמת סינגולריים, אם אנחנו רושמים את הפתרון בצורה הקומפקטית הבאה:

\[
\boxed{\, y(x) = \pm \sqrt{\tfrac{C_1 e^{-x^2}}{x^2}+1},\qquad x \neq 0. \,}
\]
שימו לב כי כעת, אם יתברר כי $C_1=0$, אז נקבל באופן אוטומטי את הפתרון שהיו חשודים להיות סינגולריים.

נציג כעת את גרף הפתרון עבור כמה קבועים שונים:
\begin{figure}[H]
    \centering
    \includegraphics[width=0.7\textwidth]{sep_one.png}
    \caption{עקומות הפתרון $y(x)=\pm \sqrt{\frac{C_1 e^{-x^2}}{x^2}+1}$ עבור ערכי פרמטר שונים $C_1=\pm 2$, 
    יחד עם הפתרונות הסינגולריים $y=\pm 1$. ניתן לראות את ההתפצלות לשני ענפים ואת תחום ההגדרה המוגבל.}
    \label{fig:sep_one}
\end{figure}

\subsubsection{בעיות גדילה ודעיכה}

\textbf{עקרון מרכזי:}  
ידוע כי בהרבה תהליכים בטבע, קצב השינוי של כמות החומר \textbf{פרופורציונלית} לכמות החומר הנוכחית.  
נסמן $f(t)$ ככמות החומר בזמן $t$, אזי:
\begin{equation}
f'(t) = \frac{df}{dt} = \pm k f(t)
\end{equation}

עבור תהליך \textbf{דעיכה} נקבל:
\[
\frac{df}{dt} = -k f
\quad \Rightarrow \quad
\ln |f| = -kt + C
\quad \Rightarrow \quad
f(t) = e^{-kt+C}
\]

עבור תהליך \textbf{גדילה} נקבל:
\[
\frac{df}{dt} = k f
\quad \Rightarrow \quad
\ln |f| = kt + C
\quad \Rightarrow \quad
f(t) = e^{kt+C}
\]

שימו לב שמדובר במקרה הפשוט ביותר ללא כל התחשבות ב\textbf{קיבולת הנשיאה} של האוכלוסיה.

\example{}
 
נוקליד של רדיום $226$ מתפרק במהירות פרופורציונלית לכמות החומר.  
ידוע כי במשך $1600$ שנים מתפרקת מחצית הכמות ההתחלתית של רדיום.

\begin{enumerate}[label=\alph*.]
  \item לאחר כמה שנים יתפרקו $20\%$ מהכמות ההתחלתית של החומר?  
  \item איזה אחוז של רדיום יישאר לאחר $300$ שנים?
\end{enumerate}

\explanation

נסמן את ריכוז החומר בזמן $t$ ב־$c(t)$ ואת הריכוז ההתחלתי ב־$c_0$.  
מודל הדעיכה נותן:
\[
c'(t) = \frac{dc}{dt} = -k c(t)
\]

\textbf{אלגוריתם לפתרון:}  
1. מחפשים את השורשים של $g(c)$ (כאן $g(c)=c$ ולכן $c \equiv 0$ פתרון סינגולרי חשוד).  
2. נגדיר פונקציות:
\[
G(c) = \int \frac{1}{g(c)}\, dc, 
\qquad 
B(t) = \int b(t)\, dt
\]
3. נרשום את הפתרון הכללי (ללא הפתרונות הסינגולריים):  
\[
G(c) = B(t) + D
\]

\textbf{פתרון המשוואה:}  
\[
\frac{dc}{c} = -k dt
\quad \Rightarrow \quad
\ln |c| = -kt + D
\quad \Rightarrow \quad
c(t) = c_0 e^{-kt}
\]

\textbf{שימוש בתנאי המחצית:}  
לאחר $1600$ שנים נותרת חצי כמות:  
\[
\frac{1}{2}c_0 = c_0 e^{-k \cdot 1600}
\quad \Rightarrow \quad
k = \frac{\ln(2)}{1600} \approx 0.000433 \,\tfrac{1}{\text{years}}
\]

\textbf{פתרון סעיף א׳:}  
נבקש שיתפרקו $20\%$ מהכמות ההתחלתית, כלומר יישאר $0.8c_0$:  
\[
0.8c_0 = c_0 e^{-kt}
\quad \Rightarrow \quad
t = \frac{\ln(0.8)}{-0.000433} \approx 515 \,\text{years}
\]

\textbf{פתרון סעיף ב׳:}  
לאחר $300$ שנים:  
\[
c(300) = c_0 e^{-0.000433 \cdot 300}
\quad \Rightarrow \quad
\frac{c(300)}{c_0} \approx 0.878
\]

\textbf{תשובה סופית:}  
\begin{itemize}
  \item[(א)] $t \approx 515$ שנים עד שיתפרקו $20\%$.  
  \item[(ב)] לאחר $300$ שנים נשארו $\approx 87.8\%$ מהחומר.  
\end{itemize}
נראה את סרטוט הפתרון עם הנקודות ה׳׳מעניינות׳׳ על גביו:
\begin{figure}[H]
  \centering
  \includegraphics[width=0.7\textwidth]{decay.png}
  \caption{עקומת דעיכה $c(t)=c_0 e^{-kt}$ עם נקודות ציון מעניינות: זמן מחצית החיים $(t=1600)$, 
  זמן בו התפרקו $20\%$ $(t \approx 515)$, וריכוז לאחר $300$ שנים $(c(300) \approx 0.878)$.}
\end{figure}

*שימו לב כי ניתן לפתור מד׳׳ר זו באמצעות שיטת גורם האינטגרציה (שכן המד׳׳ר לינארית) וכמו כן בשיטת וריאציית הפרמטר.

\example{}

ידוע כי מהירות הקירור של גוף כלשהו פרופורציונלית להפרש הטמפרטורות בין הגוף לסביבה.  
גוף בטמפרטורה של $300^\circ C$ הונח בכמות מים גדולה מאוד בטמפרטורה של $20^\circ C$.  

ידוע כי כעבור $10$ דקות טמפרטורת הגוף הייתה $150^\circ C$.  
קבעו כעבור כמה זמן טמפרטורת הגוף תהיה $50^\circ C$?

\explanation

1. נסמן: $f(t)$ הטמפרטורה ההתחלתית של הגוף כעבור $t$ דקות.  
   תנאי ההתחלה: 
   \[
   f(0) = 300, \qquad 20 < f(t) \leq 300
   \]

2. לפי הנתון:  
   \[
   f'(t) = \frac{df}{dt} = k \cdot (f-20)
   \]

3. נפתור את המשוואה הדיפרנציאלית:  
   \[
   \frac{df}{dt} = k(f-20) 
   \quad \Rightarrow \quad
   \frac{df}{f-20} = k \, dt
   \]
   \[
   \Rightarrow \quad \ln(f-20) = kt + C
   \quad \Rightarrow \quad
   f(t) - 20 = e^{kt+C}
   \]
   \[
   f(t) = 20 + e^C \cdot e^{kt}
   \]

4. נציב את תנאי ההתחלה $f(0)=300$:  
   \[
   300 = 20 + e^C 
   \quad \Rightarrow \quad
   e^C = 280
   \]
   ולכן:  
   \[
   f(t) = 20 + 280e^{kt}
   \]

\textbf{מציאת $k$:}  
מהנתון $f(10)=150$:  
\[
150 = 20 + 280 e^{10k}
\quad \Rightarrow \quad
e^{10k} = \tfrac{130}{280} = \tfrac{13}{28}
\]
\[
10k = \ln\left(\tfrac{13}{28}\right) 
\quad \Rightarrow \quad
k = \frac{1}{10}\ln\left(\tfrac{13}{28}\right) \approx -0.077
\]

\textbf{הפתרון הפרטי:}  
\[
f(t) = 20 + 280e^{\tfrac{\ln(13/28)}{10}t} \;\; \approx \;\; 20 + 280e^{-0.077t}
\]
 
כעת נדרוש $f(t)=50$:  
\[
50 = 20 + 280e^{-0.077t}
\quad \Rightarrow \quad
\frac{30}{280} = e^{-0.077t}
\]
\[
e^{-0.077t} = \tfrac{3}{28}
\quad \Rightarrow \quad
t = \frac{\ln(3/28)}{-0.077} \approx 29.1
\]

\textbf{תשובה סופית:}  
טמפרטורת הגוף תהיה $50^\circ C$ לאחר כ־$\boxed{\text{ דקות}\,\,29.1 }$.

להלן גרף הפתרון:
\begin{figure}[H]
\centering
\includegraphics[width=0.7\textwidth]{temp.png}
\caption{חוק ניוטון לקירור גוף. הפתרון $f(t) = 20 + 280e^{-0.077t}$ מתאר את הטמפרטורה כפונקציה של הזמן. מסומנות הנקודות $f(0)=300^\circ$, $f(10)=150^\circ$, והנקודה $f(t)=50^\circ$ ($t\approx 29.1$ דקות). }
\end{figure}

*שימו לב כי ניתן לפתור מד׳׳ר זו באמצעות שיטת גורם האינטגרציה (שכן המד׳׳ר לינארית) וכמו כן בשיטת וריאציית הפרמטר.

\example{}
במרכז מחקר מסוים עוקבים אחרי אוכלוסיית חיידקים הגדלה בתוך מבחנה.  
תחילה, כאשר מספר החיידקים קטן, הם מתרבים בקצב אקספוננציאלי. אולם בהדרגה, עקב מגבלות מקום ומשאבים, קצב הגידול הולך וקטן עד אשר האוכלוסייה מתייצבת סביב ערך מרבי $M$ הנקרא \textbf{קיבולת נשיאה}.  

מודל מתמטי לתיאור מצב זה הוא משוואת הגידול הלוגיסטי:
\[
y' = k y \left(1-\tfrac{y}{M}\right),
\]
כאשר:
- $y(t)$ מתאר את גודל האוכלוסייה בזמן $t$,  
- $k>0$ הוא קצב הגידול הראשוני,  
- $M>0$ הוא גודל האוכלוסייה המקסימלי שהסביבה יכולה לתמוך בו.  
\begin{enumerate}[label=\alph*.]
    \item מצאו את הפתרון הכללי למשוואה באמצעות שיטת הפרדת משתנים.  
    \item נניח שבזמן $t=0$ מספר החיידקים ההתחלתי היה $y(0)=y_0$. מצאו את הפתרון הפרטי המתאים.  
\end{enumerate}



\explanation{}
א.

\textbf{שלב א׳ – נזהה את סוג המשוואה:}  

זו משוואת פרידה: $y' = f(t)\cdot g(y)$ עם $f(t)=k$ ו־$g(y)=y(1-y/M)$.

\textbf{שלב ב׳ – נבצע הפרדת משתנים:}
\[
\frac{dy}{y(1-y/M)} = k\,dt.
\]

\textbf{שלב ג׳ – אינטגרציה:}  
כדי לחשב את האינטגרל בצד שמאל נפרק לשברים חלקיים:
\[
\frac{1}{y(1-y/M)} = \frac{1}{y} + \frac{1}{M-y}.
\]
ולכן:
\[
\int \frac{dy}{y(1-y/M)} = \int \left(\frac{1}{y} + \frac{1}{M-y}\right)dy 
= \ln|y| - \ln|M-y| + C.
\]

\textbf{שלב ד׳ – חוקי לוגים ובידוד:}

נפעיל את חוק חיסור הלוגריתמים ונקבל
\[
\ln\!\left(\frac{y}{M-y}\right) = kt + C.
\]

נעבור לאקספוננט:
\[
\frac{y}{M-y} = A e^{kt}, \qquad A = e^C >0.
\]

\textbf{שלב ה׳ – פתרון כללי עבור $y$:}
\[
y(x) = \frac{M}{1 + A e^{-kt}}, \qquad t\geq0.
\]

ב.

\textbf{שלב ו׳ – קביעת $A$ לפי תנאי התחלה:} $y(0)=y_0$, אזי:
\[
y_0 = \frac{M}{1+A}, \quad \Rightarrow \quad A = \frac{M-y_0}{y_0}.
\]

\textbf{פתרון פרטי:}
\[
\boxed{\, y(t) = \frac{M}{1 + \tfrac{M-y_0}{y_0} e^{-kt}} , \qquad t\geq0}
\]

לפניכם סרטוט המתאר כמה תרחישים שונים הכוללים מקדמי גידול שונים וכושרי נשיאה שונים:

\begin{figure}[H]
    \centering
    \includegraphics[width=0.7\textwidth]{part2.png}
    \caption{עקומות הפתרון של פונקציית הצמיחה הלוגיסטית 
    \(
    y(x)=\frac{M}{1+\frac{M-y_0}{y_0}e^{-kx}}
    \)
    עבור ערכים שונים של הפרמטרים $M=\{5000,8000,10000\}$ ו־$k=\{0.25,0.5\}$, 
    עם ערך התחלתי $y_0=1000$. ניתן לראות את נקודת ההתחלה $(0,1000)$ ואת התקרבות כל עקומה לערך הרוויה שלה $M$.}
    \label{fig:logistic_growth}
\end{figure}

%%%CUT%%%

\newpage
\underline{תרגילים}

\exercise{}
קבלו פתרון פרטי למד״ר הבאה:
\[
3y = y'x, \qquad y(1)=\tfrac{1}{2}.
\]

\exercise{}

\begin{enumerate}[label=\alph*.]
  \item מצאו את העקומה שבכל נקודה שיעור השיפוע המשיק שווה למכפלת שיעוריה.  
  \item מצאו את העקומה המקיימת את תנאי סעיף א׳ ועוברת דרך הנקודה $(\sqrt{2}, e^2)$.  
\end{enumerate}

\exercise{}
חיידקים מתרבים בקצב יחסי $k=0.3$ לשעה, עד שמגיעים לקיבולת נשיאה $M=1000$. בתחילת הניסוי יש $y(0)=10$ חיידקים.
\begin{enumerate}[label=\alph*.]
  \item כתבו את המד״ר הלוגיסטי $y'(t)=f(y)$ (המשוואה נתונה בפרק המבוא).
  \item סווגו את סוג המד״ר.
  \item פתרו את המד״ר וקבלו את הפתרון הפרטי.
  \item חשבו את מספר החיידקים לאחר $t=20$ שעות.
\end{enumerate}

\exercise{}{}

גוף שמהירותו ההתחלתית היא $v(0)=v_0 = 5 \tfrac{m}{s}$ ומסתו $m=2 \, kg$ , נכנס לתוך תווך צמיגי שמקדם הצמיגות בו הוא $\mu = 0.5 \tfrac{N \cdot s}{m}$.  
על הגוף לא פועלים כוחות נוספים. כתבו את משוואת התנועה של הגוף ואת הקשר בין הגדלים הפיזיקליים השונים,
כאשר בזמן התחלתי הגוף נמצא במיקום יחסי $0$.  
לסיום, מצאו את המהירות והמיקום של הגוף כפונקציה של הזמן.

\exercise{}

אוכלוסיית חיידקים מתרבה בצלחת פטרי. ידוע כי קצב הגידול של האוכלוסייה פרופורציונלי לגודלה.  
נסמן ב־$B(t)$ את גודל האוכלוסייה בזמן $t$, ב־$B_0 = B(0)$ את הגודל ההתחלתי של האוכלוסייה, וב־$K$ את קבוע הפרופורציה (קבוע הגידול).  

\begin{enumerate}[label=\alph*.]
  \item הציגו את המשוואה הדיפרנציאלית המתארת את פונקציית האוכלוסייה $B(t)$.  
  \item קבלו פתרון פרטי למשוואה התלוי בערכי הקבועים $B_0,K$, מהצורה $B=f(t)$. הסבירו לאורך הדרך באיזו שיטה החלטתם לפתור את המשוואה ומדוע היא מתאימה לפתרון המד״ר הנתונה. יש לנמק כל שלב ולרשום את תחום ההגדרה של הפתרון כולל פתרונות סינגולריים (באם יש כאלה).  
  \item ידוע כי לאחר שעה האוכלוסייה בצלחת הפטרי מונה $100$ חיידקים, ואילו לאחר $3$ שעות ישנם $500$ חיידקים בצלחת. מצאו את הערכים של $B_0,K$ והציגו את הפתרון הפרטי של המשוואה המתאים לנתונים אלה.
  \item חשבו כמה חיידקים ישנם בצלחת הפטרי לאחר חמש שעות מתחילת הניסוי.
\end{enumerate}

\exercise{}

פתרו באמצעות שיטת הפרדת משתנים את המשוואה המתקבלת בקירוב מתרגיל~\ref{ex:newtoncooling}.

\exercise{}

מצאו את העקומה המקיימת כי הקטע העובר בין נקודות החיתוך של המשיק לפונקציה עם הצירים, \underline{נחצה בנקודת ההשקה}. ראו סרטוט להמחשה:
\begin{center}
\begin{tikzpicture}[scale=1.2]

  % axes
  \draw[->] (-0.5,0) -- (4,0) node[right] {$x$};
  \draw[->] (0,-0.5) -- (0,5) node[above] {$y$};

  % curve y = 2/x (no label now)
  \draw[blue, thick, domain=0.5:4, samples=200] 
    plot (\x, {2/\x});

  % tangency point A at x=1, y=2
  \coordinate (A) at (1,2);
  \filldraw (A) circle (1.5pt) node[above right] {$A(x,y)$};

  % tangent slope at A
  % y' = -c/x^2 = -2/1^2 = -2
  % tangent: Y - 2 = -2(X - 1)  -> Y = -2X + 4

  % tangent line
  \draw[red, thick, domain=-0.2:2.5] plot (\x,{-2*\x+4}) node[right] {משיק};

  % intercepts
  \coordinate (B) at (2,0); % x-intercept
  \coordinate (C) at (0,4); % y-intercept
  \filldraw (B) circle (1.5pt) node[below] {$B$};
  \filldraw (C) circle (1.5pt) node[left] {$C$};

  % helper dashed lines

  % keep curve indication
  \node[blue] at (2.8,1) {עקומה};

\end{tikzpicture}
\end{center}

\newpage
\underline{פתרונות}

\solution{}
נכתוב את המשוואה בצורה המתאימה:
\[
y' = \Bigl(\tfrac{3}{x}\Bigr)\cdot (y) = f(x)\cdot g(y).
\]

\textbf{שלב א׳ – בדיקת פתרונות סינגולריים:}  
\[
g(y) = y \quad \Rightarrow \quad g(y)=0 \iff y=0.
\]
לכן $y\equiv 0$ הוא פתרון סינגולרי/נוסף/קבוע (חשוד לפחות).

\textbf{שלב ב׳ – מציאת פונקציות $F(x),G(y)$:}  
\[
F(x) = \int \tfrac{3}{x}\,dx = 3\ln|x| \quad (+c), \qquad
G(y) = \int \tfrac{1}{y}\,dy = \ln|y| \quad (+d).
\]

\textbf{הערה:} $c$ ו-$d$ אינם קבועים שחובה לכתוב אותם, שכן הם ייבלעו בביטוי הכללי שבצורה סתומה.

\textbf{שלב ג׳ – פתרון כללי:}  
\[
\ln|y| = 3\ln|x| + C.
\]

נפרש אקספוננציאלית:
\[
|y| = e^{3\ln|x|}\cdot e^C = x^3 \cdot (\pm e^C),
\]
ומכאן:
\[
y = C_1 \cdot x^3, \qquad C_1 \neq 0,\qquad x\in\mathbb{R}.
\]

לפתרון הכללי נוסיף את הפתרון הסינגולרי $y\equiv 0$.  

\textbf{סיכום:}  
הפתרון הפרטי הוא:
\[
y(x) = C_1 x^3, \quad C_1 \neq0, \qquad  y \equiv 0,\qquad x\in\mathbb{R}.
\]
פתרון סינגולרי הוא כזה שלא קיים אף קבוע ממשי שיביא לנו את הפתרון החשוד. לגבי הפתרון הכללי - ניתן לרשום את הפתרון ׳׳בבת אחת׳׳ עם ׳׳שחרור׳׳ האילוץ על $C_1$ ולהבין כי הפתרון החשוד להיות סינגולרי לא באמת סינגולרי:
\[
y(x) = C_1 x^3, \qquad x\in\mathbb{R}.
\]

\textbf{שלב ד׳ – פתרון פרטי:}  
נציב את תנאי ההתחלה $y(1)=\tfrac{1}{2}$:
\[
\tfrac{1}{2} = C_1 \cdot 1^3 \quad \Rightarrow \quad C_1 = \tfrac{1}{2}.
\]

ולכן:
\[
\boxed{\, y(x) = \tfrac{x^3}{2}, \qquad x\in\mathbb{R} \,}
\]
נשים לב כי הפתרון $y\equiv0$, לא מהווה פתרון פרטי לבעיה (שכן לא עובר בתנאי ההתחלה), ועל כן הפתרון היחידי לבעיה הוא $\tfrac{x^3}{2}$.

*שימו לב כי ניתן לפתור מד׳׳ר זו באמצעות שיטת גורם האינטגרציה (שכן המד׳׳ר לינארית) וכמו כן בשיטת וריאציית הפרמטר.


\solution

נרצה לפתור את המד״ר:  
\[
y' = xy
\]
1.
 על פי האלגוריתם שלמדנו, נחפש תחילה את השורשים של \( g(y) \). כלומר, פותרים \( g(y) = 0 \). אם \( y_0 \) שורש של \( g(y) \), כלומר \( g(y_0)=0 \), אזי \( y \equiv y_0 \) הוא פתרון סינגולרי/נוסף/קבוע (חשוד לפחות).  

2. נגדיר פונקציות:  
\[
G(y) = \int \frac{1}{g(y)} \, dy , \qquad F(x) = \int f(x)\, dx
\]

3. הפתרון הכללי בצורה סתומה (ולא כולל הפתרונות הסינגולריים):  
\[
G(y) = F(x) + C
\]

במקרה שלנו:  
\[
y' = xy \quad \Rightarrow \quad f(x) = x , \quad g(y) = y
\]

פתרון חשוד לסינגולרי יהיה \( y \equiv 0 \).  

\[
F(x) = \int x\, dx = \frac{x^2}{2}
\]

\[
G(y) = \int \frac{1}{y}\, dy = \ln |y|
\]

נשווה:  
\[
\ln |y| = \frac{x^2}{2} + C
\]

ולכן:  
\[
|y| = e^{\tfrac{x^2}{2}} \cdot e^C \quad \Rightarrow \quad y = C_1 \cdot e^{\tfrac{x^2}{2}}, \quad C_1 \neq 0
\]

נזכור כי בנוסף קיים גם הפתרון \( y \equiv 0 \).  

כלומר, הפתרון הכללי הוא:  
\[
y = C_1 \cdot e^{\tfrac{x^2}{2}}, \qquad x \in \mathbb{R}
\]

כעת נציב את תנאי המעבר דרך \((\sqrt{2}, e^2)\):  
\[
e^2 = C_1 \cdot e^1 \quad \Rightarrow \quad C_1 = e
\]

ומכאן הפתרון הפרטי הוא:  
\[
\boxed{y = e^{\tfrac{x^2}{2} + 1}, \qquad x \in \mathbb{R}}
\]
שימו לב כי הפתרון $y=0$ לא פתרון פרטי לבעיה שלנו.

*שימו לב כי ניתן לפתור מד׳׳ר זו באמצעות שיטת גורם האינטגרציה (שכן המד׳׳ר לינארית) וכמו כן בשיטת וריאציית הפרמטר.



\solution{}{}
\textbf{סעיף א׳}

המשוואה הלוגיסטית:
\[
y'(t) = k\,y(t)\Big(1 - \tfrac{y(t)}{M}\Big).
\]

נציב את הנתונים: $k=0.3$, $M=1000$:
\[
y'(t) = 0.3\,y(t)\Big(1 - \tfrac{y(t)}{1000}\Big).
\]

\textbf{סעיף ב׳ – סיווג}

\begin{itemize}
  \item סדר: ראשון (רק $y'$ מופיע).
  \item לינאריות: לא, זו משוואה לא לינארית (הכפלה של $y$ בעצמו).
  \item מנורמלת: כן (מקדם $y'$ שווה 1).
  \item סוג: פרידה.
\end{itemize}

\textbf{סעיף ג׳ – פתרון כללי ופרטי}

נפריד משתנים:
\[
\frac{dy}{dt} = 0.3\,y\Big(1 - \tfrac{y}{1000}\Big).
\]

נכתוב:
\[
\frac{dy}{y(1 - y/1000)} = 0.3\,dt.
\]

נפרק לשברים חלקיים:
\[
\frac{1}{y(1 - y/1000)} 
= \frac{1}{y} + \frac{1}{1000-y}\cdot\frac{1}{1000}.
\]

בפשטות: 
\[
\frac{1}{y(1 - y/1000)} = \frac{1}{y} + \frac{1}{1000-y}\cdot \frac{1}{1000} \cdot 1000 = \frac{1}{y} + \frac{1}{1000-y}.
\]

לכן:
\[
\int \left(\frac{1}{y} + \frac{1}{1000-y}\right) dy = \int 0.3\,dt.
\]

נבצע אינטגרציה:
\[
\ln|y| - \ln|1000-y| = 0.3t + C.
\]

כלומר:
\[
\ln\left(\frac{y}{1000-y}\right) = 0.3t + C.
\]

נעלה בחזקת $e$:
\[
\frac{y}{1000-y} = Ae^{0.3t}, \quad A=e^C.
\]

נבודד את $y$:
\[
y(t) = \frac{1000\,Ae^{0.3t}}{1 + Ae^{0.3t}}.
\]

נשתמש בתנאי ההתחלה $y(0)=10$:
\[
\frac{10}{1000-10} = A \cdot 1 \quad \Rightarrow \quad A = \frac{10}{990} = \frac{1}{99}.
\]

ולכן הפתרון הפרטי:
\[
\boxed{\, y(t) = \frac{1000}{1 + 99 e^{-0.3t}}, \quad t \geq 0 \,}
\]

\textbf{סעיף ד׳ – חישוב עבור $t=20$}

\[
y(20) = \frac{1000}{1 + 99 e^{-0.3\cdot 20}}.
\]

חישוב מספרי: $e^{-6} \approx 0.002478$,
\[
y(20) \approx \frac{1000}{1 + 99 \cdot 0.002478}
= \frac{1000}{1 + 0.2453} 
= \frac{1000}{1.2453} 
\approx \boxed{803}.
\]

\textbf{פירוש פיזיקלי:}  
לאחר 20 שעות גודל האוכלוסייה הוא כ־803 חיידקים, ומתקרב בהדרגה לקיבולת הנשיאה $M=1000$.

\begin{figure}[H]
    \centering
    \includegraphics[width=0.7\textwidth]{logistic.png}
    \caption{הגרף של הפתרון 
    $y(t)=\tfrac{1000}{1+99e^{-0.3t}}$ 
    במודל גידול לוגיסטי עם פרמטרים:
    $k=0.3\,\text{ש}^{-1}$, 
    $M=1000$, 
    $y(0)=10$. 
    הקו האדום המקווקו מייצג את קיבולת הנשיאה $M=1000$, 
    והנקודה האדומה מסמנת את $y(20)\approx 803$.}
    \label{fig:logistic_growth}
\end{figure}


\solution

נזכור כי הקשר בין המיקום, המהירות והתאוצה של הגוף הוא:
\[
x'' = \frac{d^2x}{dt^2} = \frac{dv}{dt} = a(t)
\]

ניעזר גם בחוק השני של ניוטון:
\[
\sum F = ma
\]

במקרה שלנו פועל רק כוח הצמיגות/חיכוך על הגוף:
\[
F(t) = -\mu v = ma = m v'
\]

ולכן נקבל מד׳׳ר פרידה מסדר 1:
\[
v' = -\frac{\mu}{m} v
\]
נכתוב את המשוואה בצורתה הכללית:
\[
v' = b(t) \cdot g(v), \qquad b(t) = -\frac{\mu}{m}, \quad g(v)=v
\]
המד׳׳ר למעשה אוטונומית.

\textbf{אלגוריתם לפתרון:}
\begin{enumerate}
  \item נחפש את השורשים של $g(v)$. במקרה שלנו $g(v)=v$ ולכן $v \equiv 0$ הוא פתרון חשוד סינגולרי.
  \item נגדיר פונקציות:
  \[
  G(v) = \int \frac{1}{g(v)} dv = \int \frac{1}{v} dv = \ln|v| , 
  \qquad 
  B(t) = \int b(t)\, dt = \int -\frac{\mu}{m}\, dt = -\frac{\mu}{m} t
  \]
  \item הפתרון הכללי בצורה סתומה הוא:
  \[
  G(v) = B(t) + C \quad \Rightarrow \quad \ln|v| = -\frac{\mu}{m}t + C
  \]
\end{enumerate}

ולכן:
\[
|v| = e^{-\tfrac{\mu}{m}t} \cdot e^C
\]

נגדיר $C_1 = \pm e^C\neq 0$ ונקבל:
\[
v(t) = C_1 e^{-\tfrac{\mu}{m}t}, \qquad t\geq0
\]
לצד $v=0$.

\textbf{שימוש בתנאי ההתחלה:}  
עבור $t=0$ מתקיים $v(0)=v_0$, ולכן:
\[
v_0 = C_1 \quad \Rightarrow \quad v(t) = v_0 e^{-\tfrac{\mu}{m}t}
\]

במקרה שלנו:
\[
v(t) = 5 e^{-\tfrac{1}{4}t}
\]

\textbf{מעבר למיקום:}  
נבצע אינטגרציה של $v(t)$:
\[
x(t) = \int v(t)\, dt = \int v_0 e^{-\tfrac{\mu}{m}t}\, dt
= -\frac{m}{\mu} v_0 e^{-\tfrac{\mu}{m}t} + D
\]

מתנאי ההתחלה $x(0)=0$ נקבל:
\[
D = \frac{m}{\mu} v_0
\]

ולכן:
\[
x(t) = \frac{m}{\mu} v_0 \left(1 - e^{-\tfrac{\mu}{m}t}\right)
\]

ובמקרה שלנו:
\[
\boxed{x(t) = 20 \left(1 - e^{-\tfrac{1}{4}t}\right)}
\]

\begin{figure}[H]
  \centering
  \includegraphics[width=\textwidth]{velocity_position.png}
  \caption{(a) מהירות הגוף $v(t)=5e^{-t/4}$ כפונקציה של הזמן, 
           (b) מיקום הגוף $x(t)=20\left(1-e^{-t/4}\right)$ כפונקציה של הזמן.}
\end{figure}

*שימו לב כי ניתן לפתור מד׳׳ר זו באמצעות שיטת גורם האינטגרציה (שכן המד׳׳ר לינארית) וכמו כן בשיטת וריאציית הפרמטר.

%%%CUT%%%

\solution

\begin{enumerate}[label=\alph*.]
\item
קצב הגידול של האוכלוסייה פרופורציונלי לגודלה.  
קצב גידול/שינוי הגדרתו היא הנגזרת, כלומר הנגזרת פרופורציונלית לגודל של האוכלוסיה.  
מכאן שהנגזרת שווה למקדם פרופורציה כפול כמות החיידקים, ולכן:
\[
\frac{dB}{dt} = B'(t) = K B.
\]

\item
המד״ר כבר מנורמלת וניתן לראות שהיא \textbf{פרידה}:
\[
B'(t) = \textcolor{red}{K}\textcolor{blue}{ B(t)}.
\]

בשלב הראשון ניתן לראות שקיים פתרון אשר חשוד להיות סינגולרי עבור $B=0$.  

בשלב השני נכתוב את שני האינטגרלים שלנו:
\[
F(t) = \int f(t) dt = \int K\, dt = Kt,
\qquad
G(B) = \int \frac{1}{g(B)}\, dB = \int \frac{1}{B(t)}\, dB = \ln|B|.
\]

בשלב השלישי נכתוב את הפתרון הכללי בצורה סתומה:
\[
\ln|B| = Kt + c
\;\;\;\Rightarrow\;\;\;
|B| = e^c \cdot e^{Kt}
\;\;\;\Rightarrow\;\;\;
B = \pm e^c \cdot e^{Kt}.
\]

נסמן $c_1 = \pm e^c$ (כאשר $c_1 \neq 0$).  
ולכן:
\[
B(t) = c_1 e^{Kt}, \quad t \geq 0.
\]

נזכור כי קיים פתרון חשוד סינגולרי $B=0$.  
מכאן ניתן לאחד את הפתרונות לפתרון כללי:
\[
B(t) = c_1 e^{Kt}, \quad t \geq 0.
\]

תחום ההגדרה נובע מכך שאנו פותרים בעיה פיזיקלית עם זמנים אי־שליליים.

כדי לקבל את הפתרון הפרטי נציב $B(0)=B_0$:  
\[
B(0) = B_0 = c_1.
\]

ולכן הפתרון הפרטי הוא:
\[
B(t) = B_0 e^{Kt}, \quad t \geq 0.
\]

\item
נפתור את מערכת המשוואות הבאה:
\[
\begin{cases}
100 = B_0 e^K \\
500 = B_0 e^{3K}
\end{cases}
\]

נחלק משוואות:  
\[
\frac{1}{5} = e^{-2K} \;\;\;\Rightarrow\;\;\; K = -\tfrac{1}{2}\ln\left(\tfrac{1}{5}\right) > 0.
\]

נציב (למשל) במשוואה הראשונה:
\[
100 = B_0 e^{-\tfrac{1}{2}\ln(1/5)}
\;\;\;\Rightarrow\;\;\;
B_0 = 100 e^{\tfrac{1}{2}\ln(1/5)} = 20\sqrt{5}.
\]

מכאן כי הפתרון הפרטי (ללא תלות בפרמטרים) הוא:
\[
B(t) = 20\sqrt{5}\; e^{-\tfrac{1}{2}\ln(1/5)\, t}, \quad t \geq 0.
\]

\item 
כמות החיידקים לאחר $5$ שעות היא:
\[
B(5) = 20\sqrt{5}\; e^{-\tfrac{5}{2}\ln(1/5)}=2500.
\]
\end{enumerate}
גרף הפתרון:
\begin{figure}[H]
  \centering
  \includegraphics[width=0.7\textwidth]{petri_dish.png}
  \caption{התפתחות אוכלוסיית החיידקים לפי $B(t) = B_0 e^{Kt}$.  
  מסומנים ערכים מעניינים: $B(1)=100$, $B(3)=500$, וחיזוי $B(5)$.}
\end{figure}


\solution{}

לאחר הקירוב (כפי שנעשה בתרגיל~\ref{ex:newtoncooling}) מתקבלת המשוואה:
\[
\frac{dT}{dt} = -\big(k+4\sigma T_a^3\big)(T-T_a).
\]

נסמן \(\lambda = k+4\sigma T_a^3 > 0\).  
המד״ר היא מסדר ראשון, מנורמלת, פרידה.

נבחן פתרונות סינגולריים חשודים:  
אם \(T'(t)=0\) מתקבל \(T(t)\equiv T_a\). זהו פתרון קבוע.

\item
נבודד משתנים:
\[
\frac{dT}{T-T_a} = -\lambda\, dt.
\]

כאן:
\[
f(t) = -\lambda, 
\qquad 
g(T) = T-T_a.
\]

ולכן:
\[
F(t) = \int f(t)\, dt = \int -\lambda \, dt = -\lambda t,
\]
\[
G(T) = \int \frac{1}{g(T)}\, dT = \int \frac{1}{T-T_a}\, dT = \ln|T-T_a|.
\]

נרשום את הפתרון בצורה סתומה:
\[
G(T) = F(t) + C \;\;\;\Rightarrow\;\;\; \ln|T-T_a| = -\lambda t + C.
\]

נעבור לצורה אקספוננציאלית:
\[
|T-T_a| = e^C \cdot e^{-\lambda t}.
\]

כלומר:
\[
T(t) - T_a = C_1 e^{-\lambda t}, \qquad C_1 \neq 0.
\]

יחד עם הפתרון הסינגולרי \(T(t)\equiv T_a\) נקבל את הפתרון הכללי:
\[
T(t) = T_a + C_1 e^{-\lambda t}, \qquad t \geq 0.
\]

כעת נציב את תנאי ההתחלה \(T(0)=T_0\):
\[
T_0 - T_a = C_1.
\]

ולכן הפתרון הפרטי הוא:
\[
\boxed{\,T(t) = T_a + (T_0 - T_a)e^{-\lambda t}, \quad t \geq 0\,}
\]
קיבלנו את אותו הפתרון.



\solution{}

ראשית נסביר את הסימונים:  

\begin{itemize}
  \item $A(x,y)$ היא נקודת ההשקה של העקומה עם המשיק.  
  \item $(X,Y)$ הם קואורדינטות כלליות במישור.  
  \item $B$ היא נקודת החיתוך של המשיק עם ציר $X$. 
  \item $C$ היא נקודת החיתוך של המשיק עם ציר $Y$.
  \item מאחר ולפי הנתון נקודת ההשקה חוצה את הקטע $BC$, מתקיים ש־$A$ היא האמצע של $BC$, 
\end{itemize}

משוואת המשיק בנקודת ההשקה $A(x,y)$ מתקבלת ע״י:
\[
Y - y = y'(X - x),
\]
כאשר $y'$ הוא השיפוע של העקומה בנקודה $A$.

\textbf{שלב 1 – מציאת נקודות החיתוך עם הצירים:}

\begin{itemize}
  \item חיתוך עם ציר $X$: נציב $Y=0$.  
  \[
  0 - y = y'(X - x) \;\;\;\Rightarrow\;\;\; X = x - \tfrac{y}{y'}.
  \]
  לכן נקודת החיתוך היא 
  \[
  B\left(x-\tfrac{y}{y'},\,0\right).
  \]

  \item חיתוך עם ציר $Y$: נציב $X=0$.  
  \[
  Y - y = y'(0 - x) \;\;\;\Rightarrow\;\;\; Y = y - x y'.
  \]
  לכן נקודת החיתוך היא 
  \[
  C\left(0,\,y-xy'\right).
  \]
\end{itemize}

\textbf{שלב 2 – תנאי האמצע:}  

נקודת האמצע $A$ של $BC$ היא:
\[
A = \left(\frac{(x-\tfrac{y}{y'})+0}{2},\,\frac{0+(y-xy')}{2}\right)
= \left(\tfrac{x}{2}-\tfrac{y}{2y'},\,\tfrac{y}{2}-\tfrac{xy'}{2}\right).
\]

מאחר ולפי הנתון $A(x,y)$ היא נקודת האמצע, נקבל מערכת תנאים:
\[
\begin{cases}
x = \tfrac{x}{2}-\tfrac{y}{2y'} \\
y = \tfrac{y}{2}-\tfrac{xy'}{2}.
\end{cases}
\]

\textbf{שלב 3 – בידוד המשוואה הדיפרנציאלית:}  

נכפיל ונפשט:

1. מהמשוואה הראשונה:
\[
x = \frac{x}{2}-\frac{y}{2y'} \;\;\;\Rightarrow\;\;\; \frac{x}{2} = -\frac{y}{2y'} \;\;\;\Rightarrow\;\;\; x = -\frac{y}{y'}.
\]

2. מהמשוואה השנייה:
\[
y = \frac{y}{2}-\frac{xy'}{2} \;\;\;\Rightarrow\;\;\; \frac{y}{2} = -\frac{xy'}{2} \;\;\;\Rightarrow\;\;\; y=-xy'.
\]

שתי התוצאות שקולות (זו נובעת מזו).  

ולכן המשוואה הדיפרנציאלית לעקומה היא:
\[
y' = -\frac{y}{x}.
\]

\textbf{שלב 4 – פתרון המשוואה הדיפרנציאלית :}

נזהה את המד׳׳ר כמשוואה פרידה.

\begin{enumerate}[label=\arabic*.]

\item \textbf{בדיקת פתרונות סינגולריים:}  
נבדוק מתי $g(y)=-y$ מתאפס.  
אם $y\equiv 0$, נקבל פתרון חשוד לסינגולרי:  
\[
y(x)\equiv 0.
\]

\item \textbf{זיהוי $f(x),g(y)$ וחישוב האינטגרלים:}  
נזהה:
\[
f(x)=-\frac{1}{x}, \qquad g(y)=y.
\]
ולכן:
\[
F(x)=\int f(x)\,dx=\int -\frac{1}{x}\,dx=-\ln|x|,
\]
\[
G(y)=\int \frac{1}{g(y)}\,dy=\int \frac{1}{y}\,dy=\ln|y|.
\]

\item \textbf{פתרון כללי סתום:}  
נרשום:
\[
G(y)=F(x)+C \;\;\;\Rightarrow\;\;\; \ln|y|=-\ln|x|+C.
\]

נעלה באקספוננט:
\[
|y|=\frac{C_1}{x}, \qquad C_1\neq 0.
\]

נחבר גם את הפתרון הסינגולרי, ונקבל את הפתרון הכללי:
\[
\boxed{\,y(x)=\frac{C}{x}, \quad  x\neq 0.}
\]

\end{enumerate}

%%%CUT%%%

\newpage
\subsection{משוואת ברנולי}

\textbf{הצורה הכללית המתארת את משוואת ברנולי:}
\begin{equation}\label{Bernouli}
y' + p(x)\,y = g(x)\,y^{\alpha}, 
\end{equation}
כאשר $p(x),q(x)$ רציפות בקטע $I$ ו־$\alpha \in \mathbb{R}$.

\begin{itemize}
  \item אם $\alpha>0$ אז $y\equiv 0$ הוא פתרון.
  \item עבור $\alpha=1$ נקבל משוואה לינארית מסדר 1, הומוגנית.
  \item עבור $\alpha=0$ נקבל משוואה לינארית מסדר 1, לא הומוגנית.
\end{itemize}


כדי לקבל את הפתרון למקרה הכללי, נבצע הצבה ונחשב את הנגזרת המורכבת שלה:
\[
z(x) = y^{1-\alpha}, \qquad z'(x) = (1-\alpha) y^{-\alpha} y'.
\]

נחלק את משוואה (\ref{Bernouli}) ב־$y^\alpha$ ונקבל:
\[
y^{-\alpha} y' + p(x) y^{1-\alpha} = g(x).
\]

נציב את $z(x)$ ונגזרתה:
\[
\frac{z'(x)}{1-\alpha} + p(x)z(x) = g(x).
\]

לאחר כפל ב־$(1-\alpha)$ מתקבלת משוואה לינארית, מנורמלת ב־$z$:
\begin{equation}
z'(x) + (1-\alpha)p(x)z(x) = (1-\alpha)g(x).
\end{equation}
נסמן: $\bar{p}(x)=(1-\alpha)p(x), \bar{g}(x)=(1-\alpha)g(x)$
זהו ונקבל את הפתרון הסגור למד׳׳ר בעזרת גורם אינטגרציה:
\begin{equation}
\boxed{z(x) = \frac{1}{\mu(x)} \left(C + \int \mu(x)\,\bar{g}(x)\,dx \right), 
\qquad 
\mu(x)=e^{\int \bar{p}(x)\,dx}}.
\end{equation}

לבסוף נחזור ל־$y$:
\begin{equation}
\boxed{y(x) = \Big(z(x)\Big)^{\tfrac{1}{1-\alpha}}}.
\end{equation}

\begin{insight}
אז מה בעצם היה הרעיון? ׳׳לצאת׳׳ ממישור $x,y(x)$ בו המד׳׳ר לא לינארית 
ע׳׳י הצבה פשוטה וקבלת מד׳׳ר לינארית במישור $x,z(x)$.  
מכאן, נוכל לקבל את הפתרון במישור ה׳׳חדש׳׳, ולבסוף נחזור אחורה ע׳׳י הצבה פשוטה.  
זו אחת משיטות האד־הוק הקלאסיות.
\end{insight}



\example{}

נבחן את המשוואה הדיפרנציאלית הלא־לינארית:
\[
y' + y = y^{\tfrac{1}{2}}, \qquad y(0)=0.
\]

\explanation
מדובר במשוואת \textbf{ברנולי} מהצורה:
\[
y' + p(x)y = g(x) y^\alpha, \qquad \alpha=\tfrac{1}{2}.
\]

\textbf{שלב 1:} נבדוק פתרון סינגולרי.  
מאחר ש־$\alpha>0$, הפתרון $y\equiv0$ הוא פתרון אפשרי.

\textbf{שלב 2:} נעבור לשינוי משתנה $z(x)=y^{1-\alpha}=y^{1/2}$. 
נגזור:
\[
z'(x) = \tfrac{1}{2} y^{-1/2} \, y' \quad \Rightarrow \quad y' = 2 z'(x) y^{1/2}=2 z'(x) z(x).
\]

נציב חזרה במשוואה ונשתמש בקשר $y=z^2$
\[
y' + y = y^{1/2} 
\quad \Rightarrow \quad 
2 z'(x) z(x) + z^{2} = z.
\]

נחלק ב־$2z$ (כאשר $z \neq 0$):
\[
z'(x) + \tfrac{1}{2} z(x) = \tfrac{1}{2}.
\]

קיבלנו משוואה לינארית מסדר ראשון ב־$z(x)$, מנורמלת.


\textbf{שלב 3:} קיבלנו משוואה לינארית מסדר ראשון עבור $z(x)$.  
נחשב גורם אינטגרציה:
\[
\mu(x)=e^{\int \tfrac{1}{2}\,dx}=e^{x/2}.
\]

נכפיל את המשוואה ונקבל:
\[
\big(e^{x/2} z(x)\big)' = \tfrac{1}{2} e^{x/2}.
\]

נבצע אינטגרציה:
\[
e^{x/2} z(x) = e^{x/2} + C \quad \Rightarrow \quad z(x) = Ce^{-x/2} + 1.
\]

\textbf{שלב 4:} נחזור למשתנה $y$:  
\[
y(x) = z^2(x) = \big(Ce^{-x/2}+1\big)^2.
\]

\textbf{שלב 5:} נציב תנאי התחלה $y(0)=0$:  
\[
0 = (C+1)^2 \quad \Rightarrow \quad C=-1.
\]

ולכן הפתרון הפרטי הוא:
\[
\boxed{\, y(x) = \big(1-e^{-x/2}\big)^2 , \qquad x\in\mathbb{R} \,}
\]

\textbf{סיכום:} הפתרון כולל שני חלקים – הפתרון הסינגולרי $y\equiv0$ והפתרון הפרטי לעיל, שניהם מקיימים את תנאי ההתחלה.
גרף הפתרון יראה כך:
\begin{figure}[H]
\centering
\includegraphics[width=0.7\textwidth]{first_bern.png}
\caption{השוואה בין הפתרון הסינגולרי $y\equiv0$ (קו אדום מקווקו) לבין הפתרון הפרטי 
$y(x)=(1-e^{-x/2})^2$ (קו כחול). שתי הפתרונות מקיימים את תנאי ההתחלה $y(0)=0$ המסומן בנקודה שחורה.}
\end{figure}

\newpage
\underline{תרגילים}

\exercise{}

נבחן את המשוואה הדיפרנציאלית:
\[
y' + y = \alpha y^{3}, \qquad y(0)=y_0>0,
\]
כאשר $\alpha$ הוא פרמטר ממשי.

 מצאו את הפתרון הפרטי של המשוואה. 

\exercise{}

ידוע כי המשוואה הדיפרנציאלית הרגילה הבאה יכולה לתאר את הגידול והדעיכה של אוכלוסיית חיידקים מסוימת בתוך כלי מסוים:
\[
\frac{dN}{dt} = kN - mN^2,
\]
כאשר $N$ מתאר את גודל האוכלוסייה, $t$ הוא הזמן, $k$ הוא קבוע הגידול של האוכלוסייה ($k>0$) ו־$m$ הוא קבוע התכלות המשאב (הנובע מתחרות בתוך האוכלוסייה).

נציב את הערכים:
\[
k=0.1 \,\tfrac{1}{\text{hr}}, 
\qquad 
m=0.2 \,\tfrac{1}{\text{hr}\cdot\text{germs}}.
\]

\begin{enumerate}[label=\alph*.]
  \item קבלו פתרון כללי מפורש $N=f(t)$ לבעיה. הסבירו לאורך הדרך באיזו שיטה השתמשתם, ומדוע היא מתאימה לפתרון משוואה זו. כתבו את תחום ההגדרה של הפתרון כולל פתרונות סינגולריים (אם יש כאלה).
  
  \item קבלו פתרון פרטי מפורש לבעיה עבור תנאי ההתחלה:
  \[
  N(0)=5000.
  \]
  כתבו את תחום ההגדרה של הפתרון כולל פתרונות סינגולריים (אם יש כאלה).
  
  \item חשבו את כמות החיידקים בכלי כעבור יומיים (48 שעות).
\end{enumerate}

\exercise{}

נתונה המד''ר
\[
1+y^3 = x^2 y^2 y'
\]

\begin{enumerate}[label=\alph*.]
  \item זהו את סוג המד''ר (סדר, לינאריות, מקדמים קבועים או לא, הומוגניות ונרמול).  

  \item קבלו פתרון כללי מפורש לבעיה (מהצורה $y=f(x)$). הסבירו לאורך הדרך באיזו שיטה החלטתם לפתור את המשוואה ומדוע היא מתאימה לפתרון המד''ר הנתונה. יש לנמק כל שלב ולרשום את תחום ההגדרה של הפתרון כולל פתרונות סינגולריים (באם יש כאלה).  
  \textbf{רמז:} לפתרון האינטגרל האחרון השתמשו בהצבה $t=\tfrac{3}{x}$.  

  \item קבלו פתרון פרטי מפורש לבעיה עבור תנאי ההתחלה $y(1)=1$. יש לנמק כל שלב ולרשום את תחום ההגדרה של הפתרון כולל פתרונות סינגולריים (באם יש כאלה).  
\end{enumerate}

\exercise{}

נבחן את בעיית הקירור של גוף חם לפי החוק:
\[
\frac{dT}{dt} + kT = \sigma T^4, \qquad T(0)=T_0>0,
\]
כאשר $k,\sigma>0$ הם קבועים.

\begin{enumerate}[label=\alph*.]
  \item זהו את סוג המשוואה וכתבו האם קיימים פתרונות סינגולריים.
  \item מצאו פתרון כללי מפורש לבעיה באמצעות שיטת ברנולי.
  \item הציגו פתרון פרטי כאשר $k=1$, $\sigma=0.01$, $T_0=100$ ותחום הזמן $t\geq 0$.
\end{enumerate}

%%%CUT%%%

 
\newpage
\underline{פתרונות}

\solution

מדובר במשוואת \textbf{ברנולי} מהצורה:
\[
y' + p(x)y = g(x) y^\alpha,
\]
כאשר $\alpha=3$, $p(x)=1$, ו־$g(x)=\alpha$ (קבוע).

שלב 1: נבדוק פתרון סינגולרי.  
המשוואה $y'=0$ ו־$y=0$ מראה כי $y\equiv 0$ הוא פתרון סינגולרי.

שלב 2: נחליף משתנים $z(x)=y^{1-\alpha}=y^{-2}$.  
נגזור:
\[
z'(x) = -2 y^{-3} y' .
\]

נציב במשוואה המקורית ונחלק ב־$y^3$:
\[
\frac{y'}{y^3} + \frac{1}{y^2} = \alpha.
\]

נציב $z=1/y^2$ ואת $z'(x) = -2 y^{-3} y'$: 
\[
-\frac{1}{2}z' + z = + \alpha.
\]
ננרמל ולמעשה
קיבלנו משוואה לינארית ב־$z$:
\[
z'(x) - 2z(x) = -2\alpha .
\]

שלב 3: נרצה לפתור את המשוואה הלינארית שקיבלנו.
כדי לפתור אותה ניעזר בשיטת \textbf{גורם אינטגרציה}.  
המטרה היא להפוך את הצד השמאלי לנגזרת של מכפלה.  
גורם האינטגרציה מחושב לפי:
\[
\mu(x) = e^{\int -2\, dx} = e^{-2x}.
\]

כעת נכפיל את כל המשוואה בגורם האינטגרציה:  
\[
e^{-2x}z'(x) - 2e^{-2x}z(x) = -2\alpha e^{-2x}.
\]

נשים לב שהצד השמאלי הוא בדיוק הנגזרת של המכפלה $z(x)e^{-2x}$:  
\[
\big(z(x)e^{-2x}\big)' = -2\alpha e^{-2x}.
\]

כעת אנו יכולים לבצע אינטגרציה פשוטה של שני האגפים.

שלב 4: נבצע אינטגרציה:  
\[
\int \big(z(x)e^{-2x}\big)'\, dx = \int -2\alpha e^{-2x}\, dx .
\]

מקבלים:
\[
z(x)e^{-2x} = \alpha e^{-2x} + C,
\]
כאשר $C$ הוא קבוע אינטגרציה כללי.

נכפיל חזרה ב־$e^{2x}$ ונקבל ביטוי סגור עבור $z(x)$:
\[
z(x) = \alpha + C e^{2x}.
\]

שלב 5: נזכור כי $z(x)=\tfrac{1}{y^2(x)}$.  
מכאן נקבל ישירות:
\[
y^2(x) = \frac{1}{\alpha + C e^{2x}}.
\]

כעת נרצה לעבור לפתרון עבור $y(x)$ עצמו.  
במקרה הכללי יש שתי אפשרויות:
\[
y(x) = \pm \frac{1}{\sqrt{\alpha + C e^{2x}}}\qquad , x\in\mathbb{R}.
\]

כדי לבחור את הענף הנכון נשתמש בתנאי ההתחלה.  
מאחר ש־$y(0)=y_0>0$, הפתרון המתאים הוא הענף החיובי בלבד:
\[
y(x) = \frac{1}{\sqrt{\alpha + C e^{2x}}}\qquad , x\in\mathbb{R}.
\]

כעת נחשב את הקבוע $C$.  
נציב $x=0$, $y(0)=y_0$:
\[
y_0 = \frac{1}{\sqrt{\alpha+C}} 
\quad \Rightarrow \quad 
C = \frac{1}{y_0^2} - \alpha .
\]

ולכן הפתרון הפרטי הוא:
\[
\boxed{\, y(x) = \frac{1}{\sqrt{\alpha + \Big(\tfrac{1}{y_0^2}-\alpha\Big)e^{2x}}}\qquad , x\in\mathbb{R}.}
\] 


\solution

א.
מדובר במשוואה דיפרנציאלית מהצורה:
\[
N' = kN - mN^2,
\]
שהיא וריאציה של \textbf{משוואת ברנולי}, או לחלופין ניתנת לפתרון באמצעות \textbf{הפרדת משתנים}.  
נבחר להיעזר בהצבה $z(t)=N^{-1}$ כדי להפוך את המשוואה ללינארית. כיוון ש-$\alpha=2>0$,
$N(t)\equiv 0$ פתרון סינגולרי.

\textbf{שלב 1:} נרשום את המשוואה:
\[
\frac{dN}{dt} = kN - mN^2.
\]

נחלק ב־$N^2$ (כל עוד $N\neq 0$):
\[
\frac{N'}{N^2} = \frac{k}{N} - m.
\]

נגזור את הקשר $z(t)=\frac{1}{N(t)}=N^{-1}$:
\[
z'(t) = -\frac{N'}{N^2}.
\]

נציב:
\[
-z'(t) = \frac{N'}{N^2} = \frac{k}{N} - m = kz(t) - m.
\]

נכפיל ב-$1$  ונקבל מד׳׳ר לינארית מנורמלת במישור $t,z(t)$:
\[
z'(t) + kz(t) = m.
\]

\textbf{שלב 2:} קיבלנו משוואה לינארית מסדר ראשון במשתנה $z(t)$.  
נשתמש בשיטת \textbf{גורם האינטגרציה}.  

גורם האינטגרציה:
\[
\mu(t) = e^{\int k \, dt} = e^{kt}.
\]

נכפיל:
\[
\big(z(t)e^{kt}\big)' = m e^{kt}.
\]

\textbf{שלב 3:} נבצע אינטגרציה:
\[
z(t)e^{kt} = \frac{m}{k} e^{kt} + C.
\]

ולכן:
\[
z(t) = \frac{m}{k} + Ce^{-kt}.
\]

\textbf{שלב 4:} נחזור למשתנה $N$. לכן הפתרון הכללי במישור $t,N(t)$ יהיה: 
\[
N(t) = \frac{1}{z(t)} = \frac{1}{\tfrac{m}{k} + Ce^{-kt}}, t \geq 0.
\]
שימו לב כי הפתרון $N(t)\equiv 0$ הוא אכן (אוטומטית) סינגולרי, שכן לא ניתן לבחור אף קבוע $C$ על מנת להשיב את הפתרון החשוד כסינגולרי (כביכול), מתוך הפתרון הכללי.

ב.
\textbf{שלב 5: פתרון פרטי עם תנאי התחלה.}  
נחיל את התנאי $N(0)=5000$:
\[
5000 = \frac{1}{\tfrac{m}{k}+C}.
\]

נציב
$k=0.1$, $m=0.2$ ונקבל:
$\tfrac{m}{k}=2$:
\[
5000 = \frac{1}{2+C} \quad \Rightarrow \quad C = \frac{1}{5000} - 2.
\]

כלומר:
\[
C = -\frac{9999}{5000}.
\]

ולכן הפתרון הפרטי:
\[
N(t) = \frac{1}{2 + \Big(-\tfrac{9999}{5000}\Big)e^{-0.1t}}.
\]

ג.
\textbf{שלב 6: חישוב $N(48)$}  
:
\[
N(48) = \frac{1}{2 - \tfrac{9999}{5000} e^{-4.8}}.
\]


\solution

\begin{enumerate}[label=\alph*.]

\item 
    מדובר במד״ר \textbf{מסדר ראשון}, \textbf{לא לינארית}, עם מקדמים \textbf{לא קבועים}, \textbf{לא הומוגנית}, \textbf{לא מנורמלת}. 

\item 
ננרמל את המשוואה כדי לזהות את שיטת הפתרון:  
\[
y' = \frac{1+y^3}{x^2 y^2} = \frac{1}{x^2}y^{-2}+\frac{1}{x^2}y.
\]

\underline{מדובר במשוואת ברנולי} (בשלב הזה מבחינתנו $x\neq0$). נגדיר:
\[
z(x) = y^{1-(-2)} = y^3.
\]

נגזור:
\[
z'(x) = 3y^2 y'.
\]

נחלק את המשוואה המנורמלת ב־$y^{-2}$ ונקבל:
\[
y^2 y' - \frac{1}{x^2} y^3 = \frac{1}{x^2}.
\]

נציב $z(x)=y^3$ ואת הנגזרת שקיבלנו:
\[
\frac{1}{3} z'(x) - \frac{1}{x^2}z(x) = \frac{1}{x^2}.
\]

נכפיל ב־3:  
\[
z'(x) - \frac{3}{x^2}z(x) = \frac{3}{x^2}.
\]

זו משוואה לינארית מנורמלת במישור $x,z(x)$. 
גורם האינטגרציה הוא:
\[
\mu(x) = e^{\int -3/x^2\, dx} = e^{3/x}.
\]

נכפיל ונקבל:  
\[
\big(z(x)e^{3/x}\big)' = \frac{3}{x^2}e^{3/x}.
\]

נבצע הצבה $t=3/x$, $dt=-\tfrac{3}{x^2}dx$:  
\[
\int \frac{3}{x^2} e^{3/x}\, dx = -\int e^t dt = -e^t.
\]

ולכן:
\[
z(x)e^{3/x} = -e^{3/x} + C.
\]

נחזור ל־$z(x)$:  
\[
z(x) = C e^{-3/x} - 1.
\]

נזכור $z=y^3$, ולכן הפתרון הכללי במישור $x,y(x)$:
\[
y(x) = \sqrt[3]{C e^{-3/x} - 1}, \qquad x \neq 0.
\]

\underline{גישה נוספת לפתרון – מד׳׳ר פרידה}

נחזור למד״ר המנורמלת:  
\[
y' = \frac{1+y^3}{x^2 y^2}.
\]

נפריד משתנים:
\[
y'= \textcolor{red}{\frac{1}{x^2}} \cdot \textcolor{blue}{\frac{1+y^3}{y^2}}
\]
בשלב הראשון ניתן לראות שאין פתרונות חשודים לסינגולריים.  

בשלב השני נכתוב את שני האינטגרלים שלנו:
\[
F(x)=\int f(x)\, dx = \int \frac{1}{x^2}\, dx = -\frac{1}{x}, 
\qquad 
G(y)=\int \frac{1}{g(y)}\, dy = \int \frac{y^2}{1+y^3}\, dy 
= \frac{1}{3}\ln|1+y^3|.
\]

בשלב השלישי נכתוב את הפתרון הכללי בצורה סתומה:
\[
\frac{1}{3}\ln|1+y^3| = -\frac{1}{x} + c
\;\;\;\Rightarrow\;\;\;
\ln|1+y^3| = -\frac{3}{x} + \tilde{c}
\;\;\;\Rightarrow\;\;\;
|1+y^3| = e^{\tilde{c}} e^{-3/x}.
\]

ומכאן:
\[
1+y^3 = c_1 e^{-3/x}, \quad c_1 \neq 0
\;\;\;\Rightarrow\;\;\;
y(x) = \sqrt[3]{c_1 e^{-3/x} - 1}, \qquad x \neq 0.
\]
וקיבלנו את אותו הפתרון.

\item \textbf{פתרון פרטי עבור $y(1)=1$}  

נשתמש בתנאי ההתחלה:  
\[
1 = \sqrt[3]{C e^{-3/1} - 1}.
\]

נעלה בחזקה שלישית:  
\[
1 = C e^{-3} - 1 \quad \Rightarrow \quad C = 2e^3.
\]

ולכן הפתרון הפרטי הוא:
\[
\boxed{\, y(x) = \sqrt[3]{\,2e^3 e^{-3/x} - 1\,}, \qquad x \neq 0. \,}
\]
הגרף נראה כך:
\begin{figure}[H]
  \centering
  \includegraphics[width=0.7\textwidth]{bern_telhai.png}
  \caption{הגרף של הפתרון $y(x) = \sqrt[3]{\,2e^3 e^{-3/x} - 1\,}$. מסומן תנאי ההתחלה $y(1)=1$.}
\end{figure}
\end{enumerate}


\solution

\begin{enumerate}[label=\alph*.]

\item
נכתוב את המשוואה בצורת ברנולי:
\[
T' + kT = \sigma T^4.
\]

זהו מקרה כללי של משוואת ברנולי מהצורה
\[
y' + p(x) y = g(x) y^\alpha, \qquad \alpha=4.
\]

מאחר ש־$\alpha>0$, הפתרון $T\equiv0$ הוא פתרון סינגולרי.

\item
נבצע הצבה $z(t)=T^{1-\alpha}=T^{-3}$.  
נקבל:
\[
z'(t) = -3 T^{-4}T'.
\]

נחלק את המשוואה ב־$T^4$ ונקבל:
\[
\frac{T'}{T^4} + \frac{k}{T^3} = \sigma.
\]

נציב $z(t)=T^{-3}$, $z'(t)=-3T^{-4}T'$:
\[
-\tfrac{1}{3} z'(t) + kz(t) = \sigma.
\]

נכפיל ב־$-3$:
\[
z'(t) - 3k z(t) = -3\sigma.
\]

קיבלנו משוואה לינארית מנורמלת במשתנה $z(t)$.  
גורם האינטגרציה הוא:
\[
\mu(t) = e^{\int -3k \, dt} = e^{-3kt}.
\]

נכפיל:
\[
(z(t)e^{-3kt})' = -3\sigma e^{-3kt}.
\]

נבצע אינטגרציה:
\[
z(t)e^{-3kt} = \frac{\sigma}{k} e^{-3kt} + C.
\]

ולכן:
\[
z(t) = \frac{\sigma}{k} + C e^{3kt}.
\]

מאחר ש־$z(t)=T^{-3}$, נקבל:
\[
T(t) = \Big(\tfrac{1}{\tfrac{\sigma}{k}+Ce^{3kt}}\Big)^{1/3},\qquad t\geq0
\]

זהו הפתרון הכללי.

\item
נשתמש בתנאי ההתחלה $T(0)=T_0=100$ כדי למצוא את $C$.  
ב־$t=0$:
\[
100^{-3} = \tfrac{1}{10^6} = \frac{\sigma}{k} + C.
\]

עבור $k=1$, $\sigma=0.01$:
\[
C = \frac{1}{10^6} - 0.01.
\]

ולכן הפתרון הפרטי:
\[
\boxed{\, T(t) = \Bigg(\frac{1}{0.01 + \Big(\tfrac{1}{10^6}-0.01\Big)e^{3t}}\Bigg)^{1/3}, \qquad t\geq0. \,}
\]

\end{enumerate}
נראה את גרף הפתרון:
\begin{figure}[H]
    \centering
    \includegraphics[width=0.7\textwidth]{bern_radiation.png}
    \caption{הפתרון הפרטי של המשוואה 
    עם $k=1$, $\sigma=0.01$, $T_0=100$.
    }
    \label{fig:bernoulli_particular}
\end{figure}
*שימו לב שכיוון שהמד׳׳ר אוטונומית, ניתן היה לפתור גם בשיטת הפרדת משתנים.

%%%CUT%%%

\newpage
\subsection{משוואת הקו הישר – שיטת הצבה} 

צורה כללית למד׳׳ר מסדר 1 המתאימה לטיפוס קו ישר תהא
\begin{equation}
y' = f(ax+by+c).
\end{equation}

נבצע את ההצבה $v(x)=ax+by+c$, ונקבל:
\[
v' = a+by' = a+b\cdot f(v).
\]

זוהי מד׳׳ר פרידה במישור $(x,v(x))$:  
\begin{equation}
v' = \underbrace{1}_{f(x)} \cdot \underbrace{(a+b\cdot f(v))}_{g(v)}.
\end{equation}
לכן נוכל ליישם את האלגוריתם אשר מניב את הפתרון הכללי בצורה סתומה כפי שכבר ראינו.

\textbf{אלגוריתם לפתרון:}

\begin{enumerate}
  \item מחפשים את השורשים של $g(v)$ (כלומר, פותרים $g(v)=0$).  
  אם $v_0$ שורש של $g(v)$, כלומר $g(v_0)=0$, אז $v\equiv v_0$ הוא פתרון סינגולרי/נוסף/קבוע (וחשוד לפחות).
  
  \item נגדיר פונקציות:
  \[
  G(v)=\int \tfrac{1}{g(v)}\,dv, 
  \qquad 
  F(x)=\int f(x)\,dx=\int 1\,dx\equiv x.
  \]
  
  \item הפתרון הכללי בצורה סתומה (ולא כולל את הפתרונות הסינגולריים):
  \begin{equation}
  G(v)=x+C.
  \end{equation}
  
  \item בסוף חוזרים למישור $x,y$ ע׳׳י $v(x)=ax+by+c$.
\end{enumerate}

\begin{insight}
    שימו לב כי היתרון בשיטה הוא שאנו מקבלים (שוב) פתרון סגור לבעיה שיכולה להיות לא לינארית (שכן תבנית הקו הישר יכולה להיות בכל פעולה לא לינארית באשר היא), לאחר ההצבה. זו גם כן שיטת אד-הוק.
\end{insight}

\example{}

קבלו פתרון פרטי למד׳׳ר
הבאה:
\[
y' = \frac{1}{(4x+3y+10)^2}, \qquad y(-1)=-2.
\]

\explanation  
נבצע הצבה $v(x)=4x+3y+10$, ואז:
\[
v'(x) = 4+3y' = 4+3\cdot \frac{1}{v^2} = \frac{4v^2+3}{v^2}.
\]

מדובר במד׳׳ר פרידה.  

אין פתרונות סינגולריים כי אין פתרונות ממשיים שמאפסים את $g(v)$.

נחשב:
\[
F(x)=\int 1\,dx = x, 
\qquad 
G(v)=\int \frac{v^2}{4v^2+3}\,dv.
\]

נחשב את האינטגרל:
\[
G(v) = \int \frac{v^2}{4v^2+3}\,dv.
\]

\textbf{שלב א׳ – פירוק השבר:}  
 
נרצה לכתוב את השבר בצורה פשוטה יותר שתכלול קבוע ועוד שבר עם מכנה $4v^2+3$.

נחלק את המונה והמכנה ב־$4$ כדי “לנרמל” את מקדם $v^2$ במכנה:
\[
\frac{v^2}{4v^2+3} 
= \frac{1}{4}\cdot \frac{4v^2}{4v^2+3}.
\]

כעת נבחין כי $4v^2$ שווה בדיוק ל־$(4v^2+3)-3$.  
לכן נכתוב:
\[
\frac{4v^2}{4v^2+3} 
= \frac{(4v^2+3)-3}{4v^2+3} 
= \frac{4v^2+3}{4v^2+3} - \frac{3}{4v^2+3}.
\]

האיבר הראשון מצטמצם ל־$1$, ולכן:
\[
\frac{4v^2}{4v^2+3} = 1 - \frac{3}{4v^2+3}.
\]

ולכן הביטוי המקורי הופך ל:
\[
\frac{v^2}{4v^2+3} 
= \frac{1}{4}\Bigg(1 - \frac{3}{4v^2+3}\Bigg).
\]


\textbf{שלב ב׳ – פיצול לאינטגרלים:}
\[
\int \frac{v^2}{4v^2+3}\,dv
= \frac{1}{4}\int 1\,dv - \frac{3}{4}\int \frac{1}{4v^2+3}\,dv.
\]

\textbf{שלב ג׳ – חישוב האיבר הראשון:}
\[
\frac{1}{4}\int 1\,dv = \frac{v}{4}.
\]

\textbf{שלב ד׳ – חישוב האיבר השני:}  
נשתמש בהצבה סטנדרטית לאינטגרל מהצורה $\int \tfrac{1}{av^2+b}\,dv$.

ידוע:
\[
\int \frac{1}{\alpha^2 v^2+\beta^2}\,dv 
= \frac{1}{\alpha\beta}\arctan\Big(\frac{\alpha v}{\beta}\Big).
\]

כאן: $4v^2+3 = (2v)^2+(\sqrt{3})^2$, כלומר $\alpha=2, \beta=\sqrt{3}$.  
ולכן:
\[
\int \frac{1}{4v^2+3}\,dv = \frac{1}{2\sqrt{3}} \arctan\Big(\frac{2v}{\sqrt{3}}\Big).
\]
\textbf{שלב ה׳ – הצבה חזרה:}
\[
-\frac{3}{4}\int \frac{1}{4v^2+3}\,dv
= -\frac{3}{4}\cdot \frac{1}{2\sqrt{3}}\arctan\Big(\frac{2v}{\sqrt{3}}\Big)
= -\frac{\sqrt{3}}{8}\arctan\Big(\frac{2v}{\sqrt{3}}\Big).
\]
ניתן לקבל את אותה התוצאה גם אם לא זוכרים/הנוסחה הכללית לא נגישה עבורכם.
נשים לב:
\[
4v^2+3 = 3\left(1+\frac{4}{3}v^2\right).
\]

ולכן:
\[
\int \frac{1}{4v^2+3}\,dv = \frac{1}{3}\int \frac{1}{1+\big(\tfrac{2}{\sqrt{3}}v\big)^2}\,dv.
\]

נגדיר \(u=\tfrac{2}{\sqrt{3}}v \;\Rightarrow\; du=\tfrac{2}{\sqrt{3}}dv \;\Rightarrow\; dv=\tfrac{\sqrt{3}}{2}du\).

\[
\frac{1}{3}\int \frac{1}{1+u^2}\cdot \frac{\sqrt{3}}{2}\,du
= \frac{\sqrt{3}}{6}\arctan(u)+C
= \frac{\sqrt{3}}{6}\arctan\!\Big(\tfrac{2v}{\sqrt{3}}\Big).
\]

\[
-\frac{3}{4}\cdot \Bigg(\frac{\sqrt{3}}{6}\arctan\!\Big(\tfrac{2v}{\sqrt{3}}\Big)\Bigg)
= -\frac{\sqrt{3}}{8}\arctan\!\Big(\tfrac{2v}{\sqrt{3}}\Big).
\]

\textbf{שלב ו׳ – איחוד התוצאה:}
\[
G(v) = \frac{v}{4}-\frac{\sqrt{3}}{8}\arctan\Big(\frac{2v}{\sqrt{3}}\Big)+C.
\]

ולכן:
\[
\frac{v}{4}-\frac{\sqrt{3}}{8}\arctan\Big(\tfrac{2v}{\sqrt{3}}\Big)=x+C.
\]

נחזור ל־$x,y$:  
\[
\frac{4x+3y+10}{4}-\frac{\sqrt{3}}{8}\arctan\Big(\tfrac{2}{\sqrt{3}}(4x+3y+10)\Big)=x+C.
\]

זהו הפתרון הכללי בצורה סתומה. שימו לב כי ממש לא סביר לקבל את $y(x)$ בצורה מפורשת, ועל כן נשאיר בצורה סתומה.

\textbf{פתרון פרטי:}  
נציב את תנאי ההתחלה $y(-1)=-2$. נקבל $0-0=-1+C$.
מתקבל $C=1$, ולכן, לכאורה מתקבל הפתרון הפרטי:
\[
\,\frac{4x+3y+10}{4}-\frac{\sqrt{3}}{8}\arctan\Big(\tfrac{2}{\sqrt{3}}(4x+3y+10)\Big)=x+1\,.
\]

\begin{insight}
שימו לב: תנאי ההתחלה $(-1,-2)$ לא נמצא בתחום ההגדרה של המד׳׳ר, ולכן הפתרון פרטי זה \textbf{לא תקף}.  מדובר בפתרון לא אמיתי. למעשה, לאחר מציאת הפתרון הכללי, היה ניתן לומר מראש שלא קיים פתרון פרטי עבור תנאי התחלה זה. שימו לב כי תנאי ההתחלה הנתון כמו כן מאפס את המכנה באגף ימין של המד׳׳ר. אם נרצה להרחיב, נבין כי עבור כל תנאי ההתחלה היושב על הפונקציה הלינארית $y=-\frac{4}{3}x-\frac{10}{3}$, לא קיים למד׳׳ר פתרון פרטי ממשי.
לכן למעשה לא קיים פתרון פרטי המקיים את תנאי ההתחלה הנתון.  
עם זאת, ניתן להציג את הפתרון הכללי עבור פתרונות פרטיים פוטנציאליים אחרים המגיעים מקשר שונה מזה שהוצג לעיל.  
\end{insight}
מסקנה : לא קיים פתרון פרטי לבעיה זו עבור תנאי ההתחלה הנתון.


\newpage
\underline{תרגילים}

\exercise{}

קבלו פתרון פרטי למד׳׳ר:
\[
y' = \frac{1}{2x+y}, \qquad y(0)=1.
\]

\exercise{}

פתרו את המשוואה:
\[
y' = (3x-y+1)^2, \qquad y(1)=0.
\]

\exercise{}

פתרו את המשוואה:
\[
y'=\sin(x+y), \qquad y(0)=0.
\]

\exercise{}

גרף הפונקציה $y=f(x)$ עובר דרך הנקודה $(0,4)$.  
מצאו את הפונקציה, אם שיפוע הנורמל בכל נקודה שעל גרף הפונקציה שווה לסכום של נקודת ההשקה.



\newpage
\underline{פתרונות}
\solution  

נזהה צורה כללית: $y'=f(ax+by+c)$, כאן $a=2$, $b=1$, $c=0$.  
נבצע הצבה:
\[
v(x)=2x+y, \qquad v'(x)=2+y' = 2+\frac{1}{v}.
\]

מדובר במד׳׳ר פרידה: 
\[
v' = \frac{2v+1}{v}.
\]

\textbf{שלב 1 – חיפוש סינגולרים:}  
$g(v)=\tfrac{2v+1}{v}$ מתאפס כאשר $2v+1=0 \;\Rightarrow\; v=-\tfrac{1}{2}$.  
לכן $v\equiv -\tfrac{1}{2}$ הוא פתרון סינגולרי (חשוד).

\textbf{שלב 2 – כתיבת אינטגרלים:}
\[
F(x)=\int 1\,dx = x, 
\qquad
G(v)=\int \frac{v}{2v+1}\,dv.
\]

\textbf{שלב 3 – חישוב $G(v)$:}  
נבצע פירוק: 
\[
\frac{v}{2v+1} = \frac{1}{2}-\frac{1}{4v+2}.
\]

ולכן:
\[
G(v)=\frac{v}{2}-\frac{1}{4}\ln|2v+1|.
\]

\textbf{שלב 4 – פתרון כללי:}
\[
\frac{v}{2}-\frac{1}{4}\ln|2v+1| = x+C.
\]

חזרה ל־$x,y$:
\[
\frac{2x+y}{2}-\frac{1}{4}\ln|2(2x+y)+1|=x+C.
\]

\textbf{שלב 5 – תנאי ההתחלה והגדרת התחום:}  
לשם פשטות, ניתן להציב את תנאי ההתחלה $(0,1)$ ב-$v$. נקבל כי $v=1$. נציב בפתרון הכללי ב-$v(x)$: $\tfrac{1}{2}-\tfrac{1}{4}\ln 3= C$. ולכן הפתרון הפרטי הוא ב-$y(x)$:

\[
\boxed{\;\frac{2x+y}{2}-\tfrac{1}{4}\ln|4x+2y+1|=x+\tfrac{1}{2}-\tfrac{1}{4}\ln 3\;}
\]

\textbf{פתרון סינגולרי:}  
מצאנו קודם כי $v\equiv -\tfrac{1}{2}$ הוא פתרון סינגולרי (שימו לב שהוא אכן בהכרח סינגולרי כי לא ניתן לקבל אותו מהפתרון הכללי ע׳׳י הצבת קבוע מסוים), כלומר:
\[
y(x)=-2x-\tfrac{1}{2}.
\]
אך פתרון זה לא עובר בתנאי ההתחלה ולכן הוא אינו פתרון לבעיה שלנו.
\textbf{תחום ההגדרה:}  
הפתרון הכללי אינו מוגדר כאשר $v=0 \;\Rightarrow\; 2x+y=0$.
לכן תחום ההגדרה של הפתרון הפרטי מוגבל להיות קטע ב־$\mathbb{R}$ שאינו חוצה את הישר:
\[
y=-2x.
\]


\solution  

צורה: $y'=f(ax+by+c)$, כאן $a=3$, $b=-1$, $c=1$.  
נבצע הצבה:
\[
v(x)=3x-y+1, \qquad v'=3-y' = 3-(v^2).
\]

מדובר במד׳׳ר פרידה:
\[
v' = 3-v^2.
\]

\textbf{שלב 1 – סינגולרים:}  
$g(v)=3-v^2=0 \;\Rightarrow\; v=\pm\sqrt{3}$.

\textbf{שלב 2 – כתיבת אינטגרלים:}
\[
F(x)=\int 1\,dx = x, 
\qquad
G(v)=\int \frac{1}{3-v^2}\,dv.
\]

\textbf{שלב 3 – חישוב $G(v)$:}  
נבצע פירוק חלקי:
\[
\frac{1}{3-v^2}=\frac{1}{2\sqrt{3}}\Big(\frac{1}{\sqrt{3}+v}+\frac{1}{\sqrt{3}-v}\Big).
\]

ולכן:
\[
G(v)=\frac{1}{2\sqrt{3}}\ln\left|\frac{\sqrt{3}+v}{\sqrt{3}-v}\right|.
\]

\textbf{שלב 4 – פתרון כללי:}
\[
\frac{1}{2\sqrt{3}}\ln\left|\frac{\sqrt{3}+v}{\sqrt{3}-v}\right|=x+C.
\]

חזרה ל־$x,y$: $v=3x-y+1$.  
\textbf{שלב 5 – תנאי התחלה:}  
ב־$x=1,y=0$: $v=3\cdot1-0+1=4$.  
נציב: 
\[
\frac{1}{2\sqrt{3}}\ln\left|\frac{\sqrt{3}+4}{\sqrt{3}-4}\right|=1+C
\;\;\;\Rightarrow\;\;\;
C=\frac{1}{2\sqrt{3}}\ln\left|\frac{\sqrt{3}+4}{\sqrt{3}-4}\right|-1.
\]

ולכן הפתרון הפרטי הוא:
\[
\;\frac{1}{2\sqrt{3}}\ln\left|\frac{\sqrt{3}+3x-y+1}{\sqrt{3}-(3x-y+1)}\right|=x+\frac{1}{2\sqrt{3}}\ln\left|\frac{\sqrt{3}+4}{\sqrt{3}-4}\right|-1\;
\]

\textbf{פתרונות סינגולריים:}  
מצאנו כי $v=\pm\sqrt{3}$ הם פתרונות חשודים לסינגולריים, כלומר:
\[
y(x)=3x+1-\sqrt{3}, 
\qquad 
y(x)=3x+1+\sqrt{3}.
\]
אך, נשים לב כי הם אינם פתרונות פרטיים לבעיה שכן הם אינם עוברים בתנאי ההתחלה הנתון.
\textbf{תחום ההגדרה:}  
המד׳׳ר ב-$x,y(x)$ למעשה מוגדרת עבור כל $x$ ממשי ועל כן הפתרון הפרטי הוא:
\[
\boxed{\;\frac{1}{2\sqrt{3}}\ln\left|\frac{\sqrt{3}+3x-y+1}{\sqrt{3}-(3x-y+1)}\right|=x+\frac{1}{2\sqrt{3}}\ln\left|\frac{\sqrt{3}+4}{\sqrt{3}-4}\right|-1\;,\qquad x\in\mathbb{R}}
\]
*שימו לב כי הפתרון לא מוגדר עבור קומבינציות מסוימות של $x,y$ אשר מאפסות את המונה ו/או המכנה של האיבר הלוגריתמי.

%%%CUT%%%

\solution  

צורה: $y'=f(ax+by+c)$, כאן $a=1$, $b=1$, $c=0$.  

הצבה:
\[
v(x)=x+y, \qquad v'=1+y'=1+\sin(v).
\]

מדובר במד׳׳ר פרידה:
\[
\frac{dv}{dx}=1+\sin(v).
\]

\textbf{שלב 1 – סינגולרים:}  

הפונקציה $g(v)=1+\sin v$ מתאפסת כאשר $\sin v=-1$.  
כלומר:
\[
v = \tfrac{3\pi}{2} + 2n\pi, \qquad n\in\mathbb{Z}.
\]

במונחי $x,y$: מאחר ש-$v=x+y$, הפתרונות הסינגולריים הם הישרים:
\[
y(x) = -x + \Big(\tfrac{3\pi}{2} + 2n\pi\Big), \qquad n\in\mathbb{Z}.
\]

לכן קיימת משפחה אינסופית של פתרונות חשודים לסינגולריים. 


\textbf{שלב 2 – כתיבת אינטגרלים:}
\[
F(x)=\int 1\,dx = x, 
\qquad
G(v)=\int \frac{1}{1+\sin v}\,dv.
\]
\textbf{שלב 3 – חישוב $G(v)$:}  

עלינו לחשב:
\[
G(v)=\int \frac{1}{1+\sin v}\,dv.
\]

כדי לפשט את האגף, נכפיל מונה ומכנה ב־$(1-\sin v)$:
\[
\frac{1}{1+\sin v}\cdot \frac{1-\sin v}{1-\sin v}
= \frac{1-\sin v}{1-\sin^2 v}.
\]

אבל $1-\sin^2 v=\cos^2 v$.  
ולכן:
\[
\frac{1}{1+\sin v}=\frac{1-\sin v}{\cos^2 v}.
\]

כעת נפרק את השבר לשני איברים:
\[
\frac{1-\sin v}{\cos^2 v}=\frac{1}{\cos^2 v}-\frac{\sin v}{\cos^2 v}.
\]

כלומר:
\[
\int \frac{1}{1+\sin v}\,dv = \int \frac{1}{\cos^2 v}\,dv - \int \frac{\sin v}{\cos^2 v}\,dv.
\]

האינטגרל הראשון הוא מוכר:
\[
\int \frac{1}{\cos^2 v}\,dv = \int \sec^2 v\,dv = \tan v.
\]

עבור האינטגרל השני נעשה הצבה:  
נסמן $u=\cos v \;\;\Rightarrow\;\; du=-\sin v\,dv$.  
ולכן:
\[
-\int \frac{1}{u^2}\,du = \frac{1}{u} = \frac{1}{\cos v} = \sec v.
\]

נחבר את שתי התוצאות ונקבל:
\[
G(v)=\tan v + \sec v.
\]

\textbf{שלב 4 – פתרון כללי:}  

אם כך הפתרון הכללי בצורה סתומה יהיה:
\[
\tan(v)+\sec(v)=x+C.
\]

חזרה למישור $x,y$:  
\[
\tan(x+y)+\sec(x+y)=x+C.
\]

\textbf{שלב 5 – תנאי התחלה:}  
ב־$x=0,y=0$: $v=0$, ואז $\tan(0)+\sec(0)=0+1=1$.  
כלומר:
\[
1=0+C \quad \Rightarrow \quad C=1.
\]

ולכן הפתרון הפרטי הוא:
\[
\boxed{\;\tan(x+y)+\sec(x+y)=x+1\;, \qquad x\in\mathbb{R}}
\]

\textbf{פתרונות סינגולריים:}  
כפי שמצאנו בשלב 1, $g(v)=1+\sin v=0$ כאשר $v=\tfrac{3\pi}{2}+2n\pi$, $n\in\mathbb{Z}$.  
במונחי $x,y$: 
\[
y(x)=-x+\left(\tfrac{3\pi}{2}+2n\pi\right).
\]
אף אחד מפתרונות אלה לא עובר בראשית הצירים ולכן אף אחד מהם לא מהווה פתרון פרטי לבעיה.

\textbf{תחום ההגדרה:}  
הפתרון הכללי כולל $\tan(x+y)$ ו־$\sec(x+y)$ ולכן אינו מוגדר כאשר $\cos(x+y)=0$, כלומר:
\[
x+y=\tfrac{\pi}{2}+n\pi, \qquad n\in\mathbb{Z}.
\]
למרות שהמד׳׳ר מוגדרת על כל קומבינציה של $x,y$, הפתרון אינו מוגדר על הישרים $y=-x+\tfrac{3\pi}{2}+2n\pi$ שבהם $g(v)=0$.  
לכן תחום ההגדרה של הפתרון הפרטי מוגבל להיות קטעי $\mathbb{R}$ שאינם חוצים את הקווים הללו.


\solution

\textbf{שלב 1 – כתיבת המשוואה הדיפרנציאלית:}  
שיפוע הנורמל בנקודה $(x,y)$ הוא $y'_{\perp}=-\tfrac{1}{y'}$. לפי הנתון:
\[
y'_{\perp} = x+y.
\]

ולכן:
\[
-\frac{1}{y'} = x+y \;\;\;\Rightarrow\;\;\; y' = -\frac{1}{x+y}.
\]

\textbf{שלב 2 – זיהוי המשוואה:}  
המשוואה היא מטיפוס קו ישר.  
נבצע הצבה $v=x+y \;\Rightarrow\; y=v-x$.  
נגזור: $y'=v'-1$.  
נציב:
\[
v'-1=-\frac{1}{v}.
\]

ולכן:
\[
v' = 1-\frac{1}{v}=\frac{v-1}{v}.
\]

\textbf{שלב 3 – זיהוי כמשוואה פרידה:}  
התקבלה משוואה פרידה במישור $(x,v)$:
\[
v'=\frac{1}{1}\cdot\frac{v-1}{v}, \quad f(x)=1,\;\; g(v)=\tfrac{v-1}{v}.
\]

\textbf{שלב 4 – פתרונות סינגולריים:}  
נבדוק מתי $g(v)=0$:  
\[
\frac{v-1}{v}=0 \;\;\Rightarrow\;\; v=1.
\]

כלומר $v\equiv1$ הוא פתרון חשוד לסינגולרי.  
במונחי $x,y$: $x+y=1$.

\textbf{שלב 5 – חישוב $F(x),G(v)$:}  
\[
F(x)=\int 1\,dx=x,
\]
\[
G(v)=\int \frac{v}{v-1}\,dv.
\]

נחלק ונפרק:
\[
\frac{v}{v-1}=1+\frac{1}{v-1}.
\]

ולכן:
\[
G(v)=\int \Big(1+\frac{1}{v-1}\Big)\,dv=v+\ln|v-1|.
\]

\textbf{שלב 6 – פתרון כללי:}  
\[
v+\ln|v-1|=x+C.
\]

נחזור ל־$y$: $v=x+y$,  
ולכן:
\[
x+y+\ln|x+y-1|=x+C.
\]

נפשט:
\[
y+\ln|x+y-1|=C,\qquad y=1-x, \qquad x\in \mathbb{R}.
\]

\textbf{שלב 7 – תנאי התחלה:}  
בנקודה $(0,4)$:  
\[
4+\ln|0+4-1|=C \;\;\Rightarrow\;\; C=4+\ln(3).
\]

ולכן הפתרון הפרטי הוא:
\[
\boxed{\,y+\ln|x+y-1|=4+\ln 3, \quad x\in\mathbb{R}\,}.
\]
שכן הפתרון הסינגולרי לא עובר בתנאי ההתחלה ועל כן אינו מהווה פתרון פרטי לבעיה.

\newpage
\subsection{משוואת מטיפוס הומוגני} 

\textbf{צורה כללית למד׳׳ר מטיפוס הומוגני:}
\begin{equation}
y' = f\!\left(\frac{y}{x}\right).
\end{equation}

באופן כללי, אם נזהה כי אגף ימין של המד׳׳ר מורכב כולו מתבניות חוזרות של $f\!\left(\frac{y}{x}\right)$, נבין כי מדובר במשוואה מטיפוס הומוגני. בדיקה פורמלית תכלול את התהליך הבא: 

נאמר כי פונקציה $F(x,y)$ היא \textbf{הומוגנית} אם מתקיים:
\begin{equation}
F(tx,ty)=F(x,y), \qquad \forall t>0.
\end{equation}

במקרה שלנו, אגף ימין תלוי רק ביחס $\tfrac{y}{x}$:
\[
F(x,y)=f\!\left(\frac{y}{x}\right).
\]

נבדוק:
\[
F(tx,ty)=f\!\left(\frac{ty}{tx}\right)=f\!\left(\frac{y}{x}\right)=F(x,y).
\]

כלומר, הביטוי אינו משתנה תחת כפל של $x,y$ בפרמטר $t$, ולכן המשוואה
\[
y' = f\!\left(\frac{y}{x}\right)
\]
היא אכן \textbf{משוואה דיפרנציאלית הומוגנית}.

\textbf{אלגוריתם לפתרון:}

\begin{enumerate}[label=\textbf{\Alph*.}]
  \item נציב $v(x)=\tfrac{y(x)}{x}$, כלומר $y=vx$.  
  נגזור: 
  \[
  y' = v'x + v.
  \]

  \item נציב במשוואה:
  \[
  f(v) = f\!\Big(\tfrac{y}{x}\Big) = y' = v + xv' 
  \;\;\Rightarrow\;\;
  v'(x) = \frac{1}{x}\big(f(v)-v\big).
  \]

  מתקבלת משוואה פרידה במישור $(x,v)$:
  \[
  v' = f(x)\cdot g(v).
  \]
\end{enumerate}

\textbf{תזכורת (נוספת) לגישת הפתרון של משוואה פרידה:}

\begin{enumerate}
  \item מחפשים את השורשים של $g(v)$, כלומר פותרים $g(v)=0$.  
  אם $v_0$ שורש של $g(v)$, אז $v\equiv v_0$ הוא פתרון סינגולרי/קבוע (חשוד לפחות).
  
  \item מגדירים פונקציות:
  \[
  G(v)=\int \frac{1}{g(v)}\,dv, 
  \qquad 
  F(x)=\int f(x)\,dx=\int \frac{1}{x}\,dx\equiv\ln|x|.
  \]
  
  \item הפתרון הכללי בצורה סתומה (לא כולל פתרונות סינגולריים):
  \begin{equation}
  G(v)=\ln|x|+C.
  \end{equation}
  
  \item חוזרים למישור $x,y$ ע׳׳י ההצבה $y=vx$.
\end{enumerate}

\example{}

קבלו פתרון כללי למד׳׳ר:
\[
y' = \frac{2xy}{x^2-y^2}.
\]

\explanation

המד׳׳ר לא לינארית. המשוואה לא מתאימה לאף מהשיטות שלמדנו עד כה. בשלב זה ניתן לחשוד שמדובר במשוואה מטיפוס הומוגני.

\textbf{ בדיקה שהמד׳׳ר הומוגנית}

נבדוק אם אגף ימין הומוגני באותה מידה במשתנים $x,y$.

נבצע הצבה $x\mapsto tx, \; y\mapsto ty$:  
\[
\frac{2(tx)(ty)}{(tx)^2-(ty)^2}
= \frac{2t^2xy}{t^2(x^2-y^2)}
= \frac{2xy}{x^2-y^2}.
\]

האיבר $t$ מצטמצם ולכן הביטוי נשאר זהה.  
מכאן ש־$\tfrac{2xy}{x^2-y^2}$ היא פונקציה הומוגנית, ולכן המשוואה הנתונה היא \textbf{מד׳׳ר הומוגנית}.


נסדר את המד׳׳ר הנתונה ע׳׳י חלוקה של המונה והמכנה ב-$x^{2}$ ונקבל:
\[
y' = \frac{2\frac{y}{x}}{1-\left(\frac{y}{x}\right)^{2}}.
\]

נבצע הצבה $v=\tfrac{y}{x}$, כלומר $y=xv$.  
נגזור: 
\[
y'=v+ xv'.
\]

נציב במשוואה:
\[
v+xv'=\frac{2v}{1-v^2}.
\]

ולכן:
\[
v'=\frac{1}{x}\left(\frac{2v}{1-v^2}-v\right)
= \frac{1}{x}\cdot\frac{2v-v(1-v^2)}{1-v^2}
= \frac{1}{x}\cdot\frac{v(1+v^2)}{1-v^2}.
\]

\textbf{שלב 1 – פתרונות סינגולריים}  
\[
g(v)=\frac{v(1+v^2)}{1-v^2}.
\]

השורש הממשי היחיד: $v=0$, כלומר $y\equiv 0$ הוא פתרון חשוד סינגולרי.

\textbf{שלב 2 – חישוב $F(x)$}
\[
F(x)=\int \frac{1}{x}\,dx = \ln|x|.
\]

\textbf{שלב 3 – חישוב $G(v)$}
\[
G(v)=\int \frac{1-v^2}{v(1+v^2)}\,dv.
\]

\textbf{פירוק לשברים חלקיים}

נרצה לפרק:
\[
\frac{1-v^2}{v(1+v^2)}.
\]

נניח:
\[
\frac{1-v^2}{v(1+v^2)} = \frac{A}{v} + \frac{Bv+C}{v^2+1}.
\]

נכפול במכנה $v(1+v^2)$:
\[
1-v^2 = A(1+v^2) + (Bv+C)\cdot v.
\]

נפתח סוגריים:
\[
1-v^2 = A + Av^2 + Bv^2 + Cv.
\]

נאסוף איברים לפי חזקות של $v$:
\[
1-v^2 = (A) + (C)v + (A+B)v^2.
\]

כעת נשווה מקדמים עם האגף השמאלי $1 + 0\cdot v -1\cdot v^2$:

\[
\begin{cases}
A = 1, \\
C = 0, \\
A+B = -1.
\end{cases}
\]

ממערכת זו נקבל:
\[
A=1, \quad C=0, \quad B=-2.
\]

ולכן הפירוק הוא:
\[
\frac{1-v^2}{v(1+v^2)} = \frac{1}{v} - \frac{2v}{v^2+1}.
\]


מכאן:
\[
G(v)=\int \Big(\frac{1}{v}-\frac{2v}{v^2+1}\Big)\,dv
= \ln|v|-\ln(v^2+1).
\]

\textbf{שלב 4 – פתרון כללי:}
\[
\ln\Big|\tfrac{v}{1+v^2}\Big|=\ln|x|+C.
\]

נעלה באקספוננט:
\[
\Big|\tfrac{v}{1+v^2}\Big| = C_1 |x|, \qquad C_1\neq 0.
\]

כלומר:
\[
v=C_1x(1+v^2), \qquad C_1\neq 0.
\]
לצד $V\equiv0$. נשים לב כי אפשר לשחר את האילוץ על $C_1$ ולהכיל את הפתרון החשוד לסינגולרי בתוך הפתרון הכללי ולקבל:
\[
v=C_1x(1+v^2).
\]

\textbf{שלב 5 – חזרה ל־$y$:}  
נזכור כי $y=xv$.  
נקבל:
\[
C_1xv^2-v+C_1x=0.
\]

זו משוואה ריבועית רגילה ב־$v$ עם הפתרונות:
\[
v=\frac{1\pm\sqrt{1-4C_1^2x^2}}{2C_1x}.
\]

ולכן:
\[
\boxed{y_1(x),y_2(x) = x\cdot v
= \frac{1\pm\sqrt{1-4C_1^2x^2}}{2C_1},\qquad x\in\mathbb{R}}.
\]

%%%CUT%%%

\newpage
\underline{תרגילים}
\exercise{}

קבלו פתרון כללי למד׳׳ר הבאה:
\[
y'=\frac{x^2+y^2}{2xy}.
\]

\exercise{}

קבלו פתרון כללי למד׳׳ר הבאה:
\[
y' = \frac{y^2}{x^2} + \frac{5}{4}.
\]


\exercise{}

קבלו פתרון כללי למד׳׳ר הבאה:
\[
(xy+y^2)y' = y^2.
\]

\exercise{}

נקודה $A(x,y)$ מונחת על עקומה. משיק לעקומה בנקודה $A$ חותך את הצירים בנקודות $B$ ו־$C$ (ראו סרטוט מטה).
מהנקודה $A$ מורידים את שני האנכים $AD$ ו־$AE$ לשני הצירים. נקודה $O$ ראשית הצירים.  
נתון כי שטח המשולש $BOC$ גדול פי $3.6$ משטח המלבן $ADOE$. מצאו את משפחת העקומות.  

\begin{center}
\begin{tikzpicture}[scale=1.2]

  % axes
  \draw[->] (-0.5,0) -- (6.5,0) node[right] {$x$};
  \draw[->] (0,-0.5) -- (0,6) node[above] {$y$};

  % tangent line at A
  \draw[red, thick, domain=-0.5:6.5] plot(\x,{ -0.8*\x + 4.8 });

  % actual curve: y=0.2x^2 -1.6x +5.6
  \draw[blue, thick, domain=0.5:5.5, samples=200] 
    plot(\x,{0.2*\x*\x -1.6*\x + 5.6});

  % tangent point A
  \coordinate (A) at (2,3.2);
  \filldraw (A) circle (1.5pt) node[above right] { $A$};

  % rectangle projections
  \coordinate (D) at (2,0);
  \coordinate (E) at (0,3.2);
  \coordinate (O) at (0,0);
  \filldraw (D) circle (1.5pt) node[below] {$D$};
  \filldraw (E) circle (1.5pt) node[left] {$E$};
  \filldraw (O) circle (1.5pt) node[below left] {$O$};

  % rectangle shading
  \fill[orange!60,opacity=0.8] (O)--(D)--(A)--(E)--cycle;

  % intercepts of tangent
  \coordinate (B) at (6,0);
  \coordinate (C) at (0,4.8);
  \filldraw (B) circle (1.5pt) node[below] {$B$};
  \filldraw (C) circle (1.5pt) node[left] {$C$};

  % labels
  \node[blue] at (4.2,2.6) {עקומה};
  \node[red] at (4.9,1.2) {משיק};

\end{tikzpicture}
\end{center}


\newpage
\underline{פתרונות}
\solution

\textbf{ בדיקה שהמד׳׳ר הומוגנית}  

נבדוק אם אגף ימין הומוגני.  

נבצע הצבה $x \mapsto tx, \; y \mapsto ty$:
\[
\frac{(tx)^2+(ty)^2}{2(tx)(ty)}
= \frac{t^2(x^2+y^2)}{2t^2xy}
= \frac{x^2+y^2}{2xy}.
\]

האיבר $t$ מצטמצם, ולכן אגף ימין אינו משתנה תחת כפל ב־$t$.  
מכאן שמדובר ב־\textbf{משוואה הומוגנית}.

\textbf{שלב 1 – הצבה} 

נבצע את ההצבה ההומוגנית הרגילה:
\[
v=\frac{y}{x}, \qquad y=vx.
\]

נגזור: $y'=v+xv'$, ונציב במשוואה:
\[
v+xv'=\frac{x^2+v^2x^2}{2xvx}=\frac{1+v^2}{2v}.
\]

\textbf{שלב 2 – בידוד $v'$:}
\[
xv'=\frac{1+v^2}{2v}-v=\frac{1-v^2}{2v},
\]
ולכן:
\[
v'=\frac{1}{x}\cdot \frac{1-v^2}{2v}.
\]

זהו מקרה של משוואה פרידה במישור $(x,v)$:
\[
v' = f(x)\cdot g(v), \qquad f(x)=\tfrac{1}{x}, \; g(v)=\tfrac{1-v^2}{2v}.
\]

\textbf{שלב 3 – פתרונות סינגולריים.}  
נבדוק מתי $g(v)=0$:  
\[
\frac{1-v^2}{2v}=0 \;\;\Rightarrow\;\; 1-v^2=0 \;\;\Rightarrow\;\; v=\pm 1.
\]

ולכן קיימים פתרונות חשודים לסינגולריים:  
\[
y=\pm x.
\]

\textbf{שלב 4 – חישוב $F(x), G(v)$:}  
\[
F(x)=\int \tfrac{1}{x}\,dx=\ln|x|,
\]
\[
G(v)=\int \frac{2v}{1-v^2}\,dv.
\]

נבצע הצבה $u=1-v^2$, $du=-2v\,dv$, ולכן:
\[
G(v)=\int \frac{2v}{1-v^2}\,dv=-\int \frac{1}{u}\,du=-\ln|u|=-\ln|1-v^2|.
\]

\textbf{שלב 5 – הפתרון הכללי}  
\[
G(v)=F(x)+C \;\;\Rightarrow\;\; -\ln|1-v^2|=\ln|x|+C,\qquad x\neq0
\]
אז:
\[
\ln|1-v^2|=-\ln|x|+C' 
\;\;\Rightarrow\;\; 
\ln|x(1-v^2)|=C'.
\]

נעלה באקספוננט ונקבל:
\[
|x(1-v^2)| = C_1, \qquad v=\pm 1, \qquad C_1\neq 0.
\]
אם $C_1=0$, נקבל את הפתרון החשוד להיות סינגולרי מתוך הפתרון הכללי. מכאן שניתן לשחרר את האילוץ על $C_1$ו לכתוב:


מכאן ניתן להסיר את האילוץ על $C_1$ ולרשום:
\[
x(1-v^2)=C_{1}, \qquad x\neq0.
\]

נחזור ל־$y=vx$: 
\[
x\Big(1-\Big(\tfrac{y}{x}\Big)^2\Big)=C_{1}.
\]

נפשט:
\[
x-\frac{y^2}{x}=C_{1}, \qquad x^2-y^2=C_{1}x.
\]

ולכן:
\[
y^2=x^2-C_{1}x \;\;\;\Rightarrow\;\;\; 
\,y(x)=\pm\sqrt{x^2-C_{1}x}\,.
\]

\textbf{סיכום:}  
הפתרון הכללי למשוואה ההומוגנית הוא:
\[
\boxed{y(x)=\pm\sqrt{x^2-Cx}, \qquad x\neq0}.
\]



\solution

נבצע את הצבת ההומוגניות:  
\[
v=\frac{y}{x}, \qquad y=vx.
\]

נגזור:
\[
y' = v + x v'.
\]

נציב במשוואה:
\[
v + x v' = v^2 + \frac{5}{4}.
\]

ולכן:
\[
v' = \frac{1}{x}\Big(v^2 - v + \tfrac{5}{4}\Big).
\]

זוהי משוואה פרידה במישור $(x,v)$:
\[
v' = f(x)\cdot g(v), \qquad f(x)=\tfrac{1}{x}, \quad g(v)=v^2-v+\tfrac{5}{4}.
\]

\textbf{שלב 1 – פתרונות סינגולריים:}  
נבדוק מתי $g(v)=0$:
\[
v^2-v+\tfrac{5}{4}=0.
\]

דיסקרימיננטה: $\Delta=(-1)^2-4\cdot\tfrac{5}{4}=-4<0$.  
אין פתרונות ממשיים, ולכן אין פתרונות סינגולריים.

\textbf{שלב 2 – חישוב $F(x)$:}
\[
F(x)=\int \tfrac{1}{x}\,dx=\ln|x|.
\]

\textbf{שלב 3 – חישוב $G(v)$:}
\[
G(v)=\int \frac{1}{v^2-v+\tfrac{5}{4}}\,dv.
\]

נשלים ריבוע במכנה:
\[
v^2-v+\tfrac{5}{4}=\Big(v-\tfrac{1}{2}\Big)^2+1.
\]

ולכן:
\[
G(v)=\int \frac{1}{(v-\tfrac{1}{2})^2+1}\,dv=\arctan\!\Big(v-\tfrac{1}{2}\Big).
\]

\textbf{שלב 4 – פתרון כללי:}
\[
\arctan\!\Big(v-\tfrac{1}{2}\Big)=\ln|x|+C,\qquad x\neq 0.
\]

נחזור ל־$y=vx$:
\[
\arctan\!\Big(\tfrac{y}{x}-\tfrac{1}{2}\Big)=\ln|x|+C.
\]

נפעיל טנגנס על שני האגפים:
\[
\tfrac{y}{x}-\tfrac{1}{2}=\tan\!\big(\ln|x|+C\big).
\]

ולכן:
\[
\frac{y}{x}=\tfrac{1}{2}+\tan\!\big(\ln|x|+C\big).
\]

נבודד את $y(x)$:
\[
\boxed{\,y(x)=x\Big(\tfrac{1}{2}+\tan(\ln|x|+C)\Big), \qquad x\neq 0.\,}
\]


\solution

נבודד את $y'$:
\[
y' = \frac{y^2}{xy+y^2} = \frac{1}{\tfrac{x}{y}+1}.
\]

\textbf{שלב 1 – הצבה:}  
נבצע את ההצבה $v=\tfrac{y}{x}$, כלומר $y=xv$.  

נגזור:  
\[
y' = v + x v'.
\]

נציב:
\[
v+xv' = \frac{1}{\tfrac{1}{v}+1} = \frac{v}{1+v}.
\]

ולכן:
\[
xv' = \frac{v}{1+v}-v = \frac{v-v(1+v)}{1+v} = -\frac{v^2}{1+v}.
\]

נחלק ב־$x$:
\[
v' = \frac{1}{x}\cdot \Big(-\frac{v^2}{1+v}\Big).
\]

\textbf{שלב 2 – פתרונות סינגולריים:}  
נחפש אפסים של $g(v)=-\tfrac{v^2}{1+v}$.  
\[
g(v)=0 \;\;\Rightarrow\;\; v=0 \;\;\Rightarrow\;\; y\equiv 0.
\]

\textbf{שלב 3 – חישוב $F(x),G(v)$:}  
\[
F(x)=\int \tfrac{1}{x}\,dx=\ln|x|,
\]
\[
G(v)=\int \frac{-(v^2)}{1+v}\,dv.
\]

החזקה הדומיננטית במונה גבוהה מבמכנה ולכן נבצע חילוק פולינומים עבור:
\(
\frac{v^2}{v+1}
\).

\textbf{שלב 1 – נתחיל בחזקה הגבוהה:}  
כמה פעמים נכנס $v+1$ ב־$v^2$? נכנס $v$ פעמים, כי:
\[
(v+1)\cdot v = v^2+v.
\]

\textbf{שלב 2 – חיסור:}  
\[
v^2 - (v^2+v) = -v.
\]

\textbf{שלב 3 – המשך חלוקה:}  
כמה פעמים נכנס $v+1$ ב־$-v$? נכנס $-1$ פעמים, כי:
\[
(v+1)\cdot (-1) = -v-1.
\]

\textbf{שלב 4 – חיסור:}  
\[
(-v) - (-v-1) = 1.
\]

\textbf{מסקנה:}  
\[
\frac{v^2}{v+1} = v - 1 + \frac{1}{v+1}.
\]


ולכן:
\[
G(v)=-\int \Big(v-1+\tfrac{1}{v+1}\Big)\,dv
= -\Big(\tfrac{v^2}{2}-v+\ln|v+1|\Big).
\]

כלומר:
\[
G(v)=-\tfrac{v^2}{2}+v-\ln|v+1|.
\]

\textbf{שלב 4 – הפתרון הכללי:}  
\[
G(v)=F(x)+C \;\;\Rightarrow\;\; -\tfrac{v^2}{2}+v-\ln|v+1|=\ln|x|+C.
\]

\textbf{שלב 5 – חזרה ל־$y$:}  
נציב $v=\tfrac{y}{x}$:
\[
-\tfrac{1}{2}\Big(\tfrac{y}{x}\Big)^2+\tfrac{y}{x}-\ln\!\Big|1+\tfrac{y}{x}\Big|
= \ln|x|+C.
\]

\textbf{סיכום:}  
הפתרון הכללי ניתן בצורה סתומה:
\[
\boxed{-\tfrac{1}{2}\Big(\tfrac{y}{x}\Big)^2+\tfrac{y}{x}-\ln\!\Big|1+\tfrac{y}{x}\Big|=\ln|x|+C,\qquad y\equiv0,\qquad x\neq 0.}
\]

%%%CUT%%%

\solution{}

נתון: $S_{BOC}=3.6 S_{ADOE}$.  
נקודת ההשקה $(x,y)$; שיפוע המשיק $y'$.  
נמצא את נקודות החיתוך של המשיק עם הצירים.  

נחזור על משוואת המשיק בנקודה $A(x,y)$:
\[
Y-y = y'(X-x).
\]

\textbf{חיתוך עם ציר ה-$x$:}  
בנקודה $B(X_B,0)$ מציבים $Y=0$:
\[
0-y = y'(X_B-x).
\]
ולכן:
\[
X_B-x = -\frac{y}{y'} \;\;\Longrightarrow\;\; X_B = x - \frac{y}{y'} = \frac{xy'-y}{y'}.
\]

\textbf{חיתוך עם ציר ה-$y$:}  
בנקודה $C(0,Y_C)$ מציבים $X=0$:
\[
Y_C-y = y'(0-x).
\]
ולכן:
\[
Y_C-y = -xy' \;\;\Longrightarrow\;\; Y_C = y-xy'.
\]

\textbf{מקבלים:}
\[
X_B=\frac{xy'-y}{y'}, 
\qquad 
Y_C=y-xy'.
\]

נציג את הביטויים לשטח המשולש ולשטח המלבן. מקבלים:  
\[
S_{BOC}=\frac{X_B \cdot Y_C}{2} = -\frac{(xy'-y)^2}{2y'}, 
\qquad 
S_{ADOE}=X_A \cdot Y_A = xy.
\]

לכן המשוואה הדיפרנציאלית היא:
\[
-\frac{(xy'-y)^2}{2y'}=3.6xy.
\]

נסדר:  
\[
(xy'-y)^2=-7.2xyy'.
\]

נראה כי המשוואה מטיפוס הומוגני. נציב $y=vx$ ו־$y'=v'x+v$:  
\[
(x(v'x+v)-vx)^2=-7.2x\cdot vx \cdot (v'x+v).
\]

רואים שאת שני האגפים אפשר לצמצם ב- $x^{2}$ (לאיקס יש משמעות גיאומטרית ולכן במקרה זה לא יכול להתאפס). נקבל:
\[
(v'x)^2=-7.2 v (v'x+v).
\]
המשוואה עדיין קשה לפתרון ולא נראית ׳׳מוכרת׳׳. נפעיל משתנה עזר אלגברי $t=v'x$ ונתייחס זמנית ל $v$ כאל פרמטר. ומקבלים:
נקבל: 
\[
t^2+7.2vt+7.2v^2=0.
\]

נפתור את המשוואה הריבועית:  
\[
t_{1,2}=\frac{-7.2v\pm\sqrt{(7.2v)^2-4\cdot7.2v^2}}{2}
=\{-1.2v,\,-6v\}.
\]

כלומר:  
\[
v'x=-1.2v \quad \text{או} \quad v'x=-6v.
\]

קיבלנו בשני המקרים מד׳׳ר פרידה עם פתרון חשוד להיות סינגולרי ב-$v=0$. נפתור:  
\[
\frac{dv}{v}=-1.2\frac{dx}{x} \quad\Rightarrow\quad v=\frac{c}{x^{1.2}},
\]
או
\[
\frac{dv}{v}=-6\frac{dx}{x} \quad\Rightarrow\quad v=\frac{c}{x^6}.
\]
שימו לב כי כתבנו את הפתרונות הסינגולריים בתוך הפתרון הכללי.
נחזור ל־$y=vx$:  
\[
y=\frac{c}{x^{0.2}} \quad \text{או} \quad y=\frac{c}{x^5}.
\]

\textbf{סיכום:} משפחת העקומות המקיימת את התנאי היא:
\[
\boxed{\,y=\tfrac{c}{\sqrt[5]{x}},\qquad y=\tfrac{c}{x^5},\; x\neq0\,}.
\]

\newpage
\subsection{משוואת ריקטי}

הצורה הכללית למשוואת ריקטי מסדר ראשון תהא:
\begin{equation}
y' = a(x)y^2 + b(x)y + c(x)
\end{equation}
כאשר הפונקציות $a(x), b(x), c(x)$ רציפות בתחום המשוואה. 

\subsubsection{מקרים פרטיים}

1. כל הפונקציות $a(x), b(x), c(x)$ הן קבועות.  
במקרה זה המשוואה ניתנת להפרדת משתנים, שכן המד׳׳ר היא אוטונומית. נראה דוגמה.

\example{} 

מצאו את הפתרון הכללי של 
\[
y' = y^2 - 3y + 2
\]

\explanation  
\[
y' = y^2 - 3y + 2 \;\;\Longleftrightarrow\;\; \frac{dy}{dx} = y^2 - 3y + 2
\;\;\Longleftrightarrow\;\; \frac{1}{(y-2)(y-1)}\,dy = dx \;\;\Rightarrow\;\; x + C
\]

\[
= \ln\!\left|\frac{y-2}{y-1}\right|
\]

לצד הפתרונות סינגולריים $y\equiv1,2$. כלומר, הפתרון הכללי הוא:
\[
\ln\!\left|\frac{y-2}{y-1}\right| = x + C,\qquad y\equiv1,2\qquad x\in\mathbb{R}.
\]

2.
אם $c(x)=0$, המשוואה היא משוואת ברנולי.  
למשל, המשוואה 
\[
y' = -\frac{4y}{x} - y^2
\]
היא משוואת ברנולי.  

\[
y' = -\frac{4y}{x} - y^2 
\;\;\Longleftrightarrow\;\; \frac{y'}{y^2} + \frac{4}{x}\cdot\frac{1}{y} = -1
\]
נזהה כי
$\alpha=2$. 
כעת נבצע את ההצבה:
\[
z(x) = \frac{1}{y(x)} \quad \Longrightarrow \quad z'(x) = -\frac{y'}{y^2}.
\]

נחזיר למשוואה ונקבל:
\[
-\,z'(x) + \frac{4}{x}z(x) = -1.
\]

ולאחר נרמול:
\[
z'(x) - \frac{4}{x}z(x) = 1.
\]

זהו מקרה של משוואה לינארית מסדר ראשון.  
נחשב את גורם האינטגרציה:
\[
\mu(x) = e^{\int -\tfrac{4}{x}\,dx} = e^{-4\ln|x|} = x^{-4}.
\]

נכפיל את המשוואה ב־$\mu(x)$:
\[
(x^{-4}z(x))' = x^{-4}.
\]

נבצע אינטגרציה:
\[
x^{-4}z(x) = \int x^{-4}\,dx = \int x^{-4}\,dx = -\frac{1}{3}x^{-3} + C.
\]

נכפיל ב־$x^4$ ונקבל:
\[
z(x) = -\frac{1}{3}x + Cx^4.
\]

ולבסוף, נזכור כי $z(x)=\tfrac{1}{y(x)}$, לכן:
\[
\boxed{\,y(x) = \frac{1}{Cx^4 - \tfrac{1}{3}x}\,}.
\]

\subsubsection{המשפט המרכזי – משפט על פתרון פרטי}

בהינתן המשוואה הכללית של ריקטי:  
\[
y' = a(x)y^2 + b(x)y + c(x).
\]

אם קיים פתרון פרטי $y=f(x)$ של המשוואה, אזי קיים פתרון למשוואה המובע באמצעות אינטגרלים של פונקציות אלמנטריות.  

\begin{proof} 
במשוואה הכללית נבצע הצבה:  
\[
y = t(x) + f(x).
\]
נקבל:

\[
y' = a(x)y^2 + b(x)y + c(x) 
\;\;\Longleftrightarrow\;\;
t' + f' = a(x)(t^2+2ft+f^2) + b(x)(t+f) + c(x).
\]

ולאחר סידור:  
\begin{equation}\label{riccati}
t' + \textcolor{red}{f'} = a(x)t^2 + 2a(x)ft + \textcolor{red}{a(x)f^2} + b(x)t + \textcolor{red}{b(x)f} + \textcolor{red}{c(x)}.
\end{equation}
כיוון ש-$f(x)$ נתון להיות פתרון פרטי של הבעיה, הוא למעשה מקיים את המד׳׳ר:
\[
f' = a(x)f^2 + b(x)f + c(x).
\]
משום שכל האיברים האדומים מתאפסים במשוואה (\ref{riccati}), קיבלנו את המשוואה:  
\[
t' - (2a(x)f+b(x))t = a(x)t^2.
\]
וזאת משוואת ברנולי שאנו יודעים לפתור בכפוף לאינטגרלים, עם חזקה $\alpha=2$.

\textbf{שלב 1 – נזהה את מבנה המשוואה:}
\[
t' + p(x)t = q(x) t^2, \qquad p(x) = -(2a(x)f(x)+b(x)), \quad q(x) = a(x).
\]

\textbf{שלב 2 – נבצע הצבה:}
\[
z(x) = \frac{1}{t(x)} \quad \Longrightarrow \quad z'(x) = -\frac{t'(x)}{t(x)^2}.
\]

\textbf{שלב 3 – נציב חזרה למשוואה:}
\[
t' + p(x)t = q(x)t^2 
\;\;\Longleftrightarrow\;\;
-\frac{z'}{z^2} + p(x)\frac{1}{z} = \frac{q(x)}{z^2}.
\]

נכפיל ב־$z^2$ ונקבל:
\[
-z' + p(x)z = q(x).
\]

\textbf{שלב 4 – קיבלנו משוואה לינארית:}
\[
z'(x) - p(x)z(x) = -q(x).
\]

\textbf{שלב 5 – נמצא את גורם האינטגרציה:}
\[
\mu(x) = e^{-\int p(x)\,dx} = e^{\int (2a(x)f(x)+b(x))\,dx}.
\]

\textbf{שלב 6 – פתרון ל־$z(x)$:}
\[
z(x) = \frac{1}{\mu(x)}\left(\int -q(x)\mu(x)\,dx + C\right).
\]

\textbf{שלב 7 – נחזור ל־$t(x)$:}
\[
t(x) = \frac{1}{z(x)}.
\]

\textbf{ולבסוף – הפתרון הכללי של ריקטי:}
\begin{equation}
\boxed{\,y(x) = f(x) + \frac{1}{\tfrac{1}{\mu(x)}\left(\int -q(x)\mu(x)\,dx + C\right)}\,},
\end{equation}

כאשר $f(x)$ הוא הפתרון הפרטי הנתון, $q(x)=a(x)$ ו־$\mu(x)=e^{\int (2a(x)f(x)+b(x))dx}$.

\end{proof} 

\begin{example}
נתונה המשוואה $y' = y^2 - 2e^x y + e^{2x} + e^x$.

\begin{enumerate}[label=\alph*.]
\item הראו כי $f(x)=e^x$ הוא פתרון פרטי של המשוואה.  
\item מצאו את הפתרון הכללי של המשוואה.  
\end{enumerate}

\explanation{}

א.
נבדוק ע"י הצבה. למעשה, $f(x)=e^x \Rightarrow f'(x)=e^x$.  

$e^x = e^{2x} - 2e^x e^x + e^{2x} + e^x$  

קל לראות כי קיבלנו שוויון זהותי לכל $x$ ממשי.

ב.
נציב $y=t+e^x$ ונקבל:  

\[
t' + \textcolor{red}{e^x} = t^2 + 2e^x t + \textcolor{red}{e^{2x}} - 2e^x t \; \textcolor{red}{- 2e^{2x} + e^{2x}} + \textcolor{red}{e^x}.
\]

ולאחר פישוט:
\[
t' = t^2.
\]

זהו מקרה של משוואה פרידה:
\[
\frac{dt}{dx} = t^2 \;\;\Longleftrightarrow\;\; \frac{dt}{t^2} = dx.
\]

נבצע אינטגרציה:
\[
\int \frac{dt}{t^2} = \int dx 
\;\;\;\Rightarrow\;\;\; -\frac{1}{t} = x + C.
\]

ולכן:
\[
t(x) = -\frac{1}{x+C}.
\]
לצד הפתרון החשוד להיות סינגולרי $t=0$. במקרה זה אנו חוזרים לפתרון הידוע $f=e^{x}$.
נחזיר את הפתרון הכללי שקיבלנו להצבה $y = t + f(x)$:
\[
y(x) = e^x - \frac{1}{x+C},\qquad x\neq -C.
\]
\end{example}

%%%CUT%%%

\newpage
\underline{תרגילים}
\exercise{}
נתונה המשוואה:
\[
y' = y^2 - \frac{6}{x^2}.
\]

\begin{enumerate}
[label=\alph*.]
\item הראו כי הפתרון הפרטי של המשוואה הוא מן הסוג $f(x)=\tfrac{m}{x}$ ומצאו את הערכים של $m$.
\item מצאו את הפתרון הכללי של המשוואה.
\end{enumerate}

\newpage
\underline{פתרונות}
\solution{} 

\textbf{(א)}
נציב $y=\tfrac{m}{x} \;\;\Rightarrow\;\; y'=-\tfrac{m}{x^2}$.  

נציב במשוואה הנתונה:
\[
-\frac{m}{x^2} = \frac{m^2}{x^2} - \frac{6}{x^2}
\;\;\;\Longrightarrow\;\;\;
m^2+m-6=0
\;\;\;\Longrightarrow\;\;\;
\begin{cases}
m_1=2,\\
m_2=-3.
\end{cases}
\]

כלומר, קיבלנו שני פתרונות פרטיים:
\[
f_1(x)=-\tfrac{3}{x}, \qquad f_2(x)=\tfrac{2}{x}.
\]

\textbf{(ב)}  
נשתמש במשפט המרכזי.  
נתחיל עם $f_2(x)=\tfrac{2}{x}$ ונציב $y=t+\tfrac{2}{x}$.  

אז נקבל:
\[
y' = t' - \frac{2}{x^2},
\]
ונציב במשוואה המקורית:
\[
t' - \frac{2}{x^2} = (t+\tfrac{2}{x})^2 - \frac{6}{x^2}.
\]

נפתח ונקבל:
\[
t' \textcolor{red}{- \frac{2}{x^2}} = t^2 + \frac{4}{x}t + \textcolor{red}{\frac{4}{x^2}} \textcolor{red}{- \frac{6}{x^2}}.
\]

נשים לב שהאיברים באדום מתאפסים ולכן נקבל: 
\[
t' - \frac{4}{x}t = t^2.
\]


נגדיר $z=\tfrac{1}{t}$, ונקבל:
\[
z' + \frac{4}{x}z = -1.
\]

זוהי משוואה לינארית.  
נחפש גורם אינטגרציה:
\[
\mu(x) = e^{\int \tfrac{4}{x}dx} = x^4.
\]

ולכן:
\[
z(x) = \frac{1}{x^4}\int x^4\cdot (-1)\,dx + \frac{C}{x^4}
= -\frac{x}{5}+Cx^{-4}.
\]

נחזור ל־$t$:  
\[
t=\frac{1}{z(x)}=\frac{5x^4}{5C-x^5}.
\]

ולכן:
\[
y(x)=t+\frac{2}{x} = \frac{5x^4}{5C-x^5}+\frac{2}{x}, \qquad x\neq 0.
\]

באופן דומה, עבור $f_1(x)=-\tfrac{3}{x}$, נציב $y=t-\tfrac{3}{x}$ ונקבל:
\[
t' + \frac{6}{x}t = t^2.
\]

שוב נבצע $z=\tfrac{1}{t}$ ונקבל:
\[
z' - \frac{6}{x}z = -1.
\]

כאן:
\[
\mu(x)=e^{\int -\tfrac{6}{x}dx}=x^{-6}.
\]

ולכן:
\[
z(x) = x^6\int x^{-6}\cdot (-1)\,dx + Cx^6
= -\frac{x}{5}+Cx^6.
\]

ולכן:
\[
t=\frac{1}{z(x)}=\frac{5}{5Cx^6-x}.
\]

לבסוף:
\[
y(x)=t-\frac{3}{x} = \frac{5}{5Cx^6-x}-\frac{3}{x}, \qquad x\neq 0.
\]

\textbf{סיכום:}  
המשוואה $y'=y^2-\tfrac{6}{x^2}$ מקיימת שני פתרונות כלליים שונים, בהתאם לפתרון הפרטי שנבחר:  

\[
\boxed{y(x)=\frac{5x^4}{5C-x^5}+\frac{2}{x}}
\quad \text{או} \quad
\boxed{y(x)=\frac{5}{5Cx^6-x}-\frac{3}{x}}, \qquad x\neq 0.
\]

\newpage
\subsection{משוואות מדויקות}

\textbf{נתבונן בצורה הכללית של המשוואה:}
\begin{equation}\label{exact}
P(x,y)dx + Q(x,y)dy = 0
\end{equation}

\textbf{ובצורותיה השקולות:}
\begin{equation}
Px' + Q = 0
\end{equation}
\begin{equation}
P + Qy' = 0
\end{equation}

אם מתקיים $P'_y = Q'_x$ אזי המשוואה נקראת \textbf{מדויקת}.  
אם כך, קיימת פונקציה $F(x,y)$ המקיימת:
\[
F'_y = Q, \qquad F'_x = P
\]

ולכן:
\[
F = \int P\,dx, 
\qquad 
F = \int Q\,dy
\]

והפתרון הכללי בצורה סתומה יהיה:
\[
F(x,y) = C.
\]

\begin{proof}
נניח כי קיימת פונקציה פוטנציאלית $F(x,y)$ כך ש:
\[
dF = F_x\,dx + F_y\,dy.
\]
אם נשווה זאת עם הצורה הכללית $P(x,y)dx + Q(x,y)dy$, נקבל בהכרח:
\[
F_x = P, \qquad F_y = Q.
\]
 כלומר המשוואה הנתונה שקולה ל־$dF=P(x,y)dx + Q(x,y)dy=0$, ולכן לאחר אינטגרציה נקבל: 
\[
F(x,y) = C.
\]

עבור פונקציה גזירה היטב מתקיים:
\[
F_{xy} = F_{yx}.
\]
ומכאן נובע:
\[
P_y = Q_x.
\]
לכן תנאי זה הכרחי לקיומה של פונקציה פוטנציאלית $F$, ומצד שני אם הוא מתקיים – ניתן לבנות את $F$ ע"י אינטגרציה של $P$ או $Q$, ולכן הוא גם תנאי מספיק.  
\end{proof}
\textbf{אלגוריתם לפתרון משוואות מדויקות}

נניח כי ניתנה המשוואה:
\[
P(x,y)\,dx + Q(x,y)\,dy = 0,
\]
כאשר מתקיים תנאי הדיוק:
\[
P_y = Q_x.
\]

\textbf{שיטה ראשונה לקבל פתרון כללי – ׳׳אינטגרל אחד ונגזרת אחת׳׳:}

\begin{enumerate}
\item נבצע אינטגרציה של $P(x,y)$ לפי $x$:
\[
F(x,y) = \int P(x,y)\,dx + g(y).
\]

\item נגזור לפי $y$:
\[
F_y(x,y) = \frac{\partial}{\partial y}\Big(\int P(x,y)\,dx\Big) + g'(y).
\]

\item מאחר שחייב להתקיים \(\,F_y(x,y)\equiv Q(x,y)\), נקבל:
\[
g'(y) \equiv Q(x,y) - \frac{\partial}{\partial y}\Big(\int P(x,y)\,dx\Big).
\]

שימו לב: מדובר בזהות \(\equiv\), כלומר צד שמאל וצד ימין ב-\(\,F_y(x,y)\equiv Q(x,y)\) חייבים להיות שווים \textbf{זהותית} כפונקציות של שני המשתנים $x,y$ בכל תחום ההגדרה של הבעיה. בשלב זה יש לשים לב: לאחר ביצוע האינטגרציה, יתקבלו בשני האגפים גם איברים ״מעורבים״ (תלויים גם ב־$x$ וגם ב־$y$).  
מאחר ש־$F(x,y)$ היא פונקציה מוגדרת היטב, כל האיברים המעורבים חייבים להתבטל בזהות.  
זו בעצם בדיקת העקביות של תהליך הפתרון: אם לא מתקבלת זהות, המשוואה אינה מדויקת.

\item אינטגרציה לפי $y$ תיתן את $g(y)$:
\[
g(y) = \int \left[Q(x,y) - \frac{\partial}{\partial y}\Big(\int P(x,y)\,dx\Big)\right]dy.
\]

\item הפתרון הכללי בצורה סתומה יהיה אם כך:
\[
F(x,y) = \boxed{\int P(x,y)\,dx + \int \left[Q(x,y) - \frac{\partial}{\partial y}\Big(\int P(x,y)\,dx\Big)\right]dy =  C}.
\]
\end{enumerate}

\textbf{
בצורה סימטרית לחלוטין ניתן להתחיל מאינטגרציה של $Q(x,y)$ לפי $y$ ולקבל תנאי על $h'(x)$}:

\begin{enumerate}
\item נבצע אינטגרציה של $Q(x,y)$ לפי $y$:
\[
F(x,y) = \int Q(x,y)\,dy + h(x).
\]
במקרה זה, ״קבוע״ האינטגרציה תלוי הפעם ב־$x$, ולכן הוספנו $h(x)$.

\item נגזור לפי $x$:
\[
F_x(x,y) = \frac{\partial}{\partial x}\Big(\int Q(x,y)\,dy\Big) + h'(x).
\]

\item מאחר שחייב להתקיים \(\,F_x(x,y)\equiv P(x,y)\), נקבל:
\[
h'(x) \equiv P(x,y) - \frac{\partial}{\partial x}\Big(\int Q(x,y)\,dy\Big).
\]

\item אינטגרציה לפי $x$ תיתן את $h(x)$:
\[
h(x) = \int \left[P(x,y) - \frac{\partial}{\partial x}\Big(\int Q(x,y)\,dy\Big)\right]dx.
\]

\item הפתרון הכללי בצורה סתומה יהיה אם כך:
\[
F(x,y) = \boxed{\int Q(x,y)\,dy + \int \left[P(x,y) - \frac{\partial}{\partial x}\Big(\int Q(x,y)\,dy\Big)\right]dx =  C}.
\]
\end{enumerate}

\textbf{שיטה שנייה לקבל פתרון כללי – ׳׳שתי אינטגרציות והשוואה׳׳:}

\begin{enumerate}
\item נבצע אינטגרציה של $P(x,y)$ לפי $x$:
\[
F(x,y) = \int P(x,y)\,dx + g(y).
\]

\item נבצע אינטגרציה של $Q(x,y)$ לפי $y$:
\[
F(x,y) = \int Q(x,y)\,dy + h(x).
\]

\item ההשוואה בזהות \(\equiv\) מאפשרת לחלץ את $g(y)$ או לחלופין את $h(x)$, ע"י ביטול האיברים המשותפים בין שתי הצורות הפונקציונליות שקיבלנו. 
כך אנו מקבלים תנאי מפורש לפונקציה/יות הלא ידועה/ות:  
\[
g(y) + \int P(x,y) 
\equiv
h(x) + \int Q(x,y)\,dy.
\]
\item בשלב זה יש לשים לב: לאחר ביצוע שתי האינטגרציות, יתקבלו בשני האגפים גם איברים ״מעורבים״ (תלויים גם ב־$x$ וגם ב־$y$).  
מאחר ש־$F(x,y)$ היא פונקציה מוגדרת היטב, כל האיברים המעורבים חייבים להתבטל בזהות.  
זו בעצם בדיקת העקביות של תהליך הפתרון: אם לא מתקבלת זהות, המשוואה אינה מדויקת.

\item לאחר ביטול האיברים המעורבים, נקבל משוואה שמורכבת משני חלקים: צד אחד יהיה \emph{טהור ב־$x$ בלבד}, והצד השני \emph{טהור ב־$y$ בלבד}.  
כך מתקבל פירוק חד־משמעי שמאפשר לקבוע את הפונקציות הלא־ידועות.

\item מכאן קיימות שתי דרכים להמשך:
\begin{itemize}
  \item להשתמש בצד התלוי ב־$y$ בלבד כדי לחלץ את $g(y)$.
  \item להשתמש בצד התלוי ב־$x$ בלבד כדי לחלץ את $h(x)$.
\end{itemize}

בשני המקרים נבנה בסופו של דבר את הפונקציה הפוטנציאלית $F(x,y)$ המלאה, והפתרון הכללי של המשוואה יהיה כאמור:
\[
F(x,y) = C.
\]
\end{enumerate}

\subsubsection{משוואות מדויקות – גורמי אינטגרציה של משתנה אחד בלבד}

נחזור למשוואה \ref{exact} ונתמקד בפרק זה במציאת גורם אינטגרציה חד מימדי בלבד (של $x$ או של $y$, כמקובל במרבית קורסי המד׳׳ר הניתנים לסטודנטים לתארי ההנדסה והמדעים), ולא בגורמי אינטגרציה של שני משתנים.
אם $P'_y \neq Q'_x$, נחפש פונקציה $\mu\neq 0$ עבורה:
\[
\mu P\,dx + \mu Q\,dy = 0 
\quad \Rightarrow \quad 
(\mu P)'_y = (\mu Q)'_x
\]
ניתן לחשוב על המד׳׳ר במצב זה כ:
\begin{equation}
    P^{*}\,dx + Q^{*}\,dy = 0
    \quad \Rightarrow \quad 
    (P^{*})'_{y} = (Q^{*})'_{x}
\end{equation}

\begin{enumerate}
\item אם יש גורם אינטגרציה שהוא פונקציה של $x$ בלבד:
\[
\mu(x) = e^{\int \frac{P'_y - Q'_x}{Q}\,dx}
\]


\item אם יש גורם אינטגרציה שהוא פונקציה של $y$ בלבד:
\[
\mu(y) = e^{\int \frac{Q'_x - P'_y}{P}\,dy}
\]
\end{enumerate}

\textbf{מומלץ לבדוק קודם את האינטגרנד ע״מ להבין באיזה כיוון יש ג׳׳א של משתנה אחד בלבד.}

\begin{proof}
נניח כי $\mu=\mu(x)$ בלבד. אזי:
\[
(\mu P)'_y = \mu(x)\,P'_y, 
\qquad
(\mu Q)'_x = \mu'(x)Q + \mu(x)Q'_x.
\]
תנאי הדיוק מחייב:
\[
\mu(x)P'_y = \mu'(x)Q + \mu(x)Q'_x.
\]
נחלק ב־$\mu(x)\neq 0$ ונקבל:
\[
\frac{\mu'(x)}{\mu(x)} = \frac{P'_y - Q'_x}{Q}.
\]
משוואה זו תלויה אך ורק ב־$x$ ולכן ניתנת לאינטגרציה ישירה:
\[
\mu(x) = \exp\!\left(\int \frac{P'_y - Q'_x}{Q}\,dx\right).
\]

באופן סימטרי, אם $\mu=\mu(y)$ בלבד, אזי:
\[
(\mu P)'_y = \mu'(y)P + \mu(y)P'_y, 
\qquad
(\mu Q)'_x = \mu(y)Q'_x,
\]
ולכן תנאי הדיוק הופך ל:
\[
\mu'(y)P + \mu(y)P'_y = \mu(y)Q'_x.
\]
לאחר חלוקה ב־$\mu(y)\neq 0$ נקבל:
\[
\frac{\mu'(y)}{\mu(y)} = \frac{Q'_x - P'_y}{P},
\]
שגם היא משוואה חד־ממדית, שנותנת את הנוסחה עבור $\mu(y)$.
\end{proof}

\textbf{אנקדוטה – האלגוריתם הכללי למציאת גורם אינטגרציה דו־ממדי:}

באופן עקרוני ניתן לחפש גורם אינטגרציה $\mu(x,y)$ שאינו תלוי במשתנה אחד בלבד. התנאי הוא:
\[
(\mu P)'_y = (\mu Q)'_x,
\]
שמוביל למשוואה דיפרנציאלית חלקית עבור $\mu(x,y)$:
\[
\mu_y P + \mu P_y = \mu_x Q + \mu Q_x.
\]
זהו תנאי כללי למציאת $\mu(x,y)$, אך בפועל לרוב קשה לקבל אותו. לכן בקורסים בסיסיים מתמקדים במקרים הפשוטים בהם $\mu$ תלויה \emph{במשתנה אחד בלבד} — אלו המקרים בהם ניתן להגיע לסגירות אנליטית בקלות יחסית.

%%%CUT%%%

\example
מצאו פתרון כללי למד״ר הבאה:
\[
2x\big(ye^{x^2} - 1\big)dx + e^{x^2}dy = 0.
\]

\explanation

התבנית מרמזת על כדאיות לבדיקת היותה של המד׳׳ר כמדויקת.
נשים לב כי מתקיים:
\[
P(x,y) = 2x(ye^{x^2}-1), \qquad Q(x,y) = e^{x^2}.
\]

נגזור:
\[
P_y = 2xe^{x^2}, \qquad Q_x = 2xe^{x^2}.
\]

כלומר, תנאי הדיוק מתקיים, והמד״ר מדויקת.
נבחר (שרירותית) בשיטת ׳׳שתי האינטגרציות׳׳.

מצד אחד:
\[
F(x,y) = \int P(x,y)\,dx = \int (2xye^{x^2}-2x)\,dx = ye^{x^2} - x^2 + h(y),
\]

ומצד שני:
\[
F(x,y) = \int Q(x,y)\,dy = \int e^{x^2}\,dy = ye^{x^2} + f(x).
\]

כעת נדרוש זהות בין שני הביטויים:
\[
\textcolor{red}{ye^{x^2}} - x^2 + h(y) \;\;\equiv\;\; \textcolor{red}{ye^{x^2}} + f(x).
\]

\textcolor{red}{האיברים המעורבים $ye^{x^2}$ מצטמצמים בזהות!} 
נשאר עם:
\[
h(y) \equiv f(x) + x^2.
\]

בשלב זה חשוב להדגיש: 
באגף שמאל מופיעה פונקציה שתלויה ב־$y$ בלבד, ובאגף ימין מופיעה פונקציה שתלויה ב־$x$ בלבד. 
זהות מהצורה
\[
\text{(פונקציה ב־$y$ בלבד)} \;\;\equiv\;\; \text{(פונקציה ב־$x$ בלבד)}
\]
יכולה להתקיים \textbf{אם ורק אם} שני הצדדים קבועים.  
כלומר, אם שתי פונקציות כאלו שוות זהותית על כל התחום, בהכרח:
\[
h(y) \equiv c, 
\qquad 
f(x) + x^2 \equiv c.
\]

קל יותר להחליף את $h(y)$ ב-$c$. לכן נקבל:
\[
F(x,y) = ye^{x^2} - x^2 + C = c.
\]

מכאן הפתרון הכללי בצורה סתומה הוא:
\[
ye^{x^2} - x^2 = C^*,
\]
או בצורה מפורשת:
\[
y = \big(C^* + x^2\big)e^{-x^2}\qquad , x\in\mathbb{R}.
\]
שימו לב כי היינו מקבלים את אותה התוצאה לו היינו בוחרים להציב 
$f(x)=c-x^{2}
\leftarrow 
f(x) + x^2 = c$:
\[
F(x,y) = ye^{x^2} + c-x^{2} = C.
\]
קיבלנו את אותה התוצאה ועל כן אותו פתרון כללי.

נראה שנקבל פתרון זהה אם היינו נוקטים בגישה של ׳׳שתי אינטגרציות׳׳.

\textbf{אינטגרציה ראשונה (לפי $x$):}
\[
F(x,y) = \int P(x,y)\,dx = \int (2xye^{x^2}-2x)\,dx.
\]

נחשב:
\[
\int 2xye^{x^2}\,dx = y\int 2xe^{x^2}\,dx = ye^{x^2}, 
\qquad 
\int -2x\,dx = -x^2.
\]

ולכן:
\[
F(x,y) = ye^{x^2} - x^2 + h(y).
\]

\textbf{אינטגרציה שנייה (לפי $y$):}
\[
F(x,y) = \int Q(x,y)\,dy = \int e^{x^2}\,dy.
\]

כיוון ש־$x$ קבוע כאן:
\[
F(x,y) = ye^{x^2} + f(x).
\]

\textbf{נשווה בין שתי הצורות:}
\[
\textcolor{red}{ye^{x^2}} - x^2 + h(y) \;\;\equiv\;\; \textcolor{red}{ye^{x^2}} + f(x).
\]

האיבר המעורב $\textcolor{red}{ye^{x^2}}$ מצטמצם, ונשאר:
\[
h(y) \equiv f(x) + x^2.
\]

\textbf{כעת:}  
משמאל יש פונקציה ב־$y$ בלבד, ומימין פונקציה ב־$x$ בלבד. זהות מהצורה:
\[
h(y) \equiv f(x) + x^2
\]
קיבלנו למעשה את אותו הקשר בין הפונקציות הנעלמות ועל כן נקבל את אותו הפתרון הכללי.

הערה: שימו לב כי כיוון שהמד׳׳ר לינארית, ניתן לפתור אותה באמצעות שיטת ג׳׳א ווריאציית הפרמטר.

\example{}

מצאו פתרון כללי למד״ר הבאה:
\[
(x^2 + y^2 + x)dx + xydy = 0
\]

\explanation{}
נסמן:
\[
P(x,y) = x^2 + y^2 + x, 
\qquad 
Q(x,y) = xy
\]

ונבדוק האם מתקיים $P'_y = Q'_x$:
\[
P'_y = 2y \neq Q'_x = y
\]

כלומר, המשוואה אינה מדויקת. נחפש ג׳יא של משתנה אחד בלבד. נבדוק את האינטגרנדים:

\[
X:\quad \frac{P'_y - Q'_x}{Q} = \frac{2y - y}{xy} = \frac{y}{xy} = \frac{1}{x}
\]

\[
Y:\quad \frac{Q'_x - P'_y}{P} = \frac{y - 2y}{x^2 + y^2 + x} = \frac{-y}{x^2+y^2+x}
\]

נקבל ג׳׳א במשתנה אחד בלבד ב- $x$:
\[
\mu(x) = e^{\int \tfrac{1}{x}dx} = e^{\ln|x|} = |x|
\]

\textbf{תזכורת חשובה – כלל אצבע:} אם $\mu(x)$ ג׳יא אז גם $|\mu(x)|$ ג׳יא.
כלומר, נוכל לבחור $\mu(x) = x$.
נציב את גורם האינטגרציה ונקבל מד״ר חדשה:
\[
P^{*}(x,y) + Q^{*}(x,y) = (x^3 + xy^2 + x^2)dx + x^2y\,dy = 0
\]

נבדוק את תנאי הדיוק:
\[
P^{*}_y = 2xy, 
\qquad 
Q^{*}_x = 2xy
\]

כלומר, המד״ר אכן מדויקת. נמצא עתה את $F$:
\[
\int Q^{*}\,dy = \int P^{*}\,dx \equiv F(x,y) = C
\]

נחשב:
\[
F(x,y) = \int P^{*}\,dx 
= \int (x^3 + xy^2 + x^2)\,dx
= \frac{x^4}{4} + \frac{x^2y^2}{2} + \frac{x^3}{3} + h(y)
\]

וכן:
\[
F(x,y) = \int Q^{*}\,dy = \int x^2y\,dy = \frac{x^2y^2}{2} + u(x)
\]

נשווה:
\[
\frac{x^2y^2}{2} + u(x) \equiv \frac{x^4}{4} + \frac{x^2y^2}{2} + \frac{x^3}{3} + h(y)
\]

כלומר:
\[
C = u(x) - \frac{x^4}{4} - \frac{x^3}{3} \equiv h(y) = C
\]

ולכן הפתרון הכללי בצורה סתומה:
\[
F(x,y) = \frac{x^4}{4} + \frac{x^2y^2}{2} + \frac{x^3}{3} + C.
\]

נסמן בקיצור:
\[\boxed{
\frac{x^4}{4} + \frac{x^2y^2}{2} + \frac{x^3}{3} = \hat{C}, 
\qquad x \in \mathbb{R}}
\]

וזהו הפתרון הכללי בצורה סתומה של המשוואה. אגב, קל לחלץ במקרה זה את $y(x)$.
אך יש לשים לב: הכפלנו את המד׳׳ר המקורית כולה ב-$x$. $x$ עלול להיות שווה לאפס. ברור כי $x=0$ הוא פתרון של המד׳׳ר החדשה.
לכן נותר לנו לבדוק האם $x=0$ הוא פתרון של המשוואה המקורית.

נבדוק:  
\[
(x^2 + y^2 + x)dx + xydy = 0
\]

כאשר $x=0$ נקבל:
\[
(y^2)dx = 0
\]
ניתן להבין שאם $x=0$, נגרר אוטומטית כי $dx=0$, שכן מדובר באלמנט אינפיניטיסמלי של $x$. 
על כן, $x=0$ הוא אכן פתרון לבעיה. ניתן להציג את המד׳׳ר בצורה שקולה ע׳׳י חלוקה ב-$dy$ על מנת לראות את הדברים ביותר בהירות:
\[
(x^2 + y^2 + x)x' + xy = 0
\]
$x=0$ גורר $x'=0$, ועל כן $x=0$ הוא אכן פתרון לבעיה. שימו לב לנקודה מעניינת: אם היינו בוחרים לחלק את המד׳׳ר ב-$dx$, היינו נתקלים במצב לא ברור בו לא היינו יודעים לקבוע האם $x=0$ הוא פתרון או לא (למען הסר ספק, זה לא פוטר אותנו מלחפש את התצוגה המתאימה על מנת לקבוע זאת, כפי שראינו קודם לכן):
\[
x^2 + y^2 + x + xyy' = 0
\]
נשים לב כי נותרנו עם איבר שהוא לא אפס ($y^{2}$). כאמור, לא היינו יכולים לקבוע במקרה זה.

\begin{remark}
    שימו לב כי זו הפעם הראשונה שאנו דנים במשמעות כפל המשוואה המקורית, בגורם אינטגרציה כלשהו. למה זה לא עלה עד עכשיו? כיוון שעד כמה התעסקנו עם גורמי אינטגרציה אקספוננציאליים בלבד. מטבע משוואה זו, גורם האינטגרציה, התברר כהפונקציה הפולינומיאלית $x$, אשר מתאפסת על הישר הממשי, בניגוד לאקספוננטים.
\end{remark}

\newpage
\subsection{משוואות ומשפחות אורתגונליות (ניצבות)}

\textbf{הגדרה:}  
נאמר כי שתי משפחות של עקומים במישור, המסומנות $F(x,y,C)$ ו־$G(x,y,d)$, הן \textbf{אורתגונליות} אם ורק אם עבור כל עקומה $f(x)$ מן המשפחה הראשונה ועקומה $g(x)$ מן המשפחה השנייה מתקיים התנאי הבא:  
בכל נקודת חיתוך $x_0$ שבה $f(x_0)=g(x_0)$, מכפלת השיפועים של העקומים בנקודה מקיימת:
\[
f'(x_0)\cdot g'(x_0) = -1.
\]
במילים אחרות, העקומים חותכים זה את זה בניצב בכל נקודת מפגש, ולכן:
\begin{equation}
f'(x_0) = -\frac{1}{g'(x_0)}.
\end{equation}

\textbf{\underline{אלגוריתם כללי למציאת משפחה אורתגונלית:}}
\begin{enumerate}
\item \textbf{מציאת המד״ר של המשפחה הנתונה:}  
נגזור את הביטוי הכללי של המשפחה כדי לקבל משוואה דיפרנציאלית ראשונית המתארת את שיפועי העקומים. 
קיימות שתי דרכים עיקריות:
\begin{itemize}
  \item \textbf{בידוד הפרמטר:}  
  מבודדים את הקבוע הפרמטרי מתוך המשוואה של המשפחה,  
  גוזרים את שני האגפים ומקבלים מד״ר שלא תלויה עוד בפרמטר.
  \item \textbf{גזירה ישירה:}  
  גוזרים ישירות את המשוואה הכללית של המשפחה לפי $x$,  
  ומבטלים את הקבוע באמצעות הקשר שנוצר בין $x,y,y'$.
\end{itemize}
\item \textbf{כתיבת המד״ר של המשפחה האורתגונלית:}  
נחליף את $y'$ (השיפוע של המשפחה הנתונה) בביטוי האורתגונלי לו:
\[
y' \;\;\longmapsto\;\; -\frac{1}{y_{\perp}'}.
\]
כך מתקבלת משוואה דיפרנציאלית חדשה המתארת את המשפחה האורתגונלית.
\item \textbf{פתרון המד״ר החדשה:}  
נפתור את המשוואה שהתקבלה, והמשפחה הכללית של פתרונותיה היא בדיוק המשפחה האורתגונלית לעקומים המקוריים.
\end{enumerate}

\example{}

מצאו את המשפחה האורתגונלית למשפחת העקומים הבאה:
\[
y^2 = ke^x + x + 1,
\]
כאשר $k\in\mathbb{R}$.
\explanation{}

נראה פתרון בשתי הדרכים. למעשה, נקבל את התוצאה של שלב 1 באלגוריתם ונראה שהיא זהה. משם נוכל להמשיך כרגיל.
\underline{דרך 1 – בידוד הפרמטר} 

נבודד את הפרמטר ונגזור:
\[
k = (y^2 - x - 1)e^{-x} \quad \Longrightarrow \quad 0 = [(y^2 - x - 1)e^{-x}]'
\]

נחשב את הנגזרת בעזרת כלל המכפלה:
\[
0 = (2yy' - 1)e^{-x} - (y^2 - x - 1)e^{-x}.
\]

נחלק ב-$e^{-x}\neq 0$:
\[
0 = (2yy' - 1) - (y^2 - x - 1).
\]

נסדר:
\[
2yy' - 1 - y^2 + x + 1 = 0,
\]

כלומר:
\[
y^2 = 2yy' + x.
\]

\underline{דרך 2 – גזירה ישירה}  

נגזור:
\[
2y\cdot y' = ke^x + 1 
\quad \Longrightarrow \quad 
k = (2yy' - 1)e^{-x}
\]

נחזיר את $k$ שקיבלנו במשוואה המקורית:
\[
y^2 = (2yy' - 1)e^{-x}\cdot e^x + x + 1 
\quad \Longrightarrow \quad 
y^2 = 2yy' + x
\]
קיבלנו את אותה התוצאה כמו בדרך הראשונה. מכאן נמשיך לשלב 2.

נעבור לשלב 2 באלגוריתם – \textbf{החלפת $y'$ ב־$\left(-\tfrac{1}{y'_\perp}\right)$}:
\[
y^2 = 2y_\perp \cdot \left(-\frac{1}{y'_\perp}\right) + x
\]

מטעמי נוחות נציב $y_{\perp}=y$ ונסדר את המשוואה:
\[
\frac{2y}{y'} = x - y^2 
\quad \Longrightarrow \quad 
\frac{y'}{2y} = \frac{1}{x - y^2} 
\]
נשים לב שאם נבצע החלפת תפקידי $x,y$, נקבל מד׳׳ר לינארית ב׳׳כיוון׳׳ $x$:
\[ 
\frac{1}{x'} = y' = \frac{2y}{x - y^2}
\]

מכאן:
\[
x' = \frac{x - y^2}{2y} = \frac{x}{2y} - \frac{y}{2}
\]

כלומר:
\[
x' - \frac{1}{2y}x = -\frac{y}{2}
\]
זו משוואה לינארית מסדר ראשון עבור $x=x(y)$ עם:
\[
p(y) = -\frac{1}{2y}, 
\qquad 
q(y) = -\frac{y}{2}.
\]

\textbf{שלב 1 – חישוב גורם האינטגרציה:}
\[
\mu(y) = e^{\int p(y)\,dy} 
= e^{\int -\tfrac{1}{2y}\,dy}.
\]

נחשב את האינטגרל:
\[
\int -\frac{1}{2y}\,dy = -\frac{1}{2}\int \frac{1}{y}\,dy 
= -\frac{1}{2}\ln|y|.
\]

ולכן:
\[
\mu(y) = e^{-\tfrac{1}{2}\ln|y|} = |y|^{-\tfrac{1}{2}}.
\]

\textbf{שלב 2 – כתיבת הפתרון הכללי:}
\[
x(y) = \frac{1}{\mu(y)}\left(\int \mu(y)q(y)\,dy + C\right).
\]

נציב את $\mu(y)=|y|^{-\tfrac{1}{2}}$ ואת $q(y)=-\tfrac{y}{2}$:
\[
x(y) = |y|^{\tfrac{1}{2}}\left(\int |y|^{-\tfrac{1}{2}}\cdot\left(-\tfrac{y}{2}\right)\,dy + C\right).
\]

\textbf{שלב 3 – פישוט האינטגרנד:}
\[
|y|^{-\tfrac{1}{2}}\cdot\left(-\tfrac{y}{2}\right) 
= -\tfrac{1}{2}\,y\cdot |y|^{-\tfrac{1}{2}}.
\]

\textbf{שלב 4 – חישוב האינטגרל:}  
נבצע הצבה $u=|y| \;\;\Rightarrow\;\; du = \tfrac{y}{|y|}dy$.  

כעת:
\[
-\tfrac{1}{2}\,y\cdot |y|^{-\tfrac{1}{2}}\,dy
= -\tfrac{1}{2}\,|y|^{\tfrac{1}{2}}\cdot du.
\]

ולכן:
\[
\int -\tfrac{1}{2}\,y\cdot |y|^{-\tfrac{1}{2}}\,dy
= \int -\tfrac{1}{2}\,u^{\tfrac{1}{2}}\,du
= -\tfrac{1}{2}\cdot \frac{2}{3}u^{\tfrac{3}{2}}.
\]

נחזיר $u=|y|$:
\[
= -\tfrac{1}{3}|y|^{\tfrac{3}{2}}.
\]

\textbf{שלב 5 – הצבה חזרה:}
\[
x(y) = |y|^{\tfrac{1}{2}}\left(-\tfrac{1}{3}|y|^{\tfrac{3}{2}} + C\right).
\]

נפשט:
\[
x(y) = -\tfrac{1}{3}|y|^{\tfrac{1}{2}}|y|^{\tfrac{3}{2}} + C|y|^{\tfrac{1}{2}}
= -\tfrac{1}{3}y^2 + C|y|^{\tfrac{1}{2}}.
\]
נקבל את הפתרון הכללי לבעיה:
\[
\boxed{x(y) = C|y|^{\tfrac{1}{2}} - \tfrac{1}{3}y^2, \qquad y\in\mathbb{R}.}
\]
זהו פתרון כללי שמתאר את המשפחה האורתוגונלית  למשפחה הנתונה. שימו לב על התנאי שמתקבל על $y$ כתוצאה מהימצאותה במכנה של המד׳׳ר המנורמלת ב-$x$. 
נראה סרטוט להמחשה:
\begin{figure}[H]
    \centering
    \includegraphics[width=0.7\textwidth]{orthe_tech.png}
    \caption{המחשה גרפית של משפחות עקומים אורתגונליות: המשפחה ההתחלתית 
    \lr{$y^2 = ke^x + x + 1$} עבור סט קבועים, 
    והמשפחה האורתגונלית לה עבור סט קבועים 
    \lr{$x = C|y|^{\tfrac{1}{2}} - \tfrac{1}{3}y^2$}.  
    ניתן לראות ויזואלית כי עבור כל נקודת חיתוך של המשפחות, המשיקים לעקומות יהיו מאונכים זה לזה.}
    \label{fig:orthogonal_demo}
\end{figure}

%%%CUT%%%

\newpage
\underline{תרגילים}
\exercise{}

מצאו את המשפחה האורתוגונלית למשפחת הפרבולות:
\[
y^2 = 4Cx, \qquad C>0.
\]

\newpage
\underline{פתרונות}
\solution{}

ניישם את האלגוריתם למציאת משפחה אורתוגונלית.  
המטרה בשלב ראשון היא למצוא את המד״ר (משוואה דיפרנציאלית רגילה) של המשפחה הנתונה.  
נראה שתי דרכים שונות לכך.

\underline{דרך 1 – בידוד הפרמטר}  

נבודד את $C$:
\[
C = \frac{y^2}{4x}.
\]

כיוון ש-$C$ קבוע, נגזור את הביטוי ונשווה לאפס:
\[
0 = \left(\frac{y^2}{4x}\right)'.
\]

נחשב לפי כלל המנה:
\[
0 = \frac{2yy'\cdot 4x - y^2\cdot 4}{16x^2}
= \frac{2xyy' - y^2}{4x^2}.
\]

כלומר:
\[
2xyy' - y^2 = 0 
\quad \Longrightarrow \quad
y' = \frac{y}{2x}.
\]

\underline{דרך 2 – גזירה ישירה}  

נגזור את המשוואה $y^2=4Cx$ לפי $x$:
\[
2yy' = 4C.
\]

נבודד את $C$:
\[
C = \frac{yy'}{2}.
\]

נחזיר למשוואה המקורית:
\[
y^2 = 4\cdot \frac{yy'}{2}\cdot x 
= 2xyy'.
\]

נקבל:
\[
y' = \frac{y}{2x}.
\]

זו בדיוק אותה תוצאה שהתקבלה בדרך הראשונה.  
כלומר בשתי השיטות הגענו לאותה מד״ר שמתארת את המשפחה.

\textbf{שלב 2 – מציאת המד״ר האורתוגונלית}  

האלגוריתם מחייב להחליף $y'$ בביטוי האורתוגונלי לו:
\[
y' \;\;\longmapsto\;\; -\frac{1}{y'} = -\frac{y}{2x}.
\]

לכן המד״ר של המשפחה האורתוגונלית היא:
\[
\frac{dy}{dx} = -\frac{2x}{y}.
\]

\textbf{שלב 3 – פתרון המד״ר}  

מדובר במד׳׳ר פרידה. אין פתרונות חשודים לסינגולריים.
נכפול באלכסון:
\[
y\,dy = -2x\,dx.
\]

נבצע אינטגרציה:
\[
\frac{y^2}{2} = -x^2 + C.
\]

נסדר:
\[
x^2 + \frac{y^2}{2} = C.
\]

\textbf{מסקנה:}  
המשפחה האורתוגונלית לפרבולות $y^2=4Cx$ היא משפחת האליפסות:
\[
\boxed{x^2 + \tfrac{y^2}{2} = C}.
\]

באופן ויזואלי, ניתן לראות כי משיקי האליפסות פוגשות את משיקי הפרבולות בזווית ישרה בכל נקודת חיתוך:  

\begin{figure}[H]
\centering
\begin{tikzpicture}[scale=1.0]
\begin{axis}[
    axis lines=middle,
    xlabel={$x$},
    ylabel={$y$},
    xmin=-2.5, xmax=4,
    ymin=-4, ymax=4,
    legend style={at={(1.05,1)}, anchor=north west, font=\small},
    samples=200,
    domain=-3:3,
]

% === Family 1: Parabolas y^2 = 4Cx ===
\addplot[blue,thick,solid] ({(x^2)/2},{x});
\addlegendentry{$y^2 = 4(0.5)x$};

\addplot[blue,thick,dashed] ({(x^2)/4},{x});
\addlegendentry{$y^2 = 4(1)x$};

\addplot[blue,thick,dashdotted] ({(x^2)/8},{x});
\addlegendentry{$y^2 = 4(2)x$};

% === Family 2: Orthogonal curves x^2 + y^2/2 = C ===
\addplot[red,thick,solid,domain=0:360,samples=200] ({sqrt(2)*cos(x)},{sqrt(4)*sin(x)});
\addlegendentry{$x^2+\tfrac{y^2}{2}=2$};

\addplot[red,thick,dashed,domain=0:360,samples=200] ({sqrt(4)*cos(x)},{sqrt(8)*sin(x)});
\addlegendentry{$x^2+\tfrac{y^2}{2}=4$};

\addplot[red,thick,dashdotted,domain=0:360,samples=200] ({sqrt(6)*cos(x)},{sqrt(12)*sin(x)});
\addlegendentry{$x^2+\tfrac{y^2}{2}=6$};

\end{axis}
\end{tikzpicture}
\caption{משפחות עקומים אורתגונליות: בכחול – פרבולות $y^2=4Cx$ עבור ערכים שונים של $C$, ובאדום – המשפחה האורתגונלית $x^2+\tfrac{y^2}{2}=C$.}
\label{fig:parabola_orthogonal}
\end{figure}

\newpage
\subsection{משפט קיום ויחידות}

\textbf{משפט קיום ויחידות (הלא לינארי עבור סדר 1)}  

נניח כי $f(x,y)$ פונקציה המוגדרת בסביבת הנקודה $(x_0,y_0)$.  
נניח כי $f(x,y), f_y(x,y)$ רציפות בסביבת הנקודה $(x_0,y_0)$.  
אזי, למשוואה הדיפרנציאלית
\[
y' = f(x,y), \qquad y(x_0)=y_0,
\]
יש פתרון יחיד המוגדר בסביבת $x_0$.

\textbf{\underline{משמעויות}}  
\begin{enumerate}
\item אם $f, f_y$ רציפות בכל המישור, אזי פתרונות שונים לעולם לא נחתכים.  
\item אם שני פתרונות שונים נחתכים בנקודה $(x_0,y_0)$, אזי $f$ או $f_y$ לא רציפות בסביבת $(x_0,y_0)$.  
\item אם תנאי משפט הקיום והיחידות אינם מתקיימים, אין זה אומר שבהכרח יש יותר מפתרון אחד.
\end{enumerate}

\begin{insight}
משפט הקיום והיחידות הוא כלי עוצמתי ביותר בניתוח משוואות דיפרנציאליות.  
החשיבות שלו איננה רק מתמטית–תיאורטית, אלא גם פרקטית–הנדסית: הוא מאפשר לנו לקבוע \emph{באופן איכותי} את אופי הפתרונות של בעיות מורכבות, גם מבלי שנדרש בהכרח לפתור במפורש את המשוואה (ובהרבה מקרים כאשר אנו לא מסוגלים לפתור את המשוואה בכלים אנליטיים בלבד).

במקרים רבים, כאשר הפתרון הכמותי המדויק לא נגיש, עצם הידיעה שקיים או לא קיים פתרון יחיד לבעיה, יכולה לאפשר גילוי מוקדם של תכונות מעניינות של הפתרון כמו: חסימות, מונוטוניות ועוד. 
\end{insight}
\example{}

קבלו פתרון פרטי למשוואה הבאה:
\[
y' = 5x^2(y-2)^{\tfrac{2}{5}}, 
\qquad 
y(3)=2.
\]

\explanation{}

מדובר במד״ר פרידה.  
נמצא ראשית פתרונות סינגולריים:  
\[
g(y) = (y-2)^{\tfrac{2}{5}} = 0
\quad \Longrightarrow \quad
y \equiv 2 \quad \text{(חשוד כפתרון סינגולרי)}.
\]

עתה נמשיך לצעד הבא באלגוריתם:
\[
\int \tfrac{1}{g(y)}\,dy = G(y) = F(x) + C = \int f(x)\,dx + C.
\]

נחשב:
\[
F(x) = \int 5x^2\,dx = \tfrac{5x^3}{3}, 
\qquad 
G(y) = \int (y-2)^{-\tfrac{2}{5}}\,dy 
= \tfrac{5}{3}(y-2)^{\tfrac{3}{5}}.
\]

ולכן:
\[
\tfrac{5}{3}(y-2)^{\tfrac{3}{5}} = \tfrac{5}{3}x^3 + C^{*}
\quad \Longrightarrow \quad
(y-2)^{\tfrac{3}{5}} = x^3 + C.
\]

נוציא שורש חזקה:
\[
y = 2 + (x^3 + C)^{\tfrac{5}{3}}, 
\qquad x \in \mathbb{R}.
\]

זהו הפתרון הכללי המפורש.  
כעת נשתמש בתנאי ההתחלה $y(3)=2$:  
\[
2 = 2 + (3^3 + C)^{\tfrac{5}{3}} 
\quad \Longrightarrow \quad
C = -27.
\]

\textbf{פתרון פרטי:}
\[
\boxed{y = 2 + (x^3 - 27)^{\tfrac{5}{3}}, \quad y \equiv 2, \quad x \in \mathbb{R}.}
\]

נשים לב כי קיבלנו גם פתרון סינגולרי $y \equiv 2$, וגם פתרון כללי בהתאם לתנאי ההתחלה.  
נבדוק את תנאי משפט הקיום והיחידות:  
\[
y' = f(x,y) = 5x^2(y-2)^{\tfrac{2}{5}}.
\]
נבין כי
$f \in C^1$.
נגזור לפי $y$:  
\[
f_y = 5x^2 \cdot \tfrac{2}{5}(y-2)^{-\tfrac{3}{5}} 
= \tfrac{2x^2}{(y-2)^{\tfrac{3}{5}}}.
\]

נראה כי $f_y \notin C^1$ בסביבת $y=2$, ולכן תנאי משפט הקיום והיחידות אינם מתקיימים.  
כתוצאה מכך, ייתכן יותר מפתרון אחד שעובר בתנאי ההתחלה, כפי שאכן מצאנו. נראה זאת בסרטוט:
\begin{figure}[H]
    \centering
    \includegraphics[width=0.7\textwidth]{uniqueness_existence_one.png}
    \caption{השוואה בין הפתרון הפרטי 
    \lr{$y = 2 + (x^3 - 27)^{\frac{5}{3}}$} 
    לבין הפתרון הסינגולרי 
    \lr{$y \equiv 2$}. 
    נקודת תנאי ההתחלה $(3,2)$ סומנה בגרף, וניתן לראות כי שני הפתרונות עוברים דרכה.}
    \label{fig:existence_uniqueness}
\end{figure}

\example{} התבוננו במשוואה הדיפרנציאלית הבאה: \[ y' = (y-1)\sin(xy), \qquad y(0)=\tfrac{1}{2}. \] הראו כי הפתרון $y(x)$ מקיים $0<y(x)<1$ בתחום הגדרתו. 

\explanation{} ראשית נחשוב מדוע לא מבקשים מאיתנו את הפתרון של הבעיה ונסווג את המד׳׳ר. מדובר במד׳׳ר מסדר 1, לא לינארית. לכן כל הכלים שקיבלנו עד כה לפתרון מד׳׳רים לינאריות, לא תקפים. המד׳׳ר כמו כן לא פרידה. החלפת תפקידי $x,y$ גם כן לא תעזור שכן גם $x$ וגם $y$ נמצאים בתוך פונקציה סינוסואידאלית. המד׳׳ר גם לא עונה למבנה ברנולי, ולא ריקאטי. לא נוכל לסווג אותה גם כמד׳׳ר מטיפוס הומוגני. לבסוף (בדקו בעצמכם), היא גם לא מדויקת ולא ניתן לדייק אותה בעזרת גורם אינטגרציה של משתנה אחד בלבד. אז מה כן עושים? מפנימים שאין לנו כלים אנליטיים לפתור מד׳׳ר זו, מה שגם שלא ביקשו לפתור אותה. ביקשו להראות שהיא חסימה בתחום הגדרתה. דבר זה מדליק נורה אדומה לבדיקת משפט קיום ויחידות. נזכיר כי קיום ויחידות עוזר לנו רבות לבדיקת מונוטוניות וחסימות. 

\textbf{שלב 1 – בדיקת תנאי משפט הקיום והיחידות} המשוואה היא: \[ y' = f(x,y) = (y-1)\sin(xy), \qquad y(0)=\tfrac{1}{2}. \] נחשב את הנגזרת החלקית לפי $y$: \[ f_y(x,y) = \frac{\partial}{\partial y}\big[(y-1)\sin(xy)\big]. \] בעזרת כלל המכפלה: \[ f_y(x,y) = 1\cdot \sin(xy) + (y-1)\cdot \cos(xy)\cdot x. \] נראה כי $f(x,y)$ ו־$f_y(x,y)$ הן פונקציות רציפות לכל $x,y \in \mathbb{R}$, ולכן שתיהן שייכות ל־$C^1(\mathbb{R}^2)$. על כן, תנאי משפט הקיום והיחידות מתקיימים \textbf{בכל המישור}. 

\textbf{שלב 2 – מסקנה ישירה מהמשפט} אם התנאים מתקיימים בכל המישור, הרי שלכל נקודת התחלה $(x_0,y_0)$ קיים פתרון יחיד העובר דרכה. כלומר: \textbf{שני פתרונות שונים לעולם לא יכולים להיחתך}. באופן מיוחד, הפתרון $y(x)$ שמתחיל ב־$y(0)=\tfrac{1}{2}$ לעולם לא יפגוש פתרון אחר. בינתיים, מרגיש כאילו הדבר לא באמת מקדם אותנו. במקרה זה, אין לנו ברירה אלא לנסות ו׳׳לנחש׳׳ פתרונות, בתקווה לקבל איזושהי תובנה בנוגע לחסימות של הפונקציה. למעשה, יש תמיד לבחון מד׳׳ר תחילה ולבדוק האם אנחנו מסוגלים לנחש פתרון/ות מיידית. 

\textbf{שלב 3 – זיהוי פתרונות קבועים מיידיים} נשים לב כי יש \emph{פתרונות שניתן לנחש רק מהסתכלות על המד׳׳ר}: נבדוק $y(x)\equiv 1$: \[ y' = 0, \qquad f(x,1) = (1-1)\sin(x\cdot 1) = 0. \] כלומר $y\equiv 1$ הוא פתרון. נבדוק $y(x)\equiv 0$: \[ y' = 0, \qquad f(x,0) = (0-1)\sin(x\cdot 0) = (-1)\cdot 0 = 0. \] כלומר $y\equiv 0$ גם הוא פתרון. קיבלנו שתי "מעטפות" קבועות של המשוואה – גבולות שברור כי הפתרון לעולם לא יצא מהם (על מנת לא להפר את מה שנובע מקיום משפט קיום ויחידות בכל המישור). 

\textbf{שלב 4 – ייחודיות הפתרון עם תנאי ההתחלה} כיוון שנתון $y(0)=\tfrac{1}{2}$, הפתרון מתחיל בדיוק באמצע בין שני הפתרונות הקבועים $y\equiv 0$ ו־$y\equiv 1$. מאחר שמשפט הקיום והיחידות מבטיח ייחודיות, הפתרון שנובע מתנאי ההתחלה הזה: \[ 0 < y(x) < 1, \qquad \forall x \in \mathbb{R}, \] ולעולם לא יחתוך את $y\equiv 0$ או את $y\equiv 1$. \textbf{מסקנה:} בעזרת משפט הקיום והיחידות קיבלנו \textbf{תובנה איכותית}: הפתרון קיים, יחיד, והוא "נלכד" בין שתי הפתרונות הקבועים $y\equiv 0$ ו־$y\equiv 1$. במילים אחרות – אפילו מבלי לפתור במפורש את המשוואה, ניתן להסיק שהפתרון שלנו לא יעבור את הגבולות הללו, ותנאי ההתחלה מכתיב באיזה תחום הוא יישאר. על מנת להמחיש זאת באמצעות סרטוט, נציג פתרון שהתקבל בצורה נומרית (מחוץ לתחום של ספר זה): \begin{figure}[H] \centering \includegraphics[width=0.95\textwidth]{uniqueness_existence_two_final.png} \caption{המחשה נומרית למשוואה \[ y' = (y-1)\sin(xy), \qquad y(0)=\tfrac{1}{2}. \] העקומה הכחולה היא הפתרון הייחודי שמתחיל בנקודה $(0,\tfrac{1}{2})$, הקווים המקווקווים האדום והירוק מייצגים את הפתרונות הקבועים $y\equiv 0$ ו־$y\equiv 1$, בהתאמה. ניתן לראות שהפתרון נשאר חסום בתחום $0<y(x)<1$ ואינו חותך את הפתרונות הקבועים, בהתאם למשפט הקיום והיחידות. \textbf{ההגדלה שבתוך התיבה} מראה כי למרות שהפתרון הפרטי קרוב לפתרונות הקבועים, ל־$y=0$, אין חיתוך אמיתי אלא שאיפה אסימפטוטית בלבד.} \label{fig:uniqueness_existence_demo} \end{figure}



\textbf{הרחבה – ניתוח לפי תנאי התחלה כללי $y(0)=\alpha$}  

נבחן את אותה משוואה דיפרנציאלית:
\[
y' = (y-1)\sin(xy),
\]
כעת עם תנאי התחלה כללי:
\[
y(0)=\alpha,\qquad  \alpha\in\mathbb{R}.
\]

ננתח שלושה מצבים עיקריים:

\begin{enumerate}
\item \textbf{מקרה 1 – $\alpha < 0$:}  
במצב זה תנאי ההתחלה נמצא מתחת לפתרון הקבוע $y\equiv 0$.  
משפט קיום ויחידות מבטיח שהפתרון שנובע מנקודה זו לעולם לא יחתוך את $y=0$, ולכן הוא יישאר שלילי לכל $x$.  
במילים אחרות, עבור $\alpha<0$, נקבל פתרון יחיד אשר חסום מלמעלה בתחום $y(x)<0$.  

\item \textbf{מקרה 2 – $0<\alpha<1$:}  
זהו בדיוק המקרה שראינו קודם.  
כאן תנאי ההתחלה נמצא בין שני הפתרונות הקבועים $y\equiv 0$ ו־$y\equiv 1$.  
משפט הקיום והיחידות מבטיח שהפתרון לעולם לא יחתוך אף אחד מהם, ולכן נקבל:
\[
0<y(x)<1, \qquad \forall x\in\mathbb{R}.
\]
הפתרון "נלכד" בין המעטפות ויישאר חסום ביניהן.  

\item \textbf{מקרה 3 – $\alpha>1$:}  
כאן תנאי ההתחלה נמצא מעל הפתרון הקבוע $y\equiv 1$.  
שוב, משפט הקיום והיחידות מבטיח שהפתרון לא יחתוך את $y=1$, ולכן נקבל פתרון יחיד שמקיים:
\[
y(x)>1, \qquad \forall x\in\mathbb{R}.
\]
כלומר, הפתרון חסום מלמטה בתחום $y(x)>1$. 
\end{enumerate}
נמחיש ויזואלית את המקרים:

\begin{figure}[H]
    \centering
    \includegraphics[width=0.95\textwidth]{uniqueness_existence_three.png}
    \caption{השפעת תנאי ההתחלה $y(0)=\alpha$ על הפתרון של המשוואה
    \[
    y'=(y-1)\sin(xy).
    \]
    מוצגים שלושה תרחישים: 
    \textbf{(a)} $\alpha<0$ – הפתרון נשאר חסום בתחום $y(x)<0$; 
    \textbf{(b)} $0<\alpha<1$ – הפתרון נלכד בין $y=0$ ל־$y=1$; 
    \textbf{(c)} $\alpha>1$ – הפתרון נשאר חסום בתחום $y(x)>1$.  
    הקווים בצבעים שחור ותכלת מייצגים את הפתרונות הקבועים $y\equiv 0$ ו־$y\equiv 1$, בהתאמה,  
    והעקומות הצבעוניות האחרות מצייגות את הפתרונות הפרטיים לבעיה עבור ערכים שונים של $\alpha$.}
    \label{fig:uniqueness_alpha_cases}
\end{figure}

%%%%%%%example - קיום ויחידות%%%%%%
\example{}\label{special_sine}

פתרו את המשוואה הדיפרנציאלית:
\[
y' = \sqrt{1-y^2}, 
\qquad y(0)=0.
\]

כמה פתרונות שונים מקיימים את תנאי ההתחלה?  
האם משפט הקיום והיחידות מתקיים במקרה זה?

\explanation{}

המשוואה מוגדרת עבור $-1 \leq y \leq 1$, שכן מחוץ לטווח זה הביטוי תחת השורש שלילי.  
מדובר ב\textbf{משוואה פרידה (אוטונומית)}:
\[
\frac{dy}{dx} = \sqrt{1-y^2}.
\]

\textbf{שלב 1 – בדיקת פתרונות קבועים (סינגולריים):}  
נחפש את השורשים של $g(y)=0$:
\[
\sqrt{1-y^2} = 0 
\;\;\Longrightarrow\;\; y^2 = 1.
\]
ולכן:
\[
y(x) \equiv 1, 
\qquad 
y(x) \equiv -1
\]
שניהם פתרונות קבועים אמיתיים, שכן הצבה חזרה נותנת $y'=0$ ובצד ימין $\sqrt{1-1^2}=0$.  

\textbf{שלב 2 – הפרדת משתנים:}  
\[
\frac{dy}{\sqrt{1-y^2}} = dx.
\]

\textbf{שלב 3 – אינטגרציה:}  
\[
\int \frac{1}{\sqrt{1-y^2}}\,dy = \int dx.
\]
האינטגרל השמאלי הוא $\arcsin(y)$, והימני $x+C$:
\[
\arcsin(y) = x + C.
\]

\textbf{שלב 4 – בידוד $y$:}  
\[
y(x) = \sin(x+C).
\]
מכאן ניתן להבין שהפתרונות הקבועים, אכן סינגולריים.
\textbf{שלב 5 – הצבת תנאי ההתחלה:}  
מהנתון $y(0)=0$ נקבל:
\[
0 = \sin(C).
\]
מכאן $C=k\pi, \; k\in\mathbb{Z}$.

נכתוב את הפתרון כך:
\[
y(x) =
\begin{cases}
\sin(x+2k\pi)=\sin(x), & k \in \mathbb{Z}, \\[6pt]
\sin(x+(2k+1)\pi) = -\sin(x), & k \in \mathbb{Z}.
\end{cases}
\] 
זאת לצד שני הפתרונות הסינגולריים שלנו.
\begin{insight}
נשים לב לתכונה מאוד מעניינת של המד׳׳ר. אגף ימין של המד׳׳ר הוא פונקציית שורש. פרט לכך שהמשוואה מוגדרת עבור $-1 \leq y \leq 1$, השורש עצמו חייב לתת תוצר לא שלילי. כלומר אגף ימין של המשוואה לא יכול להיות שלילי. זה גורר כי גם אף שמאל לא יכול להיות שלילי, מה שגורר שהנגזרת של הפתרון לא יכולה להיות שלילית. דבר המוביל להבנה כי הפתרון שלנו לא יכול לרדת (חייב להיות קבוע או עולה). אנו מבינים כי פונקציית הסינוס לא יכולה לספק לנו תכונה כזו על כל הישר. מכאן נוצר הצורך בדבר הנקרא \textbf{תפירה של פתרונות}. כדי שפתרונות שונים יוכלו ׳׳להיתפר׳׳, עלינו לוודא שמותר להם להיחתך. נבדוק את תנאי משפט קיום ויחידות.

\end{insight}
\textbf{שלב 6 – בדיקת תנאי משפט הקיום והיחידות:}  

ננסח את המד״ר:
\[
y' = f(x,y), 
\qquad f(x,y) = \sqrt{\,1-y^2\,}.
\]

נבדוק התנאים:
1. \textbf{רציפות $f$:}  
הפונקציה $f(x,y)=\sqrt{1-y^2}$ רציפה בתחום $D=\{(x,y)\mid -1\le y\le 1\}$.

2. \textbf{רציפות $f_y$:}  
נגזור לפי $y$:
\[
f_y(x,y) = \frac{\partial}{\partial y}\sqrt{1-y^2} 
= \frac{-y}{\sqrt{1-y^2}}.
\]

נבחין:
- עבור $-1<y<1$, הנגזרת קיימת ורציפה.  
- עבור $y=\pm 1$, הנגזרת מתבדרת ואינה רציפה.  

\textbf{מסקנה:}  
משפט הקיום והיחידות תקף בכל נקודת התחלה $(x_0,y_0)$ עם $-1<y_0<1$.  
בקטע זה מובטח פתרון \textbf{יחיד !}.  
לעומת זאת, בגבולות $y=\pm 1$ ייחודיות אינה מובטחת — ולכן מותר לנו ׳׳לתפור" פתרונות שונים.

\textbf{שלב 7 – תפירת הפתרון המלא:}  

על מנת לתפור את הפתרונות היטב, ולשמור על כל התנאים הקדחתניים שמובלעים בתוך השאלה, נצטרך לחשוב אילו פתרונות שקיבלנו פוטנציאליים עבור אילוצים אלה, ואילו לא. התרחיש שמאפשר את קיום כל התנאים בו זמנית, הוא התרחיש הבא:
הפתרון $\sin(x)$ עובר מנקודת ההתחלה $(0,0)$ עד לערך $y=1$ בנקודה $x=\tfrac{\pi}{2}$, ובכיוון השלילי עד לערך $y=-1$ בנקודה $x=-\tfrac{\pi}{2}$.  

מאותו רגע ואילך, כדי לשמור על אי־ירידה ובהיעדר ייחודיות ב־$y=\pm 1$, ניתן \textbf{להמשיך בתפר} אל הפתרונות הקבועים:  
\[
\boxed{
y(x) = 
\begin{cases}
-1, & x \leq -\tfrac{\pi}{2}, \\[6pt]
\sin(x), & -\tfrac{\pi}{2}\leq x \leq \tfrac{\pi}{2}, \\[6pt]
1, & x \geq \tfrac{\pi}{2}.
\end{cases}
}
\]

זהו הפתרון הגלובלי היחיד שעובר דרך $(0,0)$, מקיים $y'(0)=1$, אינו יורד לעולם, ומנצל את חופש התפירה המותר בקצות $y=\pm 1$. למה $-\sin(x)$ לא יכול לעבוד? כי הוא לא יכול להיתפר עם $y=1$ מלמעלה. נמחיש באמצעות סרטוט:
\begin{figure}[H]
    \centering
    \includegraphics[width=0.7\textwidth]{existence_uniqueness_sin.png}
    \caption{הפתרון המלא של בעיית התנאי ההתחלתי 
    \lr{$y'=\sqrt{1-y^2}, \; y(0)=0$}. 
    הפתרון נתפר כך שקטע הסינוס 
    \lr{$y=\sin(x)$} 
    מחובר לפתרון הקבוע 
    \lr{$y\equiv -1$} 
    עבור $x\leq -\frac{\pi}{2}$ 
    ולפתרון הקבוע 
    \lr{$y\equiv 1$} 
    עבור $x\geq \frac{\pi}{2}$. 
    סומנו גם נקודת תנאי ההתחלה $(0,0)$ ונקודות התפירה 
    \lr{$(-\frac{\pi}{2},-1)$}, \lr{$(\frac{\pi}{2},1)$}.}
    \label{fig:existence_uniqueness_sin}
\end{figure}

%%%CUT%%%

\newpage
\underline{תרגילים}
%%%%%%%ex.%%%%%%
\exercise{}

פתרו את המשוואה הדיפרנציאלית:
\[
y' = \sqrt{y}, 
\qquad y(0)=0.
\]

כמה פתרונות שונים מקיימים את תנאי ההתחלה?  
האם משפט הקיום והיחידות מתקיים במקרה זה?

%%%%%%%exercise - סימטריה בפתרונות%%%%%%
\exercise{}

נתונה המשוואה:
\[
y' = x \cdot (x^2 + y^2)^4
\]

הראו כי הפתרון של הבעיה $y(x)$, המוגדר על הקטע הסימטרי $(-a,a)$ הינו פונקציה זוגית. 

\newpage
\underline{פתרונות}
\solution{}

נתחיל נבין כי המד׳׳ר מוגדרת רק עבור $y\geq0$.
נזהה כי המשוואה היא \textbf{משוואה פרידה (ואפילו אוטונומית)}, שכן:
\[
y' = f(x)g(y), \qquad g(y)=\sqrt{y}, f(x)=1.
\]

\textbf{שלב 1 – חיפוש פתרונות סינגולריים/קבועים:}  
נפתור $g(y)=0$:
\[
\sqrt{y}=0 \;\;\Longrightarrow\;\; y=0.
\]
כלומר, $y(x)\equiv 0$ הוא פתרון \emph{חשוד סינגולרי}.  
אכן, אם נציב חזרה במשוואה:
\[
y'=0, \qquad \sqrt{0}=0,
\]
נקבל שוויון ולכן $y\equiv 0$ הוא פתרון תקף.  
נרשום:
\[
\boxed{y(x)\equiv 0}
\]
כפתרון קבוע של המשוואה.

\textbf{שלב 2 – הפרדת משתנים עבור $y>0$:}  
נבצע את אלגוריתם ההפרדה:  
\[
\frac{dy}{\sqrt{y}} = dx.
\]
נבצע אינטגרציה:
\[
\int \frac{1}{\sqrt{y}}\,dy = \int dx.
\]
האינטגרל השמאלי:
\[
\int y^{-1/2}\,dy = 2\sqrt{y}.
\]
האינטגרל הימני:
\[
\int dx = x + C.
\]
נקבל:
\[
2\sqrt{y} = x + C.
\]

\textbf{שלב 3 – בידוד $y$:}  
נעלה בריבוע:
\[
y(x) = \left(\tfrac{x+C}{2}\right)^2.
\]
בשלב זה ניתן להבין כי הפתרון הקבוע הוא אכן סינגולרי.

\textbf{שלב 4 – הצבת תנאי ההתחלה:}  
הנתון הוא $y(0)=0$:  
\[
0 = \left(\tfrac{0+C}{2}\right)^2 \;\;\Longrightarrow\;\; C=0.
\]
ולכן:
\[
\boxed{y(x) = \tfrac{x^2}{4}}.
\]
בפועל קיבלנו שני פתרונות פרטיים שעוברים באותו תנאי התחלה. האם הדבר תקין? נבדוק קיום ויחידות.

\textbf{שלב 5 – בדיקת קיום וייחודיות:}  
נגדיר $f(x,y)=\sqrt{y}$.  
הפונקציה $f$ רציפה לכל $y\geq 0$.  
לעומת זאת:
\[
f_y(x,y) = \frac{1}{2\sqrt{y}}, \qquad y>0,
\]
אינה רציפה ומתבדרת כאשר $y\to 0^+$.
מכאן, ששני הפתרונות הפרטיים יכולים להיחתך בתנאי ההתחלה.


\solution{}

נרצה להראות כי הפתרון $y(x)$ הוא פונקציה זוגית, כלומר $y(-x)=y(x)$ בתחום ההגדרה של הבעיה. 

\textbf{מוטיבציה:}  
אגף ימין של המשוואה כולל את $x$ ואת $y$ בחזקות זוגיות (מלבד המכפיל $x$ מלפנים).  
נראה כי אם $y(x)$ פתרון, גם $y(-x)$ מתנהג בצורה סימטרית.  
כדי להפוך זאת להוכחה פורמלית נשווה בין $y(x)$ ל־$y(-x)$. אם שתיהן מקיימות את אותה בעיית ערך התחלה, נוכל להשתמש ב\textbf{משפט הקיום והיחידות}, שמבטיח כי קיים \emph{פתרון יחיד} על הקטע הסימטרי $(-a,a)$.  
כל ניסיון לקבל שני פתרונות שונים יסתור את ייחודיות הפתרון.  
 
נגדיר
\[
z(x) := y(-x).
\] 
נגזור:
\[
z'(x) = \frac{d}{dx}\big[y(-x)\big] = -y'(-x).
\]

מהמשוואה הנתונה:
\[
y'(-x) = (-x)\cdot\Big(((-x)^2)+(y(-x))^2\Big)^4.
\]

נציב $z(x)=y(-x)$:
\[
y'(-x) = (-x)\cdot\big(x^2+z(x)^2\big)^4.
\]

ולכן:
\[
z'(x) = -y'(-x) 
= -\Big[(-x)\cdot\big(x^2+z(x)^2\big)^4\Big] 
= x\cdot\big(x^2+z(x)^2\big)^4.
\]

זהה למשוואה שמקיימת $y(x)$:
\[
z'(x) = x \cdot (x^2+z(x)^2)^4.
\]
נבדוק את תנאי ההתחלה שלנו:
\[
z(0) = y(-0) = y(0).
\]

כלומר $z(x)$ מקיימת את אותה משוואה ואת אותו תנאי התחלה כמו $y(x)$.

נבדוק את תנאי המשפט קיום ויחידות.
נגדיר:
\[
f(x,y) = x(x^2+y^2)^4.
\]

הפונקציה $f$ רציפה לכל $x,y \in \mathbb{R}$.  
הנגזרת החלקית לפי $y$ היא:
\[
f_y(x,y) = 8x(x^2+y^2)^3y,
\]
וגם היא רציפה בכל התחום.  

מכאן ש־$f$ ו־$f_y$ רציפות על כל המישור, ולכן תנאי המשפט מתקיימים.  
כלומר על כל קטע $(-a,a)$ הפתרון לבעיה הראשונית \emph{קיים ויחיד}.  

קיבלנו כי $y(x)$ וגם $z(x)=y(-x)$ הם פתרונות של אותה בעיית ערך התחלה.  
מאחר שעל פי המשפט הפתרון יחיד בקטע $(-a,a)$, בהכרח:
\[
z(x) \equiv y(x).
\]

כלומר:
\[
y(-x) = y(x) \quad \forall x \in (-a,a).
\]
  
מכאן כי הפתרון $y(x)$ הוא פונקציה \textbf{זוגית} בתחום הסימטרי הנתון.

\newpage
\subsection{שדה כיוונים}

אחת הדרכים הוויזואליות להבין משוואה דיפרנציאלית בצורה איכותית היא דרך \textbf{שדה כיוונים}.  
נתונה משוואה דיפרנציאלית מהצורה:
\begin{equation}\label{eq:ode1}
y' = f(x,y).
\end{equation}

בנקודה $(x_0,y_0)$ במישור, המשוואה \eqref{eq:ode1} קובעת את השיפוע של הפתרון העובר דרך נקודה זו:
\[
m = y'(x_0) = f(x_0,y_0).
\]

במקום לחשב פתרון מפורש, נוכל לסמן בכל נקודה $(x_0,y_0)$ במישור קטע בעל שיפוע $m=f(x_0,y_0)$.  
קבוצת כל הקטעים הללו מהווה את \textbf{שדה הכיוונים} של המשוואה.  
ע"י בחינה גרפית של השדה ניתן לקבל אינטואיציה לגבי התנהגות הפתרונות.

\textbf{שדה כיוונים} של המשוואה \eqref{eq:ode1} הוא אוסף של וקטורים במישור $(x,y)$
המצביעים על הכיוון שבו הפתרון $y(x)$ נוטה לנוע בכל נקודה.
הפתרונות $y(x)$ של \eqref{eq:ode1} הם \textbf{קווים משיקים לשדה הכיוונים}. אם לדוגמה נחשוב על המד׳׳ר הפשוטה $y'=x$, הרי שדוגמה לפתרון פרטי שלה יהיה $y=\frac{x^{2}}{2}$. נראה את שדה הכיוונים של משוואה זו בסרטוט:

\begin{figure}[H]
    \centering
    \begin{tikzpicture}[scale=1.2]
      % Axes
      \draw[->] (-2.5,0)--(2.5,0) node[right]{$x$};
      \draw[->] (0,-2.5)--(0,2.5) node[above]{$y$};

      % Sample slope vectors for y'=x
      \foreach \x in {-2,-1.5,...,2}
        \foreach \y in {-2,-1,...,2}
        {
          % slope = x
          \pgfmathsetmacro{\dy}{\x}
          \draw[->,blue!70] (\x,\y) -- ++(0.3,0.3*\dy);
        }

      % Example solution curve y = x^2/2
      \draw[thick,red,domain=-2:2,samples=100] plot (\x,{0.5*\x*\x})
      node[right]{$y=\tfrac{x^2}{2}$};
    \end{tikzpicture}
    \caption{שדה הכיוונים של $y'=x$ (חיצים כחולים) ופתרון לדוגמה $y=\tfrac{x^2}{2}$ (קו אדום). 
    בכל נקודה $(x,y)$ החץ הכחול מתאר את השיפוע $m=f(x,y)=x$ שבו עובר הפתרון. 
    שדה הכיוונים מראה את \textbf{התנהגות הפתרונות }: הפתרון האדום תמיד משיק לכיוון החיצים באזור שבו הוא עובר.}
\end{figure}
 לקווי הגובה של הפונקציה $f$ קוראים \textbf{איזוקלינות} של המד"ר $y'=f(x,y)$. כאשר $f(x,y)=0$, כלומר קו גובה אפס, נקבל \textbf{נולקלינה}.

\begin{itemize}
  \item כאשר הפתרון המפורש של \eqref{eq:ode1} לא נגיש, או שאנו מסתפקים בפתרון איכותי לבעיה, שדה הכיוונים מאפשר לקבל תמונה איכותית על צורת הפתרונות.
  \item ניתן לראות בצורה ברורה נקודות בהן השיפוע אפס ($f(x,y)=0$) – פתרונות קבועים.
  \item אפשר לזהות אזורים בהם הפתרונות עולים או יורדים, נקודות יציבות, התכנסות או התבדרות.
\end{itemize}

\begin{insight}
שדה כיוונים מספק ייצוג גרפי של כל המשוואה הדיפרנציאלית על פני המישור, ומאפשר הבנה גיאומטרית-אינטואיטיבית של התנהגות הפתרונות עוד בטרם מציאת נוסחה סגורה.
\end{insight}

\example{}

תהי
\[
f(x,y) = |y| - x
\]

א. ציירו קווי גובה $f(x,y) = c$ עבור $c=-2,-1,0,1,2$

ב. הסתמכו על סעיף א׳ וציירו שדה כיוונים למד״ר 
\[
y' = |y| - x
\]

ג. כיצד מתנהג פתרון הבעיה
\[
y' = |y| - x \;;\; y(2)=-1
\]

ד. ציירו פתרונות נוספים של המד׳׳ר על סמך שדה הכיוונים, ובפרט פתרון שמקיים 
\[
y(0)=1
\]

\explanation

\[
f(x,y) = |y| - x
\qquad
c=-2,-1,0,1,2
\]

א. נסרטט את הקווים $|y|=x+c$: 

\begin{figure}[H]
\centering
\begin{tikzpicture}[scale=0.9]
  % axes
  \draw[thick,->] (-3.2,0) -- (3.2,0) node[right]{$x$};
  \draw[thick,->] (0,-3.2) -- (0,3.2) node[above]{$y$};

  % values of c with colors
  \foreach \c/\col in {-2/red,-1/blue,0/green!70!black,1/orange,2/purple} {
    % domain: x >= -c
    \pgfmathsetmacro\xmin{-\c}
    \pgfmathsetmacro\xmax{3}

    % upper branch: y = x+c
    \draw[thick,\col,domain=\xmin:\xmax] plot (\x,{\x+\c});
    % lower branch: y = -(x+c)
    \draw[thick,\col,domain=\xmin:\xmax] plot (\x,{-(\x+\c)});

    % label near right end of upper branch
    \node[\col] at (\xmax+0.6,{\xmax+\c}) {$c=\c$};
  }
\end{tikzpicture}

\caption{קווי הגובה (איזוקלינות) של השדה $y' = |y|-x$ עבור $c=-2,-1,0,1,2$. 
כל קו מייצג גרפית את אחד הפתרונות של $|y|=x+c$. 
בפרט, עבור $c=0$ מתקבלת \textbf{נולקלינה}, כאשר היא מציינת את המקומות 
    בהם נגזרת הפתרון מתאפסת.}
\label{fig:isoclines_abs}
\end{figure}

ב.
על הקו $f(x,y)$ מתקיים $y'=c$ כלומר כל הפתרונות של המד"ר שחותכים את קו הגובה $f(x,y)=c$ , חותכים אותו באותו שיפוע $y'=c$. לקווי הגובה של הפונקציה $f$ קוראים איזוקלינות של המד"ר $y'=f(x,y)$ (שם נוסף: לקו הגובה אפס, שעליו $f(x,y)=0$, קוראים נולקלינה).

\begin{figure}[H]
\centering
\begin{tikzpicture}[scale=0.9]
  % axes
  \draw[thick,->] (-3.2,0) -- (3.5,0) node[right]{$x$};
  \draw[thick,->] (0,-3.2) -- (0,3.5) node[above]{$y$};

  % define constants c and colors
  \foreach \c/\col in {-2/red,-1/orange,0/blue,1/green,2/purple} {
    % domain: x >= -c
    \pgfmathsetmacro\xmin{-\c}
    \pgfmathsetmacro\xmax{3}

    % upper branch
    \draw[thick,\col,domain=\xmin:\xmax,samples=2] plot (\x,{\x+\c});
    % lower branch
    \draw[thick,\col,domain=\xmin:\xmax,samples=2] plot (\x,{-(\x+\c)});

    % label
    \node[\col] at (\xmax+1.0,{\xmax+\c}) {$c=\c$};

    % slope arrows along upper branch (inside domain only)
    \foreach \x in {\xmin,...,\xmax} {
      \pgfmathsetmacro\y{\x+\c}
      \pgfmathsetmacro\dx{0.3}
      \pgfmathsetmacro\dy{0.3*\c}
      \draw[->,\col,thick] ({\x-0.5*\dx},{\y-0.5*\dy}) -- ({\x+0.5*\dx},{\y+0.5*\dy});
    }

    % slope arrows along lower branch (inside domain only)
    \foreach \x in {\xmin,...,\xmax} {
      \pgfmathsetmacro\y{-(\x+\c)}
      \pgfmathsetmacro\dx{0.3}
      \pgfmathsetmacro\dy{0.3*\c}
      \draw[->,\col,thick] ({\x-0.5*\dx},{\y-0.5*\dy}) -- ({\x+0.5*\dx},{\y+0.5*\dy});
    }
  }
\end{tikzpicture}
\caption{שדה הכיוונים של המשוואה $y' = |y|-x$ בעזרת איזוקלינות $f(x,y)=c$ 
ל־$c=-2,-1,0,1,2$. החצים הקטנים מוצגים רק על נקודות שנמצאות ממש על כל עקומה.}
\label{fig:dirfield_abs}
\end{figure}

\begin{insight}
נשים לב לתופעה מעניינת: על החלק העליון של האיזוקלינה $c=1$ (כלומר $y=x+1$), 
וכן על החלק התחתון של האיזוקלינה $c=-1$ (כלומר $y=-(x-1)$), 
החצים שמסמנים את שיפוע השדה מקבילים ממש לעקומה עצמה.  
הסיבה לכך היא שבנקודות אלו נגזרת הפתרון $y'$ שווה בדיוק לשיפוע של הישר עליו אנו נמצאים.  
כלומר, המשוואה $y'=|y|-x$ לא רק מכתיבה שיפוע קבוע לאורך כל איזוקלינה, 
אלא במקרים מיוחדים אלו השיפוע הקבוע $c$ מתלכד בדיוק עם השיפוע הגיאומטרי של הקו. 
מכאן נובע שהפתרונות הנובעים מאותן נקודות נעים ממש \textbf{לאורך} הישר כולו.
\end{insight}

ג. נכתוב את המד׳׳ר עם תנאי ההתחלה:
\[
\begin{cases}
y' = |y| - x \\
y(2) = -1
\end{cases}
\]  

נשים לב כי הנקודה $(x,y) = (2,-1)$ נמצאת על איזוקלינה $c=-1$.  

נחזור על המסקנה מהסעיף הקודם: אם החיצים משיקים לאיזוקלינה – אז האיזוקלינה היא פתרון!  
(כי החיצים משיקים לעקומת הפתרון). נדגיש זאת בסרטוט:

\begin{figure}[H]
\centering
\begin{tikzpicture}[scale=0.9]
  % axes
  \draw[thick,->] (-3.2,0) -- (3.5,0) node[right]{$x$};
  \draw[thick,->] (0,-3.2) -- (0,3.5) node[above]{$y$};

  % constants
  \foreach \c in {-2,-1,0,1,2} {
    \pgfmathsetmacro\xmin{-\c}
    \pgfmathsetmacro\xmax{3}

    % upper branch
    \ifnum\c=1
      \draw[thick,blue,domain=\xmin:\xmax] plot (\x,{\x+\c});
      \foreach \x in {\xmin,...,\xmax} {
        \pgfmathsetmacro\y{\x+\c}
        \pgfmathsetmacro\dx{0.3}
        \pgfmathsetmacro\dy{0.3*\c}
        \draw[->,blue,thick] ({\x-0.5*\dx},{\y-0.5*\dy}) -- ({\x+0.5*\dx},{\y+0.5*\dy});
      }
    \else
      \draw[thick,black,domain=\xmin:\xmax] plot (\x,{\x+\c});
      \foreach \x in {\xmin,...,\xmax} {
        \pgfmathsetmacro\y{\x+\c}
        \pgfmathsetmacro\dx{0.3}
        \pgfmathsetmacro\dy{0.3*\c}
        \draw[->,black,thick] ({\x-0.5*\dx},{\y-0.5*\dy}) -- ({\x+0.5*\dx},{\y+0.5*\dy});
      }
    \fi

    % lower branch
    \ifnum\c=-1
      \draw[thick,blue,domain=\xmin:\xmax] plot (\x,{-(\x+\c)});
      \foreach \x in {\xmin,...,\xmax} {
        \pgfmathsetmacro\y{-(\x+\c)}
        \pgfmathsetmacro\dx{0.3}
        \pgfmathsetmacro\dy{0.3*\c}
        \draw[->,blue,thick] ({\x-0.5*\dx},{\y-0.5*\dy}) -- ({\x+0.5*\dx},{\y+0.5*\dy});
      }
    \else
      \draw[thick,black,domain=\xmin:\xmax] plot (\x,{-(\x+\c)});
      \foreach \x in {\xmin,...,\xmax} {
        \pgfmathsetmacro\y{-(\x+\c)}
        \pgfmathsetmacro\dx{0.3}
        \pgfmathsetmacro\dy{0.3*\c}
        \draw[->,black,thick] ({\x-0.5*\dx},{\y-0.5*\dy}) -- ({\x+0.5*\dx},{\y+0.5*\dy});
      }
    \fi

    % labels
    \node[black] at (\xmax+0.8,{\xmax+\c}) {$c=\c$};
  }

  % initial condition point
  \filldraw[red] (2,-1) circle (2pt);

\end{tikzpicture}
\caption{איזוקלינות של $y' = |y|-x$ עבור $c=-2,-1,0,1,2$.  
כל הקווים והחצים מוצגים בשחור, למעט החלק העליון של $c=1$ והחלק התחתון של $c=-1$, 
המודגשים בכחול. נקודת ההתחלה $(2,-1)$ של תנאי ההתחלה $y(2)=-1$ מסומנת בעיגול אדום.}
\label{fig:dirfield_abs_highlight}
\end{figure}

לפי כך, לכל $x \geq 1$ הפתרון הפרטי לבעיה נתון ע"י 
\[
\boxed{y(x) = 1 - x}.
\]

ולכן המשוואה מתקיימת: 
\[
y' = -1, \quad |y| = |1 - x| = x - 1\rightarrow -1=x-1-x
\]
ועבור $x \geq 1$ זה מתקיים באופן זהותי.

מה קורה עבור $x \leq 1$?

על־פי שדה הכיוונים, ב“מרחב” שבין האיזוקלינות 
$c=-1$ ו־$c=0$ 
השיפוע הוא בין $0$ לבין $1$, 
ובדומה בשאר התחומים שבין הזוגות של איזוקלינות. 

האם ב־$x=1$ הפתרון בעל נגזרת רציפה?  
כן, כי 
\[
y' = |y| - x,
\]
ובאגף ימין יש פונקציה רציפה.
נמחיש בסרטוט:
\begin{figure}[H]
\centering
\begin{tikzpicture}[scale=0.9]
  % axes
  \draw[thick,->] (-3.2,0) -- (3.5,0) node[right]{$x$};
  \draw[thick,->] (0,-3.2) -- (0,3.5) node[above]{$y$};

  % constants
  \foreach \c in {-2,-1,0,1,2} {
    \pgfmathsetmacro\xmin{-\c}
    \pgfmathsetmacro\xmax{3}

    % upper branch
    \draw[thick,black,domain=\xmin:\xmax] plot (\x,{\x+\c});
    % lower branch
    \draw[thick,black,domain=\xmin:\xmax] plot (\x,{-(\x+\c)});

    % labels
    \node[black] at (\xmax+0.8,{\xmax+\c}) {$c=\c$};
  }

  % highlight solution branch y=1-x for x>=1
  \draw[thick,orange,domain=1:3,samples=50] plot (\x,{1-\x});

  % slope arrows
  \foreach \c in {-2,-1,0,1,2} {
    \pgfmathsetmacro\xmin{-\c}
    \pgfmathsetmacro\xmax{3}
    % upper arrows
    \foreach \x in {\xmin,...,\xmax} {
      \pgfmathsetmacro\y{\x+\c}
      \pgfmathsetmacro\dx{0.3}
      \pgfmathsetmacro\dy{0.3*\c}
      \draw[->,black,thick] ({\x-0.5*\dx},{\y-0.5*\dy}) -- ({\x+0.5*\dx},{\y+0.5*\dy});
    }
    % lower arrows
    \foreach \x in {\xmin,...,\xmax} {
      \pgfmathsetmacro\y{-(\x+\c)}
      \pgfmathsetmacro\dx{0.3}
      \pgfmathsetmacro\dy{0.3*\c}
      \draw[->,black,thick] ({\x-0.5*\dx},{\y-0.5*\dy}) -- ({\x+0.5*\dx},{\y+0.5*\dy});
    }
  }

  % highlight arrows on solution
  \foreach \x in {1,1.5,2,2.5,3} {
    \pgfmathsetmacro\y{1-\x}
    \draw[->,orange,thick] ({\x-0.15},{\y+0.15}) -- ({\x+0.15},{\y-0.15});
  }

  % initial condition point
  \filldraw[red] (2,-1) circle (2pt);

\end{tikzpicture}
\caption{המשך פתרון סעיף ג׳: הפתרון $y(x)=1-x$ תקף עבור $x \geq 1$ ומודגש בכתום.  
נקודת ההתחלה $(2,-1)$ של תנאי ההתחלה $y(2)=-1$ מסומנת בעיגול אדום.  
בשאר התחומים השיפוע נקבע לפי רצועות בין איזוקלינות סמוכות.}
\label{fig:dirfield_abs_continuation}
\end{figure}

ד.
נראה את שדה הכיוונים המלא לבעיה ובפרט את הפתרון שעובר בתנאי ההתחלה:
\begin{figure}[H]
    \centering
    \includegraphics[width=0.7\textwidth]{null_exp.png}
    \caption{שדה הכיוונים של המשוואה 
    \(\; y' = |y| - x \;\) 
    יחד עם איזוקלינות \(f(x,y)=c\) עבור \(c=-2,-1,0,1,2\) (אפור), 
    נולקלינות \(y=\pm x\) (מקווקווים אדומים), ופתרונות מספריים שונים. 
    הפתרון הפרטי עבור \(y(0)=1\) מודגש בכתום ומשמש כקו גבול: פתרונות מעליו (באדום) מראים מקסימום מקומי ומתבדרים, 
    ואילו פתרונות מתחתיו (בצהוב) חוצים את ציר ה־\(x\) ומתכנסים בין האיזוקלינות. 
    נקודת התנאי ההתחלתי \((0,1)\) סומנה גם כן.}
    \label{fig:dirfield_abs_d_matlab}
\end{figure}
נסכם כמה תכונות שניתן לראות מהגרף:
\begin{itemize}
  \item כל פתרון שנכנס לתחום שבין שתי האיזוקלינות $c=0,2$ רוצה ׳׳לברוח׳׳ מהתחום, ורק פתרון אחד נשאר בפנים.  

  \item כל הפתרונות בסביבה מתקרבים אל הפתרון שמתאים ל־$c=-1$.  

  \item לכל פתרון שנמצא מעל לפתרון שמתאים ל־$c=-1$, יש מקסימום מקומי, ויש נקודת פיתול.  

  \item כל פתרון שנכנס לתחום שבין שתי האיזוקלינות המתאימות ל־$c=0,-2$, נשאר בתחום.  

  \item כל הפתרונות שמשמאל לנולקלינה אשר מתאימה ל־$c=1$, ׳׳מתפוצצים׳׳ ולמעשה בורחים מהפתרון שמתאים ל־$c=1$.
\end{itemize}

כעת נוסיף את הפתרונות האנליטיים של המשוואה \(
y' = |y| - x.
\) (ללא דרך, אתם יכולים לעשות זאת לבד שכן המד׳׳ר לינארית. ניתן לפתור למשל באמצעות שיטת גורם האינטגרציה בשני חלקים: $y>0$, $y<0$).
הפתרון הכללי מתפצל לשני ענפים:  

\begin{itemize}
  \item כאשר $y>0$ נקבל:
  \[
  y(x) = c_1 e^{x} + x + 1,
  \]
  שהוא פתרון בעל רכיב מעריכי \emph{מתפוצץ}, כפי שניתן לראות גם בשדה הכיוונים (החצים נוטים להתרחק כלפי מעלה).
  
  \item כאשר $y<0$ נקבל:
  \[
  y(x) = c_1 e^{-x} - x + 1,
  \]
  שהוא פתרון בעל רכיב מעריכי \emph{שדועך}, כפי שהשתקף בשדה הכיוונים – הפתרונות יורדים מטה (בגלל ה-$-x$).
\end{itemize}

באופן זה ניתן לראות התאמה מלאה בין ניתוח שדה הכיוונים לבין ההתנהגות של הפתרונות האנליטיים: מצד אחד קיים פתרון מעריכי מתבדר, ומצד שני פתרון מעריכי מתכנס.

%%%CUT%%%

\newpage
\underline{תרגילים}
\exercise{}

א. ציירו שדה כיוונים למד״ר:
\[
y' = y(1-y).
\]
השתמשו בתנאי ההתחלה \(
y(0)=\tfrac{1}{2}.
\), והוסיפו נולקלינות ופתרונות אפשריים, כולל את הפתרון הפרטי שמתאים לתנאי ההתחלה של הבעיה.

ב. מצאו פתרון פרטי למשוואה בשיטה אנליטית.


\newpage
\underline{פתרונות}
\solution{}

א. על כל איזוקלינה $f(x,y)=c$ מתקיים $y'=c$, כלומר הפתרון שחותך את האיזוקלינה עושה זאת בשיפוע קבוע $c$.  
בין האיזוקלינות ניתן להבחין:  
- אם $0<y<1$, אז $y'>0$ – הפתרונות עולים.  
- אם $y>1$, אז $y'<0$ – הפתרונות יורדים.  
- אם $y<0$, אז $y'<0$ – הפתרונות יורדים עוד יותר.

\[
\begin{cases}
y' = y(1-y), \\
y(0) = \tfrac{1}{2}.
\end{cases}
\]
כיוון שבתחום $0<y<1$ השיפוע חיובי, הפתרון יעלה עם $x$.  
בגרף שנציג מיד ניתן יהיה לראות כי הפתרון מתכנס לאט־לאט לנולקלינה $y=1$.

 קל לראות מהן הנולקלינות: $y=0$ ו־$y=1$.  
נוכל כעת לשרטט מספר פתרונות:  
- אם $y(0)>1$, הפתרון יורד כלפי מטה ומתכנס ל־$y=1$.  
- אם $0<y(0)<1$, הפתרון עולה כלפי מעלה ומתכנס ל־$y=1$.  
- אם $y(0)<0$, הפתרון יורד אל עבר אינסוף שלילי. לכן הפתרון הפרטי שצריך לעבור בתנאי ההתחלה, ׳׳כלוא׳׳ בין שתי האיזוקלינות.

לאחר ניתוח זה, נראה את הסרטוט:
\begin{figure}[H]
\centering
\includegraphics[width=0.7\textwidth]{null_y_a.png}
\caption{שדה הכיוונים של $y'=y(1-y)$ עם נולקלינות $y=0,1$.  
הפתרון עבור $y(0)=\tfrac{1}{2}$ (בכתום) מתכנס אל $y=1$.}
\label{fig:dirfield_logistic}
\end{figure}

ב. נפתור את המשוואה באמצעות שיטת הפרדת משתנים:
 
נפתור $g(y)=0$: 
\[
y(1-y)=0 \;\;\Longrightarrow\;\; y=0 \quad\text{או}\quad y=1.
\]  
ולכן הפתרונות החשודים להיות סינגולריים הם:
\[
\boxed{y(x)\equiv 0}, 
\qquad 
\boxed{y(x)\equiv 1}.
\]
 
נניח $y\neq 0,1$, ואז נפריד משתנים:
\[
\frac{dy}{y(1-y)} = dx.
\]
 
נפרק לשברים חלקיים:
\[
\frac{1}{y(1-y)} = \frac{1}{y} + \frac{1}{1-y}.
\]  
ולכן:
\[
\int \left(\frac{1}{y} + \frac{1}{1-y}\right) dy = \int dx.
\]  
מקבלים:
\[
\ln\!\left|\frac{y}{1-y}\right| = x + C.
\]
 
נעבור לאקספוננט ונבודד את $y$:  
\[
\frac{y}{1-y} = Ae^x, \qquad A=\pm e^C.
\]  
ולכן:
\[
\boxed{y(x) = \frac{1}{1+Ae^{-x}}}.
\]

\textbf{בדיקת תנאי התחלה:}  
עבור $y(0)=\tfrac{1}{2}$ נקבל $A=1$, ולכן:
\[
\boxed{y(x)=\frac{1}{1+e^{-x}}}.
\]

\textbf{סיכום:}  
- $y=0$ ו־$y=1$ הם פתרונות סינגולריים (קבועים).  
- הפתרון הכללי מתכנס ל־$y=1$ כאשר $x\to\infty$ (פתרון יציב).  
- עבור $x\to -\infty$, הפתרון מתכנס ל־$y=0$.  


\begin{insight}
ניתן לראות מהשדה שהפתרון עבור $y(0)=1/2$ מתכנס ל־$y=1$ באופן מעריכי.  
מצד שני, אם מתחילים מתחת ל־$y=0$, הפתרון ״בורח״ למינוס אינסוף.  
כלומר, $y=1$ הוא מצב יציב ואילו $y=0$ הוא מצב לא יציב מלמטה, אך יציב מלמעלה.
\end{insight}

\subsubsection{יציבות של נקודות שיווי־משקל - מקרה פרטי עבור מד׳׳ר אוטונומיות סדר 1}

נבחן מד״ר אוטונומית מסדר ראשון:
\[
y' = g(y).
\]

\textbf{נקודת שיווי־משקל} $y_0$ מתקבלת כאשר $g(y_0)=0$, כלומר $y(x)\equiv y_0$ הוא פתרון קבוע.

נאמר כי פתרון קבוע $y(x)\equiv y_0$ הוא:
\begin{itemize}
  \item \textbf {יציב} אם לכל $\varepsilon>0$ קיים $\delta>0$ כך שאם $|y(0)-y_0|<\delta$ אז לכל $x>0$ מתקיים 
  \[
  |y(x)-y_0|<\varepsilon.
  \]

  \item \textbf{אסימפטוטית יציב} אם בנוסף מתקיים
  \[
  \lim_{x\to\infty} y(x) = y_0.
  \]

  \item \textbf{לא יציב} אם אף אחת מההגדרות הנ״ל לא מתקיימת.
\end{itemize}

כדי לבדוק את \textbf{היציבות} של נקודת שיווי־מהשקל נוכל להשתמש גם במבחן הנגזרת. נבחן את $g'(y_0)$:

\begin{itemize}
  \item אם $g'(y_0) < 0$ – אזי סטייה קטנה מ-$y_0$ תגרום לכך שהנגזרת תהיה בעלת סימן שמחזיר את $y$ חזרה לנקודה.  
  כלומר, $y_0$ הוא \textbf{יציב} (לרוב יציב אסימפטוטית).
  
  \item אם $g'(y_0) > 0$ – סטייה קטנה מ-$y_0$ תגרום לכך שהנגזרת דוחפת את $y$ הלאה מהנקודה.  
  כלומר, $y_0$ הוא \textbf{לא יציב}.
  
  \item אם $g'(y_0)=0$ – המבחן אינו חד־משמעי, ויש צורך בבחינה מעמיקה יותר (למשל באמצעות איברי סדר גבוה).
\end{itemize}

\vspace{0.5cm}

\begin{insight}
הכלל הבסיסי הוא:
\[
\boxed{
\begin{cases}
g'(y_0) < 0 \;\;\Rightarrow\;\;\text{ יציב}\, y_0 , \\[6pt]
g'(y_0) > 0 \;\;\Rightarrow\;\;\text{ לא יציב}\, y_0 .
\end{cases}
}
\]

באופן אינטואיטיבי – נגזרת שלילית מבטיחה ש״חץ השיפוע״ תמיד מצביע חזרה אל הנקודה (מושך פתרונות שכנים),  
בעוד נגזרת חיובית מצביעה החוצה ודוחה פתרונות קרובים.
\end{insight}



\textbf{בחינת יציבות הפתרונות}

נבחן כעת את יציבות הפתרונות הסינגולריים של המשוואה
\[
y' = \sqrt{1-y^2}.
\]

באופן כללי, עבור מד״ר אוטונומית מהצורה
\[
y' = g(y),
\]
נאמר כי פתרון קבוע $y=y_0$ הוא:
\begin{itemize}
  \item \textbf{יציב} אם לכל $\varepsilon>0$ קיים $\delta>0$ כך שאם $|y(0)-y_0|<\delta$ אז לכל $x>0$ מתקיים $|y(x)-y_0|<\varepsilon$.
  \item \textbf{אסימפטוטית יציב} אם בנוסף $y(x)\to y_0$ כאשר $x\to\infty$.
  \item \textbf{לא יציב} אם אינו מקיים את ההגדרות הנ״ל.
\end{itemize}

דרך מעשית לבדוק יציבות:  
חישוב הנגזרת $g'(y_0)$.  
\begin{itemize}
  \item אם $g'(y_0)<0$ $\;\Rightarrow\;$ $y_0$ אסימפטוטית יציב.  
  \item אם $g'(y_0)>0$ $\;\Rightarrow\;$ $y_0$ לא יציב.  
  \item אם $g'(y_0)=0$ $\;\leftarrow\;$ יש לבדוק בדרכים נוספות (למשל פיתוח טיילור גבוה יותר).
\end{itemize}

\textbf{
נחזור לדוגמא \ref{special_sine}}:
\[
y'=\sqrt{1-y^2}=g(y).
\]

הפתרונות הסינגולריים היו $y=\pm 1$ הנגזרת:
\[
g'(y) = \frac{-y}{\sqrt{1-y^2}}.
\]

נבחן כל מקרה:
\begin{itemize}
  \item עבור $y=1$: הביטוי מתבדר ל־$-\infty$, מה שמעיד שהפתרון הקבוע $y\equiv 1$ יציב מאוד (דוחה כל ניסיון לחרוג כלפי מעלה, ומושך פתרונות קרובים מלמטה).
  \item עבור $y=-1$: הביטוי מתבדר ל־$+\infty$, כלומר $y\equiv -1$ אינו יציב (הפתרון נדחה כלפי מעלה).
\end{itemize}
נראה זאת על גבי הסרטוט:
\begin{figure}[H]
    \centering
    \includegraphics[width=0.7\textwidth]{existence_uniqueness_sin_null.png}
    \caption{שדה הכיוונים של המשוואה 
    \lr{$y'=\sqrt{1-y^2}$}, יחד עם הנולקלינות $y=\pm1$, 
    הפתרון התפור שעובר דרך $(0,0)$ (בכחול), 
    והפתרונות הסינגולריים $y\equiv -1$ (בירוק) ו־$y\equiv 1$ (באדום).}
    \label{fig:existence_uniqueness_sin_null}
\end{figure}

נסכם כמה נקודות: 
\begin{itemize}
  \item $y\equiv 1$ הוא פתרון יציב (אך לא אסימפטוטית יציב לכל הכיוונים, שכן הפתרון לא יכול להתקרב אליו מלמעלה אלא רק מלמטה).  
  \item $y\equiv -1$ הוא פתרון לא יציב.  
\end{itemize}


\newpage
\section{משוואות לינאריות מסדר $n\geq2$}

באופן כללי, משוואה דיפרנציאלית מסדר $n$ ניתנת לכתיבה בצורה:
\begin{equation}
F\big(x, y, y', y'', \dots, y^{(n)}\big) = 0,
\end{equation}
כאשר $y^{(k)}$ מציין את הנגזרת ה־$k$ של $y$ ביחס ל־$x$.

בפרט נתרכז במקרה של משוואות \textbf{לינאריות}, כלומר משוואות בהן $y(x)$ וכל נגזרותיה מופיעות באופן לינארי:
\begin{equation}
a_n(x)y^{(n)} + a_{n-1}(x)y^{(n-1)} + \dots + a_1(x)y' + a_0(x)y = g(x).
\end{equation}

לפני שנתקדם לעקרונות חשובים למד׳׳רים מסדר גבוה, נראה כמה דוגמאות פשוטות שידענו לפתור עד היום, גם ללא חשיפה לטכניקות מד׳׳ר שונות. הדבר יהווה מעין קדימון לעקרון הסופרפוזיציה, הורדת סדר ומשפט אבל (Abel)
.

\example\label{2nd_examp}
מצאו את הפתרון הכללי של המשוואה הדיפרנציאלית מהסדר השני:
\[
y'' = 6x - 2.
\]

\explanation
נבצע אינטגרציה פעם ראשונה:
\[
y' = \int (6x-2)\,dx = 3x^2 - 2x + C_1.
\]

אינטגרציה נוספת:
\[
y = \int (3x^2 - 2x + C_1)\,dx = x^3 - x^2 + C_1x + C_2.
\]

תשובה סופית:
\[
\boxed{y(x) = x^3 - x^2 + C_1 x + C_2, \quad x \in \mathbb{R}}.
\]

\example
מצאו את הפתרון הכללי של המשוואה הדיפרנציאלית מהסדר השלישי:
\[
y''' = \sin(2x).
\]

\explanation
נבצע אינטגרציה ראשונה:
\[
y'' = \int \sin(2x)\,dx = -\frac{\cos(2x)}{2} + C_1.
\]

נבצע אינטגרציה שנייה:
\[
y' = \int \left(-\frac{\cos(2x)}{2} + C_1\right)\,dx
= -\frac{\sin(2x)}{4} + C_1x + C_2.
\]

נבצע אינטגרציה שלישית:
\[
y = \int \left(-\frac{\sin(2x)}{4} + C_1x + C_2\right)\,dx
= \frac{\cos(2x)}{8} + \frac{C_1 x^2}{2} + C_2 x + C_3.
\]

תשובה סופית:
\[
\boxed{y(x) = \frac{\cos(2x)}{8} + \frac{C_1 x^2}{2} + C_2 x + C_3,
\quad x \in \mathbb{R}}.
\]

\example
 בהמשך למשוואה
\[
y'' = 6x - 2,
\]
מצאו פתרון פרטי
כאשר נתונים תנאי ההתחלה:
\[
y'(2)=1, \quad y(1)=2.
\]

\example
בהמשך לתרגיל 3: מצאו את הפתרון הפרטי של המשוואה
\[
y'' = 6x - 2,
\]
כאשר נתונים תנאי ההתחלה:
\[
y'(2)=1, \quad y(1)=2.
\]

\explanation
מדוגמא \ref{2nd_examp} מצאנו כי הפתרון הכללי הוא:
\[
y(x) = x^3 - x^2 + C_1 x + C_2.
\]

כעת נחשב את הנגזרת:
\[
y'(x) = 3x^2 - 2x + C_1.
\]

נציב בתנאי $y'(2)=1$:
\[
1 = 3\cdot (2)^2 - 2\cdot 2 + C_1
= 12 - 4 + C_1
= 8 + C_1
\;\;\Longrightarrow\;\;
C_1 = -7.
\]

נחזיר לפתרון הכללי:
\[
y(x) = x^3 - x^2 - 7x + C_2.
\]

כעת נציב בתנאי $y(1)=2$:
\[
2 = (1)^3 - (1)^2 - 7\cdot 1 + C_2
= 1 - 1 - 7 + C_2
= -7 + C_2.
\]

ולכן:
\[
C_2 = 9.
\]

תשובה סופית:
\[
\boxed{y(x) = x^3 - x^2 - 7x + 9, \quad x \in \mathbb{R}}.
\]

\example
מצאו את הפתרון הכללי של המשוואה הדיפרנציאלית:
\[
y'' - \frac{y'}{x} = x^2, \quad x \neq 0.
\]

\explanation
נבצע הצבה: נגדיר $y'=z$, ולכן $y''=z'$.

המשוואה הופכת ל:
\[
z' - \frac{z}{x} = x^2.
\]

זוהי משוואה ליניארית מהסדר הראשון עבור $z$.  
גורם אינטגרציה:
\[
\mu(x) = e^{\int -\tfrac{1}{x}\,dx} = e^{-\ln x} = \frac{1}{x}.
\]

נכפיל במשוואה ונקבל:
\[
\left(\frac{z}{x}\right)' = x.
\]

אינטגרציה:
\[
\frac{z}{x} = \int x\,dx + C_1 = \frac{x^2}{2} + C_1.
\]

נכפיל ב- $x$:
\[
z = y' = \frac{x^3}{2} + C_1 x.
\]

אינטגרציה אחרונה:
\[
y = \int \left(\frac{x^3}{2} + C_1 x\right)\,dx
= \frac{x^4}{8} + \frac{C_1 x^2}{2} + C_2.
\]

תשובה סופית:
\[
\boxed{y(x) = \frac{x^4}{8} + \frac{C_1 x^2}{2} + C_2, \quad x \neq 0}.
\]

%%%CUT%%%

\newpage
\subsection{עקרון הסופרפוזיציה}
תכונה מרכזית של משוואות לינאריות היא עקרון \textbf{הסופרפוזיציה}. נפתח במקרה ספציפי עבור $n=2$. 
אם $y_1(x)$ ו־$y_2(x)$ הם פתרונות של אותה המשוואה ההומוגנית (כלומר כאשר $g(x)\equiv 0$), אזי גם
\[
c_1 y_1(x) + c_2 y_2(x)
\]
הוא פתרון לכל $c_1,c_2 \in \mathbb{R}$.  
באופן כללי, צירוף לינארי של פתרונות הוא גם פתרון. נוכיח זאת כעת.

\begin{proof}
נחזור למשוואה הלינארית ההומוגנית מסדר שני:
\begin{equation}\label{lin2}
a_2(x)y'' + a_1(x)y' + a_0(x)y = 0,
\end{equation}
כאשר $a_2(x)\neq 0$ לכל $x$ בתחום.

נניח כי $y_1(x)$ הוא פתרון של \eqref{lin2}.  
במילים אחרות, מתקיים:
\[
a_2(x)y_1'' + a_1(x)y_1' + a_0(x)y_1 = 0.
\]

באופן דומה, אם $y_2(x)$ הוא פתרון, אזי:
\[
a_2(x)y_2'' + a_1(x)y_2' + a_0(x)y_2 = 0.
\]

כעת נבחן את הצירוף הלינארי:
\[
y(x) := c_1 y_1(x) + c_2 y_2(x).
\]

נגזור לפי $x$:
\[
y'(x) = c_1 y_1'(x) + c_2 y_2'(x), 
\qquad
y''(x) = c_1 y_1''(x) + c_2 y_2''(x).
\]

נציב במשוואה \eqref{lin2}:
\[
a_2(x)y'' + a_1(x)y' + a_0(x)y 
= a_2(x)\big(c_1 y_1'' + c_2 y_2''\big) 
+ a_1(x)\big(c_1 y_1' + c_2 y_2'\big) 
+ a_0(x)\big(c_1 y_1 + c_2 y_2\big).
\]

נכנס איברים דומים:
\[
= c_1 \big(a_2 y_1'' + a_1 y_1' + a_0 y_1\big) 
+ c_2 \big(a_2 y_2'' + a_1 y_2' + a_0 y_2\big).
\]

מאחר ש־$y_1,y_2$ הם פתרונות של \eqref{lin2}, כל סוגריים מתאפסים, ולכן:
\[
a_2(x)y'' + a_1(x)y' + a_0(x)y = 0.
\]

מכאן נובע כי $y(x)=c_1y_1(x)+c_2y_2(x)$ הוא גם פתרון.  

באופן כללי יותר, עבור כל משוואה לינארית הומוגנית מסדר $n$, צירוף לינארי של פתרונותיה הוא גם פתרון.  
\end{proof}

\subsubsection{מבנה הפתרון הכללי של משוואה לא הומוגנית}

נטען כעת כי הפתרון הכללי של כל מד׳׳ר לינארית לא הומוגנית הוא סופרפוזיציה (חיבור) של פתרון פרטי עם הפתרון הכללי של המשוואה ההומוגנית:
\begin{equation}
y(x) = y_p(x) + y_h(x),
\end{equation}
כאשר $y_p(x)$ הוא פתרון פרטי אחד של המשוואה הלא הומוגנית, ו־$y_h(x)$ הוא הפתרון הכללי של המשוואה ההומוגנית.

\begin{proof}
נחזור למשוואה הלינארית הלא הומוגנית מסדר $n$:
\begin{equation}\label{eq:nonhom}
a_n(x)y^{(n)} + a_{n-1}(x)y^{(n-1)} + \dots + a_1(x)y' + a_0(x)y = g(x),
\end{equation}
כאשר $a_n(x)\neq 0$ לכל $x$ בתחום.
 
נניח כי $y_p(x)$ פותר את \eqref{eq:nonhom}:
\[
a_n(x)y_p^{(n)} + a_{n-1}(x)y_p^{(n-1)} + \dots + a_0(x)y_p = g(x),
\]
וכי $y_h(x)$ פותר את המשוואה ההומוגנית המתאימה:
\begin{equation}\label{eq:hom}
a_n(x)y_h^{(n)} + a_{n-1}(x)y_h^{(n-1)} + \dots + a_0(x)y_h = 0.
\end{equation}

נגדיר $y(x) := y_p(x) + y_h(x)$.  
מאחר שהמשוואה לינארית, הצבה נותנת:
\[
a_n(x)y^{(n)} + \dots + a_0(x)y
= \big(a_n(x)y_p^{(n)} + \dots + a_0(x)y_p\big)
+ \big(a_n(x)y_h^{(n)} + \dots + a_0(x)y_h\big).
\]

האיבר הראשון שווה ל־$g(x)$, האיבר השני שווה ל־$0$, ולכן:
\[
a_n(x)y^{(n)} + \dots + a_0(x)y = g(x).
\]

כלומר $y(x)$ הוא פתרון של \eqref{eq:nonhom}.  
\end{proof}

\newpage
\subsection{בניית פתרון הומוגני מהפרש פתרונות פרטיים}

נניח כי $y_1(x)$ ו־$y_2(x)$ הם שני פתרונות שונים של המשוואה \emph{הלא הומוגנית}:
\[
a_n(x)y^{(n)} + a_{n-1}(x)y^{(n-1)} + \dots + a_0(x)y = g(x).
\]
כל קבוע כפול ההפרש של שני פתרונות פרטיים למשוואה, נותן פתרון למשוואה ההומוגנית המתאימה.

נבחן את הפונקציה:
\[
z(x) := c \cdot \big(y_1(x) - y_2(x)\big), \qquad c \in \mathbb{R}.
\]

\begin{proof}
נציב $y_1$ במשוואה:
\[
a_n(x)y_1^{(n)} + a_{n-1}(x)y_1^{(n-1)} + \dots + a_0(x)y_1 = g(x).
\]

באופן דומה עבור $y_2$:
\[
a_n(x)y_2^{(n)} + a_{n-1}(x)y_2^{(n-1)} + \dots + a_0(x)y_2 = g(x).
\]

נחסר אגף־אגף:
\[
a_n(x)\big(y_1^{(n)} - y_2^{(n)}\big) + a_{n-1}(x)\big(y_1^{(n-1)} - y_2^{(n-1)}\big) + \dots + a_0(x)(y_1-y_2) = g(x)-g(x).
\]

כלומר:
\[
a_n(x)(y_1-y_2)^{(n)} + a_{n-1}(x)(y_1-y_2)^{(n-1)} + \dots + a_0(x)(y_1-y_2) = 0.
\]
נוכל לכפול את המשוואה בכל קבוע ממשי שונה מאפס שנרצה ולכן, $z(x)=c\,(y_1(x)-y_2(x))$ הוא פתרון של המשוואה \textbf{ההומוגנית}.
\end{proof}

\newpage
\subsection{הורונסקיאן \lr{Wronskian}}

נראה את המשוואה הליניארית ההומוגנית מסדר $n$ (מנורמלת):
\begin{equation}\label{lin_n_norm}
y^{(n)} + p_{n-1}(x)y^{(n-1)} + \cdots + p_1(x)y' + p_0(x)y = 0.
\end{equation}

נגדיר מדד למערכות של פונקציות:  
ה\textbf{הורונסקיאן} של פונקציות $\{y_1,\dots,y_n\}$ מוגדר כך:
\begin{equation}
W(y_1,\dots,y_n)(x) =
\begin{vmatrix}
y_1 & \cdots & y_n \\
y_1' & \cdots & y_n' \\
\vdots & \ddots & \vdots \\
y_1^{(n-1)} & \cdots & y_n^{(n-1)}
\end{vmatrix}.
\end{equation}
פונקציות אלו מהוות כבסיס מסדר $n$ למערכת הפתרונות ה\textbf{יסודית} של המד׳׳ר \ref{lin_n_norm}.

\textbf{הורונסקיאן עבור סדרים 2 ו-3:}
\begin{align*}
n=2: \quad 
&W(y_1,y_2)(x) = y_1 y_2' - y_1' y_2. \\[6pt]
n=3: \quad 
&W(y_1,y_2,y_3)(x) =
y_1 \big(y_2' y_3'' - y_2'' y_3'\big)
- y_1' \big(y_2 y_3'' - y_2'' y_3\big)
+ y_1'' \big(y_2 y_3' - y_2' y_3\big).
\end{align*}

הורונסקיאן משמש כקריטריון לתלות לינארית של פתרונות שמרכיבים את מערכת הפתרונות היסודית למשוואה הדיפרנציאלית. כמה עקרונות:
\begin{itemize}
  \item אם $W(x_0)\neq 0$ בנקודה אחת $x_0$, אזי $\{y_1,\dots,y_n\}$ בלתי־תלויה ליניארית על כל התחום.
  \item אם $W(x)\equiv 0$ לכל $x$, הפונקציות תלויות ליניארית.
\end{itemize}

\textbf{נוסחת אבל (Abel)}  

נבחן משוואה ליניארית מסדר $n$, מנורמלת:
\begin{equation}
y^{(n)} + p_{n-1}(x) y^{(n-1)} + p_{n-2}(x) y^{(n-2)} + \cdots + p_0(x) y = 0.
\end{equation}

נניח כי $\{y_1,\dots,y_n\}$ היא מערכת פתרונות. נרצה לדעת כיצד $W(x)$ מתנהג.  

\textbf{שלב 1 – גזירת $W$:}  
קיים משפט חשוב (שאפשר להוכיח על ידי גזירה של הדטרמיננטה, לא נראה זאת כאן) הקובע כי
\begin{equation}
W'(x) = -p_{n-1}(x)\,W(x).
\end{equation}

כלומר, הורונסקיאן מקיים משוואה דיפרנציאלית ליניארית פשוטה מסדר ראשון.

\textbf{שלב 2 – פתרון המשוואה:}  
נכתוב:
\[
\frac{W'(x)}{W(x)} = -p_{n-1}(x).
\]

נבצע אינטגרציה:
\[
\ln|W(x)| = -\int p_{n-1}(x)\,dx + C.
\]

נעלה בחזקה:
\begin{equation}
\boxed{W(x) = C \cdot e^{-\int p_{n-1}(x)\,dx}}.
\end{equation}

\textbf{מסקנה:}  
הורונסקיאן של מערכת פתרונות איננו פונקציה שרירותית אלא בעל צורה סגורה.
הורונסקיאן בעצם אומר לנו האם הפונקציות השונות שמוצעות כפתרונות הן בלתי תלויות לינארית (בת''ל) או לא, כלומר האם הצירוף הלינארי שלהן יכול להוות בסיס למרחב הפתרונות לבעיה.

כמה נקודות: 
\begin{itemize}
  \item ההורונסקיאן לא תלוי במקדמים $p_0(x),\dots,p_{n-2}(x)$ – אלא אלא אך ורק במקדם $y^{(n-1)}$.  
  \item אם $C\neq 0$ אז $W(x)$ לעולם לא מתאפס – כלומר הפתרונות בלתי־תלויים ליניארית ומהווים בסיס למרחב הפתרונות.  
  \item אם $C=0$ אז $W(x)\equiv 0$ – כלומר הפתרונות תלויים ליניארית.  
\end{itemize}

\example
הוכיחו כי הפונקציה $x^2$ אינה יכולה להיות פתרון של אף משוואה מנורמלת מהצורה
\[
y'' + p_1(x)y' + p_2(x)y = 0
\]
עם מקדמים $p_1(x),p_2(x)$ רציפים בסביבת $x=0$.

\explanation
נניח כי למשוואה יש שני פתרונות בת״ל $y_1(x)=x^2,\, y_2(x)$ בקטע סביב $x=0$.  
אם מקדמי המשוואה רציפים אז וורונסקיאן של שני הפתרונות אינו מתאפס שם.  

אבל:
\[
W(y_1,y_2) =
\begin{vmatrix}
x^2 & y_2 \\
2x & y_2'
\end{vmatrix}.
\]

הדטרמיננטה מתאפסת עבור $x=0$.  
מכאן כי הפתרונות אינם בת״ל, ו־$y_1,y_2$ לא יכולות להוות בסיס למרחב הפתרונות של הבעיה.


\example
הוכיחו כי אם $y_1(x),y_2(x)$ הם שני פתרונות של משוואה הומוגנית מסדר שני
\[
y'' + p_1(x)y' + p_2(x)y = 0
\]
אשר מתאפסים באותה נקודה, אז הם כפולה קבועה זה של זה.

\explanation
נניח למשל כי $y_1(x),y_2(x)$ מתאפסים ב־$\alpha\in\mathbb{R}$, כלומר $y_1(\alpha)=y_2(\alpha)=0$.  

אז:
\[
W(y_1(\alpha),y_2(\alpha)) =
\begin{vmatrix}
y_1(\alpha) & y_2(\alpha) \\
y_1'(\alpha) & y_2'(\alpha)
\end{vmatrix}
=
\begin{vmatrix}
0 & 0 \\
y_1'(\alpha) & y_2'(\alpha)
\end{vmatrix}
=0.
\]

על כן $y_1(x),y_2(x)$ תלויים לינארית בקבועים $c_1,c_2$, לא שניהם אפס, כך שמתקיים
\[
c_1 y_1(x) + c_2 y_2(x) = 0.
\]

ולכן:
\[
y_2(x) = -\frac{c_1}{c_2}\,y_1(x).
\]

%%%CUT%%%

\newpage
\underline{תרגילים}
\exercise
בדקו האם הפונקציות $y_1(x)=e^{2x},\; y_2(x)=e^{-2x}$ הן פתרונות בת״ל של משוואה דיפרנציאלית מסדר שני כלשהי.  

\exercise
האם הפונקציות $y_1(x)=\cos^2(x),\; y_2(x)=1+\cos(2x)$ הן פתרונות בת״ל?

\exercise
נתון כי $y_1(x)=x^2$ הוא פתרון של המשוואה
\[
x^2 y'' - 3x y' + 4y = 0, \qquad x>0.
\]
כמו כן ידוע כי הוורונסקיאן של כל שני פתרונות בת״ל במערכת זו מקיים
\[
W(x) = C x^3, \qquad C \neq 0.
\]
מצאו פתרון שני בלתי־תלוי לינארית $y_2(x)$ בעזרת הנתון על הוורונסקיאן, מבלי לפתור את המשוואה הדיפרנציאלית.

\exercise
נתונה המשוואה
\[
y'' + \frac{2}{x}\,y' - \frac{2}{x^2}\,y = 0, \qquad x>0.
\]
הראו כי הוורונסקיאן של כל שני פתרונות $y_1,y_2$ הוא מהצורה $W(x)=\tfrac{C}{x^2}$.  


\newpage
\underline{פתרונות}

\solution
נחשב את הוורונסקיאן:
\[
W(y_1,y_2)=
\begin{vmatrix}
e^{2x} & e^{-2x} \\
2e^{2x} & -2e^{-2x}
\end{vmatrix}.
\]

נפתח:
\[
W(y_1,y_2) = -2 - 2 = -4 \neq 0.
\]

מכאן כי $y_1,y_2$ בת״ל ולכן אכן יכולות להוות בסיס למרחב הפתרונות של משוואה דיפרנציאלית מסדר שני (למשל: $y''-4y=0$).


\solution
ננסח את $y_2$ מחדש באמצעות זהות:
\[
y_2(x) = 1+\cos(2x) = 2\cos^2(x).
\]

כלומר:
\[
y_2(x) = 2\,y_1(x).
\]

משום ש־$y_2$ היא כפולה קבועה של $y_1$, הפונקציות תלויות לינארית.  
לכן הן אינן מתאימות כבסיס למרחב פתרונות.


\solution
לפי ההגדרה לסדר 2:
\[
W(x) = y_1 y_2' - y_1' y_2.
\]

נציב $y_1(x)=x^2, \; y_1'(x)=2x$:
\[
W(x) = x^2 y_2' - 2x y_2.
\]

מן הנתון:
\[
x^2 y_2' - 2x y_2 = C x^3.
\]

נחלק ב־$x^2$ ($x>0$):
\[
y_2' - \frac{2}{x}y_2 = Cx.
\]

זוהי משוואה ליניארית מדרגה ראשונה עבור $y_2$.  
נחשב גורם אינטגרציה:
\[
\mu(x) = \exp\!\Big(-\int \tfrac{2}{x}\,dx\Big) = \exp(-2\ln |x|) = \frac{1}{x^2}.
\]

נכפיל את המד׳׳ר בגורם זה ונזהה נגזרת של מכפלה בצד שמאל של המשוואה:
\[
\Big(\tfrac{y_2}{x^2}\Big)' = \frac{C}{x}.
\]

נבצע אינטגרציה:
\[
\frac{y_2}{x^2} = C \ln x + K.
\]

ולכן:
\[
y_2(x) = C x^2 \ln x + K x^2.
\]

האיבר $Kx^2$ הוא כפולה של $y_1(x)$ ולכן אינו נותן פתרון בלתי תלוי.  
לכן הפתרון השני האמיתי הוא:
\[
\boxed{y_2(x) = x^2 \ln x}.
\]

\solution
נשתמש במשפט אבל.
אם המשוואה היא
\[
y'' + p_1(x) y' + p_2(x) y = 0,
\]
אז וורונסקיאן מקיים
\[
W(x) = C \cdot \exp\Big(-\int p_1(x)\,dx\Big).
\]

במשוואתנו $p_1(x)=\tfrac{2}{x}$.  
לכן:
\[
W(x) = C \cdot \exp\left(-\int \frac{2}{x}\,dx\right)
= C \cdot \exp(-2\ln |x|).
\]

נפשט:
\[
W(x) = C \cdot x^{-2}.
\]

קיבלנו:
\[
W(x) = \frac{C}{x^2}.
\]

מכאן שכל זוג פתרונות בת״ל יקיים וורונסקיאן מהצורה הנ״ל.

\newpage
\subsection{ נוסחת Abel}

צורה כללית למד׳׳ר סדר 2, לינארית, הומוגנית, מנורמלת:
\begin{equation}
y'' + p(x)y' + q(x)y = 0.
\end{equation}

נניח כי נתון פתרון אחד $y_1(x)$ של המד׳׳ר ההומוגנית.
 אזי יש נוסחה סגורה לפתרון השני:
\begin{equation}
\boxed{
y_2(x) = C \cdot y_1(x) \cdot \int \frac{e^{-\int p(x)dx}}{y_1^2(x)}\, dx.
}
\end{equation}
\begin{remark}
    שימו לב כי על מנת להשתמש בנוסחה הסגורה, חייב לנרמל את המד׳׳ר! עבור סדר 2, הנגזרת מסדר 2 היא הקובעת, עבורה המקדם צריך להיות 1.
\end{remark}

\begin{proof}
נניח כי $y_1(x)$ הוא פתרון נתון של המשוואה
\[
y'' + p(x)y' + q(x)y = 0.
\]

נחפש פתרון נוסף מהצורה
\begin{equation}
y_2(x) = y_1(x)\,v(x).
\end{equation}

נגזור:
\[
y_2'(x) = y_1'(x)v(x) + y_1(x)v'(x),
\]
\[
y_2''(x) = y_1''(x)v(x) + 2y_1'(x)v'(x) + y_1(x)v''(x).
\]

נציב במשוואה:
\[
y_2'' + p(x)y_2' + q(x)y_2 = 0.
\]

נקבל:
\[
\big(y_1''+p(x)y_1'+q(x)y_1\big)v(x) + \big(2y_1'(x)+p(x)y_1(x)\big)v'(x) + y_1(x)v''(x) = 0.
\]

מכיוון ש-$y_1$ הוא פתרון, האיבר הראשון מתאפס:
\[
y_1(x)v''(x) + \big(2y_1'(x)+p(x)y_1(x)\big)v'(x) = 0.
\]

נסמן $u(x)=v'(x)$, ואז מתקבלת משוואה מסדר ראשון:
\[
y_1(x)u'(x) + \big(2y_1'(x)+p(x)y_1(x)\big)u(x) = 0.
\]

נחלק ב-$y_1(x)$. נוכל לעשות זאת שכן $y_1(x)\neq0$:
\[
u'(x) + \left(2\frac{y_1'(x)}{y_1(x)}+p(x)\right)u(x) = 0.
\]
  
נזהה כי מדובר במשוואה לינארית מסדר ראשון מהצורה הכללית:
\[
u'(x) + P(x)\,u(x) = 0.
\]
נחשב גורם אינטגרציה:
\[
\mu(x) = e^{\int P(x)\,dx}
= e^{\int \left(2\frac{y_1'(x)}{y_1(x)}+p(x)\right)dx}.
\]

נפרק לשני אינטגרלים:
\[
\mu(x) = e^{\int 2\frac{y_1'(x)}{y_1(x)}dx} \cdot e^{\int p(x)\,dx}.
\]

האיבר הראשון נותן:
\[
e^{\int 2\frac{y_1'(x)}{y_1(x)}dx} = e^{2\ln|y_1(x)|} = y_1^2(x).
\]

לכן:
\[
\mu(x) = y_1^2(x)\,e^{\int p(x)\,dx}.
\]

כעת, הפתרון הכללי של משוואה לינארית מסדר ראשון הוא:
\[
u(x) = \frac{1}{\mu(x)}\left(\int 0 \cdot \mu(x)\,dx + C\right) 
= \frac{C}{\mu(x)}.
\]

נציב את הביטוי עבור $\mu(x)$:
\[
u(x) = C \cdot \frac{e^{-\int p(x)\,dx}}{y_1^2(x)}.
\]

מאחר ש-$u=v'$, נקבל בהמשך:
\[
v(x) = C \int \frac{e^{-\int p(x)\,dx}}{y_1^2(x)}\,dx,
\]

ולכן:
\[
y_2(x) = y_1(x)\,v(x) = 
C\cdot y_1(x)\int \frac{e^{-\int p(x)\,dx}}{y_1^2(x)}\,dx.
\]

זהו בדיוק הביטוי הידוע כ-\textbf{נוסחת Abel}.
\end{proof}

\example{}

נתון כי $y_1(x)=x$ פותר את
\[
x^2 y'' + x y' - y = 0 \quad ; \quad x>0
\]

מצאו פתרון כללי למד״ר.

\explanation{}

ראשית ננרמל:
\[
y'' + \frac{1}{x} y' - \frac{y}{x^2} = 0
\]

זוהי מד״ר הומוגנית, ועל כן נוכל להשתמש בנוסחת אבל:
\[
y_2(x) = \mathcolor{red}{C} \cdot y_1(x) \cdot \int \frac{e^{-\int p(x)\,dx}}{y_1^2(x)} dx
= \mathcolor{red}{C} \cdot x \cdot \int \frac{e^{-\int \tfrac{1}{x} dx}}{x^2} dx
\]

\[
y_2(x) = \mathcolor{red}{C} \cdot x \int \frac{e^{-\ln|x|}}{x^2}\,dx
= \mathcolor{red}{C} \cdot x \int \frac{1}{x^3}\,dx
= \mathcolor{red}{C} \cdot x \cdot \frac{x^{-2}}{-2}
= -\frac{C}{2x}
\]

נבחר קבוע כרצוננו, במקרה זה נבחר את הערך $-2$
:
\[
\quad \; \; \; \; \; \; \textcolor{red}{C=-2 \;\;\Longrightarrow\;\; \frac{1}{x}}
\]
\begin{itemize}
  \item \textcolor{red}{לא חייבים את הקבוע $C$}
  \item \textcolor{red}{הפתרון $y_{2}(x)=\frac{1}{x}$ הוא פתרון מייצג}
\end{itemize}

הסיבה לשני הדברים לעיל היא שכל כפולה בקבוע של פתרון למשוואה ההומוגנית היא גם כן פתרון ולכן ניתן לקבוע שרירותית את הערך של הקבוע (לפני שהוא מוכפל בקבוע נוסף).  

לכן, הפתרון הכללי הוא הסופרפוזיציה של שני הפתרונות אשר מהווים מערכת פתרונות יסודית לבעיה:
\[
\boxed{y(x) = C_1 x + C_2 \cdot \tfrac{1}{x} \quad ; \quad x>0}
\]

%%%CUT%%%

\newpage
\underline{תרגילים}
\exercise{}
נתונה המשוואה הדיפרנציאלית
\[
y'' - \frac{2}{x}y' + \frac{2}{x^2}y = 0, \qquad x>0.
\]
ידוע כי $y_1(x)=x$ הוא פתרון. מצאו את הפתרון הכללי לבעיה.

\exercise{}
נבחן את המשוואה הדיפרנציאלית
\[
y'' + \tan(x)\,y' = 0, \qquad -\tfrac{\pi}{2}<x<\tfrac{\pi}{2}.
\]

\begin{enumerate}
  \item[א.] הראו כי $y_1(x)=\sin(x)$ הוא פתרון של המשוואה.
  \item[ב.] מצאו פתרון שני בת׳׳ל לינארית $y_2(x)$, וכתבו את הפתרון הכללי. הציגו שתי דרכי פתרון שונות: האחת באמצעות ההצבה $z=y'$ והשנייה באמצעות נוסחת אבל.
\end{enumerate}

\exercise{}

קבלו פתרון כללי למד״ר הבאה:
\[
y'' + 2\tan(x)y' - y = 0.
\]


\newpage
\underline{פתרונות}

\solution{}
נכתוב את המשוואה בצורה מנורמלת:
\[
y'' + p(x)y' + q(x)y = 0, \qquad p(x) = -\frac{2}{x}.
\]

לפי נוסחת אבל:
\[
y_2(x) = y_1(x) \cdot \int \frac{e^{-\int p(x)\,dx}}{y_1^2(x)} dx.
\]

נציב:
\[
y_2(x) = x \cdot \int \frac{e^{-\int -\tfrac{2}{x}\,dx}}{x^2}\,dx
= x \cdot \int \frac{e^{2\ln x}}{x^2}\,dx.
\]

מאחר ש-$e^{2\ln x} = x^2$:
\[
y_2(x) = x \cdot \int \frac{x^2}{x^2}\,dx = x \int 1\,dx = x \cdot x = x^2.
\]
כעת נוודא כי $y_1,y_2$ בלתי־תלויים ליניארית בעזרת הוורונסקיאן:
\[
W(y_1,y_2)(x) =
\begin{vmatrix}
y_1 & y_2 \\
y_1' & y_2'
\end{vmatrix}
=
\begin{vmatrix}
x & x^2 \\
1 & 2x
\end{vmatrix}.
\]

נחשב:
\[
W(x) = x\cdot (2x) - (1)(x^2) = 2x^2 - x^2 = x^2.
\]

מאחר ש-$W(x) \neq 0$ לכל $x>0$, אכן $y_1,y_2$ בלתי תלויים ליניארית.

לכן הפתרון הכללי הוא סופרפוזיציה של שני הפתרונות:
\[
\boxed{y(x) = C_1 x + C_2 x^2, \qquad x>0}.
\]


\solution{}
א.
נחשב נגזרות:
\[
y_1'=\cos(x), \qquad y_1''=-\sin(x).
\]

נציב במשוואה:
\[
y_1''+\tan(x)y_1'=-\sin(x)+\tan(x)\cos(x).
\]

מאחר ש־$\tan(x)\cos(x)=\sin(x)$, נקבל:
\[
-\sin(x)+\sin(x)=0.
\]

לכן $y_1(x)=\sin(x)$ הוא אכן פתרון.

ב.
 נתחיל עם הצבת $z=y'$.
נגדיר $z=y'$, ולכן $z'=y''$. נקבל:
\[
z'+\tan(x)z=0.
\]

זוהי מד׳׳ר לינארית מסדר ראשון. גורם אינטגרציה:
\[
\mu(x)=e^{\int \tan(x)\,dx}=e^{-\ln|\cos(x)|}=\frac{1}{\cos(x)}.
\]

נכפיל:
\[
\Big(\tfrac{z}{\cos(x)}\Big)'=0 \;\;\Longrightarrow\;\; \frac{z}{\cos(x)}=C_1.
\]

לכן:
\[
z=y'=C_1\cos(x).
\]

אינטגרציה:
\[
y=\int C_1\cos(x)\,dx=C_1\sin(x)+C_2.
\]
קיבלנו פתרון כללי הכולל את $y_1(x)=\sin(x)$ ופתרון נוסף קבוע מהצורה $y(x)=C_2$.  
חשוב להבין ש-$y_1(x)=\sin(x)$ כבר כלול בפתרון, ועל כן קיבלנו באִבְחָה אחת את הפתרון הכללי המכיל שני אברים בת״ל.  
זאת למעשה פרישת מרחב הפתרונות של בעיה זו.

נבדוק את אי־התלות הליניארית בעזרת הוורונסקיאן:  
\[
W(y_1,y_2)(x) =
\begin{vmatrix}
\sin(x) & 1 \\
\cos(x) & 0
\end{vmatrix}
= \sin(x)\cdot 0 - 1\cdot \cos(x) = -\cos(x).
\]

מאחר ש-$W(x)\neq 0$ בתחום $(-\tfrac{\pi}{2},\tfrac{\pi}{2})$, הפתרונות $\sin(x)$ ו־כל קבוע הם אכן בלתי־תלויים ליניארית. שימו לב שבחישוב הורונסיקיאן השמטנו את הקבועים משני הפתרונות, אך הדבר לא משנה. התוצאה תמיד תהיה זהה.

כעת נעבור לנוסחת אבל ונוודא שנקבל את אותו הדבר. 

המשוואה מנורמלת:
\[
y''+p(x)y'+q(x)y=0, \qquad p(x)=\tan(x), \; q(x)=0.
\]

לפי משפט אבל:
\[
W(x)=C\exp\!\Big(-\int \tan(x)\,dx\Big)=C\cos(x).
\]

בעזרת נוסחת אבל לפתרון שני:
\[
y_2(x)=y_1(x)\int \frac{W(x)}{y_1^2(x)}\,dx.
\]

נציב $y_1(x)=\sin(x)$:
\[
y_2(x)=\sin(x)\int \frac{\cos(x)}{\sin^2(x)}\,dx.
\]

הצבה $u=\sin(x),\,du=\cos(x)\,dx$:
\[
\int \frac{du}{u^2}=-\frac{1}{u}=-\frac{1}{\sin(x)}.
\]

ולכן:
\[
y_2(x)=\sin(x)\cdot\Big(-\frac{1}{\sin(x)}\Big)=-1.
\]

כל פונקציה קבועה $y(x)=C$ מהווה פתרון בלתי־תלוי נוסף.

על כן, מרחב הפתרונות של הבעיה הינו:
\[
\boxed{y(x)=C_1\sin(x)+C_2, \qquad -\tfrac{\pi}{2}<x<\tfrac{\pi}{2}}
\]

\solution{}

שימו לב ל׳׳יחודיות׳׳ של התרגיל. המד׳׳ר היא מסדר 2, לינארית, הומוגנית ומנורמלת. למעשה, מבלי שסיפקו לנו פתרון אחד לפחות, אין לנו כלי אמיתי על מנת לטפל בבעיה זו. על כן, אין לנו ברירה, אלא לנחש. \textbf{איך מנחשים ׳׳טוב׳׳?} נתבונן במד׳׳ר ונבין כי היא מכילה בסך הכל מקדם אחד שהוא לא קבוע, הלא הוא $\tan(x)$. מדובר בפונקציה טריגונומטרית, שכופלת במקרה זה את הנגזרת הראשונה של הפתרון המיוחל. במד׳׳ר קיימת גם הנגזרת השנייה של הפונקציה, והפונקציה עצמה. כיוון ש-$\tan(x)=\frac{\sin(x)}{\cos(x)}$, $\sin(x)$ ו-$\cos(x)$ מועמדות פוטנציאליות לפתרון משוואה זו.

נבדוק תחילה את הפתרון המשוער \(y_1(x) = \cos(x)\):

\[
y_1'(x) = -\sin(x), \qquad y_1''(x) = -\cos(x).
\]

נציב במד״ר:
\[
y'' + 2\tan(x)y' - y = (-\cos x) + 2\tan(x)(-\sin x) - \cos x 
= -2\cos(x) - 2\sin(x)\tan(x).
\]

נפשט את האיבר השני:
\[
-2\sin(x)\tan(x) = -2\sin(x)\frac{\sin(x)}{\cos(x)} = -2\frac{\sin^2(x)}{\cos(x)}.
\]

ולכן (לאחר מכנה משותף):
\[
y'' + 2\tan(x)y' - y = -2\frac{\cos^2(x) + \sin^2(x)}{\cos(x)} = -\frac{2}{\cos(x)} \neq 0.
\]

מסקנה:  
\(\text{ איננה פתרון למשוואה.}{y_1(x) = \cos(x)} 
\)

עתה נבדוק את \(y_2(x) = \sin(x)\):

\[
y_1'(x) = \cos(x), \qquad y_1''(x) = -\sin(x).
\]

נציב שוב:
\[
y'' + 2\tan(x)y' - y = (-\sin x) + 2\tan(x)\cos(x) - \sin(x)
= -2\sin(x) + 2\sin(x) = 0.
\]

ולכן:
\(\text{ הוא פתרון נכון של המד״ר.}{y_1(x) = \sin(x)} 
\)

כעת, נוכל לקבל את הפתרון השני לפי נוסחת אבל:
\[
y_2(x) = C \cdot y_1(x) \int \frac{e^{-\int p(x)\,dx}}{y_1^2(x)}\,dx.
\]

נציב \(p(x) = 2\tan(x)\) ו-\(y_1(x)=\sin(x)\):
\[
y_2(x) = C \cdot \sin(x) \int \frac{e^{-\int 2\tan(x)\,dx}}{\sin^2(x)}\,dx.
\]

נחשב את האינטגרל הפנימי:
\[
\int 2\tan(x)\,dx = -2\ln|\cos(x)|,
\]
ולכן:
\[
e^{-\int 2\tan(x)\,dx} = e^{2\ln|\cos(x)|} = \cos^2(x).
\]

נחזור להצבה:
\[
y_2(x) = C \cdot \sin(x) \int \frac{\cos^2(x)}{\sin^2(x)}\,dx 
= C \cdot \sin(x) \int \bigg[\frac{1}{\sin^2(x)} - 1\bigg] dx.
\]

נחשב כל חלק בנפרד:
\[
\int \frac{1}{\sin^2(x)}dx = -\cot(x), \quad \int 1\,dx = x.
\]
ולכן:
\[
y_2(x) = C \cdot \sin(x)\,[ -\cot(x) - x ] = C(\cos(x) + x\sin(x)).
\]

ולכן הפתרון הכללי הוא:
\[
\boxed{y(x) = c_1\sin(x) + c_2\big(\cos(x) + x\sin(x)\big)}, \qquad x \neq \frac{\pi}{2} \pm \pi k, \; (k = \pm1, \pm2, \dots).
\]

\newpage
\subsection{ שיטת הורדת סדר}

נתבונן במד״ר לינארית, סדר 2, מנורמלת, לא הומוגנית, בצורתה הכללית:
\begin{equation}
y'' + p(x)y' + q(x)y = g(x),
\end{equation}
כאשר $p(x),q(x),g(x)$ הן פונקציות רציפות על תחום נתון.

במקרה שהמשוואה הומוגנית ($g(x)=0$) ונתון לנו פתרון אחד $y_1(x)$ של המשוואה, נוכל למצוא פתרון נוסף בלתי תלוי ליניארית $y_2(x)$ באמצעות \textbf{שיטת אבל}, כפי שעשינו עד כה. אך אם $g(x)\neq0$, דהיינו המד׳׳ר לא הומוגנית, נפנה לשיטת \textbf{הורדת סדר}. חשוב להבין כי ניתן לבצע הורדת סדר גם אם המד׳׳ר הומוגנית ולקבל אותו פתרון כמו באבל, אך הדבר מיותר, שכן אבל הוא מקרה פרטי של הורדת סדר.

\textbf{הרעיון:}
נניח כי הפתרון השני הוא מהצורה
\begin{equation}
y(x) = v(x)\cdot y_1(x),
\end{equation}
כאשר $v(x)$ פונקציה חדשה לא ידועה. הצבה זו מורידה את הסדר של הבעיה ממד״ר מסדר שני במישור $x,y$ (למשל) למד״ר מסדר ראשון עבור $v'(x)$.

\example
נבחן את המשוואה
\[
x^2 y'' - 3x y' + 4y = 0, \qquad x>0.
\]
ידוע כי $y_1(x)=x^2$ הוא פתרון. מצאו פתרון כללי בעזרת הורדת סדר.

\explanation
נניח $y(x) = v(x)\cdot y_1(x)=v(x)\,x^2$.  
נגזור:
\[
y' = v'x^2 + 2vx, \qquad y'' = v''x^2 + 4v'x + 2v.
\]

נציב במשוואה:
\[
x^2(v''x^2 + 4v'x + 2v) - 3x(v'x^2+2vx) + 4(vx^2)=0.
\]

נפשט:
\[
x^4 v'' + (4x^3-3x^3)v' + (2x^2-6x^2+4x^2)v= x^4 v'' + x^3 v'=0.
\]

כלומר:
\[
(x v')'=0 \;\;\Longrightarrow\;\; v'=\frac{C_1}{x}, \quad v=C_1\ln x + C_2.
\]

ולכן:
\[
y(x)=(C_1\ln x + C_2)x^2.
\]

נפריד פתרונות:
\[
y_1(x)=x^2, \qquad y_2(x)=x^2\ln x.
\]
למעשה, יכולנו להשתמש בנוסחת אבל, שכן המד׳׳ר הומוגנית. נראה זאת:
\[
y_2(x) = y_1(x)\cdot \int \frac{e^{-\int p(x)\,dx}}{y_1^2(x)}\,dx,
\]
כאשר $p(x)=-\tfrac{3}{x}$.

נחשב:
\[
\int p(x)\,dx = \int -\frac{3}{x}\,dx = -3\ln x,
\qquad e^{-\int p(x)\,dx} = e^{3\ln x} = x^3.
\]

נציב:
\[
y_2(x) = x^2 \cdot \int \frac{x^3}{(x^2)^2}\,dx
= x^2 \cdot \int \frac{x^3}{x^4}\,dx
= x^2 \cdot \int \frac{1}{x}\,dx=x^2 \ln x.
\]

ולכן הפתרון הכללי הוא
\[
\boxed{y(x) = c_{1}x^2+c_{2}x^{2} \ln x,\qquad x>0}.
\]
הגענו לאותו הפתרון כמצופה. תוכלו לוודא בעצמכם שהפתרונות אכן בת׳׳ל.

%%%CUT%%%

\newpage
\underline{תרגילים}
\exercise{}
מצאו את הפתרון הכללי של המשוואה הדיפרנציאלית מהסדר השני:
\[
x^2 y'' + x y' - y = x^3, \qquad x>0,
\]
בהינתן כי $y_1(x)=x$ הוא פתרון של ההומוגנית המתאימה.

\exercise{}
מצאו את הפתרון הכללי של המשוואה
\[
x y'' - (2x+1)y' + (x+1)y = x^2,
\]
בהינתן כי $y_1(x)=e^x$ הוא פתרון של המשוואה ההומוגנית המתאימה.

\exercise{}

קבלו פתרון כללי למשוואה:
\[
2x^2 y'' + x y' - 3y = 0, \qquad x>0,
\]
בהינתן כי \( y_1(x) = x^{-1} \) הוא פתרון ידוע.

\exercise{}
נתון כי \( y_1(x) = x \) פותר את המד״ר:
\[
x^3y''' - 3x^2y'' + 6xy' - 6y = 0, \qquad x>0.
\]
מצאו פתרון כללי למד״ר.


\newpage
\underline{פתרונות}

\solution{}
שימו לב שמבין כל השיטות שנלמדו בספר עד כה, הורדת סדר היא היחידה שיכולה לתת לנו מענה בשלב זה, שכן מדובר במד׳׳ר מסדר 2, לא הומוגנית. נכתוב:
\[
y(x) = v(x)\,y_1(x) = v(x)\,x.
\]

נגזור:
\[
y' = v'x + v, \qquad y'' = v''x + 2v'.
\]

נציב במשוואה:
\[
x^2(v''x+2v') + x(v'x+v) - (vx) = x^3.
\]

פישוט נותן:
\[
x^3 v'' + 3x^2 v' = x^3.
\]

נסמן $z=v'$, ונקבל משוואה ליניארית מסדר ראשון, וננרמל:
\[
x^3 z' + 3x^2 z = x^3 
\;\;\Longrightarrow\;\; 
z' + \tfrac{3}{x}z = 1.
\]

נחשב את גורם האינטגרציה:
\[
\mu(x) = e^{\int \tfrac{3}{x}\,dx} = e^{3\ln x} = x^3.
\]

נכפול במשוואה:
\[
(x^3 z)' = x^3.
\]

אינטגרציה:
\[
x^3 z = \tfrac{x^4}{4} + C_1 
\;\;\Longrightarrow\;\; 
z = v' = \tfrac{x}{4} + C_1 x^{-3}.
\]

נבצע אינטגרציה נוספת:
\[
v(x) = \int \left(\tfrac{x}{4} + C_1 x^{-3}\right)dx
= \tfrac{x^2}{8} - \tfrac{C_1}{2x^2} + C_2.
\]

נחזיר ל-$y$:
\[
y(x) = v(x)\cdot x 
= \tfrac{x^3}{8} - \tfrac{C_1}{2x} + C_2 x.
\]

נפריד בין הפתרון הפרטי להומוגני:
\[
y_p(x)=\tfrac{x^3}{8}, 
\qquad
y_h(x)=C_1\cdot\tfrac{1}{x} + C_2 x.
\]

ולכן הפתרון הכללי הוא:
\[
\boxed{y(x) = C_1 \cdot \tfrac{1}{x} + C_2 x + \tfrac{x^3}{8}, \qquad x>0.}
\]


\solution{}
נניח $y(x) = u(x)\,e^x$.  
נקבל:
\[
y' = u'e^x + ue^x, \qquad 
y'' = u''e^x + 2u'e^x + ue^x.
\]

נציב במשוואה:
\[
x(u''e^x + 2u'e^x + ue^x) - (2x+1)(u'e^x+ue^x) + (x+1)ue^x = x^2.
\]

לאחר פישוט:
\[
(xu'' - u')e^x = x^2.
\]

נחלק ב-$e^x$ ונקבל:
\[
xu'' - u' = x^2 e^{-x}.
\]

נסמן $z=u'$. נקבל:
\[
xz' - z = x^2 e^{-x} \;\;\;\Longrightarrow\;\;\; z' - \frac{1}{x}z = x e^{-x}.
\]

זוהי משוואה ליניארית מסדר ראשון.  
גורם האינטגרציה הוא:
\[
\mu(x) = \exp\!\left(\int -\frac{1}{x}\,dx\right) = e^{-\ln x} = \frac{1}{x}.
\]

נכפיל במשוואה ונקבל:
\[
\frac{1}{x}z' - \frac{1}{x^2}z = e^{-x}.
\]

אבל אגף שמאל שקול לנגזרת:
\[
\left(\frac{z}{x}\right)' = e^{-x}.
\]

נבצע אינטגרציה:
\[
\frac{z}{x} = \int e^{-x}\,dx = -e^{-x} + C_1.
\]

ולכן:
\[
z = u' = x(-e^{-x}+C_1) = -x e^{-x} + C_1 x.
\]

נבצע אינטגרציה נוספת:
\[
u(x) = \int (-x e^{-x} + C_1 x)\,dx.
\]

נחשב כל חלק:
\[
\int -x e^{-x}\,dx = (x+1)e^{-x}, 
\qquad \int C_1 x\,dx = \tfrac{C_1}{2}x^2.
\]

לכן:
\[
u(x) = (x+1)e^{-x} + \tfrac{C_1}{2}x^2 + C_2.
\]

נחזור ל-$y$:
\[
y(x) = u(x)e^x = (x+1) + \tfrac{C_1}{2}x^2 e^x + C_2 e^x.
\]

\textbf{הפתרון הכללי יהיה:}
\[
\boxed{
y(x) = x+1 + c_1 e^x + c_2 x^2 e^x,\qquad x\in\mathbb{R}},\]
כאשר $c_1,c_2$ הם קבועים שרירותיים.


\solution{}

נשתמש בשיטת \textbf{הורדת סדר}.  
נניח כי:
\[
y(x) = v(x)\,y_1(x) = v(x)\,x^{-1}.
\]

נחשב נגזרות:
\[
y' = v'x^{-1} - v x^{-2}, \qquad
y'' = v''x^{-1} - 2v' x^{-2} + 2v x^{-3}.
\]

נציב במשוואה:
\[
2x^2\big(v''x^{-1} - 2v'x^{-2} + 2v x^{-3}\big)
+ x\big(v'x^{-1} - v x^{-2}\big)
- 3(v x^{-1}) = 0.
\]

נפשט:
\[
2x v'' - 4v' + 4v x^{-1} + v' - v x^{-1} - 3v x^{-1} = 0.
\]

איחוד איברים:
\[
2x v'' - 3v' = 0.
\]

נחלק ב-\(x\) (כי \(x>0\)):
\[
2v'' - \frac{3}{x}v' = 0.
\]

נסמן \(z = v'\).  
נקבל:
\[
2z' - \frac{3}{x}z = 0.
\]

זוהי משוואה ליניארית מסדר ראשון.  
נכתוב בצורה מנורמלת:
\[
z' - \frac{3}{2x}z = 0.
\]

נשתמש בשיטת גורם האינטגרציה:
\[
\mu(x) = e^{-\int \frac{3}{2x}dx} = e^{-\frac{3}{2}\ln x} = x^{-\tfrac{3}{2}}.
\]

נכפיל את המד׳׳ר ונקבל:
\[
(\mu z)' = 0 \;\;\Longrightarrow\;\; x^{-\tfrac{3}{2}}z = C_1.
\]

ולכן:
\[
z = v' = C_1 x^{\tfrac{3}{2}}.
\]

נבצע אינטגרציה:
\[
v = \int C_1 x^{\tfrac{3}{2}}\,dx = C_1 \frac{2}{5}x^{\tfrac{5}{2}} + C_2.
\]

נחזור ל-\(y = v x^{-1}\):
\[
y(x) = \left(C_1 \frac{2}{5}x^{\tfrac{5}{2}} + C_2\right)x^{-1} 
= \frac{2}{5}C_1 x^{\tfrac{3}{2}} + C_2 x^{-1}.
\]

נפשט את הקבועים ונקבל את הפתרון הכללי:
\[
\boxed{y(x) = C_1 x^{\tfrac{3}{2}} + C_2 x^{-1}, \qquad x>0.}
\]


\solution{}
נשים לב כי המד״ר מסדר 3, אך נתון רק פתרון אחד. לפיכך, לכאורה, אין ביכולתנו למצוא פתרון. מכל הכלים שהצגנו עד כה, עומדות בפנינו שתי אופציות: 1. ניחוש פתרון. 2. שיטת \textbf{הורדת סדר}. כיוון שהמד׳׳ר מסדר 3, האופציה הראשונה במקרה זה יכולה להיות מאתגרת ו׳׳צורכת׳׳ זמן, שכן בהנחה ומדובר בפתרון פולינומיאלי, מספר האפשרויות רב מידי. בנוגע לאופציה השנייה, לאחר הורדת סדר נקבל מד״ר מסדר שני, דבר שלא מבטיח לנו פתרון כיוון שאיו לנו פתרון ידוע שני. בהמשך, תראו שגישה קלאסית למד׳׳ר כזו בהינתן פתרונות ידועים מראש, היא פתרון בעזרת טורים. אך בשלב זה, תקוותנו היחידה היא כי לאחר שנוריד סדר, אולי נקבל מד׳׳ר מסדר שני שבה המקדם של אחת הנגזרות יתאפס, ובכך נוכל לבצע הורדת סדר נוספת (!).
נניח כי:
\[
y = x v(x).
\]
נגזור:
\[
y' = v + x v', \qquad
y'' = 2v' + x v'', \qquad
y''' = 3v'' + x v'''.
\]

נציב במשוואה ונקבל:
\[
x^3(3v'' + x v''') - 3x^2(2v' + x v'') + 6x(v + x v') - 6xv = 0, \quad x>0.
\]

נפשט ונאחד איברים:
\[
x^4 v''' = 0.
\]

נחלק ב-$x^{4}$ ונפתור עבור \(v\):
\[
v''' = 0 \;\;\Longrightarrow\;\; v'' = C_1 \;\;\Longrightarrow\;\; v' = C_1x + C_2 \;\;\Longrightarrow\;\; v(x) = C_1x^2 + C_2x + C_3.
\]

נחזור ל-\(y = x v(x)\):
\[
y(x) = C_1x^3 + C_2x^2 + C_3x, \qquad x>0.
\]

\[
\boxed{y(x) = C_1x^3 + C_2x^2 + C_3x, \quad x>0.}
\]

%%%CUT%%%

\newpage
\subsection{מד״ר מסדר גבוה – מד״ר עם מקדמים קבועים}

\subsubsection{משוואות הומוגניות}

נבחן מד״ר לינארית, הומוגנית, מסדר $n$:
\begin{equation}\label{fixed}
a_n y^{(n)} + a_{n-1}y^{(n-1)} + \dots + a_1 y' + a_0 y = 0,
\end{equation}
כאשר כל המקדמים $a_i$ הם \textbf{קבועים}.

נגדיר את הפולינום האופייני:
\begin{equation}
L(r) = a_n r^n + a_{n-1} r^{n-1} + \dots + a_1 r + a_0 = 0.
\end{equation}

שורשי הפולינום האופייני יקבעו את \textbf{צורת הפתרון הכללי} של המד״ר.
נתאים $n$ שורשים של הפ׳׳א, ל-$n$ פתרונות של המשוואה.
\textbf{האפשרויות השונות:}

1. אם $r_0$ הוא \textbf{שורש ממשי של הפולינום האופייני מריבוי אלגברי $k$}, כלומר $(r - r_0)^k$ מחלק את הפ״א ללא שארית, או באופן שקול: $L(r_{0})=L'(r_{0})=L''(r_{0})=\dots L^{(k-1)}(r_{0})=0,\qquad L^{(k)}(r_{0})\neq 0$
,
אז מתקבלות $k$ פונקציות פתרון ליניאריות בלתי תלויות:

\[
e^{r_0 x}, \quad x e^{r_0 x}, \quad x^2 e^{r_0 x}, \quad \dots, \quad x^{k-1} e^{r_0 x}.
\]

אלו יהיו הפתרונות של המד״ר.

2. אם $r_0 = a + ib$ הוא \textbf{שורש מרוכב מריבוי אלגברי $k$}, אז גם $ \bar{r}_0 = a - ib$ שורש מריבוי אלגברי $k$ (אחת ממסקנות המשפט היסודי של האלגברה).
מן השורשים המרוכבים מתקבלות $2k$ פונקציות פתרון ממשיות, אשר מתאימות ל-$n$ שורשים של הפ׳׳א:

\[
e^{ax}\cos(bx), \quad e^{ax}\sin(bx), \quad x e^{ax}\cos(bx), \quad\, x e^{ax}\sin(bx),\; \dots, \quad x^{k-1} e^{ax}\cos(bx), \quad x^{k-1} e^{ax}\sin(bx).
\]

נוכיח כעת מדוע פתרונות אקספוננציאליים, הם המועמדים הטבעיים לפתרונות עבור מד׳׳רים עם מקדמים קבועים.

\begin{proof}
\textbf{
נתחיל מהמקרה הפשוט ביותר, בו מדובר בשורש ממשי מריבוי אלגברי 1}.
נניח כי הפונקציה $y(x)=e^{rx}$ פותרת את משוואה \ref{fixed}.
נחשב את הנגזרות של הפתרון המוצע:
\[
y' = r e^{rx}, \quad y'' = r^2 e^{rx}, \quad \dots, \quad y^{(n)} = r^n e^{rx}.
\]

נציב ביטויים אלה במשוואה \ref{fixed} ונקבל:
\[
a_n r^n e^{rx} + a_{n-1} r^{n-1} e^{rx} + \dots + a_1 r e^{rx} + a_0 e^{rx} = 0.
\]

נוציא גורם משותף $e^{rx}$ (שאינו אפס) ונקבל:
\[
\big(a_n r^n + a_{n-1} r^{n-1} + \dots + a_1 r + a_0 \big) e^{rx} = 0.
\]

מכאן נובע כי הפתרון $y=e^{rx}$ קיים רק אם המקדם של $e^{rx}$ מתאפס, כלומר:
\[
L(r) = a_n r^n + a_{n-1} r^{n-1} + \dots + a_1 r + a_0 = 0.
\]

כלומר, כל שורש $r_i$ של הפולינום האופייני $L(r)$ מגדיר פתרון מהצורה:
\[
y_i(x) = e^{r_i x}.
\]
\textbf{
נוכיח גם את המקרה בו יש ריבוי אלגברי גדול מ-1 של שורשים מסוימים}.
אם $r_0$ הוא שורש מריבוי אלגברי $k$, אז בנוסף ל-$e^{r_0 x}$ גם הפונקציות
\[
x e^{r_0 x}, \quad x^2 e^{r_0 x}, \quad \dots, \quad x^{k-1} e^{r_0 x}
\]
פותרות את אותה משוואה.
אנו יודעים כי $r_0$ הוא שורש מריבוי אלגברי $k>1$, כלומר:
\[
L(r_0) = L'(r_0) = \dots = L^{(k-1)}(r_0) = 0, \qquad L^{(k)}(r_0) \neq 0.
\]

ידוע כבר כי אם $r_0$ הוא שורש של $L(r)$, אז הפונקציה $y=e^{r_0 x}$ פותרת את המד״ר, ולכן:
\begin{equation}
L(D)e^{r_0 x} = L(r_0)e^{r_0 x} = 0,
\end{equation}
כאשר $D$ מסמן את האופרטור הדיפרנציאלי $D=\tfrac{d}{dx}$.

נוכיח כעת כי גם הפונקציות $x e^{r_0 x},\, x^2 e^{r_0 x},\, \dots,\, x^{k-1} e^{r_0 x}$ הן פתרונות.  
נבחן את האופרטור $L(D)$ כפונקציה של $r$.  
מאחר שהאופרטור ליניארי, נוכל לכתוב:
\[
L(D)e^{rx} = L(r)e^{rx}.
\]

נבצע גזירה לפי הפרמטר $r$ משני האגפים:
\[
\frac{\partial}{\partial r}\big[L(D)e^{rx}\big]
= \frac{\partial}{\partial r}\big[L(r)e^{rx}\big].
\]

מאחר ש-$L(D)$ אינו תלוי בפרמטר $r$, נקבל:
\[
L(D)\frac{\partial}{\partial r}e^{rx} = L'(r)e^{rx} + L(r)\,x e^{rx}.
\]

נציב כעת $r=r_0$.  
כיוון ש-$L(r_0)=0$ ו-$L'(r_0)=0$ (בגלל ריבוי השורש), נקבל:
\[
L(D)\left.\frac{\partial e^{rx}}{\partial r}\right|_{r=r_0} = 0.
\]

אבל $\frac{\partial e^{rx}}{\partial r} = x e^{rx}$, ולכן נובע כי גם $x e^{r_0 x}$ פותר את המשוואה.
נוכל לחזור על התהליך באופן רקורסיבי.  
נגזור שוב לפי $r$ ונקבל:
\[
\frac{\partial^m}{\partial r^m} e^{rx} = x^m e^{rx}.
\]
מאחר שכל אחת מהנגזרות הללו ׳׳מוחקת׳׳ עוד תנאי של ריבוי בשורש $r_0$, נקבל שכל הפונקציות:
\[
e^{r_0 x}, \quad x e^{r_0 x}, \quad x^2 e^{r_0 x}, \; \dots, \; x^{k-1} e^{r_0 x}
\]
מקיימות $L(D)y=0$.

באופן אינטואיטיבי, כל נגזרת נוספת לפי $r$ מוסיפה גורם של $x$ לפונקציה ומאפשרת לקבל פתרונות נוספים המתאימים לאותו שורש מריבוי גבוה.  
מאחר שהאופרטור $L(D)$ ליניארי, כל הצירוף הליניארי של פונקציות אלו הוא גם פתרון תקף.

לכן, עבור שורש מריבוי אלגברי $k$ מתקבלת משפחת פתרונות בלתי־תלויים:
\[
\boxed{e^{r_0 x},\; x e^{r_0 x},\; x^2 e^{r_0 x},\; \dots,\; x^{k-1} e^{r_0 x}.}
\]

\textbf{
נוכיח גם את המקרה בו ישנם שורשים מרוכבים לפ׳׳א}.  
אם $r_{1,2}=a\pm ib$, אז שני הפתרונות המרוכבים
\[
e^{(a+ib)x}, \quad e^{(a-ib)x}
\]
ניתנים לצירוף ממשי בעזרת נוסחת אוילר $e^{i\theta}=\cos\theta+i\sin\theta$, ומתקבלים הפתרונות הממשיים:
\[
e^{ax}\cos(bx), \qquad e^{ax}\sin(bx).
\]
שימו לב כי קיים אלמנט לא ממשי $i$ שמתקבל מנוסחת אוילר. עם זאת, 
המשוואה הדיפרנציאלית היא בעלת מקדמים ממשיים בלבד, ולכן אם $e^{(a+ib)x}$ הוא פתרון, גם הצמוד המרוכב שלו $e^{(a-ib)x}$ הוא פתרון נוסף.
נראה כיצד נוכל לבנות משני הפתרונות המרוכבים הללו שני פתרונות \textbf{ממשיים}.

נשתמש בנוסחת אוילר:
\[
e^{i\theta} = \cos\theta + i\sin\theta, \qquad e^{-i\theta} = \cos\theta - i\sin\theta.
\]

נציב:
\[
e^{(a+ib)x} = e^{ax}(\cos bx + i\sin bx), \qquad
e^{(a-ib)x} = e^{ax}(\cos bx - i\sin bx).
\]

נרכיב משני הפתרונות המרוכבים צירופים ליניאריים:
\[
y_1(x) = e^{(a+ib)x} + e^{(a-ib)x} = 2 e^{ax}\cos(bx),
\]
\[
y_2(x) = \frac{1}{i}\big(e^{(a+ib)x} - e^{(a-ib)x}\big) = 2 e^{ax}\sin(bx).
\]

נשים לב כי $y_1$ ו־$y_2$ הן פונקציות \textbf{ממשיות לחלוטין}, ולפי הליניאריות של האופרטור הדיפרנציאלי, גם הן מהוות פתרונות למשוואה ההומוגנית.
לכן, נוכל להחליף את שני הפתרונות המרוכבים בזוג הפתרונות הממשיים:
\[
\boxed{
y_1(x) = e^{ax}\cos(bx), \qquad y_2(x) = e^{ax}\sin(bx).
}
\]

שתי הפונקציות הללו בלתי־תלויות ליניארית ומהוות בסיס למרחב הפתרונות המתאים לשורשים המרוכבים $r_{1,2} = a \pm ib$.

\end{proof}


\paragraph{ניחוש שורשים}    

נניח כי:
\[
L(r) = a_n r^n + \dots + a_1 r + a_0
\]

הוא פולינום עם מקדמים שלמים.  
אם $p, q$ הם מספרים שלמים זרים, אז כאשר $\tfrac{p}{q}$ שורש רציונלי של $L(r)$, נובע כי:
$p$
מחלק את $a_{0}$ ו-$q$ מחלק את $a_{n}$.

מציאת שורש כזה מאפשרת פירוק של $L(r)$ לגורמים, הפחתת סדר הפולינום, ובהתאם — הורדת סדר המד״ר בצורה אפקטיבית. 
כך מתקבל תהליך שיטתי למציאת חלק מהפתרונות, כאשר כל שורש של $L(r)$ מייצר איבר התורם לפתרון הכללי של הבעיה.

\example{}

מצאו פתרון כללי למד״ר ההומוגנית הבאה:
\[
y^{(4)} - 6y^{(3)} + 14y'' - 16y' + 8y = 0
\]

\explanation{}

המקדמים קבועים, ולכן נוכל להשתמש בשיטת הפ״א:

\[
L(r) = r^4 - 6r^3 + 14r^2 - 16r + 8 = 0.
\]

ננסה לנחש פתרונות רציונליים לפי כלל ניחוש השורשים, שכן אין לנו פתרון סגור ידוע למשוואה אלגברית מסדר 4. הקבועים הם: $a_{n}=a_{4}=1, a_{0}=8$. נזכיר כי:
$p$
מחלק את $a_{0}$ ו-$q$ מחלק את $a_{n}$. מכאן:
\[
p = \pm 1, \pm 2, \pm 4, \pm 8, \qquad q = \pm 1
\]
ומכאן:
\[
\frac{p}{q} = \pm 1, \pm 2, \pm 4, \pm 8.
\]
קיבלנו משיטת ניחוש השורשים, 8 שורשים פוטנציאליים, בהנחה וקיים/ים שורש/ים רציונליים. למעשה, מספיק לנו לנחש רק שני שורשים. למה? שכן נוכל להוריד את סדר הפ׳׳א ל-2, וכידוע יש לנו נוסחאות סגורות לפתרון כל משוואה אלגברית מסדר 2, באשר היא. כמובן שנתחיל לנחש מהקל אל הקשה. השורש $1$
כרגע הוא הפשוט ביותר לבדיקה:
\[
L(1) = 1^4 - 6(1)^3 + 14(1)^2 - 16(1) + 8 = 1 - 6 + 14 - 16 + 8 = 1\neq 0.
\]
בדיקה זו לא צלחה.
 נתקדם ל-$-1$
:
\[
L(-1) = (-1)^4 - 6(-1)^3 + 14(-1)^2 - 16(-1) + 8 = 1 + 6 + 14 + 16 + 8 = 45 \neq 0.
\]
בדיקה זו לא צלחה. נתקדם ל-$2$
:
\[
L(2) = (2)^4 - 6(2)^3 + 14(2)^2 - 16(2) + 8 = 16 - 48 + 56 - 32 + 8=0.
\]
הבדיקה צלחה, אך שימו לב, זה ממש לא העת לעצור. כאמור, אנו זקוקים לשני ניחושים מוצלחים. כרגע ניחשנו אחד. אבל, יתכן וקיים ריבוי אלגברי הגדול מ-1 עבור השורש שמצאנו. לכן, נבדוק האם שורש זה מאפס גם את הנגזרת של הפ׳׳א.
נחשב את הנגזרת הראשונה:
\[
L'(r) = 4r^3 - 18r^2 + 28r - 16.
\]

נציב $r=2$:
\[
L'(2) = 4(2)^3 - 18(2)^2 + 28(2) - 16 = 32 - 72 + 56 - 16 = 0.
\]
זו תוצאה מצוינת, שכן הריבוי האלגברי של השורש 1 הוא לפחות 2. למעשה, אין צורך בלהמשיך לנסות ולנחש פתרונות. רק כדי לוודא מהו הריבוי האלגברי של שורש זה, ניקח נגזרת נוספת:
\[
L''(2)\neq0.
\]
מכאן, כי הריבוי האלגברי של השורש 1, הוא 2.
נמצא כי $r_0 = 2$ הוא שורש מריבוי 2.  
נבצע חלוקת פולינומים:
\[
\begin{array}{c c c c c c c l}
 & &  \textcolor{blue}{r^{2}}   & \textcolor{red}{-2r}   & \textcolor{darkgreen}{+2}  \\[6pt]
\cline{1-6}
 r^4 & -6r^3 & +14r^2 & -16r & +8 &  &  \Big|(r - 2)^2 = r^2 - 4r + 4\\[-2pt]
- \textcolor{blue}{r^4} & \textcolor{blue}{-4r^3} & \textcolor{blue}{+4r^2} &  &  &  &  \\[2pt]
\cline{1-4}
  &   \textcolor{black}{-2r^3} & \textcolor{black}{+10r^2} & \textcolor{black}{-16r} & \textcolor{black}{+8}  &  \\[-2pt]
-  &  \textcolor{red}{-2r^3} & \textcolor{red}{+8r^2} & \textcolor{red}{-8r} &  &  \\[2pt]
\cline{2-5}
  &  &  \textcolor{black}{2r^2} & \textcolor{black}{-8r} & \textcolor{black}{+8}  &  \\[-2pt]
-  &  &  \textcolor{darkgreen}{2r^2} & \textcolor{darkgreen}{-8r} & \textcolor{darkgreen}{+8}  &  \\[2pt]
\cline{3-6}
  &  &  &  &  \textcolor{black}{0}
\end{array}
\]



ניתן כעת הסבר מפורט של תהליך החלוקה.  

\textbf{שלב ראשון – מציאת האיבר הראשון של המנה} 

כדי לאפס את האיבר הראשי $r^4$, נכפיל את המחלק $(r-2)^2 = r^2 - 4r + 4$ ב-\(\textcolor{blue}{r^2}\).  
נרשום:
\[
\textcolor{blue}{r^2} \cdot (r^2 - 4r + 4) = 
\textcolor{blue}{r^4 - 4r^3 + 4r^2}.
\]

נחסר ביטוי זה מהפולינום המקורי:
\[
(r^4 - 6r^3 + 14r^2 - 16r + 8)
- (\textcolor{blue}{r^4 - 4r^3 + 4r^2})
= \textcolor{black}{-2r^3 + 10r^2 - 16r + 8}.
\]

\textbf{שלב שני – מציאת האיבר הבא של המנה} 

כעת נאפס את האיבר הראשי החדש $-2r^3$.  
נכפיל את המחלק ב-\(\textcolor{red}{-2r}\):
\[
\textcolor{red}{-2r} \cdot (r^2 - 4r + 4)
= \textcolor{red}{-2r^3 + 8r^2 - 8r}.
\]

נחסר שוב:
\[
(-2r^3 + 10r^2 - 16r + 8)
- (\textcolor{red}{-2r^3 + 8r^2 - 8r})
= \textcolor{black}{2r^2 - 8r + 8}.
\]

\textbf{שלב שלישי – מציאת האיבר האחרון של המנה}  

האיבר הראשי החדש הוא $2r^2$, לכן נכפיל את המחלק ב-\(\textcolor{darkgreen}{2}\):
\[
\textcolor{darkgreen}{2} \cdot (r^2 - 4r + 4)
= \textcolor{darkgreen}{2r^2 - 8r + 8}.
\]

נחסר:
\[
(2r^2 - 8r + 8) - (\textcolor{darkgreen}{2r^2 - 8r + 8}) = 0.
\]

\textbf{מסקנה:}
קיבלנו שהשארית אפס, ולכן מתקיים:
\[
L(r) = (r - 2)^2 \cdot (\textcolor{blue}{r^2} - \textcolor{red}{2r} + \textcolor{darkgreen}{2}).
\]
עבור הביטוי השמאלי (מריבוי 2), השורשים של הפ׳׳א מתאימים לפתרונות:
\[
e^{2x}, \quad x e^{2x}.
\]

נחפש את שורשי הביטוי הימני:
\[
r_{1,2} = \frac{2 \pm \sqrt{4 - 8}}{2} = 1 \pm i.
\]
ולכן הפתרונות המתאימים הם:
\[
e^{x}\cos x, \quad e^{x}\sin x.
\]

ניקח את הסופרפוזיציה של כל הפתרונות, אשר מהווים מערכת יסודית לבעיה/מרחב הפתרונות ממימד 4 למד׳׳ר:
\[
\boxed{
y(x) = C_1 e^{2x} + C_2 x e^{2x} + C_3 e^{x}\cos x + C_4 e^{x}\sin x, 
\quad x \in \mathbb{R}.
}
\]

נזכיר כי אין הבדל בין בחירת הסינוס או הקוסינוס ראשון, משום שהמקדמים $C_3, C_4$ גמישים וניתנים להתלכד לקומבינציה ממשית אחת. בנוסף, אין זה משנה אם מתאימים את הקוסינוס או הסינוס לפתרון עם סימן הפלוס או המינוס (שמגיע מ-$\pm i$), שכן קוסינוס היא פונקציה זוגית, ולמרות שסינוס היא פונקציה אי-זוגית, המינוס שיצא לפניה, נבלע בתוך הקבוע.

\newpage
\underline{תרגילים}
\exercise{}

מצאו את הפתרון הכללי של המשוואה:
\[
y'' - y' - 2y = 0
\]

\exercise{}

מצאו את הפתרון הכללי של המשוואה:
\[
y'' + 8y' + 17y = 0
\]

\exercise{}

מצאו את הפתרון הפרטי של המשוואה:
\[
y'' + 2y' + 5y = 0, 
\qquad 
y(0) = 2, 
\quad 
y'(0) = 1
\]

\exercise{}

מצאו את הפתרון הכללי של המשוואה:
\[
y''' - y' = 0
\]

\exercise{}

מצאו את הפתרון הכללי של המשוואה:
\[
y^{(4)} - 5y'' + 4y = 0
\]

\exercise{}

נתונה המשוואה:
\[
y'' + 8y' + 16y = 0
\]

קבלו פתרון כללי לבעיה.

\exercise{}

נתונה המשוואה:
\[
y'' - 6y' + 9y = 0, 
\qquad 
y(0) = 1, 
\quad 
y'(0) = -2
\]

קבלו פתרון פרטי לבעיה.

\exercise{}

מצאו פתרון כללי למד״ר ההומוגנית הבאה:
\[
y^{(5)} - 6y^{(4)} + 2y^{(3)} - 12y'' + y' - 6y = 0
\]
אילו פתרונות מתכנסים כאשר $x\to-\infty$?  אילו פתרונות חסומים על $x\in\mathbb{R}$? 

\exercise{}

נתונה המשוואה הדיפרנציאלית ההומוגנית:
\[
y^{(4)} - 2y^{(3)} + 3y'' - 2y' + 2y = 0
\]

א. הראו כי $\sin x$ הוא פתרון.  
ב. מצאו פתרון כללי למשוואה.

\exercise{}

על שולחן אופקי חלק הצמוד לקיר מונחת קוביית עץ שמסתה \( m = 1\,\mathrm{kg} \).  
הקובייה קשורה לקיר בקפיץ שהקבוע שלו הוא \( k = 4\,\frac{N}{m} \).  
מאריכים את הקפיץ ב-\( 80\,\mathrm{cm} = 0.8\,\mathrm{m} \) ומשחררים.

\begin{enumerate}[label=\textbf{(\alph*)}]
\item כתבו את המשוואה הדיפרנציאלית המתארת את המערכת.  
\item מצאו את פונקציית המקום של הקובייה.  
\item תארו את המהירות והתאוצה כפונקציות של הזמן במשך מחזור אחד של התנועה.
\end{enumerate}

\exercise{}

על שולחן אופקי חלק הצמוד לקיר מונחת קוביית עץ שמסתה \( m = 900gr \).  
הקובייה קשורה לקיר בקפיץ שקבועו הוא \( k = 3.7 \tfrac{N}{m} \).  
כוח ההתנגדות האוויר פרופורציונלי למהירות הקובייה עם מקדם הצמיגות  
\( \mu = 0.6 \tfrac{N \cdot sec}{m} \).  
מאריכים את הקפיץ ב-\( 1m \) ומשחררים.

\begin{enumerate}[label={\alph*.}]
\item כתבו את המשוואה הדיפרנציאלית המתארת את המערכת.
\item מצאו את פונקציית המיקום של הקובייה.
\item חשבו את מהירות הקובייה כעבור 4 שניות מתחילת תנועתה.
\end{enumerate}

\exercise{}
מצאו משוואה ליניארית מסדר 5 בדויק, הומוגנית, מונורמלית עם מקדמים קבועים כך ש
\( e^{-2x}\cos(3x) \)
פתרון שלה וכל פתרונותיה יציבים באופן אסימפטוטי עבור \( x \to +\infty \).
כמה תשובות יש לשאלה זו? אם יש יותר מתשובה אחת, הסבירו מדוע ותנו 2 דוגמאות.

%%%CUT%%%

\newpage
\underline{פתרונות}
\solution{}

נבנה את הפולינום האופייני של המשוואה:
\[
L(r) = r^2 - r - 2 = 0.
\]

נפתור את הפולינום ונמצא את השורשים:
\[
r_{1,2} = \frac{1 \pm \sqrt{1 + 8}}{2} = \frac{1 \pm 3}{2}.
\]

כלומר:
\[
r_1 = 2, \qquad r_2 = -1.
\]

נכתוב את הפתרונות הפרטיים המתאימים:
\[
y_1 = e^{2x}, \qquad y_2 = e^{-x}.
\]

לכן, הפתרון הכללי של המשוואה הוא הסופרפוזיציה הליניארית של שני הפתרונות העצמאיים:
\[
\boxed{
y(x) = C_1 e^{2x} + C_2 e^{-x}, \quad x \in \mathbb{R}.
}
\]
 
השורשים של הפולינום שונים (אמיתיים וממשיים), ולכן כל אחד מהם מייצר פתרון מהצורה \( e^{r_i x} \).  
במקרה זה, הפתרונות מהווים מערכת יסודית ממימד 2, כצפוי ממשוואת דיפרנציאלית מסדר שני.



\solution{}

נכתוב את המשוואה האופיינית:
\[
L(r) = r^2 + 8r + 17 = 0.
\]

נפתור את הפולינום:
\[
r = \frac{-8 \pm \sqrt{8^2 - 4\cdot1\cdot17}}{2} 
= \frac{-8 \pm \sqrt{64 - 68}}{2} 
= \frac{-8 \pm \sqrt{-4}}{2} 
= -4 \pm i.
\]

ולכן:
\[
r_1 = -4 + i, 
\qquad 
r_2 = -4 - i.
\]

נכתוב את הפתרונות הבסיסיים:
\[
y_1(x) = e^{-4x}\cos x, 
\qquad
y_2(x) = e^{-4x}\sin x.
\]

ולכן הפתרון הכללי הוא:
\[
\boxed{
y(x) = e^{-4x}\big(C_1\cos x + C_2\sin x\big), 
\quad x \in \mathbb{R}.
}
\]



\solution{}

נכתוב את המשוואה האופיינית:
\[
L(r) = r^2 + 2r + 5 = 0.
\]

נפתור ונקבל:
\[
r = \frac{-2 \pm \sqrt{4 - 20}}{2} 
= \frac{-2 \pm \sqrt{-16}}{2} 
= -1 \pm 2i.
\]

ולכן:
\[
r_1 = -1 + 2i, 
\qquad 
r_2 = -1 - 2i.
\]

מכאן שהפתרון הכללי של המשוואה הוא:
\[
y(x) = e^{-x}(C_1\cos 2x + C_2\sin 2x).
\]

נשתמש בתנאי ההתחלה כדי למצוא את הפתרון הפרטי.

\textbf{תנאי ראשון:}
\[
y(0) = 2 
\quad \Rightarrow \quad 
e^{0}(C_1\cos 0 + C_2\sin 0) = C_1 = 2.
\]

\textbf{נגזור את הפונקציה:}
\[
y'(x) = e^{-x}\big[-(C_1\cos 2x + C_2\sin 2x)\big] 
+ e^{-x}\big[-2C_1\sin 2x + 2C_2\cos 2x\big].
\]

נציב \(x=0\):
\[
y'(0) = -C_1 + 2C_2 = 1.
\]

נציב את \(C_1 = 2\):
\[
-2 + 2C_2 = 1 
\quad \Rightarrow \quad 
2C_2 = 3 
\quad \Rightarrow \quad 
C_2 = \frac{3}{2}.
\]

ולכן הפתרון הפרטי הוא:
\[
\boxed{
y(x) = e^{-x}\left(2\cos 2x + \tfrac{3}{2}\sin 2x\right), 
\quad x \in \mathbb{R}.
}
\]


\solution{}

נבנה את הפולינום האופייני:
\[
L(r) = r^3 - r = 0.
\]

נפרק לגורמים:
\[
r(r^2 - 1) = 0 \quad \Longrightarrow \quad r(r - 1)(r + 1) = 0.
\]

ומכאן:
\[
r_1 = 0, \quad r_2 = 1, \quad r_3 = -1.
\]

הפתרונות הפרטיים המתאימים הם:
\[
y_1 = e^{0x} = 1, \qquad y_2 = e^{x}, \qquad y_3 = e^{-x}.
\]

ולכן הפתרון הכללי של המשוואה הוא:
\[
\boxed{
y(x) = C_1 + C_2 e^{x} + C_3 e^{-x}, \quad x \in \mathbb{R}.
}
\]



\solution{}

נכתוב את המשוואה האופיינית:
\[
L(r) = r^4 - 5r^2 + 4 = 0.
\]

נבצע הצבה \( r^2 = t \) :
\[
t^2 - 5t + 4 = 0.
\]

נפתור ונקבל:
\[
t_1 = 4, \qquad t_2 = 1.
\]

נחזיר ל-$r$:
\[
r^2 = 4 \;\Rightarrow\; r = \pm 2, 
\qquad
r^2 = 1 \;\Rightarrow\; r = \pm 1.
\]

לכן השורשים של הפולינום האופייני הם:
\[
r_1 = 2, \quad r_2 = -2, \quad r_3 = 1, \quad r_4 = -1.
\]

ולכן הפתרון הכללי של המשוואה הוא:
\[
\boxed{
y(x) = C_1 e^{2x} + C_2 e^{-2x} + C_3 e^{x} + C_4 e^{-x}, 
\quad x \in \mathbb{R}.
}
\]


\solution{}

נכתוב את המשוואה האופיינית:
\[
L(r) = r^2 + 8r + 16 = (r+4)^{2}= 0.
\]

מכאן שמדובר בשורש כפול (ריבוי אלגברי 2):
\[
r_1 = r_2 = -4.
\]

לכן הפתרונות הבסיסיים הם:
\[
y_1(x) = e^{-4x}, 
\qquad
y_2(x) = x e^{-4x}.
\]

ולכן הפתרון הכללי הוא:
\[
\boxed{
y(x) = C_1 e^{-4x} + C_2 x e^{-4x}, 
\quad x \in \mathbb{R}.
}
\]


\solution{}

נכתוב את המשוואה האופיינית:
\[
L(r) = r^2 - 6r + 9 = 0.
\]

נפתור ונקבל:
\[
(r - 3)^2 = 0 \quad \Longrightarrow \quad r_1 = r_2 = 3.
\]

מכאן שמדובר בשורש כפול (ריבוי אלגברי 2).

לכן הפתרון הכללי הוא:
\[
\boxed{
y(x) = C_1 e^{3x} + C_2 x e^{3x}, \quad x \in \mathbb{R}.
}
\]

נשתמש בתנאי ההתחלה כדי למצוא את הפתרון הפרטי.

\textbf{תנאי ראשון:}
\[
y(0) = 1 
\quad \Rightarrow \quad 
C_1 + 0 = 1 
\quad \Rightarrow \quad 
C_1 = 1.
\]

\textbf{נחשב נגזרת:}
\[
y'(x) = 3C_1 e^{3x} + C_2(3x + 1)e^{3x}.
\]

\textbf{תנאי שני:}
\[
y'(0) = -2 
\quad \Rightarrow \quad 
3C_1 + C_2 = -2 
\quad \Rightarrow \quad 
3(1) + C_2 = -2 
\quad \Rightarrow \quad 
C_2 = -5.
\]

ולכן הפתרון הפרטי הוא:
\[
\boxed{
y(x) = e^{3x} - 5x e^{3x}, \quad x \in \mathbb{R}.
}
\]



\solution{}

נחפש שורשים של הפ״א:
\[
L(r) = r^5 - 6r^4 + 2r^3 - 12r^2 + r - 6 = 0.
\]

נשתמש בכלל ניחוש השורשים הרציונליים:

מבין כל השורשים הפוטנציאליים,
השורש היחידי שמתגלה הוא:
\[
L(6) = 0 \quad \text{אך} \quad L'(6) \neq 0.
\]
מכאן כי \(r = 6\) הוא שורש פשוט (ללא ריבוי).
למעשה, המקסימום שנוכל לעשות בשלב זה הוא להוריד את דרגת הפולינום ל-4. אין לנו כאמור נוסחאות סגורות למציאת שורשים אלה (בכל מקרה הם צפויים להיות לא רציונליים ו/או מרוכבים), נוכל לקוות שאולי נמצא נוסחת כפל מקוצר כלשהי.
נבצע חלוקת פולינומים ונקבל:
\[
L(r) = (r - 6)(r^4 + 2r^2 + 1) = (r - 6)(r^2 + 1)^2.
\]

\textbf{נראה זאת בעזרת חילוק ארוך:}
\[
\begin{array}{r r r r r r r c l}
 &  &  \textcolor{blue}{r^{4}} &  & \textcolor{red}{+2r^{2}} &  & \textcolor{darkgreen}{+1}  \\[6pt]
\cline{2-7}
 & r^5 & -6r^4 & +2r^3 & -12r^2 & +r & -6 &  \Big|\;\; (r-6) \\[-2pt]
- & \textcolor{blue}{r^5} & \textcolor{blue}{-6r^4} &  &  &  &  &  &  \\[2pt]
\cline{2-7}
 &  &  & \textcolor{black}{+2r^3} & \textcolor{black}{-12r^2} & \textcolor{black}{+r} & \textcolor{black}{-6} &  &  \\[-2pt]
- &  &  & \textcolor{red}{+2r^3} & \textcolor{red}{-12r^2} &  &  &  &  \\[2pt]
\cline{3-7}
 &  &  &  &  & \textcolor{black}{r} & \textcolor{black}{-6} &  &  \\[-2pt]
- &  &  &  &  & \textcolor{darkgreen}{r} & \textcolor{darkgreen}{-6} &  &  \\[2pt]
\cline{6-7}
 &  &  &  &  &  &   \textcolor{black}{0}
\end{array}
\]


נמצא את שורשי הפולינום:
\[
r_1 = 6, \quad r_{2,3} = i, \quad r_{4,5} = -i.
\]
שימו לב שהריבוי האלגברי של $\pm i$ הוא 2.
ולכן המערכת היסודית של הפתרונות היא:
\[
\boxed{
y(x) = C_1 e^{6x} + C_2 \sin x + C_3 x \sin x + C_4 \cos x + C_5 x \cos x, 
\quad x \in \mathbb{R}.
}
\]

  הפתרון היחיד שמתכנס עבור $x\to-\infty$ הוא \(C_1 e^{6x}\).
    
  הפתרונות החסומים הם הפונקציות הסינוס והקוסינוס:
  \(
  C_2 \sin x, C_4 \cos x.
  \)
תזכורת: כל \textbf{קומבינציה} אפשרית של פתרונות מתוך הפתרון הכללי, מהווה פתרון בפני עצמה.



\solution{}

\textbf{א. נבדוק כי $\sin x$ הוא פתרון}

נחשב נגזרות:
\[
y = \sin x, \quad y' = \cos x, \quad y'' = -\sin x, \quad y^{(3)} = -\cos x, \quad y^{(4)} = \sin x.
\]

נציב במשוואה:
\[
(\sin x) - 2(-\cos x) + 3(-\sin x) - 2(\cos x) + 2(\sin x)
= \sin x + 2\cos x - 3\sin x - 2\cos x + 2\sin x = 0.
\]
וקיבלנו כי \(0 = 0\), ולכן אכן \(\sin x\) הוא פתרון.

באותה מידה, בהנחה ו-$\sin(x)$ הוא פתרון למד׳׳ר, $\pm i$ חייבים להיות שורשים של הפ׳׳א.
נמצא את הפ״א המתאים:
\[
L(r) = r^4 - 2r^3 + 3r^2 - 2r + 2 = 0.
\]
נציב $i$:
\[
L(i) = i^4 - 2i^3 + 3i^2 - 2i + 2 = 1+2i-3-2i+2=0.
\]
באותה מידה, לפי המשפט היסודי של האלגברה, $-i$ הוא גם שורש לפ׳׳א.

\textbf{ב. נמצא את הפתרון הכללי}

כאמור אם, $r = i, -i$ הם שורשים של הפ״א, $\sin(x), \cos(x)$ הם הפתרונות המתאימים למד׳׳ר (פתרונות צמודים שכן מקדמי הפולינום ממשיים - המשפט היסודי של האלגברה כאמור).

נחלק את הפולינום ב-\((r^2 + 1)\):
\[
r^4 - 2r^3 + 3r^2 - 2r + 2 \;\Big|\; (r^2 + 1)
\]
ונקבל:
\[
r^2 - 2r + 2 = 0.
\]
שורשיו הם:
\[
r = 1 \pm i.
\]
פתרונות אלה מתאימים לפונקציות \(e^x \cos x\) ו-\(e^x \sin x\).

\textbf{ובסופו של דבר, הפתרון הכללי הוא:}
\[
\boxed{
y(x) = C_1 \cos x + C_2 \sin x + C_3 e^{x}\cos x + C_4 e^{x}\sin x,
\quad x \in \mathbb{R}.
}
\]



\solution{}

\textbf{(א) בניית המשוואה}

אם קפיץ לא מוארך ולא מכווץ, אז הוא לא מפעיל כוח על הגוף שאותו הוא מחזיק. קפיץ כזה נקרא קפיץ רפוי.
אם מאריכים או מכווצים אותו ב – $x$ יחידות אורך, הקפיץ מפעיל כוח עפ''י חוק הוק ששווה ל $F=-kx$ הגודל k נקרא קבוע הקפיץ (מדד להתנגדות לכיווץ/התארכות). הסימן מינוס מצביע על כיוונים מנוגדים בין הכוח לבין השינוי במצב הקפיץ.


נניח שמצב שיווי המשקל הוא \( x=0 \).  
כאשר הקפיץ נמתח או נדחס, מופיע כוח אלסטי לפי חוק הוק:

\[
F = -kx.
\]

עפ״י החוק השני של ניוטון:

\[
F = ma = m x''(t).
\]

נציב את הביטוי עבור \( F \):

\[
m x'' = -kx \quad \Longrightarrow \quad x'' + \frac{k}{m}x = 0.
\]

נחשב את המקדם המספרי:  
\[
\frac{k}{m} = \frac{4}{1} = 4.
\]

ולכן המשוואה הדיפרנציאלית היא:
\[
\boxed{x'' + 4x = 0.}
\]

\textbf{(ב) פתרון המשוואה}

נפתור את המשוואה ההומוגנית \( x'' + 4x = 0 \).  
נכתוב את המשוואה האופיינית:
\[
r^2 + 4 = 0 \quad \Longrightarrow \quad r = \pm 2i.
\]

ולכן הפתרון הכללי הוא:
\[
x(t) = C_1 \cos 2t + C_2 \sin 2t.
\]

על פי הנתונים ברגע השחרור:
\[
x(0) = 0.8, \quad v(0) = x'(0) = 0.
\]

נציב בתנאי הראשון:
\[
x(0) = C_1 \cos 0 + C_2 \sin 0 = C_1 = 0.8.
\]

נגזור:
\[
x'(t) = -2C_1 \sin 2t + 2C_2 \cos 2t.
\]

נציב \( t=0 \):
\[
x'(0) = 0 = -2C_1 \sin 0 + 2C_2 \cos 0 = 2C_2 \quad \Longrightarrow \quad C_2 = 0.
\]

ולכן:
\[
\boxed{x(t) = 0.8 \cos 2t.}
\]

\textbf{(ג) מהירות ותאוצה:}

\[
v(t) = x'(t) = -1.6 \sin 2t,
\qquad
a(t) = x''(t) = -3.2 \cos 2t.
\]

המחזור של התנועה הוא:
\[
T = \frac{2\pi}{\omega} = \frac{2\pi}{2} = \pi.
\]

\[
\boxed{
x(t) = 0.8\cos 2t, \quad 
v(t) = -1.6\sin 2t, \quad 
a(t) = -3.2\cos 2t, \quad 
T = \pi.
}
\]

כעת נציג את מיקום, מהירות ותאוצת הקובייה כפונקציה של הזמן:

\begin{figure}[H]
\centering
\begin{tikzpicture}
  \begin{axis}[
    width=15cm, height=10cm,
    axis lines=middle,
    xlabel={$t\,[\mathrm{sec}]$},
    ylabel={},
    xmin=0, xmax=3.3,
    ymin=-3.5, ymax=3.5,
    xtick={0,1.57,3.14},
    xticklabels={$0$, $\tfrac{\pi}{2}$, $\pi$},
    ytick={-3,-2,-1,0,1,2,3},
    grid=both,
    legend style={at={(0.02,0.98)}, anchor=north west, font=\footnotesize},
    domain=0:3.14,
    samples=200,
    ultra thick
  ]
    % x(t)
    \addplot[blue, ultra thick] {0.8*cos(deg(2*x))};
    \addlegendentry{מיקום $x(t) = 0.8\cos(2t)$}
    
    % v(t)
    \addplot[red, ultra thick] {-1.6*sin(deg(2*x))};
    \addlegendentry{מהירות $v(t) = -1.6\sin(2t)$}
    
    % a(t)
    \addplot[green!50!black, ultra thick] {-3.2*cos(deg(2*x))};
    \addlegendentry{תאוצה $a(t) = -3.2\cos(2t)$}
    
    % Labels near peaks
    \node[blue] at (axis cs:0.1,1) {\small $x(t)$};
    \node[red] at (axis cs:0.8,-1.8) {\small $v(t)$};
    \node[green!50!black] at (axis cs:2.8,-3.0) {\small $a(t)$};
  \end{axis}
\end{tikzpicture}
\caption{גרף של מיקום, מהירות ותאוצה כפונקציה של הזמן עבור תנודת הקפיץ.}
\end{figure}


\solution{}

\textbf{א. כתיבת המשוואה הדיפרנציאלית}

נשתמש בחוק השני של ניוטון:
\[
\sum F = ma = mx'' = -kx - \mu x' = mx''.
\]
נציב את המספרים ונקבל:
\[
-3.7x - 0.6x' = 0.9x'' 
\quad \Longleftrightarrow \quad 
9x'' + 6x' + 37x = 0.
\]
לכן, המשוואה היא:
\[
\boxed{
9x'' + 6x' + 37x = 0}.

\textbf{ב. מציאת פונקציית המיקום}

נפתור את המשוואה הליניארית ההומוגנית מסדר שני עם מקדמים קבועים.

המשוואה האופיינית היא:
\[
9r^2 + 6r + 37 = 0
\quad \rightarrow \quad
r_{1,2} = -\tfrac{1}{3} \pm 2i.
\]

מכאן הפתרון הכללי הוא:
\[
x(t) = e^{-t/3}\big(C_1\cos(2t) + C_2\sin(2t)\big).
\]

\textbf{תנאי התחלה:}
\[
x(0) = 1, \qquad v(0) = x'(0) = 0.
\]

נשתמש בתנאי הראשון:
\[
x(0) = e^0(C_1\cos0 + C_2\sin0) = C_1 = 1.
\]

נחשב כעת את הנגזרת:
\[
x'(t) = -\tfrac{1}{3}e^{-t/3}(C_1\cos2t + C_2\sin2t)
       + e^{-t/3}(-2C_1\sin2t + 2C_2\cos2t).
\]

נציב את תנאי ההתחלה השני \(x'(0)=0\):
\[
0 = -\tfrac{1}{3}C_1 + 2C_2
\quad \Longrightarrow \quad
C_2 = \tfrac{C_1}{6} = \tfrac{1}{6}.
\]

ולכן הפתרון הפרטי הוא:
\[
x(t) = e^{-t/3}\Big(\cos2t + \tfrac{1}{6}\sin2t\Big).
\]

\textbf{ג. חישוב מהירות הקובייה}

נגזור את \(x(t)\):
\[
x'(t) = -\tfrac{1}{3}e^{-t/3}\Big(\cos2t + \tfrac{1}{6}\sin2t\Big)
        + e^{-t/3}\Big(-2\sin2t + \tfrac{1}{3}\cos2t\Big).
\]

לאחר סידור נקבל:
\[
x'(t) = -\tfrac{37}{18}e^{-t/3}\sin2t.
\]

ולכן:
\[
\boxed{v(t) = x'(t) = -\tfrac{37}{18}e^{-t/3}\sin2t}.
\]
לפניכם גרפי מיקום ומהירות הקובייה כפונקציה של הזמן:

\begin{figure}[H]
\centering
\begin{tikzpicture}
  \begin{axis}[
    width=15cm, height=10cm,
    axis lines=middle,
    xlabel={$t\,[\mathrm{sec}]$},
    ylabel={},
    xmin=0, xmax=15,
    ymin=-1.7, ymax=1.5,
    grid=both,
    samples=500,
    domain=0:15,
    xtick={0,3.14,6.28,9.42,12.56},
    xticklabels={$0$, $\pi$, $2\pi$, $3\pi$, $4\pi$},
    ytick={-1,0,1},
    legend style={at={(0.02,0.98)},anchor=north west, font=\small},
    thick
  ]
    % x(t)
    \addplot[blue, ultra thick] {exp(-x/3)*(cos(deg(2*x)) + (1/6)*sin(deg(2*x)))};
    \addlegendentry{מיקום $x(t) = e^{-t/3}(\cos 2t + \tfrac{1}{6}\sin 2t)$}
    
    % v(t)
    \addplot[purple, ultra thick] {-(37/18)*exp(-x/3)*sin(deg(2*x))};
    \addlegendentry{מהירות $v(t) = -\tfrac{37}{18}e^{-t/3}\sin 2t$}

  \end{axis}
\end{tikzpicture}
\caption{גרף של מיקום ומהירות כפונקציה של הזמן עבור תנודת קפיץ עם דעיכה (כוח חיכוך).}
\end{figure}

%%%CUT%%%

\solution{}
נחשב ראשית את השורשים של הפונקציה האופיינית.  
ידוע כי הפונקציה \( e^{-2x}\cos(3x) \) היא מהצורה
\[
e^{\alpha x}\cos(\beta x),
\]
כאשר \( \alpha = -2 \) ו-\( \beta = 3 \).  
לכן השורשים המרוכבים של הפונקציה האופיינית הם:
\[
r_{1,2} = \alpha \pm i\beta = -2 \pm 3i.
\]

נרצה עוד 3 שורשים שהחלק הממשי שלהם שלילי. זאת כיוון שהחלק הממשי של השורש בפ׳׳א משמש בתור הארגומנט של האקספוננט במרחב הפתרונות של הבעיה. דבר זה יבטיח יציבות באיקס מאוד גדול.
לכן נדרש:
\[
Re(r_3), Re(r_4), Re(r_5) < 0
\]
יש כמובן אינסוף דוגמאות לכך.

\textbf{דוגמא 1:}
\[
r_{3,4,5} = -1
\]

נבנה את הפולינום כך:
\[
L(r) = (r+1)^3 (r - (-2 + 3i))(r - (-2 - 3i))
\]
\[
= (r+1)^3 ((r+2) - 3i)((r+2) + 3i) = (r+1)^3 ((r+2)^2 + 9)
\]
\[
= (r^3 + 3r^2 + 3r + 1)(r^2 + 4r + 13) = r^5 + 7r^4 + 28r^3 + 52r^2 + 43r + 13
\]
מכאן כי המד׳׳ר היא:
\[
\boxed{y^{(5)} + 7y^{(4)} + 28y''' + 52y'' + 43y' + 13y = 0}.
\]

\textbf{דוגמא 2:}
\[
r_3 = -1, \quad r_{4,5} = -2 \pm 3i
\]

נבנה את הפולינום כך:
\[
L(r) = (r+1)(r - (-2 + 3i))^2 (r - (-2 - 3i))^2
\]
\[
= (r+1)((r+2)^2 + 9)^2 = (r+1)(r^{2}+4r+13)^{2} = (r+1)(r^4 + 8r^3 + 42r^2 + 104r + 169)
\]
\[
= r^5 + 9r^4 + 50r^3 + 146r^2 + 273r + 169
\]
ומכאן כי המד׳׳ר היא:
\[
\boxed{y^{(5)} + 9y^{(4)} + 50y''' + 146y'' + 273y' + 169y = 0}.
\]

\subsubsection{משוואות אי-הומוגניות - שיטת השוואת מקדמים}

נבחן מד״ר לינארית אי־הומוגנית, מנורמלת, מסדר $n$:
\begin{equation}\label{gen_n_fixed}
y^{(n)} + a_{n-1}y^{(n-1)} + \dots + a_1y' + a_0y = G(x),
\end{equation}
כאשר $G(x)$ מהווה את האיבר הלא הומוגני באגף ימין, וכל המקדמים $a_i$ הם \textbf{קבועים}.

\begin{remark}
    במד׳׳ר מסדר גבוה בו אין לנו נוסחאות סגורות לפתרון, ככלל, נרצה לקבל קודם את הפתרון למד׳׳ר ההומוגנית. כלומר, נאפס את אגף ימין במשוואה \ref{gen_n_fixed} ונדרוש $G(x)=0$. לאחר ש׳׳נחזיק׳׳ את הפתרון ההומוגני בידינו, נמשיך לפתרון החלק האי-הומוגני, בהתאם לשיטה המתאימה, ולבסוף ניקח את הסופרפוזיציה של הפתרונות להוות את הפתרון הכללי לבעיה.
\end{remark}

נגדיר את הפולינום האופייני של החלק ההומוגני:
\begin{equation}
L(r) = r^n + a_{n-1}r^{n-1} + \dots + a_1r + a_0.
\end{equation}
כפי שראינו בתת-פרק הקודם, נוכל לקבל את הפתרון ההומוגני ע׳׳י מציאת השורשים של הפ׳׳א.

באשר לחלק האי-הומוגני,
נבצע הבחנה בין שלושה מקרים עיקריים עבור האיבר הבלתי־הומוגני $G(x)$.

\vspace{0.5cm}
\textbf{מקרה 1:}  
כאשר $G(x) = P_m(x)$, פולינום ממעלה $m$.  
נניח כי $0 \le k$ הוא \textbf{הריבוי של השורש $r=0$} כשורש של הפ״א.  
אזי הפתרון הפרטי הוא מהצורה:
\begin{equation}
y_p(x) = x^k R_m(x) = x^k(d_nx^m + \dots + d_0),
\end{equation}
כאשר $d_0, d_1, \dots, d_n$ הם קבועים אותם יש למצוא ע״י הצבה במשוואה \ref{gen_n_fixed}.

\vspace{0.5cm}
\textbf{מקרה 2:}  
כאשר $G(x) = e^{ax} \cdot P_m(x)$.  
נניח כי $0 \le k$ הוא \textbf{הריבוי של $r=a$} כשורש של הפ״א.  
אזי הפתרון הפרטי הוא מהצורה:
\begin{equation}
y_p(x) = x^k R_m(x)e^{ax},
\end{equation}
כאשר $R_m(x)$ הוא פולינום ממעלה $m$ עם מקדמים שיש למצוא אותם. שימו לב כי מקרה 1, הוא מקרה פרטי של מקרה 2, בעבור $a=0$.

\vspace{0.5cm}
\textbf{מקרה 3:}  
כאשר $G(x) = e^{ax}\big(P_m(x)\cos(bx) + Q_m(x)\sin(bx)\big)$.  
נניח כי $0 \le k$ הוא \textbf{הריבוי של השורש המרוכב} $r = a + ib$ של הפ״א.  
אזי הפתרון הפרטי הוא מהצורה:
\begin{equation}
y_p(x) = x^k \big[ R_m(x)e^{ax}\cos(bx) + S_m(x)e^{ax}\sin(bx) \big],
\end{equation}
כאשר $R_m(x)$ ו-$S_m(x)$ פולינומים ממעלה $m$ עם מקדמים שיש למצוא אותם. שימו לב כי מקרה 2 הוא מקרה פרטי של מקרה 3 בעבור $b=0$, ואם בנוסף $a=0$, אנו ׳׳קורסים׳׳ למעשה למקרה 1.

\example{}
מצאו פתרון כללי למד״ר:
\[
y'' + y' + y = x^2.
\]

\explanation{}
נדרוש:
\(
y = y_H + y_P
\)
ונחשב ראשית את הפתרון ההומוגני. 
הפולינום האופייני הוא:
\[
L(r) = r^2 + r + 1 = 0
\]
נפתור:
\[
r_{1,2} = \frac{-1 \pm \sqrt{3}i}{2}
\]

כלומר, הפתרונות ההומוגניים הם:
\[
y_1 = e^{-\frac{1}{2}x}\sin\!\left(\frac{\sqrt{3}}{2}x\right),
\qquad
y_2 = e^{-\frac{1}{2}x}\cos\!\left(\frac{\sqrt{3}}{2}x\right)
\]
ולכן:
\[
y_H(x) = C_1y_1 + C_2y_2
\]

\textbf{החלק הפרטי}

על מנת לזהות את הפרמטרים השונים של החלק האי-הומוגני, נכתוב את המד׳׳ר בצורה הבאה:
\[
y'' + y' + y = x^2 \cdot \big(e^{0x}(\cos(0x) + \sin(0x))\big),
\]
ונסיק כי:
\[
G(x) = x^2 \cdot (e^{0x}\cos(0x) + \sin(0x))
\Rightarrow
m = 2,\quad a = 0,\quad b = 0.
\]
בנוסף, נבחן את כשורש $a + ib = 0$ של הפ״א.
נמצא כי ריבויו הוא $k = 0$ (כלומר $r = 0$ אינו שורש של הפ״א).

לכן:
\[
y_P(x) = x^k[R_m(x)e^{ax}\cos(bx) + S_m(x)e^{ax}\sin(bx)] = x^0 \cdot R_2(x) = R_2(x)
\]
כאשר:
\[
R_2(x) = d_2x^2 + d_1x + d_0=y_{p}.
\]

נגזור:
\[
y_P'(x) = 2d_2x + d_1, \qquad y_P''(x) = 2d_2
\]

נציב למד׳׳ר:
\[
(2d_2) + (2d_2x + d_1) + (d_2x^2 + d_1x + d_0) = x^2
\]
נכנס איברים דומים:
\[
d_2x^2 + (2d_2 + d_1)x + (2d_2 + d_1 + d_0) = x^2
\]
נשווה מקדמים:
\[
\begin{cases}
d_2 = 1 \\[3pt]
2d_2 + d_1 = 0 \\[3pt]
2d_2 + d_1 + d_0 = 0
\end{cases}
\Rightarrow
d_2 = 1,\; d_1 = -2,\; d_0 = 0
\]

מכאן נקבל:
\[
y_P(x) = x^2 - 2x
\]
כעת ניקח את הסופרפוזיציה של הפתרון ההומוגני והפרטי, כדי לקבל את הפתרון הכללי לבעיה:
\[
\boxed{
y(x) = y_H(x) + y_P(x)
= C_1 e^{-\frac{1}{2}x}\sin\!\left(\frac{\sqrt{3}}{2}x\right)
+ C_2 e^{-\frac{1}{2}x}\cos\!\left(\frac{\sqrt{3}}{2}x\right)
+ x^2 - 2x, \qquad x \in \mathbb{R}.
}
\]
\newpage
\underline{תרגילים}

\exercise{}
מצאו את הפתרון הכללי של המשוואה:
\[
y'' - 3y' + 2y = 5
\]

\exercise{}
מצאו את הפתרון הכללי של המשוואה:
\[
y'' - 3y' + 2y = x^2
\]

\exercise{}
מצאו את הפתרון הפרטי לבעיית ההתחלה הבאה:
\[
y'' + 2y' + y = e^{2x},
\qquad
\begin{cases}
y(0) = 0, \\[3pt]
y'(0) = -2.
\end{cases}
\]

\exercise{}
מצאו פתרון כללי למד״ר:
\[
y'' - 4y' + 4y = x \cdot e^{2x}
\]

\exercise{}
מצאו פתרון כללי למד״ר:
\[
y'' - y = e^{x}\cos x
\]

\exercise{}
מצאו את הפתרון הכללי של המשוואה:
\[
y'' - 3y' + 2y = \sin 4x
\]

\exercise{}
על שולחן אופקי חלק ניצבת קוביית עץ שמסתה \( m = 1\,\text{kg} \).
הקובייה קשורה לקפיץ שקבועו \( k = 4\,\tfrac{N}{m} \).
על הקובייה פועל כוח חיצוני מחזורי \( F(t) = \cos(bt) \).
בתחילת התנועה הקפיץ רפוי והקובייה במנוחה.

\begin{enumerate}[label=(\alph*)]
\item כתבו את המשוואה הדיפרנציאלית המתארת את המערכת.
\item מצאו את פונקציית המיקום של הקובייה (הביעו באמצעות \( b \)). הניחו כי הקובייה מתחילה ממיקום יחסי ׳׳0׳׳ וממנוחה.
\item הציגו את פונקציית המיקום עבור \( b = 1.4 \).
\end{enumerate}


\newpage
\underline{פתרונות}
\solution{}
נתחיל מהפתרון של המשוואה \underline{ההומוגנית המתאימה}:
\[
y'' - 3y' + 2y = 0
\]

נכתוב את המשוואה האופיינית:
\[
r^2 - 3r + 2 = 0
\quad \Longrightarrow \quad
\begin{cases}
r_1 = 2 \\[3pt]
r_2 = 1
\end{cases}
\]

ולכן הפתרון של המשוואה ההומוגנית הוא:
\[
y_h = C_1 e^{x} + C_2 e^{2x}.
\]

כעת נמצא את הפתרון הפרטי של המשוואה הלא הומוגנית.
את החלק הלא־הומוגני – נפתור באמצעות השוואת מקדמים.

האגף הימני הוא קבוע:
\[
G(x) = 5.
\]

נכתוב אותו בצורתו הכללית:
\[
G(x) = 5 = 5e^{0x}\big(\cos(0x) + \sin(0x)\big),
\]
ומכאן נזהה את הפרמטרים:
\[
a = 0, \quad b = 0, \quad m = 0.
\]

מכיוון ש-\(a+ib = 0+0i = 0\) איננו שורש של הפ״א, הריבוי האלגברי הוא \(k=0\).

ולכן נציע פתרון פרטי כזה:
\[
y_p = R_0(x).
\]

לכן:
\[
y_p = d_0,
\]
כאשר יש למצוא את $d_0$.
נחשב נגזרות ונציב:
\[
y_p' = 0, \qquad y_p'' = 0.
\]

נציב במד׳׳ר הלא הומוגנית:
\[
y_p'' - 3y_p' + 2y_p = 5
\quad \Longrightarrow \quad
0 - 0 + 2d_0 = 5
\quad \Longrightarrow \quad
d_0 = \frac{5}{2}.
\]

ולכן:
\[
\boxed{y_p(x) = 2.5.}
\]

ולבסוף, הפתרון הכללי:
\[
\boxed{
y(x) = C_1 e^{x} + C_2 e^{2x} + 2.5, \qquad x\in\mathbb{R}.
}
\]


\solution{}
נתחיל מהפתרון של המשוואה \underline{ההומוגנית המתאימה}:
\[
y'' - 3y' + 2y = 0
\]

נכתוב את המשוואה האופיינית:
\[
r^2 - 3r + 2 = 0
\quad \Longrightarrow \quad
\begin{cases}
r_1 = 2, \\[3pt]
r_2 = 1.
\end{cases}
\]

ולכן הפתרון של המשוואה ההומוגנית הוא:
\[
y_h = C_1 e^{x} + C_2 e^{2x}.
\]

כעת נמצא את הפתרון הפרטי של המשוואה הלא־הומוגנית.  
את החלק הלא־הומוגני – נפתור באמצעות השוואת מקדמים.

האגף הימני הוא פולינום מדרגה 2:
\[
G(x) = x^2.
\]
נזהה את הפרמטרים של הצורה הכללית:
\[
a = 0, \quad b = 0, \quad m = 2.
\]

מכיוון ש-\(a + ib = 0 + 0i = 0\) איננו שורש של הפ״א,  
הריבוי האלגברי הוא \(k = 0\).

נציב את הערכים הידועים לפתרון המוצע:
\[
y_p(x) = x^0\big[ R_2(x)e^{0x}\cos(0x) + S_2(x)e^{0x}\sin(0x) \big],
\]
ולכן נקבל:
\[
y_p(x) = R_2(x),
\]
כאשר
\[
R_2(x) = d_2x^2 + d_1x + d_0.
\]

נחשב נגזרות:
\[
y_p'(x) = 2d_2x + d_1, \qquad y_p''(x) = 2d_2.
\]

נציב במד״ר הלא־הומוגנית:
\[
y_p'' - 3y_p' + 2y_p = 2d_2 - 3(2d_2x + d_1) + 2(d_2x^2 + d_1x + d_0) = x^2.
\]
נפתח סוגריים ונקבץ לפי חזקות של \(x\):
\[
2d_2x^2 + (-6d_2 + 2d_1)x + (2d_2 - 3d_1 + 2d_0) = x^2.
\]

מהשוואת מקדמים נקבל:
\[
\begin{cases}
2d_2 = 1, \\[4pt]
-6d_2 + 2d_1 = 0, \\[4pt]
2d_2 - 3d_1 + 2d_0 = 0.
\end{cases}
\]

נפתור את המערכת:
\[
\begin{cases}
d_2 = \tfrac{1}{2}, \\[4pt]
-6(\tfrac{1}{2}) + 2d_1 = 0 \;\Longrightarrow\; -3 + 2d_1 = 0 \;\Longrightarrow\; d_1 = \tfrac{3}{2}, \\[6pt]
2(\tfrac{1}{2}) - 3(\tfrac{3}{2}) + 2d_0 = 0 
\;\Longrightarrow\; 1 - \tfrac{9}{2} + 2d_0 = 0 
\;\Longrightarrow\; 2d_0 = \tfrac{7}{2} 
\;\Longrightarrow\; d_0 = \tfrac{7}{4}.
\end{cases}
\]

ולכן:
\[
y_p(x) = \tfrac{1}{2}x^2 + \tfrac{3}{2}x + \tfrac{7}{4}.
\]

ולבסוף, הפתרון הכללי של המשוואה הנתונה הוא:
\[
\boxed{
y(x) = C_1 e^{x} + C_2 e^{2x} + \tfrac{1}{2}x^2 + \tfrac{3}{2}x + \tfrac{7}{4},
\qquad x \in \mathbb{R}.
}
\]

%%%CUT%%%

\solution{}
נתחיל מהפתרון של המשוואה \underline{ההומוגנית המתאימה}:
\[
y'' + 2y' + y = 0.
\]

נכתוב את המשוואה האופיינית:
\[
r^2 + 2r + 1 = 0
\quad \Longrightarrow \quad (r+1)^2 = 0
\quad \Longrightarrow \quad r_1 = r_2 = -1.
\]

ולכן הפתרון ההומוגני הוא:
\[
y_h = C_1 e^{-x} + C_2 x e^{-x}.
\]

כעת עבור הפתרון הפרטי של המשוואה הלא־הומוגנית:
\[
y'' + 2y' + y = e^{2x},
\]
נזהה את הפרמטרים של הצורה הכללית:
\[
a = 2, \quad b = 0, \quad m = 0.
\]

 \(a+ib = 2+0i = 2\) הוא איננו שורש של הפ״א.
מכאן נקבל:
\(
k = 0
\).

נציב את הערכים בתבנית הכללית:
\[
y_p(x) = x^k\big[R_m(x)e^{ax}\cos(bx) + S_m(x)e^{ax}\sin(bx)\big],
\]
ונקבל:
\[
y_p(x) = x^0 \big[R_0(x)e^{2x}\cos(0x) + S_0(x)e^{2x}\sin(0x)\big]
= R_0(x)e^{2x}=Ae^{2x},
\]
כאשר $A$ הוא הקבוע שיש לגלות.
לכן:
\[
y_p(x) = A e^{2x}.
\]

נחשב נגזרות:
\[
y_p'(x) = 2A e^{2x}, \qquad y_p''(x) = 4A e^{2x}.
\]

נציב במשוואה:
\[
y_p'' + 2y_p' + y_p = 4Ae^{2x} + 4Ae^{2x} + Ae^{2x} = e^{2x}.
\]

נפשט:
\[
9A e^{2x} = e^{2x} \quad \Longrightarrow \quad A = \tfrac{1}{9}.
\]

ולכן:
\[
y_p(x) = \tfrac{1}{9}e^{2x}.
\]

נרכיב את הפתרון הכללי של המשוואה הנתונה:
\[
\boxed{y(x) = C_1 e^{-x} + C_2 x e^{-x} + \tfrac{1}{9}e^{2x}}.
\]

\textbf{נמצא את הקבועים לפי תנאי ההתחלה:}

נציב \(x=0\) במשוואה:
\[
y(0) = C_1 e^{0} + C_2 \cdot 0 \cdot e^{0} + \tfrac{1}{9}e^{0} = C_1 + \tfrac{1}{9} = 0
\quad \Longrightarrow \quad
C_1 = -\tfrac{1}{9}.
\]

נחשב נגזרת כללית:
\[
y'(x) = -C_1 e^{-x} + C_2 e^{-x} - C_2 x e^{-x} + \tfrac{2}{9}e^{2x}.
\]

נציב \(x=0\):
\[
y'(0) = -C_1 + C_2 + \tfrac{2}{9} = -2.
\]

נציב את הערך של \(C_1\):
\[
-(-\tfrac{1}{9}) + C_2 + \tfrac{2}{9} = -2
\quad \Longrightarrow \quad
\tfrac{1}{9} + C_2 + \tfrac{2}{9} = -2
\quad \Longrightarrow \quad
C_2 = -\tfrac{7}{3}.
\]

ולכן הפתרון הפרטי לבעיה הוא:
\[
\boxed{
y(x) = -\tfrac{1}{9}e^{-x} - \tfrac{7}{3}x e^{-x} + \tfrac{1}{9}e^{2x},
\qquad x \in \mathbb{R}.
}
\]



\solution{}
נתחיל מהחלק ההומוגני:
\[
y'' - 4y' + 4y = 0
\]

נגדיר פ״א:
\[
L(r) = r^2 - 4r + 4 = 0 \quad \Longrightarrow \quad L(r) = (r - 2)^2
\]

והפתרונות המתאימים הם:
\[
y_1 = e^{2x}, \qquad y_2 = x \cdot e^{2x}
\]
את החלק הלא הומוגני – נפתור באמצעות השוואת מקדמים.

אנו נמצאים במקרה של שורש כפול של הפ״א. נגדיר את הפרמטרים:
\[
a = 2, \quad b = 0, \quad m = 1
\]
במקרה הזה, הריבוי האלגברי של $a+ib$, הלא הוא $2+0i=2$,  הוא 2. על כן $k=2$.
לכן:
\[
y_p = x^2 \cdot R_1(x) \cdot e^{2x} = x^2 (d_1 x + d_0)e^{2x} = e^{2x}(d_1x^3 + d_0x^2)
\]

נגזור ונציב במד״ר ונשווה מקדמים:
\[
y_p(x) = e^{2x}(d_1x^3 + d_0x^2)
\]

נבצע גזירה והשוואת מקדמים לקבלת ערכי $d_0, d_1$.
נחשב את הנגזרות:
\[
\begin{aligned}
y_p'(x) &= e^{2x}\big[2(d_1x^3 + d_0x^2) + (3d_1x^2 + 2d_0x)\big] 
= e^{2x}\big(2d_1x^3 + 2d_0x^2 + 3d_1x^2 + 2d_0x\big) \\[4pt]
&= e^{2x}\big(2d_1x^3 + (2d_0 + 3d_1)x^2 + 2d_0x\big),
\\[6pt]
y_p''(x) &= e^{2x}\big[2(2d_1x^3 + (2d_0 + 3d_1)x^2 + 2d_0x) + (6d_1x^2 + 2(2d_0 + 3d_1)x + 2d_0)\big] \\[4pt]
&= e^{2x}\big(4d_1x^3 + 2(2d_0 + 3d_1)x^2 + 4d_0x + 6d_1x^2 + 2(2d_0 + 3d_1)x + 2d_0\big) \\[4pt]
&= e^{2x}\big(4d_1x^3 + (4d_0 + 6d_1 + 6d_1)x^2 + (4d_0 + 4d_0 + 6d_1)x + 2d_0\big) \\[4pt]
&= e^{2x}\big(4d_1x^3 + (4d_0 + 12d_1)x^2 + (8d_0 + 6d_1)x + 2d_0\big).
\end{aligned}
\]

נציב כעת  את $y_p, y_p', y_p''$: במשוואה המקורית:
\[
e^{2x}\big(4d_1x^3 + (4d_0 + 12d_1)x^2 + (8d_0 + 6d_1)x + 2d_0\big)
-4e^{2x}\big(2d_1x^3 + (2d_0 + 3d_1)x^2 + 2d_0x\big)
\]\[
+4e^{2x}\big(d_1x^3 + d_0x^2\big)
= e^{2x}x.
\]

נחלק את שני האגפים ב-$e^{2x}$ (שאינו אפס) ונקבץ לפי חזקות של $x$:
\[
\begin{aligned}
x^3: & \quad (4d_1 - 8d_1 + 4d_1) = 0, \\[4pt]
x^2: & \quad (4d_0 + 12d_1) - 4(2d_0 + 3d_1) + 4d_0 = 0, \\[4pt]
x^1: & \quad (8d_0 + 6d_1) - 8d_0 = 1, \\[4pt]
x^0: & \quad 2d_0 = 0.
\end{aligned}
\]

נפשט כל שורה:

\[
\begin{cases}
0=0, \\[4pt]
(4d_0 + 12d_1) - 8d_0 - 12d_1 + 4d_0 = 0, \\[4pt]
8d_0 + 6d_1 - 8d_0 = 1, \\[4pt]
2d_0 = 0.
\end{cases}
\]
מהשורות הראשונה והשנייה נקבל זהות $0=0$, ולכן נותרות שתי משוואות רלוונטיות:
\[
\begin{cases}
6d_1 = 1, \\[4pt]
2d_0 = 0.
\end{cases}
\]

נמצא את ערכי המקדמים:
\[
d_1 = \frac{1}{6}, \qquad d_0 = 0.
\]

נחזיר לביטוי של הפתרון הפרטי:
\[
\boxed{y_p(x) = \frac{1}{6}x^3 e^{2x}}.
\]

ולכן הפתרון הכללי של המשוואה הוא:
\[
\boxed{
y(x) = C_1 e^{2x} + C_2 x e^{2x} + \frac{1}{6}x^3 e^{2x}, \quad x \in \mathbb{R}.
}
\]


\solution{}
נתחיל מהחלק ההומוגני:
\[
y'' - y = 0
\]

נגדיר את הפ״א:
\[
L(r) = r^2 - 1 = 0
\]
ולכן:
\[
r_{1,2} = \pm 1
\]

מכאן הפתרון ההומוגני הוא:
\[
y_H(x) = C_1 e^{x} + C_2 e^{-x}
\]

נעבור כעת אל החלק הלא־הומוגני.  
נשתמש בשיטת \textbf{השוואת המקדמים}.

באגף ימין מופיעה פונקציה מהצורה \( e^{ax}\cos(bx) \),  
ולכן אנו נמצאים במקרה ה־3 מתוך שלושת המקרים שלמדנו (למרות שמופיע רק קוסינוס, ללא סינוס!).

נזהה את הפרמטרים:
\[
a = 1, \quad b = 1, \quad m = 0
\]

נבדוק אם \(a+ib = 1+i\) הוא לא שורש של הפ״א, שכן השורשים הם $\pm1$, ולכן
\(
k = 0
\).

נשתמש בצורה הכללית של הפתרון הפרטי:
\[
y_p = x^k \big[R_m(x)e^{ax}\cos bx + S_m(x)e^{ax}\sin bx\big]
\]

מכיוון ש-\(m=0\) ו-\(k=0\), נקבל:
\[
y_p = e^{x}(a_1 \cos x + b_1 \sin x)
\]

נחשב נגזרות:

\[
\begin{aligned}
y_p' &= e^{x}\big[a_1(\cos x - \sin x) + b_1(\sin x + \cos x)\big] \\[4pt]
y_p'' &= 2e^{x}\big[b_1\cos x-a_1\sin x)\big]
\end{aligned}
\]

נציב במשוואה המקורית ונקבל:
\[
2e^{x}(b_1\cos x - a_1\sin x) - e^{x}(a_1\cos x + b_1\sin x) = e^{x}\cos x.
\]

נחלק את שני האגפים ב-\(e^{x}\) (שאינו אפס), ונקבל:
\[
(2b_1 - a_1)\cos x + (-2a_1 - b_1)\sin x = \cos x.
\]

נשווה מקדמים של \(\cos x\) ושל \(\sin x\):
\[
\begin{cases}
2b_1 - a_1 = 1, \\[4pt]
-2a_1 - b_1 = 0.
\end{cases}
\]

נפתור את המערכת ע׳׳י הצבה של הקשר מהמשוואה הראשונה, לשנייה:
\[
\begin{aligned}
-2a_1 - b_1 &= 0 \quad \Longrightarrow \quad b_1 = -2a_1, \\[4pt]
2(-2a_1) - a_1 &= 1 \quad \Longrightarrow \quad -5a_1 = 1 \quad \Longrightarrow \quad a_1 = -\tfrac{1}{5}, \\[4pt]
b_1 &= -2a_1 = \tfrac{2}{5}.
\end{aligned}
\]

נחזיר לביטוי של הפתרון הפרטי:
\[
\boxed{
y_p(x) = e^{x}\!\left(-\tfrac{1}{5}\cos x + \tfrac{2}{5}\sin x\right)
}
\]

ולכן הפתרון הכללי הוא:
\[
\boxed{
y(x) = C_1 e^{x} + C_2 e^{-x} - \tfrac{1}{5}e^{x}\cos x + \tfrac{2}{5}e^{x}\sin x,
\quad x \in \mathbb{R}.
}
\]

\textbf{הערה:}  
החלק הלא־הומוגני כולל איבר \(e^{x}\cos x\), ולכן יש לקחת גם את \(e^{x}\sin x\) בביטוי של הפתרון הפרטי —  
גם אם באגף ימין מופיע רק קוסינוס.



\solution{}
נמצא תחילה את הפתרון של המשוואה \underline{ההומוגנית המתאימה}:
\[
y'' - 3y' + 2y = 0.
\]

נכתוב את המשוואה האופיינית:
\[
r^2 - 3r + 2 = 0
\quad \Longrightarrow \quad
\begin{cases}
r_1 = 2, \\[3pt]
r_2 = 1.
\end{cases}
\]

ולכן הפתרון של המשוואה ההומוגנית הוא:
\[
y_h = C_1 e^{x} + C_2 e^{2x}.
\]

כעת נטפל בחלק הלא הומוגני של המשוואה.
האגף הימני הוא פונקציה טריגונומטרית:
\[
G(x) = \sin 4x.
\]

נזהה את הפרמטרים עבור שיטת השוואת המקדמים:
\[
a = 0, \quad b = 4, \quad m = 0.
\]

נבדוק האם \(a + ib = 0 + 4i\) הוא שורש של הפ״א.
שורשי הפ״א הם \(1, 2\), ולכן \(a + ib\) אינו שורש של הפ״א.
מכאן נקבל:
\(
k = 0
\).
לכן נציע פתרון פרטי מן הצורה הכללית:
\[
y_p(x) = x^k\big[R_m(x)e^{ax}\cos(bx) + S_m(x)e^{ax}\sin(bx)\big].
\]

נציב את הערכים הידועים:
\[
y_p(x) = R_0(x)e^{0x}\cos(4x) + S_0(x)e^{0x}\sin(4x)
= A\cos 4x + B\sin 4x.
\]

נחשב נגזרות:
\[
\begin{aligned}
y_p'(x) &= -4A\sin 4x + 4B\cos 4x, \\[3pt]
y_p''(x) &= -16A\cos 4x - 16B\sin 4x.
\end{aligned}
\]

נציב במשוואה הנתונה:
\[
y_p'' - 3y_p' + 2y_p = (-16A\cos 4x - 16B\sin 4x)
 - 3(-4A\sin 4x + 4B\cos 4x)
 + 2(A\cos 4x + B\sin 4x).
\]

נפתח ונקבץ לפי פונקציות סינוס וקוסינוס:
\[
(-16A - 12B + 2A)\cos 4x + (-16B + 12A + 2B)\sin 4x = \sin 4x.
\]
נפשט את המקדמים:
\[
(-14A - 12B)\cos 4x + (-14B + 12A)\sin 4x = \sin 4x.
\]

נשווה מקדמים משני אגפי המשוואה:

\[
\begin{cases}
-14A - 12B = 0, \\[4pt]
-14B + 12A = 1.
\end{cases}
\]

נפתור את המערכת:
מהמשוואה הראשונה נקבל:
\[
A = -\tfrac{6}{7}B.
\]

נציב במשוואה השנייה:
\[
-14B + 12\!\left(-\tfrac{6}{7}B\right) = 1 
\quad \Longrightarrow \quad
-14B - \tfrac{72}{7}B = 1 
\quad \Longrightarrow \quad
-\tfrac{170}{7}B = 1 
\quad \Longrightarrow \quad
B = -\tfrac{7}{170}.
\]

נציב חזרה כדי למצוא \(A\):
\[
A = -\tfrac{6}{7}B = -\tfrac{6}{7}\left(-\tfrac{7}{170}\right) = \tfrac{6}{170} = \tfrac{3}{85}.
\]

ולכן:
\[
\boxed{
A = \tfrac{3}{85}, \qquad B = -\tfrac{7}{170}.
}
\]

נציב חזרה לביטוי של הפתרון הפרטי:
\[
y_p(x) = A\cos 4x + B\sin 4x
= \tfrac{3}{85}\cos 4x - \tfrac{7}{170}\sin 4x.
\]

ולכן הפתרון הכללי של המשוואה הנתונה הוא:
\[
\boxed{
y(x) = C_1 e^{x} + C_2 e^{2x}
 + \tfrac{3}{85}\cos 4x - \tfrac{7}{170}\sin 4x,
\qquad x \in \mathbb{R}.
}
\]


\solution{}
(א)
הכוחות הפועלים הם הכוח החיצוני $F_{1}$ והכוח האלסטי של הקפיץ $F_{2}$: 
\[
F_1 = \cos(bt), \qquad F_2 = -kx.
\]

לפי החוק השני של ניוטון:
\[
\sum F = ma = m x'' \quad \Longrightarrow \quad \cos(bt) - kx = m x''.
\]

נציב את המספרים הנתונים:
\[
\cos(bt) - 4x = x'' \quad \Longrightarrow \quad x'' + 4x = \cos(bt).
\]

קיבלנו משוואה דיפרנציאלית עם מקדמים קבועים, מסדר 2, לא הומוגנית, מהצורה:
\[\boxed{
x'' + 4x = \cos(bt)}.
\]

(ב)
\textbf{מהחלק ההומוגני נקבל:}
\[
x'' + 4x = 0
\quad \Longrightarrow \quad
r^2 + 4 = 0
\quad \Longrightarrow \quad
r_{1,2} = \pm 2i.
\]

ולכן:
\[
x_h(t) = C_1\cos(2t) + C_2\sin(2t).
\]

כעת נעבור לחלק האי-הומוגני.
נזהה את הפרמטרים:
\[
a = 0, \quad b = b, \quad m = 0.
\]

נבדוק האם \(a + ib = 0 + ib = ib\) הוא שורש של הפ״א.
שורשי הפ״א הם \(\pm 2i\).
נבחין בין שני מצבים אפשריים:

נבחין כי אם \(b \ne \pm 2\), אז \(ib\) איננו שורש של הפ״א ולכן \(k = 0\);  
ואילו אם \(b = \pm 2\), אז \(ib\) הוא שורש של הפ״א ולכן \(k = 1\).

כלומר
- במקרה (\(b \ne \pm 2\)) נקבל פתרון פרטי מן הצורה:
  \[
  x_p(t) = R_0(t)\cos(bt) + S_0(t)\sin(bt)
  = a\cos(bt) + \beta\sin(bt).
  \]

- ואילו במקרה של תהודה/רזוננס (\(b = \pm 2\)) יש להכפיל את הביטוי ב-\(t^k = t\):
  \[
  x_p(t) = t\big[a\cos(2t) + \beta\sin(2t)\big].
  \]

\textbf{
 מקרה 1 — \(b \ne \pm 2\)  (ללא תהודה)}

נכתוב את נוסחת הצורה הכללית:
\[
x_p(t) = t^k \big[R_m(t)e^{at}\cos(bt) + S_m(t)e^{at}\sin(bt)\big].
\]

נציב את הערכים \(a=0,\, b=b,\, m=0,\, k=0\):
\[
x_p(t) = R_0(t)\cos(bt) + S_0(t)\sin(bt).
\]

מכיוון ש-\(R_0,S_0\) הם פולינומים ממעלה אפס:
\[
R_0(t)=a, \qquad S_0(t)=\beta.
\]

ולכן נציע פתרון פרטי מן הצורה:
\[
x_p(t) = a\cos(bt) + \beta\sin(bt).
\]

נחשב נגזרות:
\[
x_p'(t) = -ab\sin(bt) + \beta b\cos(bt),
\quad
x_p''(t) = -ab^2\cos(bt) - \beta b^2\sin(bt).
\]

נציב במשוואה:
\[
x_p'' + 4x_p = \cos(bt),
\]

ונקבל:
\[
(-ab^2\cos bt - \beta b^2\sin bt) + 4(a\cos bt + \beta\sin bt) = \cos bt.
\]

נקבץ איברים:
\[
(-ab^2 + 4a)\cos bt + (-\beta b^2 + 4\beta)\sin bt = \cos bt.
\]

נשווה מקדמים:
\[
\begin{cases}
-ab^2 + 4a = 1, \\[3pt]
-\beta b^2 + 4\beta = 0.
\end{cases}
\]

נפתור ונקבל:
\[
a = \frac{1}{4 - b^2}, \qquad \beta = 0.
\]

ולכן:
\[
\boxed{x_p(t) = \frac{\cos(bt)}{4 - b^2}.}
\]

\textbf{
 מקרה 2 — \(b = \pm 2\)  (תהודה)}

כאשר \(b = \pm 2\), יש להכפיל את הביטוי ב-\(t^k = t\):
\[
x_p(t) = t\big[a\cos(2t) + \beta\sin(2t)\big].
\]

נחשב נגזרות:
\[
x_p'(t) = a\cos(2t) + \beta\sin(2t) - 2at\sin(2t) + 2\beta t\cos(2t).
\]
\[
x_p''(t)
= \frac{d}{dt}\big[a\cos(2t) + \beta\sin(2t)\big]
  + \frac{d}{dt}\big[-2a t\sin(2t) + 2\beta t\cos(2t)\big].
\]

נחשב כל נגזרת בנפרד:

1.  
\[
\frac{d}{dt}\big[a\cos(2t) + \beta\sin(2t)\big]
= -2a\sin(2t) + 2\beta\cos(2t).
\]

2.  
\[
\begin{aligned}
\frac{d}{dt}\big[-2a t\sin(2t)\big]
&= -2a\sin(2t) - 4a t\cos(2t), \\[4pt]
\frac{d}{dt}\big[2\beta t\cos(2t)\big]
&= 2\beta\cos(2t) - 4\beta t\sin(2t).
\end{aligned}
\]

נחבר את כל האיברים יחד:

\[
\begin{aligned}
x_p''(t)
&= [-2a\sin(2t) + 2\beta\cos(2t)]
   + [-2a\sin(2t) - 4a t\cos(2t)]
   + [2\beta\cos(2t) - 4\beta t\sin(2t)] \\[6pt]
&= (-4a t\cos(2t) - 4\beta t\sin(2t))
   + (-4a\sin(2t) + 4\beta\cos(2t)).
\end{aligned}
\]

נסדר:
\[
x_p''(t)
= -4a t\cos(2t) - 4\beta t\sin(2t)
  - 4a\sin(2t) + 4\beta\cos(2t).
\]

נציב במשוואה:
\[
x_p'' + 4x_p = \cos(2t).
\]

נשתמש בביטויים שמצאנו קודם:
\[
x_p''(t) = -4a t\cos(2t) - 4\beta t\sin(2t)
            - 4a\sin(2t) + 4\beta\cos(2t),
\]
ו-
\[
x_p(t) = t\big[a\cos(2t) + \beta\sin(2t)\big].
\]

נכפיל את \(x_p\) ב-4:
\[
4x_p(t) = 4a t\cos(2t) + 4\beta t\sin(2t).
\]

נציב את שני הביטויים במשוואה:
\[
\begin{aligned}
x_p'' + 4x_p
&= \big[-4a t\cos(2t) - 4\beta t\sin(2t)
        - 4a\sin(2t) + 4\beta\cos(2t)\big]
   + \big[4a t\cos(2t) + 4\beta t\sin(2t)\big].
\end{aligned}
\]

נכנס איברים דומים ונקבל:
\[
x_p'' + 4x_p
= (-4a\sin(2t) + 4\beta\cos(2t)).
\]

נשווה לאגף ימין של המשוואה הנתונה:
\[
(-4a\sin(2t) + 4\beta\cos(2t)) = \cos(2t).
\]

כעת נשווה מקדמים לפי פונקציות סינוס וקוסינוס:

\[
\begin{cases}
-4a = 0, \\[3pt]
4\beta = 1.
\end{cases}
\]

נפתור:
\[
a = 0, \qquad \beta = \tfrac{1}{4}.
\]

ולכן נקבל:
\[
x_p(t) = \tfrac{t}{4}\sin(2t).
\]

לסיכום:

\[
x(t) =
\begin{cases}
C_1\cos(2t) + C_2\sin(2t) + \dfrac{\cos(bt)}{4 - b^2}, & b \ne \pm 2, \vspace{10pt}\\
C_1\cos(2t) + C_2\sin(2t) + \dfrac{t}{4}\sin(2t), & b = \pm 2.
\end{cases}
\]

\textbf{תנאי התחלה:}
\[
x(0) = 0, \qquad x'(0) = 0.
\]

נשתמש בביטוי הכללי (למקרה \(b \ne \pm2\)):
\[
x(t) = C_1\cos(2t) + C_2\sin(2t) + \frac{\cos(bt)}{4 - b^2}.
\]

נציב את התנאי הראשון \(x(0)=0\):
\[
x(0) = C_1\cos(0) + C_2\sin(0) + \frac{\cos(0)}{4 - b^2}
= C_1 + \frac{1}{4 - b^2} = 0.
\]
מכאן:
\[
C_1 = -\frac{1}{4 - b^2}.
\]

כעת נגזור את הביטוי הכללי כדי לחשב את \(x'(t)\):
\[
x'(t) = -2C_1\sin(2t) + 2C_2\cos(2t) - \frac{b\sin(bt)}{4 - b^2}.
\]

נציב \(t=0\):
\[
x'(0) = -2C_1\sin(0) + 2C_2\cos(0) - \frac{b\sin(0)}{4 - b^2}.
\]

מאחר ש-\(\sin(0)=0\) ו-\(\cos(0)=1\), נקבל:
\[
x'(0) = 2C_2.
\]

כעת נשתמש בתנאי ההתחלה השני \(x'(0)=0\):
\[
2C_2 = 0 \quad \Longrightarrow \quad C_2 = 0.
\]

מהתנאים נובע:
\[
C_1 = -\frac{1}{4 - b^2}, \qquad C_2 = 0.
\]


ולכן במקרה \(b \ne \pm 2\):
\[
x(t) = \frac{\cos(bt) - \cos(2t)}{4 - b^2}.
\]

במקרה של תהודה \(b = \pm 2\):
\[
x(t) = C_1\cos(2t) + C_2\sin(2t) + \tfrac{t}{4}\sin(2t).
\]

\textbf{תנאי התחלה:}
\[
x(0) = 0, \qquad x'(0) = 0.
\]

נחשב את הערך ההתחלתי של \(x(0)\):
\[
x(0) = C_1\cos(0) + C_2\sin(0) + \tfrac{0}{4}\sin(0)
= C_1.
\]
מהתנאי \(x(0)=0\) נקבל:
\[
C_1 = 0.
\]

כעת נגזור כדי למצוא את \(x'(t)\):

\[
\begin{aligned}
x'(t)
&= -2C_1\sin(2t) + 2C_2\cos(2t)
   + \frac{d}{dt}\Big[\tfrac{t}{4}\sin(2t)\Big].
\end{aligned}
\]

נחשב את הנגזרת של האיבר האחרון לפי כלל המכפלה:
\[
\frac{d}{dt}\Big[\tfrac{t}{4}\sin(2t)\Big]
= \tfrac{1}{4}\sin(2t) + \tfrac{t}{2}\cos(2t).
\]

נציב הכול חזרה:
\[
x'(t) = -2C_1\sin(2t) + 2C_2\cos(2t)
        + \tfrac{1}{4}\sin(2t) + \tfrac{t}{2}\cos(2t).
\]

נחשב ב-\(t=0\):
\[
x'(0)
= -2C_1\sin(0) + 2C_2\cos(0)
  + \tfrac{1}{4}\sin(0) + \tfrac{0}{2}\cos(0)
= 2C_2.
\]

נשתמש בתנאי \(x'(0)=0\):
\[
2C_2 = 0 \quad \Longrightarrow \quad C_2 = 0.
\]

\textbf{ולכן הפתרון הסופי במקרה של תהודה הוא:}
\[
\boxed{
x(t) = \tfrac{t}{4}\sin(2t).
}
\]

לסיכום:
\[
x_p(t) =
\begin{cases}
\frac{\cos(bt) - \cos(2t)}{4 - b^2}, & b \ne \pm 2, \vspace{10pt}\\
\dfrac{t}{4}\sin(2t), & b = \pm 2.
\end{cases}
\]


(ג)
\textbf{עבור } \( b = 1.4 \):
\[
x(t) = \frac{\cos(1.4t) - \cos(2t)}{2.04}.
\]

\textbf{ייצוג גרפי של } \( x(t) \):

\begin{figure}[H]
\centering
\begin{tikzpicture}
\begin{axis}[
    width=15cm, height=10cm,
    grid=both,
    xlabel={$t$}, ylabel={$x(t)$},
    axis lines=middle,
    samples=400,
    domain=0:30,
    smooth, thick,
]
\addplot[orange, thick]
  {(cos(deg(1.4*x)) - cos(deg(2*x)))/2.04};
\end{axis}
\end{tikzpicture}
\caption{%
גרף הפתרון \(x(t) = \frac{\cos(1.4t) - \cos(2t)}{2.04}\)
המתאר תנודות של מערכת בתדירות \(b = 1.4\).
}
\end{figure}

%%%CUT%%%

\newpage
\subsection{שיטת וריאציית הפרמטר}

לאחר שראינו בפרק המשוואות מסדר ראשון, כיצד ניתן לקבל פתרון למד׳׳ר לינארית לא הומוגנית מסדר 1,
נרחיב כעת את יריעת השיטה לסדר $n\geq2$.
נתבונן בצורה  הכללית למד׳׳ר מסדר $n$, מנורמלת, לא הומוגנית, ללא מקדמים קבועים בהכרח:
\begin{equation}\label{var}
y^{(n)}(x) + a_{n-1}(x)y^{(n-1)}(x) + \cdots + a_1(x)y'(x) + a_0(x)y(x) = f(x),
\end{equation}

ונתונים לנו פתרונות בסיס של המשוואה ההומוגנית המתאימה:
\[
u_1(x),\, u_2(x),\, \ldots,\, u_n(x),
\]
כאשר $u_i(x)$ הן פונקציות בת׳׳ל.
אזי, קיים פתרון פרטי מהצורה:
\begin{equation}\label{private}
y_p(x) = c_1(x)u_1(x) + c_2(x)u_2(x) + \cdots + c_n(x)u_n(x),
\end{equation}
כאשר $c_{i}(x)$ הם הפונקציות הנעלמות אותן יש למצוא.

\begin{remark}
   חובה לנרמל את המד״ר כשמשתמשים בשיטת וריאציית הפרמטר.
\end{remark}

\textbf{אלגוריתם לפתרון}

כדי למצוא את \( c_1(x),\ldots,c_n(x) \) נפתור את מערכת המשוואות הבאה ב-\( n \) נעלמים:

\[
\begin{cases}
c_1'(x)u_1(x) + c_2'(x)u_2(x) + \cdots + c_n'(x)u_n(x) = 0, \\[6pt]
c_1'(x)u_1'(x) + c_2'(x)u_2'(x) + \cdots + c_n'(x)u_n'(x) = 0, \\[6pt]
\vdots \\[6pt]
c_1'(x)u_1^{(n-1)}(x) + c_2'(x)u_2^{(n-1)}(x) + \cdots + c_n'(x)u_n^{(n-1)}(x) = f(x).
\end{cases}
\]

מכאן מוצאים את \( c_1'(x),\,\ldots,\,c_n'(x) \) ומבצעים אינטגרציה. נוכיח מדוע הפתרון הפרטי מקבל צורה זו.

\begin{proof}

נוכיח את נכונות הפתרון המוצג במשוואה \ref{private}.
נבחן את המשוואה הדיפרנציאלית הלא־הומוגנית הכללית מסדר $n$:
\begin{equation}
y^{(n)}(x) + \sum_{i=0}^{n-1} a_i(x)\,y^{(i)}(x) = f(x).
\label{eq:nonhom_general}
\end{equation}
זו למעשה משוואה \ref{var}, רק בכתיב הכולל סכימה.

יהיו 
\[
u_1(x),\,u_2(x),\,\ldots,\,u_n(x)
\]
פתרונות בסיסיים של המשוואה ההומוגנית המתאימה:
\begin{equation}
y^{(n)}(x) + \sum_{i=0}^{n-1} a_i(x)\,y^{(i)}(x) = 0.
\label{eq:hom_general}
\end{equation}

נניח כי לפתרון הפרטי של המשוואה (\ref{eq:nonhom_general}) יש את הצורה:
\begin{equation}
y_p(x) = \sum_{i=1}^{n} c_i(x)\,u_i(x),
\label{eq:yp_general}
\end{equation}
כאשר הפונקציות \(c_i(x)\) גזירות ונדרש למצוא אותן.

נניח כי
\(c_i(x)\) מקיים את התנאים:
\begin{equation}
\sum_{i=1}^{n} c_i'(x)\,u_i^{(j)}(x) = 0,
\qquad j = 0,1,\ldots,n-2.
\label{eq:aux_conditions}
\end{equation}

בהתאם לכך, נובע מתוך גזירה חוזרת של (\ref{eq:yp_general}) ושימוש בתנאים (\ref{eq:aux_conditions}) כי:
\begin{equation}
y_p^{(j)}(x) = \sum_{i=1}^{n} c_i(x)\,u_i^{(j)}(x),
\qquad j = 0,1,\ldots,n-1.
\label{eq:yp_ders}
\end{equation}

לעומת זאת, הנגזרת מהסדר ה-$n$ תכיל גם את $c_i'(x)$:
\begin{equation}
y_p^{(n)}(x) = \sum_{i=1}^{n} c_i'(x)\,u_i^{(n-1)}(x)
               + \sum_{i=1}^{n} c_i(x)\,u_i^{(n)}(x).
\label{eq:yp_n}
\end{equation}

נציב את הביטויים (\ref{eq:yp_general}), (\ref{eq:yp_ders}) ו-(\ref{eq:yp_n}) במשוואה (\ref{eq:nonhom_general}).

מכיוון שכל \(u_i(x)\) מהווים פתרונות של המשוואה ההומוגנית (\ref{eq:hom_general}),
הסכום \(\sum c_i(x)\,u_i(x)\) יתאפס באגף שמאל, ונקבל כי רק איברי \(c_i'(x)\) נותרים:
\begin{equation}
\sum_{i=1}^{n} c_i'(x)\,u_i^{(n-1)}(x) = f(x).
\label{eq:last_condition}
\end{equation}

כעת נוכל לנסח את המערכת המלאה של $n$ משוואות ב-$n$ נעלמים \(c_i'(x)\):
\begin{equation}
\left\{
\begin{aligned}
\sum_{i=1}^{n} c_i'(x)\,u_i(x) &= 0, \\[3pt]
\sum_{i=1}^{n} c_i'(x)\,u_i'(x) &= 0, \\[3pt]
\vdots \\[3pt]
\sum_{i=1}^{n} c_i'(x)\,u_i^{(n-2)}(x) &= 0, \\[3pt]
\sum_{i=1}^{n} c_i'(x)\,u_i^{(n-1)}(x) &= f(x).
\end{aligned}
\right.
\label{eq:var_system}
\end{equation}

נפתור את המערכת בעזרת נוסחת קרמר.  
נגדיר את דטרמיננטת הורונסקיאן:
\begin{equation}
W(x) =
\begin{vmatrix}
u_1(x) & u_2(x) & \cdots & u_n(x) \\[3pt]
u_1'(x) & u_2'(x) & \cdots & u_n'(x) \\[3pt]
\vdots & \vdots & \ddots & \vdots \\[3pt]
u_1^{(n-1)}(x) & u_2^{(n-1)}(x) & \cdots & u_n^{(n-1)}(x)
\end{vmatrix}.
\label{eq:wronskian}
\end{equation}

ונגדיר את $W_i(x)$ על־ידי החלפת העמודה ה-$i$ בוקטור אגף ימין \((0,0,\ldots,f(x))^T\):
\begin{equation}
W_i(x) =
\begin{vmatrix}
u_1(x) & \cdots & u_{i-1}(x) & 0 & u_{i+1}(x) & \cdots & u_n(x) \\[3pt]
u_1'(x) & \cdots & u_{i-1}'(x) & 0 & u_{i+1}'(x) & \cdots & u_n'(x) \\[3pt]
\vdots & \ddots & \vdots & \vdots & \vdots & \ddots & \vdots \\[3pt]
u_1^{(n-2)}(x) & \cdots & u_{i-1}^{(n-2)}(x) & 0 & u_{i+1}^{(n-2)}(x) & \cdots & u_n^{(n-2)}(x) \\[3pt]
u_1^{(n-1)}(x) & \cdots & u_{i-1}^{(n-1)}(x) & f(x) & u_{i+1}^{(n-1)}(x) & \cdots & u_n^{(n-1)}(x)
\end{vmatrix}.
\label{eq:Wi}
\end{equation}

בעזרת נוסחת קרמר נקבל:
\begin{equation}
c_i'(x) = \frac{W_i(x)}{W(x)}, \qquad i=1,2,\ldots,n.
\label{eq:ci_prime}
\end{equation}

לבסוף, נבצע אינטגרציה:
\begin{equation}
c_i(x) = \int \frac{W_i(x)}{W(x)}\,dx.
\label{eq:ci_integral}
\end{equation}

נציב את הביטוי (\ref{eq:ci_integral}) בחזרה לפתרון (\ref{eq:yp_general}) ונקבל את הנוסחה הסופית של וריאציית הפרמטר:
\begin{equation}
\boxed{
y_p(x) = \sum_{i=1}^{n} u_i(x) \int \frac{W_i(x)}{W(x)}\,dx.
}
\label{eq:final_vop}
\end{equation}
\end{proof}

\example

נתון כי \(u_1(x) = x\), \(u_2(x) = x^2\) הם פתרונות של המשוואה ההומוגנית המתאימה ל:
\[
x^2y'' - 2xy' + 2y = x^4, \qquad x > 0
\]
מצאו פתרון כללי למד״ר.

\explanation

\textbf{שלב 1 – נרמול המשוואה}

נחלק ב-\(x^2\) ונקבל:
\[
y'' - \frac{2}{x}y' + \frac{2}{x^2}y = x^2
\]

\textbf{שלב 2 – כתיבת הפתרון הפרטי לפי וריאציית הפרמטר}
\[
y_p(x) = C_1(x)u_1(x) + C_2(x)u_2(x)
       = C_1(x)\cdot x + C_2(x)\cdot x^2
\]

\textbf{הערה:}
הפונקציות \(C_1(x)\) ו-\(C_2(x)\) אינן קבועות אלא פונקציות גזירות, ונמצאות ע״י מערכת וריאציית הפרמטר.

\textbf{שלב 3 – הצבה במערכת הורונסקיאן}

נשתמש בתנאים:
\[
\begin{cases}
C_1'(x)\cdot u_1(x) + C_2'(x)\cdot u_2(x) = 0, \\[4pt]
C_1'(x)\cdot u_1'(x) + C_2'(x)\cdot u_2'(x) = f(x),
\end{cases}
\]
ונציב \(u_1(x)=x\), \(u_2(x)=x^2\), \(f(x)=x^2\):

\[
\begin{cases}
C_1'(x)\cdot x + C_2'(x)\cdot x^2 = 0, \\[4pt]
C_1'(x) + 2xC_2'(x) = x^2.
\end{cases}
\]

\textbf{שלב 4 – פתרון המערכת}

מהמשוואה הראשונה נקבל:
\[
C_1'(x) = -xC_2'(x).
\]

נציב זאת במשוואה השנייה:
\[
-xC_2'(x) + 2xC_2'(x) = x^2 \quad\Longrightarrow\quad C_2'(x) = x.
\]

נבצע אינטגרציה:
\[
C_2(x) = \int x\,dx = \frac{x^2}{2} + d_2.
\]

נחשב גם:
\[
C_1'(x) = -x\cdot x = -x^2 \quad\Longrightarrow\quad C_1(x) = -\int x^2dx = -\frac{x^3}{3} + d_1.
\]

\[
\begin{aligned}
y_p(x)
&= \Big(-\frac{x^3}{3} + d_1\Big)\cdot x
 + \Big(\frac{x^2}{2} + d_2\Big)\cdot x^2 \\[4pt]
&= d_1x + d_2x^2 + \frac{x^4}{6}, \qquad x > 0.
\end{aligned}
\]

\underline{בוריאציית הפרמטר} אין דרישה ללקיחת קבועי האינטגרציה
\(d_1,d_2\), מכיוון שהם מתלכדים עם מקדמי הפתרון ההומוגני:
\[
y_h(x) = A_1x + A_2x^2.
\]
לכן הפתרון הכללי הוא:
\[
\boxed{
y(x) = \textcolor{red}{y_h(x)} + \textcolor{blue}{y_p(x)}
      = \textcolor{red}{A_1x + A_2x^2} + \textcolor{blue}{\frac{x^4}{6}}, \qquad x > 0.
}
\]
    
\begin{remark}

שיטת וריאציית הפרמטר שימושית במיוחד כשקשה לנחש את הצורה של הפתרון הפרטי.
היא מבוססת על עקרון של החלפת הקבועים ב״פרמטרים משתנים״, ולכן מספקת שיטה כללית לפתרון כל משוואה ליניארית לא הומוגנית, כל עוד הפתרונות ההומוגניים ידועים.
\end{remark}

%%%CUT%%%

\example{}

מצאו פתרון כללי למד״ר:
\[
y'' - 2y' + y = xe^x + 2\tan x + \sin^2x + \frac{x}{e^x} + \frac{e^x}{x}.
\]

\explanation{}

לפנינו מד׳׳ר לינארית מסדר 2, עם מקדמים קבועים, לא הומוגנית (מאוד לא הומוגנית). כיוון שהמקדמים קבועים, נוכל לפתרון את החלק ההומוגני די בקלות באמצעות שיטת הפולינום האופייני. באשר לחלק האי-הומוגני, המצב קצת יותר מורכב. קיימים 5 איברים שיש ׳׳לטפל בהם׳׳. לא כל האיברים שמופיעים שם, יכולים להיות מטופלים באמצעות שיטת השוואת מקדמים (שהיא קלה יותר מאשר וריאציית הפרמטר), שכן לא כל האיברים עונים למבנה אשר נתנו עבור השוואת מקדמים (פולינומים, אקספוננטים, וסינוס/קוסינוס). לכן, נזדקק \textbf{גם} לשיטת וריאציית הפרמטר, אשר לא מגבילה אותנו בסוג הפונקציה בחלק האי-הומוגני, כמובן בכפוף ליכולות האינטגרציה שלנו. מכאן, מגיעה המוטביציה \textbf{לשלב} בין שתי השיטות על מנת לקבל את הפתרון הפרטי למשוואה.

מכיוון שמדובר במד״ר ליניארית, הפתרון הפרטי שלה הוא סופרפוזיציה של הפתרונות הפרטיים לכל אחד מאגפי הימין בנפרד.  
נפתור אם כך 5 מד״רים, כל אחת עבור אחד מן הגורמים שבאגף ימין.

\textbf{שלב 1 – הפתרון ההומוגני}

נכתוב את הפ״א:
\[
L(r) = r^2 - 2r + 1 = (r - 1)^2 = 0
\]

ולכן \(r_1 = r_2 = 1\), ומכאן:
\[
y_h(x) = (C_1 + C_2x)e^x.
\]

\textbf{שלב 2 – חלק ראשון: } \(y'' - 2y' + y = xe^x\)

עבור
\(xe^x\)
נוכל להשתמש בשיטת השוואת מקדמים עם הפרמטרים הבאים:
\[
a = 1, \quad b = 0, \quad m = 1, \quad k = 2.
\]
נניח פתרון פרטי:
\[
y_{p1}(x) = x^2R_1(x)e^x,
\]
כאשר \(R_1(x)\) פולינום ממעלה 1:
\[
R_1(x) = a_1x + b_1.
\]נחשב נגזרות:

\[
\begin{aligned}
y_{p1}'(x) &= e^x\big[2xR_1 + x^2R_1' + x^2R_1\big], \\[6pt]
y_{p1}''(x) &= e^x\big[x^2R_1'' + 2x^2R_1' + 4xR_1' + x^2R_1 + 4xR_1 + 2R_1\big].
\end{aligned}
\]

נציב קשרים אלה במשוואה ונחלק אותה ב-$e^{x}$:

\[
\big[x^2R_1'' + 2x^2R_1' + 4xR_1' + x^2R_1 + 4xR_1 + 2R_1\big]
- 2\big[2xR_1 + x^2R_1' + x^2R_1\big]
+ \big[x^2R_1\big]
= x.
\]

נציב \(R_1(x) = a_1x + b_1\), \(R_1'(x) = a_1\), \(R_1''(x) = 0\) במשוואה:

\[
\begin{aligned}
&\big[x^2(0) + 2x^2(a_1) + 4x(a_1) + x^2(a_1x + b_1) + 4x(a_1x + b_1) + 2(a_1x + b_1)\big] \\[4pt]
&\quad - 2\big[2x(a_1x + b_1) + x^2(a_1) + x^2(a_1x + b_1)\big]
+ \big[x^2(a_1x + b_1)\big]
= x.
\end{aligned}
\]

נפתח סוגריים ונקבץ לפי חזקות של \(x\):

\[
\begin{aligned}
&\big[2a_1x^2 + 4a_1x + a_1x^3 + b_1x^2 + 4a_1x^2 + 4b_1x + 2a_1x + 2b_1\big] \\[4pt]
&\quad - 2\big[2a_1x^2 + 2b_1x + a_1x^2 + a_1x^3 + b_1x^2\big]
+ \big[a_1x^3 + b_1x^2\big]
= x.
\end{aligned}
\]

כעת נפתח את הסוגריים השניים:

\[
\begin{aligned}
&2a_1x^2 + 4a_1x + a_1x^3 + b_1x^2 + 4a_1x^2 + 4b_1x + 2a_1x + 2b_1 \\[4pt]
&\quad -4a_1x^2 -4b_1x -2a_1x^2 -2a_1x^3 -2b_1x^2
+ a_1x^3 + b_1x^2 = x.
\end{aligned}
\]

נאסוף איברים דומים לפי חזקות של \(x\):

- איברי \(x^3\):
\[
a_1x^3 - 2a_1x^3 + a_1x^3 = 0.
\]

- איברי \(x^2\):
\[
2a_1x^2 + 4a_1x^2 + b_1x^2 - 4a_1x^2 - 2a_1x^2 - 2b_1x^2 + b_1x^2 = 0.
\]

- איברי \(x^1\):
\[
(4a_1x + 4b_1x + 2a_1x) - 4b_1x = 6a_1x.
\]

- איברי $x^{0}$:
\[
2b_1=0.
\]

נשווה מקדמים:
\[
6a_1 = 1 \quad\Longrightarrow\quad a_1 = \tfrac{1}{6}, 
\qquad 
2b_1 = 0 \quad\Longrightarrow\quad b_1 = 0.
\]

ולכן:
\[
R_1(x) = \tfrac{1}{6}x 
\quad\Longrightarrow\quad
\boxed{y_{p1}(x) = \tfrac{1}{6}x^3e^x.}
\]


\textbf{שלב 3 – חלק שני: } \(y'' - 2y' + y = 2\tan x\)

נשתמש בוריאציית הפרמטר, שכן $\tan$ לא נכללת בפונקציות שניתן לנחש במסגרת השוואת מקדמים.

הפתרונות ההומוגניים הם \(u_1 = e^x,\, u_2 = xe^x\). נכתוב את מערכת המשוואות עבור סדר 2:

\[
\begin{cases}
C_1'(x)e^x + C_2'(x)xe^x = 0, \\[4pt]
C_1'(x)e^x + C_2'(x)(1+x)e^x = 2\tan x.
\end{cases}
\]
מהראשונה נקבל \(C_1'(x) = -xC_2'(x)\).  
נציב בשנייה:
\[
(-xC_2' + (1+x)C_2')e^x = 2\tan x \Rightarrow C_2'(x)e^x = 2\tan x.
\]
מכאן:
\[
C_2'(x) = 2e^{-x}\tan x \quad\Longrightarrow\quad
C_2(x) = 2\int e^{-x}\tan x\,dx.
\]

באופן דומה:
\[
C_1'(x) = -xC_2'(x) = -2x e^{-x}\tan x \quad\Longrightarrow\quad
C_1(x) = -2\int x e^{-x}\tan x\,dx.
\]

נבטא את הפתרון הפרטי באמצעות אינטגרלים (פתרונם הוא פתרון מרוכב והרכבה של פונקציות ׳׳לא סטנדרטיות׳׳, ועל כן לא נראה אותו כאן שכן הוא מחוץ למסגרת של ספר זה):
\[
y_{p2}(x) = e^xC_1(x) + xe^xC_2(x)
          = e^x\left[-2\int x e^{-x}\tan x\,dx\right]
           + xe^x\left[2\int e^{-x}\tan x\,dx\right].
\]

\textbf{שלב 4 – חלק שלישי: } \(y'' - 2y' + y = \sin^2x\)

יש כאן שתי גישות, אחת מהן קלה יותר.
נכתוב את \(\sin^2x\) בעזרת זהות טריגונומטרית, שכן כרגע הביטוי לא מתאים לשיטת השוואת מקדמים:
\[
\sin^2x = \frac{1}{2} - \frac{1}{2}\cos(2x).
\]
כעת המקרה מתאים להשוואת מקדמים.
נפרק לשני חלקים:

\textbf{1. עבור האיבר } \(\tfrac{1}{2}\):

האגף הימני הוא קבוע:
\[
G(x) = \tfrac{1}{2}.
\]

נרשום אותו בצורתו הכללית:
\[
G(x) = \tfrac{1}{2} = \tfrac{1}{2}e^{0x}\big(\cos(0x) + \sin(0x)\big).
\]

מכאן נזהה את הפרמטרים:
\[
a = 0, \qquad b = 0, \qquad m = 0.
\]

מכיוון ש-\(a + ib = 0\) \textbf{אינו} שורש של הפ״א, נקבל:
\(
k = 0
\).

לכן נציע פתרון פרטי מן הצורה:
\[
y_p(x) = e^{0x}\left\[R_0(x)\cos(0x) + S_0(x)\sin(0x)\right]
= R_0(x).
\]

מכיוון ש-\(R_0(x)\) הוא פולינום ממעלה אפס, נרשום:
\[
R_0(x) = A,
\]
כאשר \(A\) הוא קבוע שטרם נקבע.

נחשב נגזרות:
\[
y_p'(x) = 0, \qquad y_p''(x) = 0.
\]

נציב במשוואה:
\[
y_p'' - 2y_p' + y_p = \tfrac{1}{2}
\quad\Longrightarrow\quad
A = \tfrac{1}{2}
\quad\Longrightarrow\quad
A = \tfrac{1}{2}.
\]

ולכן הפתרון הפרטי הוא:
\[
\boxed{y_{p3a}(x) = \tfrac{1}{2}.}
\]
\textbf{2. עבור האגף } \(-\tfrac{1}{2}\cos(2x)\):

נכתוב את האיבר בצורתו הכללית:
\[
G(x) = -\tfrac{1}{2}\cos(2x)
      = -\tfrac{1}{2}e^{0x}\big(\cos(2x) + 0\cdot\sin(2x)\big).
\]

מכאן נזהה את הפרמטרים:
\[
a = 0, \qquad b = 2, \qquad m = 0.
\]

מכיוון ש-\(a + ib = 2i\) \textbf{איננו} שורש של הפ״א, נקבל:
\(
k = 0
\).

לכן נציע פתרון פרטי מהצורה:
\[
y_p(x) = R_0(x)\cos(2x) + S_0(x)\sin(2x)
       = A\cos(2x) + B\sin(2x),
\]
כאשר \(A,B\) הם קבועים שיש לקבוע.

נחשב נגזרות:
\[
\begin{aligned}
y_p'(x) &= -2A\sin(2x) + 2B\cos(2x), \\[4pt]
y_p''(x) &= -4A\cos(2x) - 4B\sin(2x).
\end{aligned}
\]

נציב במשוואה:
\[
y_p'' - 2y_p' + y_p = -\tfrac{1}{2}\cos(2x),
\]
ונקבל:
\[
(-4A\cos(2x) - 4B\sin(2x))
- 2(-2A\sin(2x) + 2B\cos(2x))
+ (A\cos(2x) + B\sin(2x))
= -\tfrac{1}{2}\cos(2x).
\]

נפתח סוגריים ונקבץ לפי \(\cos(2x)\) ו-\(\sin(2x)\):
\[
(-4A - 4B + A)\cos(2x) + (-4B + 4A + B)\sin(2x)
= -\tfrac{1}{2}\cos(2x).
\]
נשווה מקדמים של הפונקציות הבלתי־תלויות \(\cos(2x)\) ו-\(\sin(2x)\):

\[
\begin{cases}
-3A - 4B = -\tfrac{1}{2}, \\[4pt]
4A - 3B = 0.
\end{cases}
\]

נפתור את המערכת:

מן המשוואה השנייה:
\[
4A = 3B
\quad\Longrightarrow\quad
A = \tfrac{3}{4}B.
\]

נציב במשוואה הראשונה:
\[
-3\left(\tfrac{3}{4}B\right) - 4B = -\tfrac{1}{2}
\quad\Longrightarrow\quad
-\tfrac{9}{4}B - 4B = -\tfrac{1}{2}.
\]

נפשט:
\[
-\tfrac{25}{4}B = -\tfrac{1}{2}
\quad\Longrightarrow\quad
B = \tfrac{1}{2} \cdot \tfrac{4}{25} = \tfrac{2}{25}.
\]

ולכן:
\[
A = \tfrac{3}{4}B = \tfrac{3}{4}\cdot\tfrac{2}{25} = \tfrac{3}{50}.
\]

נחזיר לביטוי:
\[
y_{p3b}(x) = \tfrac{3}{50}\cos(2x) + \tfrac{2}{25}\sin(2x).
\]

ולבסוף, נשלב את שני הרכיבים שמצאנו — הקבוע והטריגונומטרי:
\[
\boxed{
y_{p3}(x) = \tfrac{1}{2} + \tfrac{3}{50}\cos(2x) + \tfrac{2}{25}\sin(2x).
}
\]
כעת נציג כיצד ניתן למצוא את הפתרון הפרטי עבור \( \sin^2(x) \) באמצעות שיטת וריאציית הפרמטרים (פחות מומלץ).

הפתרון ההומוגני של המשוואה
\[
y'' - 2y' + y = 0
\]
הוא:
\[
y_h(x) = (C_1 + C_2x)e^x.
\]
לכן נבחר פתרון פרטי מהצורה:
\[
y_p(x) = u_1(x)e^x + u_2(x)xe^x,
\]
כאשר \(u_1(x)\) ו-\(u_2(x)\) פונקציות גזירות שיש לקבוע.

נקבל מערכת של שתי משוואות בשני נעלמים \(u_1'(x), u_2'(x)\):
\[
\begin{cases}
u_1'(x)e^x + u_2'(x)xe^x = 0, \\[6pt]
u_1'(x)e^x + u_2'(x)e^x(x+1) = \sin^2(x).
\end{cases}
\]
נפתור את המערכת לפי נוסחת קרמר.

באופן כללי, עבור משוואה דיפרנציאלית מסדר שני מן הצורה
\[
y'' + P(x)y' + Q(x)y = f(x),
\]
ונניח כי \(y_1(x), y_2(x)\) הם פתרונות הומוגניים בלתי־תלויים,  
אז לפי נוסחת קרמר נקבל:
\[
u_1'(x) = -\frac{y_2(x)f(x)}{W(x)}, 
\qquad
u_2'(x) = \frac{y_1(x)f(x)}{W(x)},
\]
כאשר \(W(x)\) הוא הורונסקיאן:
\[
W(x) = 
\begin{vmatrix}
y_1(x) & y_2(x) \\[4pt]
y_1'(x) & y_2'(x)
\end{vmatrix}.
\]

במקרה שלנו \(y_1(x)=e^x, \ y_2(x)=xe^x, \ f(x)=\sin^2x\), ולכן:
\[
W(x) =
\begin{vmatrix}
e^x & xe^x \\[4pt]
e^x & e^x(1+x)
\end{vmatrix}
= e^{2x}.
\]

נציב בנוסחאות הכלליות:
\[
\begin{aligned}
u_1'(x) &= -\frac{y_2(x)f(x)}{W(x)} 
= -\frac{xe^x \sin^2x}{e^{2x}} 
= -x e^{-x}\sin^2x, \\[6pt]
u_2'(x) &= \frac{y_1(x)f(x)}{W(x)} 
= \frac{e^x \sin^2x}{e^{2x}} 
= e^{-x}\sin^2x.
\end{aligned}
\]

נחשב את האינטגרלים:
\[
u_1(x) = -\int xe^{-x}\sin^2x\,dx,
\qquad
u_2(x) = \int e^{-x}\sin^2x\,dx.
\]

נשתמש בזהות \(\sin^2x = \frac{1 - \cos(2x)}{2}\):

\[
u_2(x) = \tfrac{1}{2}\int e^{-x}(1 - \cos(2x))\,dx
= \tfrac{1}{2}\left(\int e^{-x}\,dx - \int e^{-x}\cos(2x)\,dx\right).
\]

נחשב כל אינטגרל בנפרד:

\[
\int e^{-x}\,dx = -e^{-x}.
\]

נבצע אינטגרציה בחלקים עבור \(\int e^{-x}\cos(2x)\,dx\):

נבחר \(u = \cos(2x)\), \(dv = e^{-x}dx\), נקבל \(du = -2\sin(2x)\,dx\), \(v = -e^{-x}\).

\[
\int e^{-x}\cos(2x)\,dx = -e^{-x}\cos(2x) - \int (-e^{-x})(-2\sin(2x))\,dx
= -e^{-x}\cos(2x) - 2\int e^{-x}\sin(2x)\,dx.
\]

כעת נחשב \(\int e^{-x}\sin(2x)\,dx\) באותו אופן:
נבחר \(u = \sin(2x)\), \(dv = e^{-x}dx\), נקבל \(du = 2\cos(2x)dx\), \(v = -e^{-x}\):
\[
\int e^{-x}\sin(2x)\,dx = -e^{-x}\sin(2x) + 2\int e^{-x}\cos(2x)\,dx.
\]

נסמן \(I = \int e^{-x}\cos(2x)\,dx\), \(J = \int e^{-x}\sin(2x)\,dx\).

מהשוויונות:
\[
I = -e^{-x}\cos(2x) - 2J, \qquad J = -e^{-x}\sin(2x) + 2I.
\]
נציב את \(J\) מהשנייה בראשונה:
\[
I = -e^{-x}\cos(2x) - 2[-e^{-x}\sin(2x) + 2I] = -e^{-x}\cos(2x) + 2e^{-x}\sin(2x) - 4I.
\]
ולכן:
\[
5I = e^{-x}(-\cos(2x) + 2\sin(2x)) \quad\Longrightarrow\quad
I = \frac{e^{-x}}{5}(-\cos(2x) + 2\sin(2x)).
\]

נחזיר ל־\(u_2\):
\[
u_2(x) = \tfrac{1}{2}\Big[-e^{-x} - I\Big]
= \tfrac{1}{2}\Big[-e^{-x} - \tfrac{e^{-x}}{5}(-\cos(2x) + 2\sin(2x))\Big]
= \tfrac{e^{-x}}{10}\big(-5 + \cos(2x) - 2\sin(2x)\big).
\]

כעת נעבור ל־\(u_1(x)\):
\[
u_1(x) = -\int x e^{-x}\sin^2x\,dx
= -\tfrac{1}{2}\int x e^{-x}(1 - \cos(2x))\,dx
= -\tfrac{1}{2}(I_1 - I_2),
\]
כאשר:
\[
I_1 = \int x e^{-x}\,dx, \qquad I_2 = \int x e^{-x}\cos(2x)\,dx.
\]

נחשב את \(I_1\) לפי חלקים:

נבחר \(u=x,\ dv=e^{-x}dx\), לכן \(du=dx,\ v=-e^{-x}\):
\[
I_1 = -xe^{-x} + \int e^{-x}\,dx = -xe^{-x} - e^{-x} = -(x+1)e^{-x}.
\]

כעת נחשב את \(I_2\) באותה שיטה:

נבחר \(u=x,\ dv = e^{-x}\cos(2x)\,dx\).  
לכן \(du=dx\), \(v = I\) מהחישוב הקודם.

\[
I_2 = xI - \int I\,dx.
\]
כבר ידוע:
\[
I = \frac{e^{-x}}{5}(-\cos(2x) + 2\sin(2x)).
\]

נחשב את \(\int I\,dx\) בעזרת אינטגרציה נוספת בחלקים:
\[
\int I\,dx = \int \frac{e^{-x}}{5}\big(-\cos(2x) + 2\sin(2x)\big)\,dx
= \frac{1}{5}\int e^{-x}\big(-\cos(2x) + 2\sin(2x)\big)\,dx.
\]

נפריד לשני אינטגרלים:
\[
\int I\,dx
= \frac{1}{5}\Big[-\int e^{-x}\cos(2x)\,dx + 2\int e^{-x}\sin(2x)\,dx\Big].
\]

ידוע מהחישוב הקודם:
\[
\begin{aligned}
\int e^{-x}\cos(2x)\,dx &= \frac{e^{-x}}{5}\big(-\cos(2x) + 2\sin(2x)\big), \\[4pt]
\int e^{-x}\sin(2x)\,dx &= \frac{e^{-x}}{5}\big(-2\cos(2x) - \sin(2x)\big).
\end{aligned}
\]

נציב ונחשב:

\[
\begin{aligned}
\int I\,dx
&= \frac{1}{5}\!\left[
-\frac{e^{-x}}{5}\big(-\cos(2x) + 2\sin(2x)\big)
+ 2\cdot\frac{e^{-x}}{5}\big(-2\cos(2x) - \sin(2x)\big)
\right] \\[6pt]
&= \frac{e^{-x}}{25}\!\left[
(\cos(2x) - 2\sin(2x)) + (-4\cos(2x) - 2\sin(2x))
\right].
\end{aligned}
\]

נחבר איברים דומים:
\[
\int I\,dx = \frac{e^{-x}}{25}\big(-3\cos(2x) - 4\sin(2x)\big)
= -\frac{e^{-x}}{25}\big(3\cos(2x) + 4\sin(2x)\big).
\]

נציב זאת בחזרה בביטוי של \(I_2\):

\[
I_2 = xI - \int I\,dx
= \frac{x e^{-x}}{5}\big(-\cos(2x) + 2\sin(2x)\big)
+ \frac{e^{-x}}{25}\big(3\cos(2x) + 4\sin(2x)\big).
\]

נציב ל־\(u_1(x)\):

\[
\begin{aligned}
u_1(x)
&= -\tfrac{1}{2}\big[I_1 - I_2\big]
= -\tfrac{1}{2}\Big[-(x+1)e^{-x} - I_2\Big]
= \tfrac{1}{2}\Big[(x+1)e^{-x} + I_2\Big].
\end{aligned}
\]

נציב את \(I_2\):
\[
\begin{aligned}
u_1(x)
&= e^{-x}\!\Bigg[
\tfrac{1}{2}(x+1)
+ \tfrac{1}{10}\big(-x\cos(2x) + 2x\sin(2x)\big)
+ \tfrac{1}{50}\big(3\cos(2x) + 4\sin(2x)\big)
\Bigg].
\end{aligned}
\]

כעת נציב את שני הפונקציות \(u_1(x)\) ו-\(u_2(x)\) בביטוי של הפתרון הפרטי:
\[
y_p(x) = u_1(x)e^x + u_2(x)xe^x=\]\[
e^x\!\left[
e^{-x}\!\Big(
\tfrac{1}{2}(x+1)
+ \tfrac{1}{10}(-x\cos(2x) + 2x\sin(2x))
+ \tfrac{1}{50}(3\cos(2x) + 4\sin(2x))
\Big)
\right] + \]\[x e^x\!\left[\tfrac{e^{-x}}{10}\big(-5 + \cos(2x) - 2\sin(2x)\big)\right]
\]

נכנס איברים:

\[
\begin{aligned}
y_p(x)
&= \tfrac{1}{2}(x+1)
+ \tfrac{1}{10}(-x\cos(2x) + 2x\sin(2x))
+ \tfrac{1}{50}(3\cos(2x) + 4\sin(2x)) \\[3pt]
&\quad + \tfrac{x}{10}\big(-5 + \cos(2x) - 2\sin(2x)\big).
\end{aligned}
\]

לאחר כינוס וסידור, נקבל את הפתרון הפרטי (הזהה למה שקיבלנו בהשוואת מקדמים):

\[
y_{p3}(x) = \tfrac{1}{2} + \tfrac{3}{50}\cos(2x) + \tfrac{2}{25}\sin(2x).
\]


\textbf{שלב 5 – חלק רביעי: } \(y'' - 2y' + y = \tfrac{x}{e^x}\)

נפתור בעזרת שיטת השוואת מקדמים.
נכתוב את אגף ימין כך:
\[
G(x) = x e^{-x}.
\]

נזהה את הפרמטרים:
\[
a = -1, \qquad b = 0, \qquad m = 1.
\]

כיוון ש-\(a = -1\) שונה מהשורש \(r = 1\), נקבל:
\(
k = 0
\).

נציע פתרון פרטי מהצורה:
\[
y_p(x) = R_1(x)e^{-x}, \quad R_1(x) = a_4x + b_4.
\]

נחשב נגזרות:

\[
\begin{aligned}
y_p'(x)
&= e^{-x}(R_1'(x) - R_1(x))
= e^{-x}(a_4 - (a_4x + b_4))
= e^{-x}(-a_4x + (a_4 - b_4)), \\[4pt]
y_p''(x)
&= e^{-x}(R_1'' - 2R_1' + R_1)
= e^{-x}(0 - 2a_4 + (a_4x + b_4))
= e^{-x}(a_4x + b_4 - 2a_4).
\end{aligned}
\]

נציב במשוואה:
\[
y_p'' - 2y_p' + y_p = x e^{-x}.
\]

נחשב את אגף שמאל:

\[
\begin{aligned}
y_p'' - 2y_p' + y_p
&= e^{-x}\Big[(a_4x + b_4 - 2a_4)
- 2(-a_4x + (a_4 - b_4))
+ (a_4x + b_4)\Big].
\end{aligned}
\]

נפתח ונכנס איברים:

\[
=e^{-x}\Big[
a_4x + b_4 - 2a_4
+ 2a_4x - 2a_4 + 2b_4
+ a_4x + b_4
\Big] = e^{-x}\Big[
(4a_4x) + (4b_4 - 4a_4)
\Big].
\]

נשווה לאגף ימין \(x e^{-x}\):

\[
4a_4x + (4b_4 - 4a_4) = x.
\]

נשווה מקדמים:

\[
\begin{cases}
4a_4 = 1, \\[3pt]
4b_4 - 4a_4 = 0.
\end{cases}
\]

נפתור:

\[
a_4 = \tfrac{1}{4}, \qquad b_4 = a_4 = \tfrac{1}{4}.
\]

נחזיר לביטוי:
\[
R_1(x) = \tfrac{1}{4}x + \tfrac{1}{4}.
\]

ולכן הפתרון הפרטי הוא:
\[
y_{p4}(x) = \left(\tfrac{1}{4}x + \tfrac{1}{4}\right)e^{-x}.
\]

\textbf{שלב 6 – חלק חמישי: } \(y'' - 2y' + y = \dfrac{e^x}{x}\)

נשתמש בשיטת וריאציית הפרמטרים כדי למצוא פתרון פרטי.  
נזכור כי הפתרון ההומוגני של המשוואה הוא:
\[
y_h(x) = (C_1 + C_2x)e^x.
\]
מכאן נבחר פתרון פרטי מהצורה:
\[
y_p(x) = C_1(x)e^x + C_2(x)xe^x,
\]
כאשר \(C_1(x)\) ו-\(C_2(x)\) הן פונקציות גזירות שעלינו למצוא.
המערכת שעלינו לפתור היא:
\[
\begin{cases}
C_1'(x)e^x + C_2'(x)xe^x = 0, \\[4pt]
C_1'(x)e^x + C_2'(x)(1+x)e^x = f(x),
\end{cases}
\]
כאשר במקרה שלנו \(f(x) = \dfrac{e^x}{x}\).

נחלק את שתי המשוואות ב-\(e^x\) (שאינו מתאפס) ונקבל מערכת פשוטה יותר:
\[
\begin{cases}
C_1'(x) + xC_2'(x) = 0, \\[4pt]
C_1'(x) + (1+x)C_2'(x) = \dfrac{1}{x}.
\end{cases}
\]

נחסיר את המשוואה הראשונה מהשנייה כדי לבודד את \(C_2'(x)\):

\[
\big[(1+x) - x\big]C_2'(x) = \frac{1}{x}
\quad\Longrightarrow\quad
C_2'(x) = \frac{1}{x}.
\]

נבצע אינטגרציה:
\[
C_2(x) = \int \frac{1}{x}\,dx = \ln|x|.
\]

כעת נחזור למשוואה הראשונה:
\[
C_1'(x) + xC_2'(x) = 0
\quad\Longrightarrow\quad
C_1'(x) = -xC_2'(x) = -x \cdot \frac{1}{x} = -1.
\]

נבצע אינטגרציה:
\[
C_1(x) = -x.
\]

נציב את \(C_1(x)\) ו-\(C_2(x)\) בפתרון הפרטי ונקבל:
\[
y_{p5}(x) = e^x\big(x\ln|x| - x\big).
\]

\textbf{שלב 7 – סיכום:}

הפתרון ההומוגני:
\[
y_h(x) = (C_1 + C_2x)e^x.
\]

הפתרון הכללי הוא סכום של הפתרון ההומוגני והפרטיים שמצאנו בכל השלבים:
\[
\boxed{
y(x) = c_1e^x + c_2xe^x + \sum_{i=1}^{5} y_{pi}(x), {\,x\in\mathbb{R}\;|\;x \neq 0,\ x \neq \tfrac{\pi}{2} + k\pi,\ k\in\mathbb{Z}}}.
\]
כאשר:
\[
\begin{aligned}
y_{p1}(x) &= \tfrac{1}{6}x^3 e^x, \\[6pt]
y_{p2}(x) &= e^x\!\left[-2\!\int x e^{-x}\tan x\,dx + 2x\!\int e^{-x}\tan x\,dx\right], \\[6pt]
y_{p3}(x) &= \tfrac{1}{2} + \tfrac{3}{50}\cos(2x) + \tfrac{2}{25}\sin(2x), \\[6pt]
y_{p4}(x) &= \left(\tfrac{1}{4}x + \tfrac{1}{4}\right)e^{-x}, \\[6pt]
y_{p5}(x) &= e^x(x\ln|x| - x).
\end{aligned}
\]

\begin{remark}
    שימו לב לחסרון וליתרון של שיטת וריאצית הפרמטר, על פני שיטת השוואת מקדמים. החסרון הוא שאנו צריכים לפתור אינטגרלים, שלעיתים עלולים להיות מסובכים. בהשוואת מקדמים אנחנו בסה׳׳כ גוזרים ומבצעים השוואת מקדמים במסגרת של משוואות אלגבריות. היתרון הוא שוריאצית הפרמטר למעשה, יכולה ׳׳לטפל׳׳ בכל אגף אי-הומוגני של משוואה ולהתמודד עם כל פונקציה (בכפוף ליכולות האינטגרציה שלנו כמובן), תכונה שאין בידי שיטת השוואת המקדמים, אשר נותנת לנו מענה מצומצם (אך יעיל מאוד) לפונקציות אקספוננציאליות, פולינומים וסינוס/קוסינוס בלבד.
\end{remark}

%%%CUT%%%

\newpage
\underline{תרגילים}
\exercise{}

מצאו פתרון כללי למד״ר:
\[
y'' - 2y' + y = \frac{e^x}{1 + x^2}.
\]

\exercise{}

מצאו פתרון כללי למשוואה:
\[
t\,y'' - (t+1)\,y' + y = t^2,
\]
כאשר ידוע כי \(y_1(t)=e^t,\ y_2(t)=t+1\) הם פתרונות בלתי־תלויים של המשוואה ההומוגנית. $t\in\mathbb{R}$.



\newpage
\underline{פתרונות}
\solution{}

מדובר במשוואה ליניארית מסדר שני עם מקדמים קבועים, אך אגף ימין \(\dfrac{e^x}{1+x^2}\) אינו מתאים לשיטת השוואת מקדמים, שכן הוא אינו מהצורות \(P_m(x)e^{ax}\cos bx\) או \(P_m(x)e^{ax}\sin bx\).  
לכן ניעזר בשיטת וריאציית הפרמטרים כדי למצוא את הפתרון הפרטי.

\textbf{שלב 1 – הפתרון ההומוגני}

נכתוב את הפ״א:
\[
L(r) = r^2 - 2r + 1 = (r - 1)^2 = 0.
\]

ולכן:
\[
r_1 = r_2 = 1
\quad\Longrightarrow\quad
y_h(x) = (C_1 + C_2x)e^x.
\]

\textbf{שלב 2 – מציאת הפתרון הפרטי בשיטת וריאציית הפרמטרים}

נניח כי הפתרון הפרטי מהצורה:
\[
y_p(x) = u_1(x)e^x + u_2(x)xe^x,
\]
כאשר \(u_1(x)\) ו-\(u_2(x)\) פונקציות גזירות שיש למצוא.

נכתוב את מערכת המשוואות:
\[
\begin{cases}
u_1'(x)e^x + u_2'(x)xe^x = 0, \\[4pt]
u_1'(x)e^x + u_2'(x)(1+x)e^x = \dfrac{e^x}{1+x^2}.
\end{cases}
\]
נפתור את המערכת בעזרת נוסחת קרמר.

הוורונסקיאן:
\[
W(x) =
\begin{vmatrix}
e^x & xe^x \\[4pt]
e^x & (1+x)e^x
\end{vmatrix}
= e^{2x}.
\]

נציב בנוסחאות הכלליות:
\[
\begin{aligned}
u_1'(x) &= -\frac{y_2(x)f(x)}{W(x)} 
= -\frac{xe^x \cdot \frac{e^x}{1+x^2}}{e^{2x}}
= -\frac{x}{1+x^2}, \\[6pt]
u_2'(x) &= \frac{y_1(x)f(x)}{W(x)} 
= \frac{e^x \cdot \frac{e^x}{1+x^2}}{e^{2x}}
= \frac{1}{1+x^2}.
\end{aligned}
\]

נחשב את האינטגרלים:

\[
u_1(x) = -\int \frac{x}{1+x^2}\,dx, 
\qquad
u_2(x) = \int \frac{1}{1+x^2}\,dx.
\]

נבצע הצבות:

- עבור \(u_1(x)\): נציב \(t = 1+x^2 \Rightarrow dt = 2x\,dx\)
\[
u_1(x) = -\tfrac{1}{2}\int \frac{dt}{t} = -\tfrac{1}{2}\ln|1+x^2|.
\]

- עבור \(u_2(x)\):
\[
u_2(x) = \arctan(x).
\]

\textbf{שלב 3 – הצבה לביטוי הכללי של הפתרון הפרטי}

נחזיר לביטוי:
\[
y_p(x) = u_1(x)e^x + u_2(x)xe^x.
\]

נציב את התוצאות שמצאנו:
\[
y_p(x) = e^x\!\left[-\tfrac{1}{2}\ln(1+x^2) + x\arctan(x)\right].
\]

\textbf{שלב 4 – סיכום הפתרון הכללי}

הפתרון ההומוגני:
\[
y_h(x) = (C_1 + C_2x)e^x.
\]

נחבר את הפתרון הפרטי ונקבל את הפתרון הכללי:
\[
\boxed{
y(x) = (C_1 + C_2x)e^x
+ e^x\!\left[x\arctan(x) - \tfrac{1}{2}\ln(1+x^2)\right]
,\qquad x\in\mathbb{R}.}
\]


\solution{}

נחלק תחילה את המשוואה ב-\(t\) כדי לקבל את הצורה המנורמלת ($t\neq0$):
\[
y'' - \left(1 + \frac{1}{t}\right)y' + \frac{1}{t}y = t.
\]

הפתרון ההומוגני ידוע:
\[
y_h(t) = C_1e^t + C_2(t+1).
\]

\textbf{שלב 1 – נחשב את הורונסקיאן}
\[
W(t) =
\begin{vmatrix}
e^t & t+1 \\[4pt]
e^t & 1
\end{vmatrix}
= e^t - e^t(t+1) = -t e^t.
\]

\textbf{שלב 2 – נשתמש בשיטת וריאציית הפרמטרים}

לפי הנוסחה הכללית:
\[
y_p(t)
= -y_1(t)\int \frac{y_2(t)f(t)}{W(t)}\,dt
  + y_2(t)\int \frac{y_1(t)f(t)}{W(t)}\,dt,
\]
כאשר \(f(t) = t\).

נחשב כל אחד מהאיברים.

\[
\frac{y_2 f}{W}
= \frac{(t+1)t}{-t e^t}
= -\frac{t+1}{e^t},
\qquad
\frac{y_1 f}{W}
= \frac{e^t t}{-t e^t}
= -1.
\]

נציב:
\[
y_p(t)
= -e^t\int\!\left(-\frac{t+1}{e^t}\right)dt
+ (t+1)\int(-1)\,dt
= e^t\int (t+1)e^{-t}\,dt - (t+1)t.
\]

\textbf{שלב 3 – נחשב את האינטגרל:}
\[
\int (t+1)e^{-t}\,dt
= \int te^{-t}\,dt + \int e^{-t}\,dt.
\]

נחשב את הראשון לפי חלקים:  
נבחר \(u=t,\ dv=e^{-t}dt \Rightarrow du=dt,\ v=-e^{-t}\).

\[
\int te^{-t}\,dt = -te^{-t} + \int e^{-t}\,dt = -te^{-t} - e^{-t}.
\]

ולכן:
\[
\int (t+1)e^{-t}\,dt = (-te^{-t} - e^{-t}) - e^{-t} = -e^{-t}(t+2).
\]

\textbf{שלב 4 – נציב חזרה ל-\(y_p(t)\):}
\[
y_p(t) = e^t\big[-e^{-t}(t+2)\big] - (t+1)t
= -(t+2) - t(t+1)
= -t^2 - 2t - 2.
\]

\textbf{שלב 5 – נכתוב את הפתרון הכללי:}
\[
\boxed{
y(t) = C_1e^t + C_2(t+1) - t^2 - 2t - 2,\,\qquad t\neq0.
}
\]

%%%CUT%%%

\newpage

\subsection{משוואת אוילר}

נביט במשוואה הדיפרנציאלית הלינארית, לא הומוגנית, מסדר $n$, עם הסימטריה הבאה:
\begin{equation}
a_n x^n y^{(n)} + a_{n-1} x^{n-1} y^{(n-1)} + \dots + a_1 x y' + a_0 y = b(x).
\end{equation}
שימו לב כי בכל איבר מופיעה חזקה של \(x\) הזהה לסדר הנגזרת של \(y\).  
זהו \textbf{מאפיין הסימטריה} של משוואות אוילר — כל נגזרת מוכפלת באותו סדר חזקה של \(x\).  

תכונה זו מעניקה למשוואות אוילר כוח רב:  
באמצעות הצבה מתאימה ניתן להפוך את המשוואה \textbf{שאינה בעלת מקדמים קבועים}  
למשוואה בעלת \textbf{מקדמים קבועים} (במישור אחר), ובכך להשתמש בכלים מוכרים כמו פולינום אופייני,  
מבלי להזדקק לידיעה מוקדמת של הפתרון ההומוגני כפי שנדרש למשל בשיטת וריאציית הפרמטרים.

עם זאת — הסימטריה גם \textbf{מגבילה אותנו}:  
אם לא מתקיים מבנה כזה (כלומר, החזקה של \(x\) אינה תואמת לסדר הנגזרת),  
לא נוכל להשתמש בשיטה כלל, והיא מאבדת את יתרונה.

על מנת לעבור למשוואה עם מקדמים קבועים, נבצע את ההצבה הבאה:
\[
x = \pm e^t \quad \Longleftrightarrow \quad t = \ln|x|.
\]
הטרנספורמציה שלנו תהיה בהתאם לערכים של $x$:
\[
\left\{
\begin{array}{ll}
x = e^{t}, & x > 0, \\[6pt]
x = -e^{t}, & x < 0.
\end{array}
\right\}\]
מכאן נקבל את הקשר ההפוך:
\[
Y(t) = y(\pm e^t),ֿ\qquad y(x) = Y(\ln|x|).
\]

על מנת לעבור מ-$x$ ל-$t$,
נשתמש בכלל השרשרת. נקבל את הקשרים בין הנגזרות:
\[
\frac{dy}{dx} = \frac{dY}{dt} \cdot \frac{dt}{dx} = \frac{1}{x}Y'(t),
\]
נגזור שוב לפי \( x \):
\[
\frac{d^2y}{dx^2} = \frac{d}{dx}\left(\frac{1}{x}Y'(t)\right).
\]

נשתמש בכלל המכפלה:
\[
\frac{d^2y}{dx^2} = 
\underbrace{\frac{d}{dx}\left(\frac{1}{x}\right)}_{-\frac{1}{x^2}}Y'(t)
+ 
\frac{1}{x}\underbrace{\frac{dY'(t)}{dx}}_{Y''(t)\cdot\frac{1}{x}}.
\]

נפשט:
\[
\frac{d^2y}{dx^2} 
= -\frac{1}{x^2}Y'(t) + \frac{1}{x^2}Y''(t)
= \frac{1}{x^2}\big[Y''(t) - Y'(t)\big].
\]
ניתן להמשיך לקבל נגזרות מסדרים גבוהים יותר בצורה דומה.

נגזור שוב לפי \(x\):
\[
\frac{d^3y}{dx^3} = 
\frac{d}{dx}\left(\frac{1}{x^2}[Y''(t) - Y'(t)]\right).
\]
נשתמש בכלל המכפלה:
\[
\frac{d^3y}{dx^3} =
\underbrace{\frac{d}{dx}\left(\frac{1}{x^2}\right)}_{-2/x^3}[Y''(t) - Y'(t)]
+ \frac{1}{x^2}\underbrace{\frac{d}{dx}[Y''(t) - Y'(t)]}_{(Y'''(t)-Y''(t))/x}.
\]
נפשט:
\[
\frac{d^3y}{dx^3}
= -\frac{2}{x^3}[Y'' - Y'] + \frac{1}{x^3}[Y''' - Y'']
= \frac{1}{x^3}\big[Y'''(t) - 3Y''(t) + 2Y'(t)\big].
\]

נגזור שוב:
\[
\frac{d^4y}{dx^4} = \frac{d}{dx}\left(\frac{1}{x^3}[Y''' - 3Y'' + 2Y']\right).
\]
נשתמש בכלל המכפלה:
\[
\frac{d^4y}{dx^4} =
\underbrace{\frac{d}{dx}\left(\frac{1}{x^3}\right)}_{-3/x^4}[Y''' - 3Y'' + 2Y']
+ \frac{1}{x^3}\underbrace{\frac{d}{dx}[Y''' - 3Y'' + 2Y']}_{(Y^{(4)} - 3Y''' + 2Y'')/x}.
\]
נפשט:
\[
\frac{d^4y}{dx^4}
= -\frac{3}{x^4}[Y''' - 3Y'' + 2Y'] + \frac{1}{x^4}[Y^{(4)} - 3Y''' + 2Y''].
\]
נפתח סוגריים:
\[
\frac{d^4y}{dx^4}
= \frac{1}{x^4}\big[Y^{(4)}(t) - 6Y'''(t) + 11Y''(t) - 6Y'(t)\big].
\]

נציב ביטויים אלו במשוואת אוילר המקורית, ונקבל משוואה חדשה עבור \(Y(t)\) עם \textbf{מקדמים קבועים}.

\textbf{מקרה לדוגמה: משוואת אוילר מסדר שני}


נניח כי \(x>0\), ונבצע את ההצבה:
\[
x = e^{t} \quad \Longleftrightarrow \quad t = \ln x, \qquad Y(t) = y(e^{t}).
\]

נחשב את הנגזרות לפי כלל השרשרת:
\[
\frac{dy}{dx} = \frac{1}{x}Y'(t), 
\qquad
\frac{d^2y}{dx^2} = \frac{1}{x^2}\big[Y''(t) - Y'(t)\big].
\]

נכתוב את המשוואה:
\begin{equation}\label{euler_2}
a_2 x^2 y'' + a_1 x y' + a_0 y = b(x).
\end{equation}
נחליף את הנגזרות שבמונחי \(x\) בביטויים שבמונחי \(t\):

\[
\begin{aligned}
x^2 y'' &= x^2 \cdot \frac{1}{x^2}\big[Y''(t) - Y'(t)\big]
= Y''(t) - Y'(t), \\[6pt]
x y' &= x \cdot \frac{1}{x} Y'(t)
= Y'(t), \\[6pt]
y &= Y(t).
\end{aligned}
\]

נציב במשוואה \ref{euler_2} :
\begin{equation}
a_2 [Y''(t) - Y'(t)] + a_1 Y'(t) + a_0 Y(t) = b(e^{t}).
\end{equation}

נפשט:
\[
a_2 Y''(t) + (a_1 - a_2) Y'(t) + a_0 Y(t) = b(e^{t}).
\]


 הצבת \(Y = e^{rt}\) מניבה את הפולינום האופייני :
\begin{equation}\label{polyn}
L(r) = a_n r(r-1)(r-2)\cdots(r-n+1)
+ a_{n-1} r(r-1)\cdots(r-n+2)
+ \dots + a_1 r + a_0.
\end{equation}
מכאן נקבל את ההומוני ב-$t$, כאשר ברקע ניתן לחזור בקלות ל-$x$. לאחר מכן, פותרים את החלק האי-הומוגני מאחת השיטות שנלמדו. לבסוף, מבצעים סופרפוזיציה של הפתרונות.


\textbf{גישה חלופית — הצבה ישירה $y = x^r$}

ראינו כי בעזרת ההצבה \(x = e^t\) ניתן להעביר את משוואת אוילר
למשוואה עם מקדמים קבועים במשתנה \(t\),
שפתרונה ההומוגני הוא מהצורה \(Y = e^{rt}\).
נשים לב כעת לעובדה חשובה:  
אם \(Y = e^{rt}\) הוא פתרון במשתנה \(t\),
אזי במשתנה המקורי \(x = e^{t}\) מתקיים:
\[
Y = e^{rt} = (e^{t})^{r} = x^{r}.
\]
מכאן נובע שנוכל \textbf{להניח ישירות} במשתנה \(x\) כי
\[
y(x) = x^{r}
\]
הוא פתרון הומוגני אפשרי של משוואת אוילר.

גישה זו מאפשרת לנו לוותר על המעבר למשתנה \(t\)
ולפתור את המשוואה \textbf{באופן ישיר ופשוט יותר}.

\textbf{הצבה כללית עבור סדר $n$}

נניח אפוא כי \(y = x^{r}\).
נחשב את הנגזרות:
\[
\begin{aligned}
y' &= r x^{r-1},\\
y'' &= r(r-1)x^{r-2},\\
&\vdots\\
y^{(n)} &= r(r-1)(r-2)\cdots(r-n+1)x^{r-n}.
\end{aligned}
\]

נחזור למשוואת אוילר ההומוגנית:
\[
a_n x^n y^{(n)} + a_{n-1}x^{n-1}y^{(n-1)} + \dots + a_1 x y' + a_0 y = 0.
\]

נחליף את הנגזרות שחישבנו:
\[
\begin{aligned}
a_n x^n \big[r(r-1)\cdots(r-n+1)x^{r-n}\big]
&+ a_{n-1}x^{n-1}\big[r(r-1)\cdots(r-n+2)x^{r-n+1}\big] \\[4pt]
&+ \dots + a_1 x(r x^{r-1}) + a_0 x^r = 0.
\end{aligned}
\]

נפשט — בכל איבר מופיעה החזקה \(x^r\):
\[
x^r\big[a_n r(r-1)\cdots(r-n+1)
+ a_{n-1}r(r-1)\cdots(r-n+2)
+ \dots + a_1 r + a_0\big] = 0.
\]

מאחר ש-\(x^r \neq 0\) עבור \(x>0\), נקבל את המשוואה האופיינית:
\begin{equation}\tag{\ref{polyn}}
L(r) = a_n r(r-1)(r-2)\cdots(r-n+1)
+ a_{n-1} r(r-1)\cdots(r-n+2)
+ \dots + a_1 r + a_0 = 0. 
\end{equation}

שורשי הפולינום \(r_1, r_2, \dots, r_n\)
קובעים את הפתרונות של המשוואה ההומוגנית:
\begin{equation}
y_h(x) = c_1 x^{r_1} + c_2 x^{r_2} + \dots + c_n x^{r_n}.
\end{equation}

\begin{remark}
    שימו לב כי שיטה זו אינה עוזרת לנו במקרה ובו יש לפחות שורש אחד מרוכב לפולינום האופייני. במקרה זה, יש לפנות לשיטה הראשונה.
\end{remark}

\textbf{המשך לפתרון האי-הומוגני}

חשוב להדגיש כי הצבה זו מניבה רק את \textbf{הפתרון ההומוגני} של משוואת אוילר.  
לאחר שמצאנו את \(y_h(x)\), נמשיך למציאת הפתרון הפרטי \(y_p(x)\)
למשוואה האי-הומוגנית
\[
a_n x^n y^{(n)} + a_{n-1}x^{n-1}y^{(n-1)} + \dots + a_1 x y' + a_0 y = b(x),
\]
באחת מן השיטות המוכרות.
לבסוף, נקבל את הפתרון הכללי:
\[
y(x) = y_h(x) + y_p(x).
\]

%%%CUT%%%

\example{}

מצאו פתרון כללי למשוואה:
\[
x^2y'' - 3xy' + 3y = 0, \qquad x > 0
\]

\explanation{} 
נבחין כי זוהי משוואה לינארית הומוגנית, אך \textbf{אינה בעלת מקדמים קבועים}.
בשל כך, הכלים שקיבלנו עד כה לפתרון משוואות דיפרנציאליות מסדר גבוה \emph{אינם ישימים כאן}:

\begin{itemize}
  \item \textbf{הנוסחה של אבל (Abel)} – רלוונטית של כאשר אנו יודעים (במקרה של סדר 2) פתרון אחד פרטי.

  \item \textbf{הורדת סדר } – ניתנת לשימוש רק כאשר ידוע פתרון אחד מראש, אך כאן אין לנו פתרון ידוע או ניחוש רציונלי לצורה של \(y\).

  \item \textbf{שיטת וריאציית הפרמטרים} – מיועדת ל-\textbf{משוואות אי-הומוגניות}.  
  "הדלק" של השיטה הוא הפתרון ההומוגני \(y_h\);  
  מאחר שהמשוואה כאן כבר \textbf{הומוגנית}, אין רכיב חיצוני \(b(x)\) שיכול לייצר פתרון פרטי \(y_p\),
  ולכן אין טעם להשתמש בה בשלב זה.
\end{itemize}

מתוך הסימטריה הזו נובעת \textbf{שיטת אוילר},  
שבה נשתמש בהצבה \(t = \ln x\) או לחלופין בהנחה \(y = x^r\)
כדי להמיר את המשוואה למקרה שקול עם \textbf{מקדמים קבועים}.
זוהי משוואת אוילר. נציב:
\[
t = \ln(x), \qquad x = e^t.
\]
כפי שנובע מהפולינום האופייני הכללי:
\begin{equation}\tag{\ref{polyn}}
L(r) = a_n r(r-1)(r-2)\cdots(r-n+1)
+ a_{n-1} r(r-1)\cdots(r-n+2)
+ \dots + a_1 r + a_0 = 0.
\end{equation}

במקרה שלפנינו, עבור \(n=2\),
נקבל ממשוואת אוילר:
\[
a_2 x^2 y'' + a_1 x y' + a_0 y = 0.
\]

לאחר ההצבה \(x = e^t\) והמעבר לפונקציה \(Y(t)\),
נקבל משוואה עם מקדמים קבועים:
\[
a_2 Y''(t) + (a_1 - a_2) Y'(t) + a_0 Y(t) = 0.
\]

ובמקרה שלנו: \(a_2=1,\, a_1=-3,\, a_0=3,\)
נמצא:
\[
Y''(t) - 4Y'(t) + 3Y(t) = 0.
\]

נמצא את הפולינום האופייני (של ההומוגנית במשתנה \(t\)):
\[
L(r) = r^2 - 4r + 3 = (r - 1)(r - 3).
\]

ומכאן הפתרון:
\[
Y(t) = C_1 e^{t} + C_2 e^{3t}.
\]

נחזיר למשתנה המקורי \(x = e^{t}\):
\[
e^{t} = e^{\ln(x)} = x, \qquad e^{3t} = e^{3\ln(x)} = x^3.
\]

ולכן:
\[
\boxed{y(x) = C_1 x + C_2 x^3, \qquad x > 0.}
\]

\textbf{דרך שניה — הצבה ישירה \(y = x^k\):}

נניח כי הפתרון הוא מהצורה \(y = x^k\).  
אזי:
\[
y' = kx^{k-1}, \qquad y'' = k(k-1)x^{k-2}.
\]

נציב במד׳׳ר:
\begin{equation}\label{euler_1}
x^2y'' - 3xy' + 3y = 0,
\end{equation}

ונקבל:
\[
x^2[k(k-1)x^{k-2}] - 3x[kx^{k-1}] + 3x^k = 0.
\]

נפשט:
\[
k(k-1)x^k - 3kx^k + 3x^k = 0.
\]

נחלק ב-\(x^k\) (שאינו אפס עבור \(x>0\), גם כיוון שאנחנו לא מעוניינים בפתרון הטריוויאלי $y=0$):
\[
k(k-1) - 3k + 3 = 0.
\]

נפשט את הביטוי:
\[
k^2 - 4k + 3 = 0 \quad \Longrightarrow \quad (k - 1)(k - 3) = 0.
\]

ולכן:
\[
k = 1, \quad k = 3.
\]

קיבלנו את אותו הפתרון ההומוגני:
\[
\boxed{y_h(x) = C_1 x + C_2 x^3, \qquad x > 0.}
\]

\example{}
מצאו פתרון כללי למשוואה:
\[
x^2y'' + xy' + y = \ln(x), \qquad x > 0
\]

\explanation{}
זוהי משוואת אוילר \textbf{לא הומוגנית}.  
נתחיל בפתרון \textbf{החלק ההומוגני}:
\[
x^2y'' + xy' + y = 0, \qquad x > 0.
\]

נזהה כי מדובר במשוואה מהצורה הכללית:
\[
a_2 x^2 y'' + a_1 x y' + a_0 y = 0,
\]
שמתאימה לפולינום האופייני:
\[
L(r) = a_2 r(r-1) + a_1 r + a_0 = 0.
\]

נציב \(a_2 = 1,\; a_1 = 1,\; a_0 = 1\):
\[
L(r) = r(r-1) + r + 1 = r^2 + 1 = 0.
\]

נמצא את השורשים:
\[
r = \pm i.
\]

מאחר והשורשים מרוכבים, הפתרון ההומוגני במשתנה \(t\) הוא:
\[
Y_H(t) = C_1 \sin(t) + C_2 \cos(t).
\]

כעת, נציב \(x = e^t\), כלומר \(t = \ln(x)\), ונקבל:
\[
\boxed{
y_H(x) = C_1 \sin(\ln x) + C_2 \cos(\ln x), \qquad x > 0.
}
\]

\textbf{דרך 1 — וריאציית פרמטרים במישור $(x,y)$ }

נחלק את המשוואה ב-\(x^2\) כדי להביאה לצורה הנורמלית:
\[
y'' + \frac{1}{x}y' + \frac{1}{x^2}y = \frac{\ln(x)}{x^2}.
\]

נבחר את הפתרון הפרטי בצורת וריאציית פרמטרים:
\[
y_p(x) = C_1(x)\sin(\ln x) + C_2(x)\cos(\ln x).
\]

נגזור לפי \(x\):
\[
\begin{cases}
C_1'(x)\sin(\ln x) + C_2'(x)\cos(\ln x) = 0, \\[4pt]
C_1'(x)\cos(\ln x)\cdot \frac{1}{x} + 
C_2'(x)\big(-\sin(\ln x)\big)\cdot \frac{1}{x} = \frac{\ln(x)}{x^2}.
\end{cases}
\]

כעת יש לפתור את מערכת שתי המשוואות לשני הנעלמים \(C_1'(x),\, C_2'(x)\).  
הפתרון קיים עקרונית, אך כולל חישובים ארוכים.


\textbf{דרך 2 — משוואה עם מקדמים קבועים במישור $(t,Y)$}

נבצע את ההצבה:
\[
x = e^t, \qquad t = \ln(x), \qquad Y(t) = y(e^t).
\]
אזי נקבל:
\[
Y''(t) + Y(t) = t.
\]

נזהה את צורת הגורם הימני ונרשום:
\[
a = 0, \quad b = 0, \quad m = 1, \quad k = 0.
\]

נניח פתרון פרטי מהצורה:
\[
Y_p(t) = P_1(t) = a_1 t + a_0,
\]
ולכן:
\[
Y_p'(t) = a_1, \qquad Y_p''(t) = 0.
\]

נציב במשוואה:
\[
0 + (a_1 t + a_0) = t.
\]

מכאן נקבל את ערכי המקדמים:
\[
a_1 = 1, \qquad a_0 = 0.
\]
ולכן:
\[
Y_p(t) = t.
\]

שימו לב לפשטות של הפתרון במישור זה. 
הפתרון הכללי במשתנה \(t\) אם כן הוא סכום הפתרון ההומוגני והפרטי:
\[
Y(t) = C_1\cos(t) + C_2\sin(t) + t.
\]

נחזיר למשתנה המקורי \(x = e^t \Rightarrow t = \ln(x)\):
\[
\boxed{
y(x) = C_1 \cos(\ln x) + C_2 \sin(\ln x) + \ln(x), \qquad x > 0.
}
\]

על מנת \textbf{להדגיש} את החשיבות של בחירה נכונה במישור שבו אנו ׳׳מייצרים׳׳ את הפתרון הפרטי, נראה כאן עד כמה הפתרון האי-הומוגני ב-$x$ קדחתני ולא יעיל (במיוחד בבחינה).

נחזור למערכת המשוואות שקיבלנו משיטת וריאציית הפרמטר עבור \(C_1'(x)\) ו-\(C_2'(x)\):
\[
\begin{cases}
C_1'(x)\sin(\ln x) + C_2'(x)\cos(\ln x) = 0, \\[6pt]
C_1'(x)\cos(\ln x)\cdot \frac{1}{x} - 
C_2'(x)\sin(\ln x)\cdot \frac{1}{x} = \frac{\ln(x)}{x^2}.
\end{cases}
\]

\textbf{נבודד את \(C_2'(x)\) מן המשוואה הראשונה:}
\[
C_2'(x)\cos(\ln x) = -C_1'(x)\sin(\ln x)
\quad \Longrightarrow \quad
C_2'(x) = -C_1'(x)\tan(\ln x).
\]

נכפיל את המשוואה השנייה ב-\(x\):
\[
C_1'(x)\cos(\ln x) - C_2'(x)\sin(\ln x) = \frac{\ln(x)}{x}.
\]

נחליף את \(C_2'(x)\) לפי הביטוי שמצאנו:
\[
C_1'(x)\cos(\ln x) - \big[-C_1'(x)\tan(\ln x)\big]\sin(\ln x)
= \frac{\ln(x)}{x}.
\]

נסדר:
\[
C_1'(x)\big[\cos(\ln x) + \tan(\ln x)\sin(\ln x)\big]
= \frac{\ln(x)}{x}.
\]

נפשט את הסוגריים:
\[
\tan(\ln x)\sin(\ln x)
= \frac{\sin^2(\ln x)}{\cos(\ln x)} \quad \Longrightarrow \quad
\cos(\ln x) + \frac{\sin^2(\ln x)}{\cos(\ln x)}
= \frac{1}{\cos(\ln x)}.
\]

ולכן:
\[
C_1'(x) = \frac{\ln(x)}{x}\cos(\ln x).
\]

נחזיר למשוואה הראשונה:
\[
C_2'(x) = -C_1'(x)\tan(\ln x)
= -\frac{\ln(x)}{x}\cos(\ln x)\tan(\ln x)
= -\frac{\ln(x)}{x}\sin(\ln x).
\]

\textbf{נבצע אינטגרציה כדי למצוא את \(C_1(x)\) ו-\(C_2(x)\):}
\[
\begin{aligned}
C_1(x) &= \int \frac{\ln(x)}{x}\cos(\ln x)\,dx, \\[4pt]
C_2(x) &= -\int \frac{\ln(x)}{x}\sin(\ln x)\,dx.
\end{aligned}
\]

נבצע הצבה \(t = \ln x \Rightarrow dt = \frac{dx}{x}\):
\[
\begin{aligned}
C_1(x) &= \int t\cos t\,dt, \\[4pt]
C_2(x) &= -\int t\sin t\,dt.
\end{aligned}
\]

נחשב בחלקים:

\[
\begin{aligned}
\int t\cos t\,dt &= t\sin t - \int \sin t\,dt = t\sin t + \cos t, \\[4pt]
\int t\sin t\,dt &= -t\cos t + \int \cos t\,dt = -t\cos t + \sin t.
\end{aligned}
\]

נציב בחזרה \(t = \ln x\):
\[
\begin{aligned}
C_1(x) &= \ln(x)\sin(\ln x) + \cos(\ln x), \\[4pt]
C_2(x) &= \ln(x)\cos(\ln x) - \sin(\ln x).
\end{aligned}
\]

כעת נציב בביטוי של \(y_p(x)\):
\[
y_p(x) = C_1(x)\sin(\ln x) + C_2(x)\cos(\ln x).
\]

נפתח סוגריים:
\[
\begin{aligned}
y_p(x)
&= [\ln(x)\sin(\ln x) + \cos(\ln x)]\sin(\ln x)
+ [\ln(x)\cos(\ln x) - \sin(\ln x)]\cos(\ln x) \\[4pt]
&= \ln(x)\sin^2(\ln x) + \cos(\ln x)\sin(\ln x)
+ \ln(x)\cos^2(\ln x) - \sin(\ln x)\cos(\ln x).
\end{aligned}
\]

נראה כי שני האיברים המעורבים מתבטלים:
\[
\cos(\ln x)\sin(\ln x) - \sin(\ln x)\cos(\ln x) = 0,
\]
ולכן:
\[
y_p(x) = \ln(x)\big[\sin^2(\ln x) + \cos^2(\ln x)\big].
\]

נקבל כפי שציפינו:
\[
\boxed{
y_p(x) = \ln(x), \qquad x > 0.
}
\]

%%%CUT%%%

\newpage
\underline{תרגילים}
\exercise{}
קבלו פתרון כללי למד׳׳ר הבאה:
\[
x^2 y'' - 3x y' + 4y = 2x^2 + \ln(x),\qquad x>0.
\]

\exercise{}
הפתרון של המשוואה הדיפרנציאלית:
\[
x^2 y'' - 3x y' + 4y = 2x^3,
\]
מקיים את התנאים ההתחלתיים:
\[
y(1) = 2, \qquad y'(1) = -1.
\]

\textbf{מצאו את הערך:} \(y(e)\).

\exercise{}
נתונה המשוואה הדיפרנציאלית:
\[
x^2 y'' - x y' + y = \frac{\ln x}{x}, \qquad x>0,
\]

המקיימת את התנאים ההתחלתיים:
\[
y(1) = 0, \qquad y'(1) = 1.
\]

מצאו את הפתרון הפרטי לבעיה.

\exercise{}
נתונה המשוואה הדיפרנציאלית:
\[
x^2 y'' - 8x y' + 20y = 5x^4 \cos(2\ln x), \qquad x > 0.
\]

מה תהיה צורתו הכללית של הפתרון הפרטי לבעיה?



\newpage
\underline{פתרונות}
\solution{}
ראשית, נפתור את המשוואה ההומוגנית:
\[
x^2 y'' - 3x y' + 4y = 0.
\]

נניח כי \( y = x^r \), ומכאן:
\[
y' = r x^{r-1}, \qquad y'' = r(r-1)x^{r-2}.
\]

נציב במשוואה:
\[
x^2 \cdot r(r-1)x^{r-2} - 3x \cdot r x^{r-1} + 4x^r = 0.
\]

נחלק ב-\(x^r\) (שאינו אפס עבור \(x>0\)):
\[
r(r-1) - 3r + 4 = 0.
\]

נפשט:
\[
r^2 - 4r + 4 = 0 \quad \Longrightarrow \quad (r - 2)^2 = 0.
\]

ולכן קיבלנו שורש כפול \(r = 2\).

הפתרון ההומוגני הוא:
\[
\boxed{y_h(x) = C_1 x^2 + C_2 x^2 \ln(x).}
\]

\textbf{נמצא פתרון פרטי עבור האיבר \(2x^2\):}

נציב \(x = e^t\), כלומר \(t = \ln(x)\), ונגדיר \(Y(t) = y(e^t)\).

נקבל:
\[
Y'' - 4Y' + 4Y = 2e^{2t}.
\]

השורש הכפול של הפולינום האופייני הוא \(r = 2\),
ולכן נניח פתרון פרטי מהצורה:
\[
Y_p = a t^2 e^{2t}.
\]

נחשב:
\[
Y_p' = a e^{2t}(2t + 2t^2), \qquad Y_p'' = a e^{2t}(4t^2 + 8t + 2).
\]

נציב במשוואה:
\[
Y_p'' - 4Y_p' + 4Y_p = a e^{2t}\big[(4t^2 + 8t + 2) - 4(2t^2 + 2t) + 4t^2\big] = 2a e^{2t}.
\]

נדרוש ש-\(2a e^{2t} = 2e^{2t}\), ומכאן:
\[
\boxed{a = 1.}
\]

ולכן:
\[
Y_p = t^2 e^{2t} \quad \Longrightarrow \quad y_p = x^2 (\ln x)^2.
\]

\textbf{נמצא פתרון פרטי עבור האיבר \(\ln(x)\):}

נציב שוב \(x = e^t\), ונקבל:
\[
Y'' - 4Y' + 4Y = t.
\]

נניח פתרון פרטי פולינומי:
\[
Y_p = a t + b.
\]

נחשב נגזרות:
\[
Y_p' = a, \qquad Y_p'' = 0.
\]

נציב במשוואה:
\[
0 - 4a + 4(a t + b) = t.
\]

נשווה מקדמים:
\[
\begin{cases}
4a = 1 \Rightarrow a = \tfrac{1}{4}, \\[4pt]
-4a + 4b = 0 \Rightarrow b = \tfrac{1}{4}.
\end{cases}
\]

ולכן:
\[
Y_p = \frac{1}{4}t + \frac{1}{4} \quad \Longrightarrow \quad y_p = \frac{1}{4}\ln(x) + \frac{1}{4}.
\]

\textbf{הפתרון הכללי:}
\[
\boxed{
y(x) = C_1 x^2 + C_2 x^2 \ln(x) + x^2 (\ln x)^2 + \frac{1}{4}\ln(x) + \frac{1}{4}, \qquad x > 0.
}
\]


\solution{}

נבחין כי זוהי משוואת אוילר מסדר שני.  
נמצא תחילה את הפתרון ההומוגני:

\[
x^2 y'' - 3x y' + 4y = 0.
\]

נניח כי \(y = x^r\). נגזור:
\[
y' = r x^{r-1}, \qquad y'' = r(r-1)x^{r-2}.
\]

נציב במשוואה:
\[
x^2[r(r-1)x^{r-2}] - 3x[r x^{r-1}] + 4x^r = 0.
\]

נחלק ב-\(x^r\):
\[
r(r-1) - 3r + 4 = 0.
\]

נפשט:
\[
r^2 - 4r + 4 = 0 \quad \Longrightarrow \quad (r - 2)^2 = 0.
\]

ולכן קיבלנו שורש כפול \(r = 2\).

מכאן הפתרון ההומוגני הוא:
\[
y_h(x) = C_1 x^2 + C_2 x^2 \ln(x).
\]

\textbf{נמצא פתרון פרטי עבור האיבר \(2x^3\):}

כיוון שהאיבר בצד ימין הוא מהצורה \(x^m\) עם \(m = 3\),  
נניח פתרון פרטי מהצורה:
\[
y_p = A x^3.
\]

נגזור:
\[
y_p' = 3A x^2, \qquad y_p'' = 6A x.
\]

נציב במשוואה:
\[
x^2(6A x) - 3x(3A x^2) + 4(A x^3) = 2x^3.
\]

נפשט:
\[
(6A - 9A + 4A)x^3 = 2x^3.
\]

נחלק ב-\(x^3\) (שאינו אפס):
\[
A = 2.
\]

ולכן:
\[
y_p = 2x^3.
\]

\textbf{נרכיב את הפתרון הכללי:}
\[
y(x) = C_1 x^2 + C_2 x^2 \ln(x) + 2x^3.
\]

נחשב את הנגזרת:
\[
y'(x) = 2C_1 x + C_2(2x\ln x + x) + 6x^2.
\]

נשתמש בתנאים ההתחלתיים:
\[
\begin{cases}
y(1) = 2,\\[4pt]
y'(1) = -1.
\end{cases}
\]

נציב \(x = 1\):
\[
\begin{cases}
C_1 + 2 = 2,\\[4pt]
2C_1 + C_2 + 6 = -1.
\end{cases}
\]

נפתור את המערכת:
\[
\begin{aligned}
&C_1 = 0,\\[4pt]
&2(0) + C_2 + 6 = -1 \quad \Rightarrow \quad C_2 = -7.
\end{aligned}
\]

ולכן:
\[
y(x) = -7x^2 \ln(x) + 2x^3.
\]

\textbf{נחשב את \(y(e)\):}

נציב \(x = e\):
\[
\begin{aligned}
y(e) &= -7e^2 \ln(e) + 2e^3 \\[4pt]
&= -7e^2(1) + 2e^3 \\[4pt]
&= e^2(2e - 7).
\end{aligned}
\]

\[
\boxed{y(e) = e^2(2e - 7).}
\]



\solution{}

נמצא תחילה את הפתרון ההומוגני:
\[
x^2 y'' - x y' + y = 0.
\]

נניח כי \(y = x^r\).  
אזי:
\[
y' = r x^{r-1}, \qquad y'' = r(r-1)x^{r-2}.
\]

נציב במשוואה:
\[
x^2[r(r-1)x^{r-2}] - x[r x^{r-1}] + x^r = 0.
\]

נחלק ב-\(x^r\):
\[
r(r-1) - r + 1 = 0.
\]

נפשט:
\[
r^2 - 2r + 1 = 0 \quad \Longrightarrow \quad (r - 1)^2 = 0.
\]

ולכן \(r = 1\) הוא שורש כפול.

מכאן הפתרון ההומוגני הוא:
\[
y_h(x) = C_1 x + C_2 x \ln(x),\qquad x>0.
\]

\textbf{נמצא פתרון פרטי עבור האיבר} \(\displaystyle \frac{\ln x}{x}\).

נשתמש בהצבה:
\[
x = e^t, \qquad t = \ln x, \qquad Y(t) = y(e^t).
\]

נקבל משוואה עם מקדמים קבועים:
\[
Y'' - 2Y' + Y = t e^{-t}.
\]
 
נניח פתרון פרטי מהצורה:
\[
Y_p = e^{-t}(a t + b).
\]

נחשב נגזרות:
\[
\begin{aligned}
Y_p' &= e^{-t}\big[a - (a t + b)\big] = e^{-t}(-a t + a - b),\\[4pt]
Y_p'' &= e^{-t}\big[a t - 2a + b\big].
\end{aligned}
\]

נציב במשוואה:
\[
Y_p'' - 2Y_p' + Y_p = e^{-t}t.
\]

נחשב את אגף שמאל:
\[
e^{-t}\big[(a t - 2a + b) - 2(-a t + a - b) + (a t + b)\big] = e^{-t}t.
\]

נשווה מקדמים לפי חזקות של \(t\):

\[
a + 2a + a = 4a,\qquad 
 (-2a + b - 2a + 2b + b) = -4a + 4b.
\]


נדרוש:
\[
4a = 1, \qquad -4a + 4b = 0.
\]

נפתור:
\[
a = \frac{1}{4}, \qquad b = \frac{1}{4}.
\]

ולכן:
\[
Y_p = e^{-t}\left(\frac{1}{4}t + \frac{1}{4}\right).
\]

נחזיר למשתנה \(x = e^t\):
\[
y_p(x) = \frac{1}{4x}\big(\ln x + 1\big).
\]

\textbf{הפתרון הכללי:}
\[
y(x) = C_1 x + C_2 x \ln(x) + \frac{1}{4x}\big(\ln x + 1\big).
\]

\textbf{נשתמש בתנאים ההתחלתיים:}
\[
y(1) = 0, \qquad y'(1) = 1.
\]

נחשב את הנגזרת:
\[
\begin{aligned}
y'(x) &= C_1 + C_2(1 + \ln x)
- \frac{1}{4x^2}(\ln x + 1) + \frac{1}{4x^2}.
\end{aligned}
\]


נציב \(x=1\):
\[
\begin{cases}
y(1) = C_1 + \frac{1}{4} = 0,\\[4pt]
y'(1) = C_1 + C_2 = 1.
\end{cases}
\]

נפתור:
\[
\begin{aligned}
&C_1 = -\frac{1}{4},\\[4pt]
&-\frac{1}{4} + C_2 = 1 \quad \Rightarrow \quad C_2 = \frac{5}{4}.
\end{aligned}
\]

ולכן:
\[
\boxed{
y(x) = -\frac{1}{4}x + \frac{5}{4}x\ln(x) + \frac{1}{4x}\big(\ln x + 1\big).
}
\]


\solution{}

המשוואה הנתונה היא משוואת אוילר אי-הומוגנית.  
נתחיל מהחלק ההומוגני. נתחיל מהצורה הכללית של הפולינום האופייני:

\begin{equation}\tag{\ref{polyn}}
L(r) = a_n r(r-1)(r-2)\cdots(r-n+1)
+ a_{n-1} r(r-1)\cdots(r-n+2)
+ \dots + a_1 r + a_0 = 0.
\end{equation}

במקרה שלפנינו, עבור \(n=2\), מתקבלת משוואת אוילר מהצורה:
\[
a_2 x^2 y'' + a_1 x y' + a_0 y = 0.
\]

נציב את המקדמים המתאימים מהמשוואה הנתונה:
\[
x^2 y'' - 8x y' + 20y = 0.
\]

נקבל את הפולינום האופייני:
\[
L(r) = r(r-1) - 8r + 20 = 0.
\]

נפשט:
\[
r^2 - 9r + 20 = 0 \quad \Longrightarrow \quad r_{1,2} = 4 \pm 2i.
\]

שורשים אלו הם \textbf{מרוכבים}, ולכן לא נוכל להציב $y=x^{r}$ ונגזרותיה ישירות למד׳׳ר.

נבצע את ההצבה:
\[
x = e^t, \qquad t = \ln x, \qquad Y(t) = y(e^t).
\]

נכתוב את המשוואה במשתנה \(t\):
\[
Y'' - 9Y' + 20Y = 5e^{4t}\cos(2t).
\]
נוכל לבצע השוואת מקדמים שכן אגף ימין מתאים לשיטה זו. נזהה את הפרמטרים:
\[
a = 4, \qquad b = 2, \qquad m = 0.
\]

נבחין כי \(a = 4+2i\) פותר את הפולינום האופייני מריבוי אלגברי 1,
ולכן
\(
k = 1
\).

נציע פתרון פרטי מהצורה:
\[
Y_p(t) = t e^{4t}\big(A\cos(2t) + B\sin(2t)\big).
\]

נחזיר למשתנה המקורי \(x = e^t \Rightarrow t = \ln x\):
\[
\boxed{
y_p(x) = x^4 (\ln x)\big(A\cos(2\ln x) + B\sin(2\ln x)\big).
}
\]

%%%CUT%%%

\newpage
\subsection{פתרון מד״ר באמצעות טורי חזקות}

\textbf{ המוטיבציה לשיטה}

נבחן את הצורה הכללית של משוואה לינארית אי־הומוגנית מסדר \(n\):

\begin{equation}
a_n(x)y^{(n)} + a_{n-1}(x)y^{(n-1)} + \dots + a_1(x)y' + a_0(x)y = b(x),
\end{equation}

כאשר \(a_i(x)\) הן פונקציות של \(x\), ולא קבועים (לא בהכרח).

נבחין כי זוהי \textbf{משוואה לינארית}, אך \textbf{אינה בעלת מקדמים קבועים}.  
בשל כך, הכלים שעמדו לרשותנו עד כה לפתרון משוואות דיפרנציאליות מסדר גבוה  
\textbf{אינם ישימים כאן}:

\begin{itemize}
  
  \item \textbf{הנוסחה של אבל (Abel)} – רלוונטית רק למקרים שבהם ידועים פתרונות של המשוואה ההומוגנית,  
  דבר שאינו מתקיים כאן.

  \item \textbf{הורדת סדר} – ניתנת ליישום רק כאשר ידועים פתרונות של המשוואה ההומוגנית מראש,  
  אך אין לנו פתרון ידוע או אפילו ניחוש רציונלי לצורה של \(y\).

  \item \textbf{שיטת וריאציית הפרמטרים} – פועלת רק עבור \textbf{משוואות אי־הומוגניות},  
  ודרוש בה הפתרון ההומוגני \(y_h\) כבסיס.  
  כאן איננו יודעים את \(y_h\), ולכן לא ניתן להפעילה כלל.

  \item \textbf{שיטת אוילר} – דרשה מבנה סימטרי של \(x^k y^{(k)}\),  
  אך כאן החזקה של \(x\) אינה תואמת את סדר הנגזרת, ולכן הסימטריה חסרה.

\end{itemize}

לכן, במצב זה — כאשר:

\begin{itemize}
  \item המשוואה אינה בעלת מקדמים קבועים,
  \item לא קיימת סימטריה (כמו באוילר),
  \item ואין בידינו פתרון ידוע מראש —
\end{itemize}

ניאלץ לפנות לשיטה חדשה ועוצמתית יותר:  
\textbf{פיתוח הפתרון בצורת טור חזקה} סביב נקודה \(x_0\),  
ובלבד שרדיוס ההתכנסות של הטור מאפשר זאת.  

זוהי הגישה הכללית ביותר לפתרון משוואות לינאריות עם מקדמים לא קבועים,  
כאשר כל יתר השיטות נכשלו.

נניח כי הפתרון יכול להיכתב בצורה הבאה
\begin{equation}\label{series_sol}
y(x) = \sum_{n=0}^{\infty} a_n (x - x_0)^n.
\end{equation}
זהו למעשה פיתוח טור טיילור של הפתרון סביב הנקודה $x_{0}$, וזאת כאמור בהנחה כי רדיוס ההתכנסות של הטור מכיל את הנקודה.
מכאן נפתח את הרעיון הכללי של שיטת טורי החזקה, נחשב נגזרות איבר־איבר,  
ונראה כיצד נוכל לקבוע את המקדמים \(a_n\) באמצעות הצבה לתוך המשוואה.

\vspace{0.5cm}
\textbf{צורה כללית לפתרון הנתון ע״י טור חזקות (פיתוח סביב 0 - טור מקלורן):}

\begin{equation}
f(x) = \sum_{n=0}^{\infty} a_n x^n = a_0 + a_1x + a_2x^2 + \dots
\end{equation}

כאשר הקשר בין המקדמים לפונקציה הנתונה ע״י טור טיילור הוא:
\begin{equation}\label{coeff_taylor}
\frac{f^{(n)}(0)}{n!} = a_n.
\end{equation}

מגזירת איבר–איבר של טור החזקות נקבל את שתי הנגזרות הראשונות:
\[
f'(x) = 
\left(\sum_{n=0}^{\infty} a_n x^n\right)' 
= \sum_{n=0}^{\infty} (a_n x^n)' 
= \sum_{n=0}^{\infty} n a_n x^{n-1} 
= \textcolor{red}{\sum_{n=1}^{\infty} n a_n x^{n-1}},
\]

\[
f''(x) = 
\sum_{n=0}^{\infty} n(n-1)a_n x^{n-2} 
= \color{red}{\sum_{n=1}^{\infty} n(n-1)a_n x^{n-2}} 
= \textcolor{red}{\sum_{n=2}^{\infty} n(n-1)a_n x^{n-2}}.
\]

\textbf{ ננתח את המקרה של סדר שני:}
\begin{equation}\label{2nd_norm}
y''(x) + p(x)y'(x) + q(x)y(x) = g(x),
\end{equation}

כאשר הפונקציות $p(x), q(x), g(x)$ ניתנות לכתבה ע׳׳י טורי חזקות:
\[
p(x) = \sum_{n=0}^{\infty} p_n x^n, 
\qquad
q(x) = \sum_{n=0}^{\infty} q_n x^n,
\qquad
g(x) = \sum_{n=0}^{\infty} g_n x^n.
\]

אלו הן הגדרות של פונקציות אנליטיות.  
לכן, לכל פתרון \(y(x)\) קיים טור חזקות כך ש:

\begin{equation}
y(x) = \sum_{n=0}^{\infty} a_n x^n.
\end{equation}

ע״י הצבת \(y(x)\) ונגזרותיה לתוך המד״ר, נקבל נוסחת \textbf{רקורסיה} (נוסחת נסיגה)  
עבור המקדמים — נוסחת רקורסיה שבה האיבר \(a_{n}\) תלוי בקודמיו.

\vspace{0.5cm}
\textbf{הזחת אינדקסים והשגת נוסחת רקורסיה}

נחשב את הנגזרת הראשונה של טור החזקה:
\[
y'(x) = 
\sum_{n=0}^{\infty} n a_n x^{n-1}.
\]

נזכור כי עבור \(n=0\) מתקבל איבר אפס, ולכן ניתן להתחיל את הסכימה מ-\(n=1\):
\[
y'(x) =
\sum_{n=1}^{\infty} n a_n x^{n-1}.
\]

כעת נרצה לכתוב את הסכום כך שכל החזקות תהיינה \(x^n\) (ולא \(x^{n-1}\)),  
כדי לאפשר השוואה ישירה בין איברים בהמשך.

נבצע שינוי אינדקס:
\[
k = n-1 \quad \Longrightarrow \quad n = k+1.
\]
נחזיר את הסימון הרגיל \(n\) במקום \(k\) (כן, מותר לעשות את זה. מדובר בסימון ותו לא, ולא בביטוי אלגברי שמוביל לסתירה):
\begin{equation}
\boxed{y'(x) = 
\sum_{n=0}^{\infty} (n+1)a_{n+1}x^n}.
\end{equation}

אותו עיקרון ניישם גם עבור הנגזרת השנייה:
\[
y''(x) =
\sum_{n=2}^{\infty} n(n-1)a_n x^{n-2}.
\]

נזיח גם כאן את האינדקסים כך שכל איבר יכיל חזקה \(x^n\):
\[
k = n-2 \quad \Longrightarrow \quad n = k+2,
\]
ונקבל:
\begin{equation}
\boxed{y''(x) =
\sum_{n=0}^{\infty} (n+2)(n+1)a_{n+2}x^n}.
\end{equation}

נבצע כעת הצבה \textbf{כללית} עבור משוואה לינארית מסדר שני, עם שני תנאי התחלה, ועם פיתוח סביב נקודה רגולרית $x_{0}$:
\begin{equation}\label{2nd_norm}
\begin{cases}
y'' + p(x)y' + q(x)y = g(x), \\[6pt]
y(x_0) = \alpha, \qquad y'(x_0) = \beta.
\end{cases}
\end{equation}

נניח כי הפונקציות ניתנות לפיתוח כטורי חזקה סביב \(x_0\):
\[
\begin{aligned}
y(x)= \sum_{n=0}^{\infty} a_n (x-x_0)^n, 
p(x)= \sum_{n=0}^{\infty} p_n (x-x_0)^n,
q(x)= \sum_{n=0}^{\infty} q_n (x-x_0)^n, 
g(x)= \sum_{n=0}^{\infty} g_n (x-x_0)^n.
\end{aligned}
\]

לאחר גזירה והזחת אינדקסים נקבל:
\[
\begin{aligned}
y'(x) &= \sum_{n=0}^{\infty} (n+1)a_{n+1}(x-x_0)^n,\\[4pt]
y''(x) &= \sum_{n=0}^{\infty} (n+2)(n+1)a_{n+2}(x-x_0)^n.
\end{aligned}
\]

נרצה להציב ביטויים אלה במשוואה \eqref{2nd_norm}. נבחן תחילה את האיברים השונים.

נבחן את המכפלה \(p(x)y'(x)\):
\[
p(x)y'(x) = 
\Big(\sum_{k=0}^{\infty} p_k (x-x_0)^k\Big)
\Big(\sum_{m=0}^{\infty} (m+1)a_{m+1}(x-x_0)^m\Big).
\]

כפל של שני טורי חזקות מתבצע ע״י כפל איבר–איבר
ואיסוף כל האיברים בעלי אותה חזקה של \((x-x_0)\).  
נפתח את המכפלה הכפולה:
\[
p(x)y' = 
\sum_{k=0}^{\infty}\sum_{m=0}^{\infty}
p_k (m+1)a_{m+1}(x-x_0)^{k+m}.
\]

כעת נרצה לרכז את כל האיברים בעלי אותה חזקה כוללת,
ולכן נגדיר:
\[
n = k + m \quad \Longrightarrow \quad m = n - k.
\]
נחליף ביטוי זה בתוך הסכום ונקבל:
\begin{equation}
p(x)y' = 
\sum_{n=0}^{\infty}
\Big(\sum_{k=0}^{n}
p_k (n-k+1)a_{n-k+1}\Big)(x-x_0)^n.
\end{equation}

באופן דומה, עבור המכפלה \(q(x)y\) נקבל:
\begin{equation}
q(x)y =
\Big(\sum_{k=0}^{\infty} q_k (x-x_0)^k\Big)
\Big(\sum_{m=0}^{\infty} a_m (x-x_0)^m\Big)
= \sum_{n=0}^{\infty}
\Big(\sum_{k=0}^{n} q_k a_{n-k}\Big)(x-x_0)^n.
\end{equation}

נציב כעת את כל הביטויים במשוואה \eqref{2nd_norm} ונאסוף לפי חזקות של \((x-x_0)^n\):

\begin{equation}
\sum_{n=0}^{\infty} 
\Big[
(n+2)(n+1)a_{n+2} 
+ \sum_{k=0}^{n} p_k (n-k+1)a_{n-k+1}
+ \sum_{k=0}^{n} q_k a_{n-k}
\Big](x-x_0)^n
=
\sum_{n=0}^{\infty} g_n (x-x_0)^n.
\end{equation}

מאחר שהפיתוח תקף עבור כל ערך של \(x\) בתחום ההתכנסות,
נשווה מקדמים לפי חזקות זהות של \((x-x_0)^n\),  
ונקבל את \textbf{נוסחת הרקורסיה הכללית עבור סדר 2}:

\begin{equation}\label{recu}
\boxed{a_{n+2} =
\frac{1}{(n+2)(n+1)}
\left[
g_n
- \sum_{k=0}^{n} p_k (n-k+1)a_{n-k+1}
- \sum_{k=0}^{n} q_k a_{n-k}
\right]}.
\end{equation}

\vspace{1em}
באמצעות נוסחה זו, וידיעת שני המקדמים הראשונים מהתנאים ההתחלתיים:
\[
a_0 = \alpha, \qquad a_1 = \beta,
\]
נוכל לחשב בהדרגה את יתר המקדמים:
\[
a_2,\, a_3,\, a_4,\, \dots
\]
וכך \textbf{לבנות את טור הפתרון} כולו סביב הנקודה \(x_0\) כפי שכבר הוצג:
\begin{equation}\tag{\ref{series_sol}}
y(x) = \sum_{n=0}^{\infty} a_n (x-x_0)^n.
\end{equation}

הנוסחה במשוואה \ref{recu} שקיבלנו היא הבסיס לשיטת טור החזקות \textbf{עבור סדר 2}: איבר ה- \(a_{n+2}\) תלוי באיברים הקודמים שלו ובמקדמי הפיתוח של הפונקציות \(p(x)\), \(q(x)\) ו-\(g(x)\).

%%%CUT%%%

\example{}

נתונה המשוואה:
\[
y'' - 2xy' + \lambda y = 0
\]

א.
מצאו נוסחה רקורסיבית למשוואה המבטאת קשר בין קבועי טור החזקות המייצג את הפתרון סביב \(x_0 = 0\), 
שהיא \textbf{נקודה רגולרית} של המשוואה.

ב. עבור $\lambda=10, y(0)=0, y'(0)=3$, מצאו את $y(\frac{1}{2})$.

\explanation

א.
נניח כי הפתרון ניתן לפיתוח מהצורה:
\[
y(x) = \sum_{n=0}^{\infty} a_n x^n.
\]

נגזור ונבצע הזחות אינדקסים:
\[
y'(x) = \textcolor{blue}{\sum_{n=0}^{\infty} n a_n x^{n-1}}
= \sum_{n=0}^{\infty} (n+1)a_{n+1}x^n,
\]
\[
y''(x) = \sum_{n=0}^{\infty} n(n-1)a_n x^{n-2}
= \sum_{n=0}^{\infty} (n+2)(n+1)a_{n+2}x^n.
\]

איך יודעים כמה הזזות אינדקסים עלינו לעשות?  הכל תלוי במקדם של הנגזרות במד"ר – כל הזזת האינדקסים מטרתה להשוות את המעריך של x בין הטורים, וכך גם בחירת הטור בו נשתמש.

נבצע כעת הצבה במשוואה הנתונה:
\[
y'' - 2xy' + \lambda y = 0.
\]

נציב ונאחד חזקות:
\[
\sum_{n=0}^{\infty}(n+2)(n+1)a_{n+2}x^n
- 2x \textcolor{blue}{\sum_{n=0}^{\infty} n a_n x^{n-1}}
+ \lambda \sum_{n=0}^{\infty} a_n x^n = 0.
\]
שימו לב לאיבר \textcolor{blue}{בכחול} שנבחר לנגזרת השנייה. למה לא ביצענו הזזה? כיוון שאין צורך. $x$ כבר מכפיל איבר זה, ולכן החזקה $x^{n-1}$ תהפוך ל- $x^{n}$ כפי שאנחנו רוצים מטעמי נוחות. לבצע כאן הזזה, יגרור אותנו לאיבר $x^{n+1}$, מה שיאלץ אותנו לבצע הזזה חוזרת ׳׳אחורה׳׳.

לאחר כפל ב-\(x\) באיבר השני:
\[
\sum_{n=0}^{\infty}(n+2)(n+1)a_{n+2}x^n
- 2 \sum_{n=0}^{\infty} n a_n x^n
+ \lambda \sum_{n=0}^{\infty} a_n x^n = 0.
\]
שימו לב כי מותר לנו ׳׳להכניס׳׳ את ה-$x$ פנימה כיוון שאינו מכיל אינדקס מחוץ לטור.
נאסוף איברים דומים:
\[
\sum_{n=0}^{\infty}
\Big[(n+2)(n+1)a_{n+2} + (\lambda - 2n)a_n\Big]x^n = 0.
\]

מאחר שהמשוואה מתקיימת לכל \(x\), נקבל את הקשר הרקורסיבי:
\[
(n+2)(n+1)a_{n+2} + (\lambda - 2n)a_n = 0.
\]

ומכאן נוסחת הרקורסיה:
\[
\boxed{a_{n+2} = \frac{2n - \lambda}{(n+1)(n+2)}\,a_n, \qquad n \ge 0}.
\]

ב.
נשתמש ב\textbf{תנאי ההתחלה}.

ממשוואה (\ref{coeff_taylor}) נוכל לתרגם את שני קבועי ההתחלה שלנו למקדמים הראשונים של הטור:
\[
a_0 = \frac{y(0)}{0!}=y(0), \qquad a_1 = \frac{y'(0)}{1!}=y'(0).
\]

$a_0$, $a_1$ לא נקבעים ע"י יחס הרקורסיה – הם תנאי ההתחלה.  
נוכל לראות מנוסחת הרקורסיה שכל איבר זוגי תלוי ב-\(a_0\),  
וכל איבר אי-זוגי תלוי ב-\(a_1\).  
נחשב את האיברים בהדרגה לפי הצורך.

נציב בנוסחת הרקורסיה את \(
\lambda = 10, \quad y(0)=0, \quad y'(0)=3.
\):
\[
a_{n+2} = \frac{2n - 10}{(n+1)(n+2)}\,a_n,
\qquad a_0=0, \quad a_1=3.
\]

כיוון ש-\(a_0 = 0\), כל האיברים הזוגיים מתאפסים:
\[
a_{2n} = 0 \quad \forall n \ge 0.
\]

נחשב איברים אי-זוגיים:
\[
\begin{aligned}
a_3 &= \frac{2(1) - 10}{(2)(3)} a_1 = -\frac{8}{6} a_1 = -4, \\[6pt]
a_5 &= \frac{2(3) - 10}{(4)(5)} a_3 = \frac{-4}{20}(-4) = \frac{4}{5}, \\[6pt]
a_7 &= 0, \quad a_9 = 0, \dots
\end{aligned}
\]
כלומר, אנו שמים לב כי גם כאן, החל ממקום מסוים כלל האיברים באינדקסים האי-זוגיים הם גם כן 0:
\[
a_{2n+1} = 0 \quad \forall n \ge 3.
\]

ומכאן:
\[
y(x) = a_1 x + a_3 x^3 + a_5 x^5
= 3x - 4x^3 + \frac{4}{5}x^5.
\]

ולכן:
\[
\boxed{
y\!\left(\tfrac{1}{2}\right)
= 3\!\left(\tfrac{1}{2}\right)
- 4\!\left(\tfrac{1}{2}\right)^3
+ \tfrac{4}{5}\!\left(\tfrac{1}{2}\right)^5
= \frac{41}{40}.
}
\]

\begin{remark}
    שימו לב כי $n=5$ הוא למעשה \textbf{תנאי עצירה רקורסיבי. אני אוהב לכנות אותו בתור ׳׳תנאי עצירה רקורסיבי טבעי׳׳, שכן הוא מתקבל כתוצאה מהמקדמים השונים במד׳׳ר, אותם אנו לא יכולים לשנות.}
    לעומת זאת, לעיתים אנו שולטים בתנאי ההתחלה של הבעיה. במקרה זה, ׳׳בחירת׳׳ $a_{0}=0$, היא למעשה
    \textbf{׳׳תנאי עצירה רקורסיבי מאולץ׳׳}, כאשר לעיתים אנו נשאלים מה צריך להיות תנאי ההתחלה על מנת שנוכל לקבל פתרון פולינומיאלי סופי.
\end{remark}

\example{}

נתונה המשוואה:
\[
y'' + x^4 y = 0,
\qquad y(0)=1, \quad y'(0)=0.
\]

מצאו פתרון פרטי לבעיה והראו כי ל-\(y(x^{1/6})\) יש טור חזקות מתכנס עבור \(x \ge 0\).

\explanation

נניח כי הפתרון ניתן לפיתוח בצורת טור חזקות סביב \(x_0=0\) 
(שהיא נקודה רגולרית):
\[
y(x) = \sum_{n=0}^{\infty} a_n x^n.
\]

נגזור ונבצע הזחת אינדקסים:
\[
y'(x) = \sum_{n=0}^{\infty} (n+1)a_{n+1}x^n,
\qquad
y''(x) = \sum_{n=0}^{\infty} (n+2)(n+1)a_{n+2}x^n.
\]

נבצע כעת הצבה במשוואה הנתונה:
\[
y'' + x^4 y = 0.
\]

נציב ונאחד חזקות:
\[
\sum_{n=0}^{\infty} (n+2)(n+1)a_{n+2}x^n
+ x^4 \sum_{n=0}^{\infty} a_n x^n = 0.
\]

נרצה שכל איברי הטור יהיו בעלי אותה חזקה של \(x\).  
נבצע הזחת אינדקסים באיבר השני:
\[
\textcolor{blue}{\sum_{n=0}^{\infty} (n+2)(n+1)a_{n+2}x^n}
+ \sum_{n=4}^{\infty} a_{n-4}x^n = 0.
\]

על כן נאחד את שני הטורים ו"נוציא" את ארבעת האיברים הראשונים מהטור \textcolor{blue}{השמאלי} על מנת להתחיל את ריצת הטור מאותו ה-$n$:
\[
2a_2 + 6a_3x + 12a_4x^2 + 20a_5x^3 +
\sum_{n=4}^{\infty}[(n+2)(n+1)a_{n+2} + a_{n-4}]x^n = 0.
\]

נבצע השוואת מקדמים:

\[
\begin{aligned}
x^0 &: 2a_2 = 0,\\
x^1 &: 6a_3 = 0,\\
x^2 &: 12a_4 = 0,\\
x^3 &: 20a_5 = 0.
\end{aligned}
\]

כלומר:
\[
a_2 = a_3 = a_4 = a_5 = 0.
\]

לשאר האיברים (\(n \ge 4\)):
\[
(n+2)(n+1)a_{n+2} + a_{n-4} = 0.
\]

ומכאן נוסחת הרקורסיה:
\[
\boxed{a_{n+2} = -\frac{1}{(n+1)(n+2)}a_{n-4}, \qquad n \ge 4.}
\]
נוכל להזיז אינדקסים ולרשום:
\[
\boxed{a_{n+6} = -\frac{1}{(n+5)(n+6)}a_n, \qquad n \ge 0}.
\]

נכתוב את תנאי ההתחלה שלנו:
\[
a_0 = y(0) = 1, \qquad a_1 = y'(0) = 0.
\]

למעשה קיבלנו כי \(
a_1 = a_2 = a_3 = a_4 = a_5 = 0.
\). מתוך נוסחת הנסיגה שמצאנו לעיל, נשים לב כי רק האיברים עם אינדקס שהוא כפולה של 6, אינם מתאפסים. 
כלומר, נוכל לבטא את הפתרון באופן הבא:
\[
y(x) = a_0 + a_6x^6 + a_{12}x^{12} + \dots
= \sum_{n=0}^{\infty} a_{6n}x^{6n}.
\]

נבחן את הפונקציה \(y(x^{1/6})\):
\[
y(x^{1/6}) = \sum_{n=0}^{\infty} a_{6n}(x^{1/6})^{6n}
= \sum_{n=0}^{\infty} a_{6n}x^n.
\]
שימו לב כי במהלך הפתרון הנחנו טור חזקות מתכנס לחזקות של $x^{n}$!, וקיבלנו נוסחה רקורסיבית לכך.
מכאן כי \(y(x^{1/6})\) עצמה ניתנת לפיתוח כטור חזקות מתכנס!
מדוע? לאור צורת המד"ר, טור החזקות המקורי \(
y(x) = \sum_{n=0}^{\infty} a_n x^n.
\) הוא פונקציה אנליטית, ולכן בהכרח מתכנס בתוך רדיוס ההתכנסות שלו.  כלומר, מובטח לנו כי כל טור של $x^n$ הוא טור מתכנס. 

\newpage
\underline{תרגילים}
\exercise{}

נתונה המשוואה:
\[
y'' - (x - 1)y' + y = 0
\]

ויהי 
\[
y(x) = \sum_{n=0}^{\infty} a_n (x - 2)^n
\]
פתרון שלה שמקיים את תנאי ההתחלה:
\[
y(2) = 12, \qquad y'(2) = 6.
\]

חשבו את
\(
a_1 + a_3 + a_4
\)

\exercise{}

מצאו נוסחה רקורסיבית למקדמי הפולינום הפותר את המד״ר הבא (סביב 0):
\[
y'' - 2x^2y' + 4xy = x^2 + 2x + 2,
\qquad y(0)=3, \quad y'(0)=12.
\]

\exercise{}
  
מצאו פתרון פרטי של המשוואה הדיפרנציאלית:
\[
y'' - 2x^2y' + 6xy = 0,
\]
המקיים את תנאי ההתחלה:
\[
y(0)=\alpha, \qquad y'(0)=0,
\]
כתלות ב- \(\alpha \in \mathbb{R}\).

\exercise{}
 
נתונה המשוואה:
\[
(1 - x^2)y'' + 2y = 0,
\]

ומחפשים פתרון מצורת טור החזקות הבא:
\[
y(x) = \sum_{n=0}^{\infty} a_n x^n.
\]
 
אילו תנאי התחלה \(y(0), y'(0)\) יש לבחור כדי שהפתרון יהיה פולינום סופי (לא הטריוויאלי)?

\exercise{}

נתונה המשוואה:
\[
y'' + (\sin x)y' + (\cos x)y = 0,
\]
עם תנאי ההתחלה:
\[
y(0) = 0, \qquad y'(0) = 1.
\]

ניתן לכתוב את הפתרון בצורה:
\[
y(x) = \sum_{n=0}^{\infty} a_n x^n,
\]
ונתון כי:
\[
\sin x = \sum_{n=0}^{\infty} \frac{(-1)^n x^{2n+1}}{(2n+1)!}, 
\qquad
\cos x = \sum_{n=0}^{\infty} \frac{(-1)^n x^{2n}}{(2n)!}.
\]

יש למצוא את מקדם הטור
\( a_3 \)
.

\exercise{}

נתונה המשוואה:
\[
y'' + (\sin x)y' + (\cos x)y = 0,
\]
עם תנאי ההתחלה:
\[
y(0) = 0, \qquad y'(0) = 1.
\]

נניח פתרון בצורת טור חזקות:
\[
y(x) = \sum_{n=0}^{\infty} a_n x^n,
\]
וכן ידוע כי:
\[
\sin x = \sum_{n=0}^{\infty} \frac{(-1)^n x^{2n+1}}{(2n+1)!},
\qquad
\cos x = \sum_{n=0}^{\infty} \frac{(-1)^n x^{2n}}{(2n)!}.
\]

מצאו את המקדם $a_{3}$ של טור הפתרון מתוך נוסחת הרקורסיה.

%%%CUT%%%

\newpage
\underline{פתרונות}
\solution

נזכור כי עבור טור חזקות כללי סביב \(x_0 = 2\),
הקשר בין המקדמים לנגזרות הפתרון הוא:
\[
a_n = \frac{y^{(n)}(2)}{n!}.
\]

\vspace{0.7em}
נחשב לפי סדר את הנגזרות בנקודה \(x=2\) ע׳׳י הצבה ישירה במד׳׳ר.

\textbf{שלב ראשון – מציאת \(a_2\)}

נחשב ב-\(x = 2\):
\[
y''(2) - (2 - 1)y'(2) + y(2) = 0
\quad\Rightarrow\quad
y''(2) - 6 + 12 = 0
\quad\Rightarrow\quad
y''(2) = -6.
\]

ולכן:
\[
a_2 = \frac{y''(2)}{2!} = \frac{-6}{2} = -3.
\]

\textbf{שלב שני – מציאת \(a_3\)}

נגזור את המשוואה ונקבל:
\[
y''' - y' - (x - 1)y'' + y' = 0
\quad\Rightarrow\quad
y''' - (x - 1)y'' = 0.
\]

נציב \(x = 2\):
\[
y'''(2) - (2 - 1)y''(2) = 0
\quad\Rightarrow\quad
y'''(2) - (-6) = 0
\quad\Rightarrow\quad
y'''(2) = -6.
\]

ולכן:
\[
a_3 = \frac{y'''(2)}{3!} = \frac{-6}{6} = -1.
\]

\textbf{שלב שלישי – מציאת \(a_4\)}

נגזור פעם נוספת:
\[
y'''' - y'' - (x - 1)y''' = 0.
\]

נציב שוב \(x = 2\):
\[
y''''(2) - y''(2) - (2 - 1)y'''(2) = 0
\quad\Rightarrow\quad
y''''(2) + 6 + 6 = 0
\quad\Rightarrow\quad
y''''(2) = -12.
\]

ולכן:
\[
a_4 = \frac{y''''(2)}{4!} = \frac{-12}{24} = -\frac{1}{2}.
\]

\textbf{שלב רביעי – חישוב הסכום המבוקש}
\[
a_1 + a_3 + a_4 = 6 - 1 - \frac{1}{2} = \boxed{4.5}.
\]



\solution{}

נרצה למצוא נוסחה רקורסיבית למקדמי הפולינום הפותר את המד״ר הלא הומוגנית (סביב 0).  
נניח פתרון מהצורה:
\[
y(x) = \sum_{n=0}^{\infty} a_n x^n.
\]

נחשב נגזרות:
\[
y'(x) = \sum_{n=0}^{\infty} n a_n x^{n-1},
\qquad
y''(x) = \sum_{n=0}^{\infty} (n+2)(n+1)a_{n+2}x^n.
\]

נציב במד״ר:
\[
\sum_{n=0}^{\infty}(n+2)(n+1)a_{n+2}x^n
- 2x^2\sum_{n=0}^{\infty} n a_n x^{n-1}
+ 4x\sum_{n=0}^{\infty} a_n x^n
= x^2 + 2x + 2.
\]

נפשט:
\[
\sum_{n=0}^{\infty}(n+2)(n+1)a_{n+2}x^n
+ \sum_{n=0}^{\infty}(-2n)a_n x^{n+1}
+ \sum_{n=0}^{\infty}4a_n x^{n+1}
= x^2 + 2x + 2.
\]

\textbf{נזיז אינדקסים אחורה כדי להשוות חזקות:}
\[
\sum_{n=0}^{\infty}(n+2)(n+1)a_{n+2}x^n
+ \sum_{n=1}^{\infty}\big[-2(n-1)a_{n-1}\big]x^n
+ \sum_{n=1}^{\infty}4a_{n-1}x^n
= x^2 + 2x + 2.
\]

נכנס את שני הטורים:
\[
\textcolor{blue}{\sum_{n=0}^{\infty}(n+2)(n+1)a_{n+2}x^n}
+ \sum_{n=1}^{\infty}\big[-2(n-1)a_{n-1} + 4a_{n-1}\big]x^n
= x^2 + 2x + 2.
\]

׳׳נמשוך׳׳ איברים מהטור ה\textcolor{blue}{שמאלי} לפי הצורך כדי להתחיל מאותו אינדקס רץ:
\[
2a_2 + \sum_{n=1}^{\infty}x^n\Big[(n+2)(n+1)a_{n+2} - 2(n-1)a_{n-1} + 4a_{n-1}\Big] = x^2 + 2x + 2.
\]

נבצע השוואת מקדמים עד האיבר הריבועי ואז נכליל נוסחת רקורסיה:

חזקת \(x^0\):
\[
2a_2 = 2 \quad\Rightarrow\quad a_2 = 1.
\]

חזקת \(x^1\):
\[
2\cdot3a_3 + 4a_0 = 2 \quad\Rightarrow\quad a_3 = -\frac{5}{3}.
\]

חזקת \(x^2\):
\[
3\cdot4a_4 + 2a_1 = 1 \quad\Rightarrow\quad a_4 = -\frac{23}{12}.
\]

\textbf{לחזקות 3 ומעלה:}
\[
(n+2)(n+1)a_{n+2} + (-2n+6)a_{n-1} = 0, \qquad n \ge 3.
\]

ומכאן:
\[
\boxed{
a_{n+2} = \frac{2n - 6}{(n+1)(n+2)}\,a_{n-1}, \qquad n \ge 3.
}
\]

\textbf{מהזחת אינדקסים נקבל גם:}
\[
a_{n+3} = \frac{2(n+1) - 6}{(n+2)(n+3)}\,a_n = \boxed{\frac{2n - 4}{(n+2)(n+3)}\,a_n, \qquad n \ge 2}.
\]


\solution{}

נניח פתרון מהצורה:
\[
y(x) = \sum_{n=0}^{\infty} a_n x^n.
\]

נחשב נגזרות:
\[
y'(x) = \sum_{n=0}^{\infty} n a_n x^{n-1}, 
\qquad
y''(x) = \sum_{n=0}^{\infty} n(n-1)a_n x^{n-2}.
\]

נבצע הצבה במשוואה:
\[
\sum_{n=0}^{\infty} n(n-1)a_n x^{n-2}
- 2x^2 \sum_{n=0}^{\infty} n a_n x^{n-1}
+ 6x \sum_{n=0}^{\infty} a_n x^n = 0.
\]

נסדר:
\[
\sum_{n=0}^{\infty} n(n-1)a_n x^{n-2}
- 2 \sum_{n=0}^{\infty} n a_n x^{n+1}
+ 6 \sum_{n=0}^{\infty} a_n x^{n+1} = 0.
\]

נבצע הזזת אינדקסים כדי לאחד את כל הסכומים:

\[
\sum_{k=-3}^{\infty} a_{k+3}(k+3)(k+2)x^{k+1}
- 2\sum_{k=0}^{\infty} a_k k x^{k+1}
+ 6\sum_{k=0}^{\infty} a_k x^{k+1} = 0.
\]

נוציא את האיבר הראשון (שמתקבל ע׳׳י הצבה של $k=-1$) \(2a_2\):
\[
2a_2 + 
\sum_{k=0}^{\infty}
\Big[a_{k+3}(k+3)(k+2) - 2a_k k + 6a_k\Big]x^{k+1} = 0\rightarrow a_{2}=0.
\]

מהשוואת מקדמים נקבל:
\[
a_{k+3}(k+3)(k+2) - 2a_k k + 6a_k = 0
\quad\Longrightarrow\quad
\boxed{a_{k+3} = \frac{2a_k(k-3)}{(k+3)(k+2)}}.
\]

\textbf{תנאי התחלה:}
\[
a_0 = \alpha, \qquad a_1 = 0.
\]

נחשב את המקדמים הראשונים:

\[
\begin{aligned}
k=0 &:\quad a_3 = \frac{2a_0(0-3)}{(3)(2)} = -a_0 = -\alpha, \\[6pt]
k=1 &:\quad a_4 = \frac{2a_1(1-3)}{(4)(3)} = 0, \\[6pt]
k=2 &:\quad a_5 = \frac{2a_2(2-3)}{(5)(4)} = 0, \\[6pt]
k=3 &:\quad a_6 = \frac{2a_3(3-3)}{(6)(5)} = 0.
\end{aligned}
\]

כל המקדמים פרט ל-$a_{0}$ ו- $a_{3}$ מתאפסים. 
לפיכך נקבל את הפתרון הפרטי:

\[
\boxed{y(x) = \alpha - \alpha x^3 = \alpha(1 - x^3)}.
\]


\solution{}

נחשב נגזרות:
\[
y'(x) = \sum_{n=0}^{\infty} n a_n x^{n-1}, \qquad
y''(x) = \sum_{n=2}^{\infty} n(n-1)a_n x^{n-2}.
\]
שימו לב שעבור הנגזרת השנייה, נבחר להתחיל את הסכימה מ-$n=2$, מטעמי נוחות (חוקי לחלוטין שכן מתקבלת אותה התוצאה).
נציב במשוואה:
\[
\sum_{n=2}^{\infty} n(n-1)a_n x^{n-2}
- \sum_{n=0}^{\infty} n(n-1)a_n x^n
+ \sum_{n=0}^{\infty} 2a_n x^n = 0.
\]

נבצע הזחות אינדקסים ונאחד חזקות:
\[
\sum_{k=0}^{\infty} a_{k+2}(k+2)(k+1)x^k
- \sum_{k=0}^{\infty} a_k k(k-1)x^k
+ \sum_{k=0}^{\infty} 2a_k x^k = 0=\sum_{k=0}^{\infty}
\Big[a_{k+2}(k+2)(k+1)
- a_k k(k-1)
+ 2a_k\Big]x^k
\]

נשווה מקדמים ונקבל נוסחת רקורסיה:
\[
a_{k+2}(k+2)(k+1) = a_k\big[k(k-1) - 2\big] = a_k\big[k^{2}-k - 2\big] = a_k\big(k+1\big)\big(k-2\big).
\]

ולכן:
\[
\boxed{a_{k+2} = a_k\,\frac{(k-2)}{(k+2)}}.
\]

\textbf{תנאי התחלה:}
\[
y(0) = a_0, \qquad y'(0) = a_1.
\]

נחשב את האיברים הראשונים לפי הרקורסיה:

\[
\begin{aligned}
k=0: &\quad a_2 = a_0 \cdot \frac{0 - 2}{0 + 2} = -a_0, \\[6pt]
k=1: &\quad a_3 = a_1 \cdot \frac{1 - 2}{1 + 2} = -\frac{a_1}{3}, \\[6pt]
k=2: &\quad a_4 = a_2 \cdot \frac{2 - 2}{2 + 2} = 0, \\[6pt]
k=3: &\quad a_5 = a_3 \cdot \frac{3 - 2}{3 + 2} = \frac{a_3}{5} = -\frac{a_1}{15}.
\end{aligned}
\]

כדי שהטור יהיה סופי (פתרון פולינומי), נדרוש ש-\(a_5 = 0\), כלומר:
\[
\boxed{a_1 = 0}.
\]

אם \(a_1 = 0\) (כלומר \(y'(0)=0\)), אז כל האיברים האי-זוגיים מתאפסים,  
ומקבלים ביטוי סופי:

\[
y(x) = a_0 - a_0 x^2 = a_0(1 - x^2).
\]

אם \(a_1 \neq 0\), יתקבל טור אינסופי.

\textbf{נסכם:}
\[
\boxed{
y(0)=a_0 \ne 0, \quad y'(0)=0
\quad\Longrightarrow\quad
y(x) = a_0(1 - x^2).
}
\]



\solution{}

נגזור את \(y(x)\):
\[
y'(x) = \sum_{n=1}^{\infty} n a_n x^{n-1}, 
\qquad
y''(x) = \sum_{n=2}^{\infty} n(n-1)a_n x^{n-2}.
\]

נבצע הצבה במשוואה:
\[
y'' + (\sin x)y' + (\cos x)y = 0.
\]

נכתוב כל איבר כסכום כפול:

\[
\sum_{n=2}^{\infty} n(n-1)a_n x^{n-2}
+ 
\Big(\sum_{m=0}^{\infty} \frac{(-1)^m x^{2m+1}}{(2m+1)!}\Big)
\Big(\sum_{k=1}^{\infty} k a_k x^{k-1}\Big)
+
\Big(\sum_{p=0}^{\infty} \frac{(-1)^p x^{2p}}{(2p)!}\Big)
\Big(\sum_{r=0}^{\infty} a_r x^r\Big)
= 0.
\]

נרצה לאחד את כל האיברים לפי חזקות זהות של \(x\).
נחשב כל מכפלה באמצעות הזחת אינדקסים:

1.  
האיבר הראשון הוא כבר בצורה נוחה:
\[
y'' = \sum_{n=0}^{\infty} (n+2)(n+1)a_{n+2} x^n.
\]

2.  
המכפלה \((\sin x)y'\):
\[
(\sin x)y' = 
\sum_{m=0}^{\infty}\sum_{k=1}^{\infty}
\frac{(-1)^m k a_k}{(2m+1)!}\,x^{2m+k}.
\]

נגדיר \(n = 2m + k\), ואז:
\[
(\sin x)y' =
\sum_{n=0}^{\infty}
\Big(\sum_{m=0}^{\lfloor n/2 \rfloor}
\frac{(-1)^m (n-2m) a_{n-2m}}{(2m+1)!}\Big)x^n.
\]

3.  
המכפלה \((\cos x)y\):
\[
(\cos x)y = 
\sum_{p=0}^{\infty}\sum_{r=0}^{\infty}
\frac{(-1)^p a_r}{(2p)!}\,x^{2p+r}.
\]

נגדיר \(n = 2p + r\), ונקבל:
\[
(\cos x)y =
\sum_{n=0}^{\infty}
\Big(\sum_{p=0}^{\lfloor n/2 \rfloor}
\frac{(-1)^p a_{n-2p}}{(2p)!}\Big)x^n.
\]

נחבר את שלושת הביטויים ונשווה את מקדמי \(x^n\):

\[
(n+2)(n+1)a_{n+2}
+ \sum_{m=0}^{\lfloor n/2 \rfloor}\frac{(-1)^m (n-2m)a_{n-2m}}{(2m+1)!}
+ \sum_{p=0}^{\lfloor n/2 \rfloor}\frac{(-1)^p a_{n-2p}}{(2p)!} = 0.
\]

נשתמש בתנאי ההתחלה:
\[
a_0 = 0, \qquad a_1 = 1.
\]

כעת נחשב שלב אחר שלב את המקדמים הראשונים.

\textbf{עבור \(n=0\):}
\[
(2)(1)a_2 + \frac{(-1)^0 (0)a_0}{1!} + \frac{(-1)^0 a_0}{0!} = 0
\quad\Rightarrow\quad
2a_2 = 0 \Rightarrow a_2 = 0.
\]

\textbf{עבור \(n=1\):}
\[
(3)(2)a_3 
+ \sum_{m=0}^{0}\frac{(-1)^m (1-2m)a_{1-2m}}{(2m+1)!}
+ \sum_{p=0}^{0}\frac{(-1)^p a_{1-2p}}{(2p)!} = 0.
\]
קיים רק איבר אחד בכל סכום:
\[
6a_3 + (1)(a_1) + (a_1) = 0
\quad\Rightarrow\quad
6a_3 + 2a_1 = 0
\quad\Rightarrow\quad
\boxed{a_3 = -\frac{a_1}{3}}.
\]

ולפי \(a_1 = 1\):
\[
\boxed{a_3 = -\frac{1}{3}}.
\]

\textbf{תוצאה סופית:}
\[
\boxed{a_3 = -\tfrac{1}{3}}, \qquad y(0)=0, \ y'(0)=1.
\]




\solution{}

נחשב את הנגזרות של \(y(x)\):
\[
y'(x) = \sum_{n=1}^{\infty} n a_n x^{n-1}, 
\qquad
y''(x) = \sum_{n=2}^{\infty} n(n-1)a_n x^{n-2}.
\]

נקח רק כמה איברים בודדים של הפונקציות \(\sin x\) ו-\(\cos x\) עד סדר שלישי, שכן אנחנו צריכים רק את $a_{3}$ ועל כן אין טעם לפתח את הנוסחה הרקורסיבית המלאה:
\[
\sin x \approx x - \frac{x^3}{6}, \qquad
\cos x \approx 1 - \frac{x^2}{2}.
\]

נכתוב את \(y(x)\) כסדרה עד הסדר השלישי:
\[
y(x) = a_0 + a_1 x + a_2 x^2 + a_3 x^3 + \dots
\]

ונחשב את הנגזרות:
\[
y' = a_1 + 2a_2 x + 3a_3 x^2 + \dots, 
\qquad
y'' = 2a_2 + 6a_3 x + \dots
\]

נבצע הצבה במשוואה:
\[
y'' + (\sin x)y' + (\cos x)y = 0.
\]

נחליף את הפונקציות בטורי החזקה:
\[
(2a_2 + 6a_3 x)
+ (x - \tfrac{x^3}{6})(a_1 + 2a_2 x + 3a_3 x^2)
+ (1 - \tfrac{x^2}{2})(a_0 + a_1 x + a_2 x^2) = 0.
\]

נאסוף לפי חזקות \(x\).

\textbf{חזקה \(x^0\):}
\[
2a_2 + a_0 = 0.
\]

\textbf{חזקה \(x^1\):}
\[
6a_3 + a_1 + a_1 = 0 
\quad \Rightarrow \quad 6a_3 + 2a_1 = 0
\quad \Rightarrow \boxed{a_3 = -\frac{a_1}{3}}.
\]

\textbf{נציב את תנאי ההתחלה:}
\[
y(0)=a_0=0, \qquad y'(0)=a_1=1.
\]

ולכן:
\[
a_2 = 0, \qquad a_3 = -\frac{1}{3}.
\]

\textbf{תוצאה סופית:}
\[
\boxed{a_3 = -\tfrac{1}{3}}.
\]

%%%CUT%%%

\newpage
\subsection{ התמרת לפלס (עבור מקדמים קבועים)}

שיטת \textbf{התמרת לפלס} היא אחת מהכלים המרכזיים בפתרון משוואות דיפרנציאליות, 
בעיקר בתחום ההנדסה, הפיזיקה והבקרה.  
הרעיון הבסיסי הוא להמיר בעיה בתחום הזמן \(t\) — שבו נגזרות מתארות שינוי —  
לבעיה שקולה בתחום חדש הנקרא \textbf{תחום לפלס}, המתואר באמצעות המשתנה \(s\).  

במקום לפתור משוואה הכוללת נגזרות (כמו \(y'' + y' + y = f(t)\)),  
נבצע התמרה שתהפוך את הנגזרות לאיברים אלגבריים פשוטים (כמו \(s^2Y(s)\) או \(sY(s)\)),  
כך שהמשוואה תהפוך ממשוואה דיפרנציאלית למשוואה אלגברית.  
לאחר הפתרון בתחום לפלס, ניתן להחזיר את הפתרון חזרה לתחום הזמן בעזרת ההתמרה ההפוכה.

\textbf{הגדרה
פורמלית}

\begin{equation}\label{Lap}
\boxed{
\mathcal{L}\{f(t)\} = F(s) = \int_{0}^{\infty} e^{-st} f(t)\,dt,
\qquad s \in \mathbb{C},\ \Re(s) > s_0.
}
\end{equation}

במילים אחרות — ההתמרה היא אינטגרל של הפונקציה \(f(t)\) מוכפלת בפונקציה המעריכית \(e^{-st}\).  
הגורם \(e^{-st}\) "מנחית" את התרומות של \(f(t)\) בערכים גדולים של \(t\), 
וכך הופך את האינטגרל לבעל התכנסות במקרים רבים.  
המשתנה \(s\) הוא \textbf{משתנה מרוכב}, המייצג שילוב של תדר (החלק המדומה של \(s\)) 
וקצב דעיכה (החלק הממשי של \(s\)).

באופן אינטואיטיבי, נוכל לומר כי:
- \(f(t)\) מתארת את הפונקציה בתחום הזמן.
- \(F(s)\) מתארת את אותו תהליך אך במרחב התדר–דעיכה. נראה זאת גרפית באיור הבא:
\begin{figure}[H]
\centering
\begin{tikzpicture}[scale=1.6]
  % axes
  \draw[->, thick] (0,0)--(5.5,0) node[right]{$t$};
  \draw[->, thick] (0,0)--(0,2.6) node[above]{$f(t)$};

  % original f(t)
  \draw[domain=0:4.5, smooth, variable=\x, blue, ultra thick]
    plot ({\x},{2*exp(-0.5*\x)}) node[right, blue, xshift=-6cm, yshift=2cm] {$f(t)$};

  % weighting exp(-st)
  \draw[domain=0:4.5, smooth, variable=\x, red, dashed, thick]
    plot ({\x},{exp(-0.5*\x)}) node[right, red, xshift=-4cm, yshift=0.6cm] {$e^{-st}$};

  % weighted product f(t)e^{-st}
  \draw[domain=0:4.5, smooth, variable=\x, darkgreen, thick]
    plot ({\x},{2*exp(-1*\x)}) node[right, darkgreen, xshift=-5.5cm, yshift=0.25cm] {$f(t)e^{-st}$};

  % highlight initial equality at t=0
  \filldraw[fill=black] (0,2) circle (0.04);

\end{tikzpicture}
\caption{\textbf{השפעת המקדם \(e^{-st}\) על הפונקציה \(f(t)\):}  
הפונקציה המקורית \(f(t)\) (בכחול) מוכפלת במקדם דעיכה מעריכי \(e^{-st}\) (באדום).  
לאחר השקילה מתקבלת פונקציה חדשה \(f(t)e^{-st}\) (בירוק), השואפת לאפס מהר יותר.}
\end{figure}

לאחר ביצוע ההתמרה, נגזרות בזמן הופכות לפעולות אלגבריות פשוטות:

\[
\begin{aligned}
\mathcal{L}\{f'(t)\} &= sF(s) - f(0), \\[4pt]
\mathcal{L}\{f''(t)\} &= s^2F(s) - s f(0) - f'(0), \\[4pt]
\mathcal{L}\{f^{(n)}(t)\} &= s^nF(s) - s^{n-1}f(0) - \dots - f^{(n-1)}(0),
\end{aligned}
\]
כאשר $f(0)$ הוא תנאי ההתחלה.

זוהי תכונה עוצמתית במיוחד — כל נגזרת בזמן הופכת למכפלה ב-\(s\) בלבד.  
כך משוואות דיפרנציאליות הופכות למשוואות אלגבריות שקל לפתור.

חשוב לציין בשלב זה את
\textbf{הלינאריות של התמרת לפלס:}
\begin{equation}\label{lin_lap}
\mathcal{L}[a\,f(t) + b\,g(t)](s)
= a\,\mathcal{L}[f(t)](s) + b\,\mathcal{L}[g(t)](s).
\end{equation}
נראה כעת כמה דוגמאות להעברת פונקציות מישר הזמן לישר התדר באמצעות התמרת לפלס.

\vspace{1cm}
\textbf{דוגמה 1: פונקציה מעריכית}

נחשב:
\[
\mathcal{L}\{e^{at}\} = \int_0^{\infty} e^{-st} e^{at}\,dt
= \int_0^{\infty} e^{-(s-a)t}\,dt
= \left[ \frac{e^{-(s-a)t}}{-(s-a)} \right]_{0}^{\infty}
= \frac{1}{s-a}\Big( e^{-(s-a)\cdot 0} - \lim_{t\to\infty} e^{-(s-a)t} \Big).
\]

בהנחה ש-\(\Re(s) > a\), האיבר \(e^{-(s-a)t}\) שואף ל-0 כאשר \(t \to \infty\),
ולכן:
\[
\mathcal{L}\{e^{at}\}
= \frac{1}{s-a}\big(1 - 0\big)
= \boxed{\frac{1}{s-a}}, \qquad \Re(s) > a.
\]

\textbf{דוגמה 2: פונקציה סינוסואידלית}

נחשב את התמרת לפלס של הפונקציה \(\sin(at)\):
\[
\mathcal{L}\{\sin(at)\} = \int_0^{\infty} e^{-st}\sin(at)\,dt.
\]

נשתמש בזהות המעריכית:
\[
\sin(at) = \frac{e^{iat} - e^{-iat}}{2i}.
\]

נציב באינטגרל:
\[
\mathcal{L}\{\sin(at)\}
= \frac{1}{2i}\int_0^{\infty} e^{-st}\big(e^{iat} - e^{-iat}\big)\,dt
= \frac{1}{2i}\int_0^{\infty}\big(e^{-(s - ia)t} - e^{-(s + ia)t}\big)\,dt.
\]

נחשב כל אינטגרל בנפרד:

\[
\int_0^{\infty} e^{-(s - ia)t}\,dt
= \left[\frac{e^{-(s - ia)t}}{-(s - ia)}\right]_0^{\infty}
= \frac{1}{s - ia}\Big(1 - \lim_{t \to \infty} e^{-(s - ia)t}\Big).
\]

בהנחה ש-\(\Re(s) > 0\), האיבר \(e^{-(s - ia)t}\) שואף לאפס עבור \(t \to \infty\),
ולכן:
\[
\int_0^{\infty} e^{-(s - ia)t}\,dt = \frac{1}{s - ia}.
\]

באופן דומה נקבל:
\[
\int_0^{\infty} e^{-(s + ia)t}\,dt = \frac{1}{s + ia}.
\]

נציב חזרה:
\[
\mathcal{L}\{\sin(at)\}
= \frac{1}{2i}\left(\frac{1}{s - ia} - \frac{1}{s + ia}\right).
\]

נפשט:
\[
\mathcal{L}\{\sin(at)\}
= \frac{1}{2i} \cdot \frac{(s + ia) - (s - ia)}{(s - ia)(s + ia)}
= \frac{1}{2i} \cdot \frac{2ia}{s^2 + a^2}
= \boxed{\frac{a}{s^2 + a^2}}, \qquad \Re(s) > 0.
\]


\textbf{דוגמה 3: פולינום בזמן}

נחשב עבור \(f(t) = t^n\):
\[
\mathcal{L}\{t^n\} = \int_0^{\infty} e^{-st}t^n\,dt.
\]
נשתמש בהגדרת פונקציית $\Gamma$ (שאולי ידועה לחלקכם בשלב זה, או אולי לא):
\[
\Gamma(n+1) = \int_0^{\infty} e^{-u}u^n\,du = n!,
\]
ומציבים \(u = st \Rightarrow du = s\,dt\):
\[
\mathcal{L}\{t^n\}
= \frac{1}{s^{n+1}}\int_0^{\infty} e^{-u}u^n\,du
= \boxed{\frac{n!}{s^{n+1}}}, \qquad s > 0.
\]

נציג כמה נקודות חשובות וחידודים:

\begin{itemize}
  \item בספר זה נציג את השיטה רק עבור משוואות עם מקדמים קבועים.
  \item 
  השימוש בשיטה נפוץ כשאחת מהפונקציות במד"ר רציפה למקוטעין. למעשה, ניתן לעבוד עם אי רציפות מסוג 0 וסוג 1 (אי רציפות מסוג סליקה ומסוג וקפיצה, בהתאמה).

  \item התמרת לפלס $\mathcal{L}[f(t)](s)$ מאפשרת לנו להעביר את הבעיה מתחום הזמן לתחום התדירות, שם למעשה אנו מטפלים במשוואות אלגבריות, ולא במשוואות דיפרנציאליות. זוהי למעשה מוטיביציה מספקת שכן משוואה המערבת משתנה תלוי ובלתי תלוי בלבד (ללא נגזרות), היא למעשה משוואה אלגברית, או משוואה דיפרנציאלית ׳׳מסדר 0׳׳.
  \item לאחר מכן נשתמש בהתמרת לפלס ההפוכה $\mathcal{L}^{-1}[F(s)](t)$ כדי לחזור ולמצוא את $f(t)$.
  \item בפיזיקה – $t$ הוא הזמן, $s$ הוא תדר, וישר לפלס הוא ישר התדר.
\end{itemize}

נציג רשימת התמרות מיידיות (אשר ניתן להשתמש בהם ישירות במרבית הקורסים במשוואות דיפרנציאליות רגילות) אשר מתקבלות מפתרון משוואה \ref{Lap} עבור אותן הפונקציות, בטבלה \ref{lap_table}. 

\begin{table}\label{lap_table}
\centering
\caption{\textbf{טבלת התמרת לפלס}}
\renewcommand{\arraystretch}{2.6}  % increase vertical spacing
\setlength{\tabcolsep}{8pt}

\begin{tabular}{|c|c|c|}
\hline
$N$ &
\textbf{פונקציה בתחום הזמן} \(f(t)=\mathcal{L}^{-1}[F(s)](t)\) &
\textbf{פונקציה בתחום לפלס} \(F(s)=\mathcal{L}[f(t)](s)\) \\
\hline
1 & \(k_1 f_1(t) + k_2 f_2(t)\) & \(k_1 F_1(s) + k_2 F_2(s)\) \\
\hline
2 & \(f(kt),\; k>0\) & \(\dfrac{1}{k}F\!\left(\dfrac{s}{k}\right)\) \\
\hline
3 & \(f^{(n)}(t)\) & \(s^n F(s) - s^{n-1}f(0) - s^{n-2}f'(0) - \dots - f^{(n-1)}(0)\) \\
\hline
4 & \(t^{n}\cdot f(t)\) & \((-1)^n F^{(n)}(s)\) \\
\hline
5 & \(e^{at}f(t)\) & \(F(s - a)\) \\
\hline
6 & \(u_c(t)\) & \(\dfrac{e^{-cs}}{s}, \; s>0\) \\
\hline
7 & \(u_c(t)f(t - c)\) & \(e^{-cs}F(s)\) \\
\hline
8 & \(\displaystyle \int_0^{t} f(\tau)\,d\tau\) & \(\dfrac{F(s)}{s}\) \\
\hline
9 & \(\displaystyle \int_0^t f(t-\tau)g(\tau)\,d\tau\) & \(F(s)G(s)\) \\
\hline
10 & \(1\) & \(\dfrac{1}{s}, \; s>0\) \\
\hline
11 & \(e^{at}\) & \(\dfrac{1}{s - a}, \; s>a\) \\
\hline
12 & \(t^n\) & \(\dfrac{n!}{s^{n+1}}, \; s>0\) \\
\hline
13 & \(t^n e^{at}\) & \(\dfrac{n!}{(s - a)^{n+1}}, \; s>a\) \\
\hline
14 & \(\sin(at)\) & \(\dfrac{a}{s^2 + a^2}, \; s>0\) \\
\hline
15 & \(\cos(at)\) & \(\dfrac{s}{s^2 + a^2}, \; s>0\) \\
\hline
16 & \(e^{at}\sin(bt)\) & \(\dfrac{b}{(s - a)^2 + b^2}\) \\
\hline
17 & \(e^{at}\cos(bt)\) & \(\dfrac{s - a}{(s - a)^2 + b^2}\) \\
\hline
\end{tabular}
\end{table}

\subsubsection{פונקציית הביסייד}

פונקציית הביסייד, 
היא אחת הפונקציות הבסיסיות והחשובות ביותר בניתוח מערכות לינאריות ובתורת האותות.  
היא מייצגת הפעלה פתאומית של מערכת בזמן מסוים – לדוגמה, כאשר מפעילים מקור מתח, או כאשר גוף מתחיל לנוע ברגע \(t=c\).  

באופן אינטואיטיבי, ניתן לחשוב עליה כפונקציה שעוברת מערך \(0\) לערך \(1\) בדיוק ברגע \(t=c\).  
ערך זה קובע את \textbf{נקודת הקפיצה} של הפונקציה, ומשמש לתיאור מצבים שבהם מערכת "נדלקת" או "מופעלת" לאחר זמן השהיה קבוע.  
פונקציה זו משמשת גם לבניית פונקציות אחרות בחלקים, באמצעות שילובים והזזות בזמן.
הגדרתה המתמטית היא כדלקמן
\begin{equation}
u_c(t) = 
\begin{cases}
0, & 0 \le t < c,\\[4pt]
1, & t \ge c.
\end{cases}
\end{equation}
ובצורה גרפית
\begin{figure}[H]
\centering
\begin{tikzpicture}[scale=1.4]
  % axes
  \draw[->, thick] (0,0)--(5,0) node[right]{$t$};
  \draw[->, thick] (0,0)--(0,2) node[above]{$u_c(t)$};

  % function
  \draw[blue, ultra thick] (0,0)--(2,0);
  \draw[blue, ultra thick] (2,1)--(5,1);
  \draw[dashed] (2,0)--(2,1);

  % labels
  \node[below] at (2,0){$t=c$};
  \node[right, blue] at (4.5,1.25){$u_c(t)$};
  \node[below left] at (0,0){$0$};
  \node[left] at (0,1){$1$};
\end{tikzpicture}
\caption{\textbf{פונקציית הביסייד \(u_c(t)\):}  
הפונקציה נשארת כבויה עד לרגע \(t=c\), ולאחר מכן "נדלקת" לערך קבוע של \(1\).}
\end{figure}

פונקציה זו מהווה גם את אבני היסוד של ההתמרת לפלס:  
היא מאפשרת לתאר בצורה נוחה מערכות המתפקדות רק לאחר זמן מסוים,  
כאשר ההשהיה מתבטאת בפקטור \(e^{-cs}\) בתחום לפלס.

מטבלה \ref{lap_table}, נראה את התמרת לפלס של פונקציית הביסייד כפול פונקציה מוזזת בזמן
\begin{equation}
\mathcal{L}\big[u_c(t)f(t-c)\big](s) = e^{-cs}\,\mathcal{L}[f(t)](s).
\end{equation}
כעת נרצה ליצור מניפולציה מסוימת, בעזרת פונקציות הביסייד, אשר תוכל להביא פונקציות מסוימות לידי ביטוי במקטעים תחומים. הייצוג המתמטי אפוא יהיה

\begin{equation}
u_c(t)-u_d(t)
= f(t) =
\begin{cases}
0, & t<c,\\[3pt]
1, & c\le t<d,\\[3pt]
0, & t\ge d.
\end{cases}
\end{equation}

ובצורה גרפית:
\begin{figure}[H]
\centering
\begin{tikzpicture}[scale=1.4]
  % axes
  \draw[->, thick] (0,0)--(6,0) node[right]{$t$};
  \draw[->, thick] (0,0)--(0,2) node[above]{$f(t)$};

  % yellow filled area (pulse)
  \fill[yellow!40] (1,0) rectangle (3,1);

  % pulse outline
  \draw[blue, ultra thick] (0,0)--(1,0);
  \draw[blue, ultra thick] (1,1)--(3,1);
  \draw[blue, ultra thick] (3,0)--(6,0);

  % dashed verticals
  \draw[dashed] (1,0)--(1,1);
  \draw[dashed] (3,0)--(3,1);

  % labels
  \node[below] at (1,0){$t=c$};
  \node[below] at (3,0){$t=d$};
  \node[right, blue] at (4.5,1){$u_c(t)-u_d(t)$};
  \node[left] at (0,1){$1$};
  \node[below left] at (0,0){$0$};
\end{tikzpicture}
\caption{\textbf{פולס מלבני באמצעות פונקציות הביסייד:}  
ההפרש \(u_c(t)-u_d(t)\) מייצר פולס "מואר" בין הזמנים \(t=c\) ל־\(t=d\),  
כלומר הדלקה ב-\(t=c\) וכיבוי ב-\(t=d\).}
\end{figure}
שימו לב כי כעת ניתן להכפיל כל פונקציה על ישר הזמן בפונקציה הביסייד זו, ולהביא אותה לידי ביטוי \textbf{רק} באותו האינטרוול $(c,d)$.

\textbf{דוגמה לשימוש בפונקציית הביסייד}

נניח כי ברצוננו לתאר פונקציה $g(t)$ המתנהגת באופן שונה בפרקי זמן שונים.  
לדוגמה, פונקציה המתארת מערכת שנדלקת, פועלת במשך זמן מסוים, ולאחר מכן דועכת אקספוננציאלית.  
נגדיר את $g(t)$ כך:

\[
g(t) =
\begin{cases}
t^2, & 0 \le t < 1,\\[4pt]
\sin t, & 1 \le t < 8\pi,\\[4pt]
e^{-4t}, & t \ge 8\pi.
\end{cases}
\]

ניתן לראות כי $g(t)$ \textbf{אינה רציפה בנקודות השבירה} $t=1$ ו־$t=8\pi$.  
במילים אחרות, קיימות נקודות "קפיצה" שבהן הביטוי של $g(t)$ משתנה לפתע.  
אם היינו רוצים לפתור בעיה דיפרנציאלית שבה הפונקציה הזו מהווה את החלק האי-הומוגני של משוואה דיפרנציאלית,  
היינו נדרשים להשתמש בהתמרת לפלס.  
כדי לעשות זאת, עלינו לרשום את $g(t)$ באמצעות פונקציות הביסייד,  
כך שכל קטע בזמן ייוצג על ידי ביטוי הכולל את ההפעלה או הכיבוי של פונקציה אחרת.

נשים לב למעברים:
\begin{itemize}
  \item ב־\(t=0\): הפונקציה \(t^2\) "נדלקת".
  \item ב־\(t=1\): הפונקציה \(t^2\) "נכבית" והפונקציה \(\sin t\) "נדלקת".
  \item ב־\(t=8\pi\): הפונקציה \(\sin t\) "נכבית" והפונקציה \(e^{-4t}\) "נדלקת".
\end{itemize}

מכאן ניתן לבטא את הפונקציה $g(t)$ בעזרת פונקציות הביסייד באופן הבא:
\[
g(t) = [u_0(t) - u_1(t)]t^2 + [u_1(t) - u_{8\pi}(t)]\sin t + u_{8\pi}(t)e^{-4t}.
\]

בכתיב זה, כל ביטוי מהווה את הפעלת הקטע המתאים של הפונקציה בזמן המתאים:
- $u_0(t) - u_1(t)$ מפעיל את $t^2$ רק כאשר \(0 \le t < 1\).
- $u_1(t) - u_{8\pi}(t)$ מפעיל את $\sin t$ בין $1$ ל־$8\pi$.
- $u_{8\pi}(t)$ מפעיל את $e^{-4t}$ רק לאחר $t \ge 8\pi$. נפרק את החישובים לשלושה מקטעים ונסביר. תחילה, עבור מקרים בהם אנו דנים בזמנים אי-שליליים בלבד, יש לקבל את ההגדרה המתמטית $u_{0}(t)\equiv1$. נכתוב את פונקציות הביסייד המתאימות לכל מקטע ונראה את הערך שהן נותנות:

\[
\begin{aligned}
u_0(t) &\equiv 1, \\[4pt]
u_1(t) &=
  \begin{cases}
  0, & 0 \le t < 1,\\
  1, & t \ge 1,
  \end{cases}
\qquad
u_{8\pi}(t) =
  \begin{cases}
  0, & 1 \le t < 8\pi,\\
  1, & t \ge 8\pi.
  \end{cases}
\end{aligned}
\]

באמצעות ייצוג זה ניתן לבצע התמרת לפלס לפונקציות שאינן רציפות,  
ולחשב את הפתרון עבור כל תחום זמן בנפרד.
בצורה גרפית, כך הדבר נראה:
\begin{figure}[H]
\centering
\begin{tikzpicture}[scale=1.2]
  % axes
  \draw[->, thick] (0,0)--(10,0) node[right]{$t$};
  \draw[->, thick] (0,0)--(0,3) node[above]{$g(t)$};

  % t^2 region
  \fill[yellow!30] (0,0) rectangle (1,1);
  \draw[domain=0:1, smooth, variable=\x, blue, ultra thick]
    plot ({\x},{\x*\x});

  % sin region
  \fill[orange!25] (1,0) rectangle (8,2.5);
  \draw[domain=1:8, smooth, variable=\x, red, ultra thick]
    plot ({\x},{sin(deg(\x))+1.5});

  % exp region
  \fill[green!20] (8,0) rectangle (10,1.5);
  \draw[domain=8:10, smooth, variable=\x, green!70!black, ultra thick]
    plot ({\x},{exp(-0.4*\x)});

  % dashed dividers
  \draw[dashed] (1,0)--(1,3);
  \draw[dashed] (8,0)--(8,3);

  % labels
  \node[below] at (1,0){$1$};
  \node[below] at (8,0){$8\pi$};
  \node[blue] at (0.6,1){$t^2$};
  \node[red] at (4,2.4){$\sin t$};
  \node[green!70!black] at (9,1){$e^{-4t}$};
\end{tikzpicture}
\caption{\textbf{דוגמה לפונקציה מקטעית \(g(t)\):}  
הפונקציה מורכבת משלושה תחומים: \(t^2\) בתחום הראשון,  
\(\sin t\) בתחום האמצעי, ו-\(e^{-4t}\) בתחום האחרון.}
\end{figure}

%%%CUT%%%

\example{}

מצא פתרון פרטי למשוואה הבאה:
\[
y'' - 6y' + 15y = 2\sin(3t) \quad ; \quad y(0)=-1,\; y'(0)=-4
\]

\explanation{}
נפתור באמצעות התמרת לפלס (ולשם תרגול בלבד – השיטה העדיפה במקרה זה היא כמובן השוואת מקדמים).

אם כן, נפעיל התמרת לפלס על שני אגפי המשוואה:
\[
L[y'' - 6y' + 15y] = L[2\sin(3t)](s)
\]
ומליניאריות אופרטור לפלס (ראו משוואה (\ref{lin_lap})) נוכל לפרק את הביטוי:
\[
L[y''](s) - 6L[y'](s) + 15L[y](s) = 2L[\sin(3t)](s)
\]

לפי נוסחאות ההתמרה מדף הנוסחאות (ראו טבלה~\ref{lap_table}, שורה~3), נקבל את ההתמרה של הנגזרות הבאות:
\[
L[y''] = s^2L[y] - s\,y(0) - y'(0), 
\qquad 
L[y'] = sL[y] - y(0)
\]

נציב במשוואה הדיפרנציאלית המקורית ונקבל:
\[
\big(s^2L[y](s) - s\cdot y(0) - y'(0)\big)
- 6\big(sL[y](s) - y(0)\big)
+ 15L[y](s)
= 2\cdot L[\sin(3t)](s)
\]

לפי טבלת ההתמרות (טבלה~\ref{lap_table}, שורה~14) נקבל:
\[
L[\sin(3t)](s) = \frac{3}{s^2 + 9}
\]

ולכן:
\[
(s^2L[y](s) - s\cdot y(0) - y'(0)) - 6(sL[y](s) - y(0)) + 15L[y](s)
= 2\cdot \frac{3}{s^2 + 9}
\]

לאחר הצבת תנאי ההתחלה, כינוס איברים דומים והעברת אגפים, נקבל: 
\[
y(0) = -1, \qquad y'(0) = -4
\]
נקבל:
\[
(s^2 - 6s + 15)L[y](s) = \frac{6}{s^2 + 9} - s + 2
\]
נבודד את התמרת לפלס של הפתרון המיוחל ולאחר קצת אלגברה נקבל:
\[
L[y](s) = \frac{6}{(s^2 + 9)(s^2 - 6s + 15)} - \frac{s - 2}{s^2 - 6s + 15}
\]
שימו לב כי בשלב זה, האיברים השונים שמרכיבים את אגף ימין של המשוואה, לא ניתנים לזיהוי בקלות מהטבלה שלנו. בשלב זה, כשיש לנו באגף שמאל את התמרת הפתרון המבוקש, היינו רוצים להפעיל את ההתמרה ההפוכה ולסיים את התרגיל. כאמור, לאור העובדה שאגף ימין לא מתאים עדיין להתמרה הפוכה, ניאלץ לפתח אותו מבחינה אלגברית לביטויים מוכרים מהטבלה. לשם כך, נשתמש בשיטת הפירוק לשברים חלקיים (זאת על מנת, להוריד ברוב המקרים את דרגת הפולינום שבמכנה של השבר לדרגה מקסימלית של 2 ולהגעה למשהו מוכר מהטבלה).

\textbf{פירוק לשברים חלקיים/פירוק הביסייד}

כעת נפרק את הביטוי:
\[
\frac{6}{(s^2 + 9)(s^2 - 6s + 15)} = \frac{As + B}{s^2 + 9} + \frac{Cs + D}{s^2 - 6s + 15}
\]

נכפול ונשווה מקדמים:
\[
(A s + B)(s^2 - 6s + 15) + (C s + D)(s^2 + 9) = 6
\]

נפרוס:
\[
A s^3 - 6A s^2 + 15A s + B s^2 - 6B s + 15B + C s^3 + 9C s + D s^2 + 9D = 6
\]

נשווה מקדמים ונקבל מערכת משוואות:
\[
\begin{cases}
A + C = 0 \\
-6A + B + D = 0 \\
15A - 6B + 9C = 0 \\
15B + 9D = 6
\end{cases}
\]

פתרון המערכת נותן:
\[
A = -\frac{1}{10}, \quad B = \frac{1}{2}, \quad C = \frac{1}{10}, \quad D = \frac{1}{10}
\]

נחזיר לביטוי:
\[
\frac{6}{(s^2 + 9)(s^2 - 6s + 15)} = \frac{5 - s}{10(s^2 - 6s + 15)} + \frac{s + 1}{10(s^2 + 9)}
\]

נחזור למשוואת לפלס הכוללת:
\[
L[y](s) = \frac{5 - s}{10(s^2 - 6s + 15)} + \frac{s + 1}{10(s^2 + 9)} - \frac{s - 2}{s^2 - 6s + 15}
\]

נכנס איברים דומים ונקבל:
\[
L[y](s) = \frac{1}{10}\left( \frac{-11s + 25}{s^2 - 6s + 15} + \frac{s + 1}{s^2 + 9} \right)
\]

כדי לבצע התמרה הפוכה ולהגיע למשהו מוכר מהטבלה, נרצה להשלים לריבוע:
\[
s^2 - 6s + 15 = (s - 3)^2 + 6.
\]
המוטיבציה כאן היא להגיע ע׳׳י השלמה לריבוע לשורות 16-17 בטבלה~\ref{lap_table}. 
ולכן:
\[
L[y](s) = \frac{1}{10}\left( \frac{-11s + 25}{(s - 3)^2 + 6} + \frac{s}{s^2 + 9} + \frac{1}{s^2 + 9} \right)
\]

נבצע מניפולציה נוספת:
\[
-11s + 25 = -11(s - 3) - 8
\]
ולכן:
\[
L[y](s) = \frac{1}{10}\left( \frac{-11(s - 3) - 8}{(s - 3)^2 + 6} + \frac{s}{s^2 + 9} + \frac{1}{s^2 + 9} \right)
\]
ולבסוף, עוד קצת מניפולציות אלגבריות – הכפלה וחלוקה באותו המספר, הוצאת גורם משותף, כל זאת על מנת להפוך את הביטויים לצורות שקיימות בטבלת הנוסחאות:

\[
L[y](s) =
-\frac{11}{10}\cdot\textcolor{blue}{\frac{(s - 3)}{(s - 3)^2 + (\sqrt{6})^2}}
-\frac{1}{10}\cdot\frac{8}{\sqrt{6}}\cdot\textcolor{red}{\frac{\sqrt{6}}{(s - 3)^2 + (\sqrt{6})^2}}
+\frac{1}{10}\cdot\textcolor{darkgreen}{\frac{s}{s^2 + 9}}
+\frac{1}{10}\cdot\frac{1}{3}\cdot\textcolor{darkpurple}{\frac{3}{s^2 + 9}}
\]

ניעזר שוב בטבלת ההתמרות \ref{lap_table}, שורות 14-17 ונקבל את הפתרון הפרטי לבעיה:
\[
\boxed{y(t) = -\frac{11}{10}\textcolor{blue}{e^{3t}\cos(\sqrt{6}t)}
- \frac{4}{5\sqrt{6}}\textcolor{red}{e^{3t}\sin(\sqrt{6}t)}
+ \frac{1}{10}\textcolor{darkgreen}{\cos(3t)}
+ \frac{1}{30}\textcolor{darkpurple}{\sin(3t)},
\quad t \ge 0}
\]

נבחין כי הפתרון זהה לפתרון שהיינו מקבלים בשיטת השוואת המקדמים.

\textbf{שלב 1: הפתרון ההומוגני}

נרשום את המשוואה ההומוגנית המתאימה:
\[
y'' - 6y' + 15y = 0.
\]

הפולינום האופייני הוא:
\[
r^2 - 6r + 15 = 0.
\]

נפתור:
\[
r_{1,2} = 3 \pm i\sqrt{6}.
\]

ולכן הפתרון ההומוגני הוא:
\[
y_H(t) = e^{3t}\big(C_1\cos(\sqrt{6}t) + C_2\sin(\sqrt{6}t)\big).
\]

\textbf{שלב 2: הפתרון הפרטי}

נבחן את אגף ימין של המשוואה:
\[
G(t) = 2\sin(3t).
\]
נזהה את הפרמטרים:
\[
a = 0, \qquad b = 3, \qquad m = 0.
\]

 \(r = a + ib = 0 + 3i\) הוא לא שורש של הפולינום האופייני
ולכן \(k = 0\).

נציע לפיכך פתרון פרטי מהצורה:
\[
y_P(t) = R_0(t)e^{0t}\cos(3t) + S_0(t)e^{0t}\sin(3t)= A\cos(3t) + B\sin(3t).
\] 

נגזור את הפתרון:
\[
y_P'(t) = -3A\sin(3t) + 3B\cos(3t),
\qquad
y_P''(t) = -9A\cos(3t) - 9B\sin(3t).
\]

נציב במשוואה המקורית:
\[
(-9A\cos(3t) - 9B\sin(3t))
- 6(-3A\sin(3t) + 3B\cos(3t))
+ 15(A\cos(3t) + B\sin(3t))
= 2\sin(3t).
\]

נאסוף איברים של \(\cos(3t)\) ושל \(\sin(3t)\):

\[
\big(-9A - 18B + 15A\big)\cos(3t)
+
\big(-9B + 18A + 15B\big)\sin(3t)
= 2\sin(3t).
\]

נשווה מקדמים:
\[
\begin{cases}
-9A - 18B + 15A = 0 \\[4pt]
-9B + 18A + 15B = 2
\end{cases}
\quad \Rightarrow \quad
\begin{cases}
6A - 18B = 0 \\[3pt]
18A + 6B = 2
\end{cases}
\]

נפתור את המערכת ונקבל:
\[
A = \frac{1}{10}, \qquad B = \frac{1}{30}.
\]

ולכן:
\[
y_P(t) = \frac{1}{10}\cos(3t) + \frac{1}{30}\sin(3t).
\]

\textbf{שלב 3: חיבור הפתרונות}

נחבר את שני החלקים:
\[
y(t) = y_H(t) + y_P(t)
= e^{3t}\big(C_1\cos(\sqrt{6}t) + C_2\sin(\sqrt{6}t)\big)
+ \frac{1}{10}\cos(3t) + \frac{1}{30}\sin(3t).
\]

כעת נציב את תנאי ההתחלה \(y(0)=-1,\; y'(0)=-4\) ונפתור עבור \(C_1,C_2\).

נחשב תחילה נגזרת:
\[
y'(t) = 3e^{3t}\big(C_1\cos(\sqrt{6}t) + C_2\sin(\sqrt{6}t)\big)
+ e^{3t}\big(-C_1\sqrt{6}\sin(\sqrt{6}t) + C_2\sqrt{6}\cos(\sqrt{6}t)\big)
- \tfrac{3}{10}\sin(3t) + \tfrac{1}{10}\cos(3t).
\]

נציב \(t=0\):
\[
\begin{cases}
y(0) = C_1 + \tfrac{1}{10} = -1 \\[4pt]
y'(0) = 3C_1 + \sqrt{6}C_2 + \tfrac{1}{10} = -4
\end{cases}
\]

נקבל:
\[
\begin{cases}
C_1 = -\tfrac{11}{10} \\[3pt]
C_2 = -\tfrac{4}{5\sqrt{6}}
\end{cases}
\]

ולכן הפתרון הפרטי זהה למה שקיבלנו בהתמרת לפלס:
\[
\boxed{
y(t) = -\frac{11}{10}e^{3t}\cos(\sqrt{6}t)
- \frac{4}{5\sqrt{6}}e^{3t}\sin(\sqrt{6}t)
+ \frac{1}{10}\cos(3t)
+ \frac{1}{30}\sin(3t),
\quad t \ge 0.
}
\]


\begin{remark}
שימו לב כי בהתמרת לפלס, כאשר קיימים תנאי התחלה, הם נבלעים באופן טבעי בתוך ההתמרה, ואנו מקבלים ׳׳בבת אחת׳׳ את הפתרון הפרטי, בניגוד לשיטת השוואת המקדמים בה אנו צריכים לחפש את הפתרון ההומוגני והפרטי. לעומת זאת, החסרון של התמרת לפלס הוא האלגברה הכבדה והסיזיפית שכרוכה בחיפוש ההתמרה המתאימה בטבלה.
\end{remark}

\example{}
קבלו פתרון פרטי למשוואה הבאה:

\[
y'' - y' - 2y = g(t), \qquad y(0) = 1,\; y'(0) = 2
\]
כאשר:
\[
g(t) =
\begin{cases}
1, & 0 \le t < 1, \\[4pt]
0, & t \ge 1.
\end{cases}
\]

\explanation{}
נפתור את הבעיה הבאה בעזרת התמרת לפלס, שכן אגף ימין הוא פונקציה לא רציפה.

\textbf{שלב 1 – כתיבת } \( g(t) \) \textbf{בעזרת פונקציות יחידה:}
\[
g(t) = u_0(t) - u_1(t) = 1 - u_1(t)
\]

\begin{center}
\begin{tikzpicture}[scale=1.3, thick]
  % axes
  \draw[->] (-0.3,0) -- (3,0) node[right] {$t$};
  \draw[->] (0,-0.2) -- (0,1.5) node[above] {$g(t)$};

  % horizontal lines
  \draw[blue, ultra thick] (0,1) -- (1,1);
  \draw[blue, ultra thick] (1,0) -- (2.5,0);

  % vertical jump
  \draw[blue, ultra thick] (1,0)--(1,1);
  \filldraw[white, thick] (1,1) circle (1.5pt); % open circle
  \filldraw[blue] (0,1) circle (1.5pt); % filled start

  % labels
  \node[anchor=south west, blue] at (-0.25,1) {$1$};
  \node[anchor=north east, blue] at (0.09,0) {$0$};
  \node[anchor=north] at (1,-0.1) {$1$};

  % text label
  \node[anchor=west, align=left] at (1.4,1) {\small $\displaystyle g(t)=u_0(t)-u_1(t)$};
\end{tikzpicture}
\end{center}


\textbf{שלב 2 – הפעלת אופרטור לפלס:}
\[
L[y''] - L[y'] - 2L[y] = L[1 - u_1(t)](s)
\]

נשתמש בנוסחאות מהטבלה \ref{lap_table}:  
שורה~3 (נגזרות), שורה~10 (\(1 \mapsto \tfrac{1}{s}\)), ושורה~6 (\(u_c(t) \mapsto \dfrac{e^{-cs}}{s}\)) ונקבל:

\[
(s^2L[y](s) - s\cdot1 - 2) - (sL[y](s) - 1) - 2L[y](s)
= \frac{1}{s} - \frac{e^{-s}}{s}.
\]

\textbf{נבצע פישוט:}
\[
L[y](s) = \frac{1}{s(s-2)(s+1)} - \frac{e^{-s}}{s(s-2)(s+1)} + \frac{1}{s-2}.
\]

\textbf{שלב 3 – פירוק לשברים חלקיים:}
\[
\frac{1}{s(s-2)(s+1)} = -\frac{1}{2}\cdot\frac{1}{s}
+ \frac{1}{6}\cdot\frac{1}{s-2}
+ \frac{1}{3}\cdot\frac{1}{s+1}.
\]

נציב בחזרה:
\[
L[y](s)
= \left(-\frac{1}{2}\cdot\textcolor{blue}{\frac{1}{s}}
+ \frac{1}{6}\cdot\textcolor{red}{\frac{1}{s-2}}
+ \frac{1}{3}\cdot\textcolor{darkgreen}{\frac{1}{s+1}}\right)
- e^{-s}\!\left(-\frac{1}{2}\cdot\frac{1}{s}
+ \frac{1}{6}\cdot\frac{1}{s-2}
+ \frac{1}{3}\cdot\frac{1}{s+1}\right)
+ \textcolor{red}{\frac{1}{s-2}}.
\]

\textbf{שלב 4 – ביצוע התמרה הפוכה:}

ניעזר שוב בטבלה \ref{lap_table}:  
שורה~10 (\(1/s \mapsto 1\)), שורה~11 (\(1/(s-a) \mapsto e^{at}\)), ושורות~6-7, עליהן נרחיב מעט יותר עכשיו.

נפריד לכל איבר:

\[
y(t)
= -\tfrac{1}{2}\cdot\textcolor{blue}{1} + \tfrac{1}{6}\textcolor{red}{e^{2t}} + \tfrac{1}{3}\textcolor{darkgreen}{e^{-t}}
+ \tfrac{1}{2}\textcolor{orange}{L^{-1}\!\left[\frac{e^{-s}}{s}\right]}
- \tfrac{1}{6}\textcolor{darkpurple}{L^{-1}\!\left[\frac{e^{-s}}{s-2}\right]}
- \tfrac{1}{3}\textcolor{darkpurple}{L^{-1}\!\left[\frac{e^{-s}}{s+1}\right]}
+ e^{2t}.
\]

הביטויים \textcolor{orange}{הכתום} ו-\textcolor{darkpurple}{הסגולים} מתאימים להתמרות עם הזזה בזמן (שורות~6–7 בטבלה~\ref{lap_table})  
ולאקספוננטים מהצורה \(e^{at}\) (שורה~11).  
נחשב אותם כעת אחד־אחד:

\[
\textcolor{orange}{
\tfrac{1}{2}L^{-1}\!\left[\frac{e^{-s}}{s}\right]
= \tfrac{1}{2}u_1(t)\cdot 1
= \tfrac{1}{2}u_1(t)
}
\]


\[
\textcolor{darkpurple}{
-\tfrac{1}{6}L^{-1}\!\left[\frac{e^{-s}}{s-2}\right]
= -\tfrac{1}{6}u_1(t)e^{2(t-1)}
}
\]

\[
\textcolor{darkpurple}{
-\tfrac{1}{3}L^{-1}\!\left[\frac{e^{-s}}{s+1}\right]
= -\tfrac{1}{3}u_1(t)e^{-(t-1)}
}
\]

\textbf{פירוט להתמרות עם הזזה בזמן (שורה~7):}

ניזכר בכלל ההזזה מטבלת ההתמרות \ref{lap_table}, שורה~7:
\[
\mathcal{L}\big[u_c(t)f(t-c)\big] = e^{-cs}F(s)
\]
כאשר:
- \(c\) הוא גודל ההזזה בזמן (המספר שמופיע במעריך של \(e^{-cs}\)),
- \(F(s)\) היא ההתמרה של פונקציה כלשהי \(f(t)\),
- כלומר \(F(s) = \mathcal{L}[f(t)](s)\).

כדי למצוא את ההתמרה ההפוכה של ביטוי מהצורה \(e^{-cs}F(s)\),
עלינו לזהות תחילה את \(c\) ואת \(F(s)\),
לאחר מכן לגלות מהי \(f(t)\),
ולבסוף לבנות את \(u_c(t)f(t-c)\).

ניישם זאת כעת על שני הביטויים הסגולים:

\[
\textcolor{darkpurple}{
-\tfrac{1}{6}L^{-1}\!\left[\frac{e^{-s}}{s-2}\right]
}
\]

בביטוי זה:
\[
e^{-s}F(s) = e^{-s}\cdot\frac{1}{s-2}.
\]

מכאן נזהה:
\[
c = 1, \qquad F(s) = \frac{1}{s-2}.
\]

ניעזר בשורה~11 מהטבלה (התמרה של \(e^{at}\)):
\[
\mathcal{L}[e^{at}] = \frac{1}{s-a}.
\]
מכאן:
\[
f(t) = e^{2t}.
\]

כעת נחזור לשורה~7 (כלל ההזזה):
\[
\mathcal{L}^{-1}\!\big[e^{-cs}F(s)\big]
= u_c(t)\,f(t-c).
\]

נציב \(c=1\) ו-\(f(t)=e^{2t}\):
\[
L^{-1}\!\left[e^{-s}\cdot\frac{1}{s-2}\right]
= u_1(t)e^{2(t-1)}.
\]

ולכן:
\[
\textcolor{darkpurple}{
-\tfrac{1}{6}L^{-1}\!\left[\frac{e^{-s}}{s-2}\right]
= -\tfrac{1}{6}u_1(t)e^{2(t-1)}.
}
\]

באופן זהה ננתח את האיבר השני:

\[
\textcolor{darkpurple}{
-\tfrac{1}{3}L^{-1}\!\left[\frac{e^{-s}}{s+1}\right]
}
\]

נזהה שוב:
\[
c = 1, \qquad F(s) = \frac{1}{s+1}.
\]

משורה~11:
\[
\mathcal{L}[e^{at}] = \frac{1}{s - a}
\quad \Rightarrow \quad
a = -1 \Rightarrow f(t) = e^{-t}.
\]

נחזור לשורה~7:
\[
L^{-1}\!\left[e^{-s}\cdot\frac{1}{s+1}\right]
= u_1(t)e^{-(t-1)}.
\]

ולכן:
\[
\textcolor{darkpurple}{
-\tfrac{1}{3}L^{-1}\!\left[\frac{e^{-s}}{s+1}\right]
= -\tfrac{1}{3}u_1(t)e^{-(t-1)}.
}
\]

\textbf{שלב 5 - הפתרון הפרטי}

כעת נרשום את הפתרון הפרטי לבעיה:

\[
\fbox{$
y(t)
= -\tfrac{1}{2}\cdot\textcolor{blue}{1}
+ \tfrac{1}{6}\textcolor{red}{e^{2t}}
+ \tfrac{1}{3}\textcolor{darkgreen}{e^{-t}}
+ \textcolor{orange}{\tfrac{1}{2}u_1(t)}
- \textcolor{darkpurple}{\tfrac{1}{6}u_1(t)e^{2(t-1)}}
- \textcolor{darkpurple}{\tfrac{1}{3}u_1(t)e^{-(t-1)}}
+ \textcolor{red}{e^{2t}},
\quad t \ge 0.
$}
\]
ובצורה של ׳׳ענפים׳׳:
\[
\fbox{$
y(t) =
\begin{cases}
-\tfrac{1}{2} + \tfrac{7}{6}e^{2t} + \tfrac{1}{3}e^{-t}, & 0 \le t < 1, \\[6pt]
\tfrac{7}{6}e^{2t} + \tfrac{1}{3}e^{-t}
- \tfrac{1}{6}e^{2(t-1)} - \tfrac{1}{3}e^{-(t-1)}, & t \ge 1.
\end{cases}
$}
\]
נראה את הפתרון בצורה גרפית:

\begin{figure}[H]
\centering
\begin{tikzpicture}[xscale=2.0, yscale=0.2, thick]
  % axes
  \draw[->] (-0.05,0) -- (1.6,0) node[right] {$t$};
  \draw[->] (0,-2) -- (0,26) node[above] {$y(t)$};

  % dashed line at t=1
  \draw[dashed, gray] (1,0)--(1,25);
  \node[anchor=north] at (1,-0.1) {$t=1$};

  % left branch: 0 <= t < 1
  \draw[domain=0:1, smooth, variable=\t, ultra thick, blue]
    plot ({\t},{-0.5 + (7/6)*exp(2*\t) + (1/3)*exp(-\t)});

  % right branch: 1 <= t <= 1.5 (extended for visibility)
  \draw[domain=1:1.5, smooth, variable=\t, ultra thick, red]
    plot ({\t},{(7/6)*exp(2*\t) + (1/3)*exp(-\t)
               - (1/6)*exp(2*(\t-1)) - (1/3)*exp(-(\t-1))});

  % meeting point (exact)
  \pgfmathsetmacro{\yone}{-0.5 + (7/6)*exp(2*1) + (1/3)*exp(-1)}
  \filldraw[black] (1,\yone) circle (2pt);

  % labels
  \node[blue, anchor=south east] at (0.8,4.2) {\small $0\le t<1$};
  \node[red, anchor=south west] at (1.15,10) {\small $t\ge1$};

\end{tikzpicture}
\caption{איור גרפי של הפתרון $y(t)$.
העקומה \textcolor{blue}{הכחולה} מתארת את תחום $0\le t<1$,
והעקומה \textcolor{red}{האדומה} את תחום $t\ge1$.}
\label{fig:laplace_piecewise_exact_hebrew}
\end{figure}

%%%CUT%%%

\newpage
\underline{תרגילים}
\exercise{}
קבלו פתרון פרטי למשוואה הבאה:
\[
y'' + 2y' + y = h(t), \qquad y(0)=0,\; y'(0)=0
\]
כאשר:
\[
h(t) =
\begin{cases}
e^{-t}, & 0 \le t < 1, \\[4pt]
3e^{-t}, & t \ge 1.
\end{cases}
\]

\exercise{}
נתונה המשוואה הדיפרנציאלית:
\[
y'' - y = f(t), \qquad
y(0)=1,\; y'(0)=1,\quad t\ge0,
\]
כאשר:
\[
f(t) =
\begin{cases}
\alpha, & 0 \le t < 1, \\[4pt]
\beta, & 1 \le t < 2, \\[4pt]
\gamma, & t \ge 2,
\end{cases}
\]
ו־$\alpha, \beta, \gamma$ הם מספרים ממשיים חיוביים כך ש־
\(\alpha < \gamma < \beta.\)

א׳.
סרטטו את $f(t)$.

ב׳.
מצאו פתרון פרטי לבעיה כתלות ב- $\alpha, \beta, \gamma$.

\exercise{}
נתונה המשוואה הדיפרנציאלית:
\[
y'' - 2y' + y = f(t),
\]
כאשר:
\[
f(t) =
\begin{cases}
e^{t}, & 0 \le t < 1, \\[3pt]
t e^{t}, & t \ge 1,
\end{cases}
\]
והתנאים ההתחלתיים:
\[
y(0) = 0, \qquad y'(0) = 0.
\]
מצאו את הפתרון הפרטי לבעיה.

\exercise{}
פתרו את המשוואה הדיפרנציאלית הבאה עבור \(t \ge 0\):
\[
y' + y = g(t), \qquad y(0)=0,
\]
כאשר:
\[
g(t) =
\begin{cases}
\sin t, & \pi \le t < 2\pi, \\[3pt]
0, & \text{אחרת}
\end{cases}
\]


\newpage
\underline{פתרונות}
\solution{}
נפתור את הבעיה בעזרת \textbf{התמרת לפלס}, שכן אגף ימין כולל פונקציה מקוטעת בזמן.

\textbf{שלב 1 – כתיבת } \( h(t) \) \textbf{בעזרת פונקציות יחידה:}
\[
h(t) = e^{-t} + (3e^{-t} - e^{-t})u_1(t)
= e^{-t} + 2e^{-t}u_1(t)
= e^{-t} + 2u_1(t)e^{-(t-1)}e^{-1}.
\]
סידרנו כאן את האיבר האקספוננציאלי מבעוד מועד כדי שיתאים להתמרת לפלס המופיעה בטבלה.
\textbf{שלב 2 – הפעלת אופרטור לפלס:}
\[
L[y''] + 2L[y'] + L[y] = L[h(t)](s)
\]
נשתמש בשורות~3, 7 ו־11 מהטבלה \ref{lap_table}:
\[
(s^2Y - s\cdot0 - 0) + 2(sY - 0) + Y = \frac{1}{s+1} + 2e^{-s-1}\frac{1}{s+1}.
\]

שימו לב כי נהוג לסמן לעיתים $Y(s)$, במקום $F(s)$.
נבצע כינוס איברים:
\[
\textcolor{blue}{Y(s)(s^2 + 2s + 1)} = \frac{1}{s+1} + 2e^{-s-1}\frac{1}{s+1}=\textcolor{blue}{Y(s)(s+ 1)^{2}}.
\]

ולכן:
\[
Y(s) = \frac{1}{(s+1)^3} + 2e^{-s-1}\frac{1}{(s+1)^3}.
\]

\textbf{שלב 3 – ביצוע התמרה הפוכה:}

ניעזר שוב בטבלה \ref{lap_table}:
שורה~13 \(\left(\dfrac{n!}{(s - a)^{n+1}}\mapsto t^n e^{at}\right)\)
ושורה~7:

\[
L^{-1}\!\left[\frac{1}{(s+1)^3}\right]
= \frac{t^2}{2}e^{-t}.
\]

כעת עבור האיבר השני:
\[
L^{-1}\!\left[e^{-s-1}\frac{1}{(s+1)^3}\right]
= e^{-1}u_1(t)\cdot \frac{(t-1)^2}{2}e^{-(t-1)}.
\]

\textbf{שלב 4 –הפתרון הפרטי:}
\[
\boxed{
y(t)
= \frac{1}{2}t^{2}e^{-t}
+ e^{-1}u_1(t)(t-1)^2e^{-(t-1)}.}
\]
ובצורת ׳׳ענפים׳׳:
\[
\boxed{
y(t)=
\begin{cases}
\dfrac{1}{2}t^2e^{-t}, & 0 \le t < 1, \\[3pt]
\vspace{6pt}\\[-3pt]
\dfrac{1}{2}t^2e^{-t} + e^{-1}(t-1)^2e^{-(t-1)}, & t \ge 1.
\end{cases}
}
\]


גרף הפתרון יראה כך:

\begin{figure}[H]
\centering
\begin{tikzpicture}[xscale=4.0, yscale=4.0, thick]
  % axes
  \draw[->] (-0.05,0) -- (2,0) node[right] {$t$};
  \draw[->] (0,-0.1) -- (0,1.2) node[above] {$y(t)$};

  % dashed line at t=1
  \draw[dashed, gray] (1,0)--(1,1.1);
  \node[anchor=north] at (1,-0.05) {$t=1$};

  % left branch
  \draw[domain=0:1, smooth, variable=\t, ultra thick, blue]
    plot ({\t},{0.5*\t*\t*exp(-\t)});
  % right branch
  \draw[domain=1:2, smooth, variable=\t, ultra thick, red]
    plot ({\t},{0.5*\t*\t*exp(-\t) + exp(-1)*( (\t-1)*(\t-1)*exp(-(\t-1)) )});

  % meeting point
  \pgfmathsetmacro{\yone}{0.5*1*1*exp(-1)}
  \filldraw[black] (1,\yone) circle (1pt);

  % labels
  \node[blue, anchor=south east] at (0.9,0.4) {\small $0\le t<1$};
  \node[red, anchor=south west] at (1.05,0.6) {\small $t\ge1$};

\end{tikzpicture}
\caption{איור גרפי של הפתרון $y(t)$.
הענף \textcolor{blue}{הכחול} מייצג את תחום $0\le t<1$,
והענף \textcolor{red}{האדום} את תחום $t\ge1$.
הנקודה השחורה מציינת את רציפות הפתרון בזמן $t=1$.}
\label{fig:laplace_piecewise_expdecay}
\end{figure}



\solution{}
א׳.
\begin{center}
\begin{tikzpicture}[scale=1.2, thick]
  % axes
  \draw[->] (-0.2,0) -- (3,0) node[right] {$t$};
  \draw[->] (0,-0.2) -- (0,3.5) node[above] {$f(t)$};

  % horizontal levels
  \draw[blue, ultra thick] (0,0.7) -- (1,0.7);
  \draw[blue, ultra thick] (1,2.5) -- (2,2.5);
  \draw[blue, ultra thick] (2,1.4) -- (2.7,1.4);

  % vertical jumps
  \draw[blue, ultra thick] (1,0.7)--(1,2.5);
  \draw[blue, ultra thick] (2,2.5)--(2,1.4);

  % open/closed circles
  \filldraw[blue] (0,0.7) circle (1.5pt);
  \filldraw[white] (1,0.7) circle (1.5pt);
  \filldraw[blue] (1,2.5) circle (1.5pt);
  \filldraw[white] (2,2.5) circle (1.5pt);
  \filldraw[blue] (2,1.4) circle (1.5pt);

  % labels
  \node[anchor=east, blue] at (0,0.7) {$\alpha$};
  \node[anchor=east, blue] at (0,1.4) {$\gamma$};
  \node[anchor=east, blue] at (0,2.5) {$\beta$};
  \node[anchor=north] at (1,-0.1) {$1$};
  \node[anchor=north] at (2,-0.1) {$2$};
\end{tikzpicture}
\end{center}

ב׳.
\textbf{שלב 1 – כתיבת } \( f(t) \) \textbf{בעזרת פונקציות יחידה}
\[
f(t) = \alpha
+ (\beta - \alpha)u_1(t)
+ (\gamma - \beta)u_2(t).
\]

\textbf{שלב 2 – הפעלת התמרת לפלס על המשוואה}
\[
L[y''] - L[y] = L[f(t)](s),
\quad
y(0)=1,\; y'(0)=1.
\]
לפי נוסחאות ההתמרה (שורה~3 מהטבלה~\ref{lap_table}):
\[
(s^2Y - s - 1) - Y = L[f(t)](s).
\]

נכתוב תחילה את \(L[f(t)](s)\) (שורות 6 ו-10 בטבלה):
\[
L[f(t)](s)
= \frac{\alpha}{s}
+ (\beta - \alpha)\frac{e^{-s}}{s}
+ (\gamma - \beta)\frac{e^{-2s}}{s}.
\]

נציב:
\[
(s^2 - 1)Y(s) - s - 1
= \frac{\alpha}{s}
+ (\beta - \alpha)\frac{e^{-s}}{s}
+ (\gamma - \beta)\frac{e^{-2s}}{s}.
\]
ומכאן:
\[
Y(s)
= \frac{s+1}{s^2 - 1}
+ \frac{1}{s(s^2 - 1)}
\big[
\alpha + (\beta - \alpha)e^{-s} + (\gamma - \beta)e^{-2s}
\big].
\]
\textbf{שלב 3 – נפרק לשברים חלקיים}

תחילה נבצע פירוק לשבר חלקי עבור שני הביטויים הדרושים.

(1) נחשב:
\[
\frac{1}{s(s^2 - 1)}
= \frac{1}{s(s-1)(s+1)}
= -\frac{1}{s}
+ \frac{1}{2}\cdot\frac{1}{s-1}
+ \frac{1}{2}\cdot\frac{1}{s+1}.
\]

(2) באופן טריוויאלי, נפרק גם את:
\[
\frac{s+1}{s^2 - 1}
= \frac{s+1}{(s-1)(s+1)}=\frac{1}{(s-1)}.

נרשום עתה את $Y(s)$ במלואו:

\[
Y(s)
= \frac{1}{s-1}
+ \Big(-\frac{1}{s}
+ \frac{1}{2}\frac{1}{s-1}
+ \frac{1}{2}\frac{1}{s+1}\Big)
\Big[
\alpha + (\beta - \alpha)e^{-s} + (\gamma - \beta)e^{-2s}
\Big].
\]

\textbf{שלב 4 – ביצוע ההתמרה ההפוכה ומציאת הפתרון הפרטי}

ניעזר שוב בטבלה~\ref{lap_table}, שורות~7 ו־11:
\[
L^{-1}\!\left[\frac{1}{s-a}\right] = e^{at},
\qquad
L^{-1}\!\left[e^{-cs}F(s)\right] = u_c(t)\,f(t-c).
\]

נחשב לפי כל רכיב:

\[
L^{-1}\!\left[\frac{1}{s-1}\right] = e^{t}, \quad
L^{-1}\!\left[\frac{1}{s}\right] = 1, \quad
L^{-1}\!\left[\frac{1}{s+1}\right] = e^{-t}.
\]

וכן עבור האיברים עם הזזה בזמן:
\[
L^{-1}\!\big[e^{-s}F(s)\big] = u_1(t)f(t-1), \qquad
L^{-1}\!\big[e^{-2s}F(s)\big] = u_2(t)f(t-2).
\]

נפעיל את ההתמרה ההפוכה על שני אגפי המשוואה ונקבל את הפתרון הפרטי:

\[
\boxed{
\begin{aligned}
y(t)
&= e^{t}
+ \alpha\!\left(-1 + \tfrac{1}{2}e^{t} + \tfrac{1}{2}e^{-t}\right) \\[4pt]
&\quad + (\beta - \alpha)u_1(t)\!\left(-1 + \tfrac{1}{2}e^{t-1} + \tfrac{1}{2}e^{-(t-1)}\right)
+ (\gamma - \beta)u_2(t)\!\left(-1 + \tfrac{1}{2}e^{t-2} + \tfrac{1}{2}e^{-(t-2)}\right).
\end{aligned}}
\]



\solution{}

\textbf{שלב 1 – כתיבת } \( f(t) \) \textbf{בעזרת פונקציות יחידה}
\[
f(t) = e^{t} + u_1(t)\!\big(t e^{t} - e^{t}\big)
= e^{t} + u_1(t)\!\big(e^{t}(t - 1)\big).
\]

\textbf{שלב 2 – הפעלת התמרת לפלס}
\[
L[y''] - 2L[y'] + L[y] = L[f(t)].
\]

נשתמש בנוסחאות:
\[
L[y''] = s^2Y(s) - sy(0) - y'(0), \qquad
L[y'] = sY(s) - y(0),
\]
ונציב את התנאים ההתחלתיים \(y(0)=0,\; y'(0)=0\):

\[
(s^2 - 2s + 1)Y(s) = L[f(t)] = L[e^{t}] + L\big[u_1(t)e^{t}(t - 1)\big]\rightarrow Y(s) = \frac{L[e^{t}] + eL\big[u_1(t)e^{t-1}(t - 1)\big]}{(s-1)^{2}} .
\]

\textbf{שלב 3 – התמרת אגף ימין}

נשתמש בכלל ההזזה בזמן (שורה~7 בטבלה~\ref{lap_table}):
\[
L\!\big[u_c(t)f(t-c)\big] = e^{-cs}F(s),
\]
כאשר \(F(s) = L[f(t)]\).

כאן \(c=1\) ו-\(f(t-1)=e^{t-1}(t-1)\), ולכן:
\[
L\!\big[u_1(t)e^{t-1}(t - 1)\big] = e^{-s}L\!\big[te^{t}\big].
\]

לכן
(שורה 13):
\[
L\!\big[u_1(t)e^{t-1}(t - 1)\big]
= e^{-s}\,\frac{1}{(s - 1)^2}.
\]

נחשב גם את התמרת האיבר הראשון:
\[
L[e^{t}] = \frac{1}{s - 1}.
\]

\textbf{שלב 4 – הצבה במשוואה עבור } \(Y(s)\)

נציב במשוואת לפלס:

\[
Y(s)
= \frac{1}{(s - 1)^3}
+ ee^{-s}\,\frac{1}{(s - 1)^4}.
\]

\textbf{שלב 5 – ביצוע ההתמרה ההפוכה}

ניעזר בנוסחאות משורה~13 ו־7 בטבלה~\ref{lap_table}:
\[
L^{-1}\!\left[\frac{1}{(s - a)^{n+1}}\right] = \frac{t^{n}e^{at}}{n!}, 
\qquad
L^{-1}\!\left[e^{-cs}F(s)\right] = u_c(t)f(t-c).
\]

לכן:
\[
L^{-1}\!\left[\frac{1}{(s - 1)^3}\right] = \frac{t^2 e^{t}}{2},
\qquad
L^{-1}\!\left[\frac{1}{(s - 1)^4}\right] = \frac{t^3 e^{t}}{6}.
\]

נציב בחזרה:
\[
y(t)
= \frac{1}{2}t^2 e^{t}
+ eu_1(t)\,\frac{(t-1)^3 e^{t-1}}{6}.
\]

נכתוב את הפתרון הפרטי בצורת ׳׳ענפים׳׳:
\[
\boxed{
y(t)=
\begin{cases}
\dfrac{1}{2}t^2 e^{t}, & 0 \le t < 1, \\[3pt]
\vspace{6pt}\\[-3pt]
\dfrac{1}{2}t^2 e^{t} + \dfrac{e}{6}(t-1)^3 e^{t-1}, & t \ge 1.
\end{cases}
}
\]

%%%CUT%%%

\solution{}

\textbf{שלב 1 – נכתוב את } \( g(t) \) \textbf{בעזרת פונקציות יחידה}
\[
g(t) = \sin t\,[u_\pi(t) - u_{2\pi}(t)]
= u_\pi(t)\sin(t) - u_{2\pi}(t)\sin(t).
\]
כיוון שנתקלנו בתבנית של פונקציית הביסייד כפול פונקציה טריגונומטרית במקרה זה, נכוון לשורה 7 בטבלה ועל כן נבצע הזזה בזמן.
נשתמש בזהויות:
\[
\sin(t \pm \pi) = -\sin t, 
\qquad
\cos(t \pm \pi) = -\cos t.
\]

ולכן:
\[
g(t) = -u_\pi(t)\sin(t - \pi) - u_{2\pi}(t)\sin(t - 2\pi).
\]

\textbf{שלב 2 – הפעלת התמרת לפלס על שני אגפי המשוואה}
\[
L[y'] + L[y] = L[g(t)].
\]

נציב ונקבל (תוך שימוש בשורות 3, 7 ו-16):
\[
sL[y](s) - y(0) + L[y](s)
= -e^{-\pi s}\frac{1}{s^2 + 1} - e^{-2\pi s}\frac{1}{s^2 + 1}.
\]

מאחר ש-\(y(0)=0\), נקבל:
\[
(s+1)L[y](s)
= -\frac{e^{-2\pi s} + e^{-\pi s}}{s^2 + 1}.
\]

\textbf{שלב 3 – בידוד } \(L[y](s)\)
\[
L[y](s)
= -\frac{e^{-2\pi s} + e^{-\pi s}}{(s^2 + 1)(s+1)}.
\]

נרצה לפרק את הביטוי:
\[
\frac{1}{(s^2 + 1)(s + 1)}.
\]

נניח:
\[
\frac{1}{(s^2 + 1)(s + 1)} = \frac{As + B}{s^2 + 1} + \frac{C}{s + 1}.
\]

נכפיל במכנה המשותף \((s^2 + 1)(s + 1)\):
\[
1 = (As + B)(s + 1) + C(s^2 + 1)
= (A + C)s^2 + (A + B)s + (B + C).
\]

נשווה מקדמים ונקבל:
\[
\begin{cases}
A + C = 0, \\[3pt]
A + B = 0, \\[3pt]
B + C = 1.
\end{cases}
\]

נפתור:
\[
A = -\tfrac{1}{2}, \qquad B = \tfrac{1}{2}, \qquad C = \tfrac{1}{2}.
\]

ולכן:
\[
\frac{1}{(s^2 + 1)(s + 1)}
= -\tfrac{1}{2}\cdot\frac{s - 1}{s^2 + 1}
+ \tfrac{1}{2}\cdot\frac{1}{s + 1}.
\]

נציב את התוצאה בביטוי של השלב הקודם:
\[
L[y](s)
= -(e^{-2\pi s} + e^{-\pi s})
\!\left(
-\tfrac{1}{2}\frac{s - 1}{s^2 + 1}
+ \tfrac{1}{2}\frac{1}{s + 1}
\right).
\]

נפשט מעט:
\[
L[y](s)
= \tfrac{1}{2}(e^{-\pi s} + e^{-2\pi s})
\!\left(
\frac{s}{s^2 + 1} - \frac{1}{s^2 + 1} - \frac{1}{s + 1}
\right).
\]

\textbf{שלב 4 – ביצוע ההתמרה ההפוכה}

ניעזר שוב בטבלה~\ref{lap_table}, שורות~11, 14, 15:
\[
L^{-1}\!\left[\frac{s}{s^2 + 1}\right] = \cos t, 
\qquad
L^{-1}\!\left[\frac{1}{s^2 + 1}\right] = \sin t,
\qquad
L^{-1}\!\left[\frac{1}{s+1}\right] = e^{-t}.
\]

כעת ניישם את ההתמרה ההפוכה בשורה 7 ונקבל:

\[
y_p(t)
= \tfrac{1}{2}u_\pi(t)
\big[\cos(t - \pi) - \sin(t - \pi) - e^{-(t - \pi)}\big]
+ \tfrac{1}{2}u_{2\pi}(t)
\big[\cos(t - 2\pi) - e^{-(t - 2\pi)}\big].
\]

\textbf{שלב 5 – הפתרון הפרטי}
\[
\boxed{
y_p(t)
= \tfrac{1}{2}u_\pi(t)\!\left[\cos(t - \pi) - \sin(t - \pi) - e^{-(t - \pi)}\right]
+ \tfrac{1}{2}u_{2\pi}(t)\!\left[\cos(t - 2\pi) - e^{-(t - 2\pi)}\right].
}
\]

\newpage
\subsection{מערכת משוואות דיפרנציאליות (מקדמים קבועים)}

מערכת משוואות דיפרנציאליות מתארת קבוצה של פונקציות שתלויות בזמן,  
כאשר הנגזרת של כל אחת מהן תלויה גם בשאר הפונקציות במערכת.  
במילים אחרות, המשוואות \textbf{מצומדות} אחת לשנייה.

בדרך כלל נכתוב מערכת מסדר ראשון בצורה וקטורית, כך שכל המשוואות מאוגדות יחד בייצוג קומפקטי.

הרעיון נובע מכך שניתן לכתוב כל מד׳׳ר לינארית מסדר $n$, באמצעות $n$ מד׳׳רים לינאריות מסדר 1, ולמעשה על ידי מציאת הערכים העצמיים והוקטורים העצמיים של המערכת, לקבל את הפתרון לאותה מד׳׳ר מסדר $n$, או לחילופין, לקבל את הפתרונות לכל $n$ המד׳׳רים שמרכיבות את המערכת.

נוכיח אם כן כי ניתן ׳׳לשבור׳׳ מד׳׳ר לינארית מסדר $n$ ל- $n$ מד׳׳רים מסדר 1 וייצוגם ע׳׳י מערכת משוואות.

\begin{proof}

נבחן משוואה לינארית מסדר $n$ מהצורה:
\begin{equation}\label{high_order_ode}
a_n y^{(n)}(t) + a_{n-1}y^{(n-1)}(t) + \dots + a_1 y'(t) + a_0 y(t) = f(t),
\end{equation}
כאשר $a_0, a_1, \dots, a_n$ הם מקדמים קבועים ו-$f(t)$ היא פונקציה נתונה.

נגדיר משתנים חדשים המייצגים את הנגזרות של $y(t)$ עד לסדר $n-1$:
\[
\begin{aligned}
x_1(t) &= y(t), \\[4pt]
x_2(t) &= y'(t), \\[4pt]
x_3(t) &= y''(t), \\[4pt]
&\vdots \\[4pt]
x_n(t) &= y^{(n-1)}(t).
\end{aligned}
\]

נמצא את הנגזרות של המשתנים החדשים:
\[
\begin{aligned}
x_1'(t) &= x_2(t), \\[4pt]
x_2'(t) &= x_3(t), \\[4pt]
&\vdots \\[4pt]
x_{n-1}'(t) &= x_n(t).
\end{aligned}
\]

כעת נחשב את הנגזרת של $x_n(t)$, כלומר את $y^{(n)}(t)$:
מתוך משוואה \eqref{high_order_ode} נקבל:
\[
y^{(n)}(t)
= \frac{1}{a_n}\Big(f(t) - a_{n-1}y^{(n-1)}(t) - a_{n-2}y^{(n-2)}(t) - \dots - a_0 y(t)\Big).
\]
נחליף את כל הנגזרות בייצוג החדש:
\[
x_n'(t)
= \frac{1}{a_n}\Big(f(t) - a_{n-1}x_n(t) - a_{n-2}x_{n-1}(t) - \dots - a_0x_1(t)\Big).
\]

נכתוב את כל המערכת יחד:
\begin{equation}\label{first_order_system}
\boxed{
\begin{aligned}
x_1'(t) &= x_2(t), \\[4pt]
x_2'(t) &= x_3(t), \\[4pt]
&\vdots \\[4pt]
x_{n-1}'(t) &= x_n(t), \\[4pt]
x_n'(t) &= -\frac{a_0}{a_n}x_1(t) - \frac{a_1}{a_n}x_2(t) - \dots - \frac{a_{n-1}}{a_n}x_n(t) + \frac{1}{a_n}f(t).
\end{aligned}
}
\end{equation}

נכתוב את המערכת בצורת מטריצה קומפקטית:
\begin{equation}\label{nonhom.sys}
\boxed{
\vec{x}'(t)
= A\vec{x}(t) + \vec{b}(t),
}
\end{equation}
כאשר:
\[
A =
\begin{pmatrix}
0 & 1 & 0 & \cdots & 0\\
0 & 0 & 1 & \cdots & 0\\
\vdots & \vdots & \vdots & \ddots & \vdots\\
0 & 0 & 0 & \cdots & 1\\
-\frac{a_0}{a_n} & -\frac{a_1}{a_n} & -\frac{a_2}{a_n} & \cdots & -\frac{a_{n-1}}{a_n}
\end{pmatrix},
\qquad
\vec{b}(t) =
\begin{pmatrix}
0\\
0\\
\vdots\\
\frac{1}{a_n}f(t)
\end{pmatrix}.
\]

\textbf{מסקנה:}
משוואה לינארית מסדר $n$ שקולה למערכת של $n$ משוואות לינאריות מסדר ראשון.
פתרון המערכת $\vec{x}(t)$ נותן מיידית גם את הפתרון $y(t) = x_1(t)$ של המשוואה המקורית.

\textbf{במקרה ההומוגני:}
אם $f(t)=0$, המערכת \eqref{first_order_system} הופכת למערכת \textbf{הומוגנית} מהצורה:
\begin{equation}\label{hom.sys}
\boxed{
\vec{x}'(t) = A\vec{x}(t).
}
\end{equation}
\end{proof}

%%%CUT%%%

\subsubsection{מערכת משוואות הומוגנית}

נבחן את המערכת ה\textbf{הומוגנית} (כלומר, ללא איבר חופשי), לצד וקטור תנאי ההתחלה:
\begin{equation}\tag{\ref{hom.sys}}
\boxed{
\vec{x}'(t) = A_{n\times n}\,\vec{x}(t),
\qquad
\vec{x}(t_0) = \vec{v}_0.
}
\end{equation}

כאשר:
\begin{itemize}
  \item \(\vec{x}(t)\) – וקטור הנעלמים, המורכב מפונקציות \(x_1(t), x_2(t), \dots, x_n(t)\).
  \item \(A_{n\times n}\) – מטריצת המקדמים של המערכת.
  \item \(\vec{v}_0\) – וקטור תנאי ההתחלה.
\end{itemize}

לדוגמה, עבור מערכת מסדר שלישי נקבל:
\[
\begin{pmatrix}
x_1'(t)\\[4pt]
x_2'(t)\\[4pt]
x_3'(t)
\end{pmatrix}
=
\begin{pmatrix}
a_{11} & a_{12} & a_{13}\\[4pt]
a_{21} & a_{22} & a_{23}\\[4pt]
a_{31} & a_{32} & a_{33}
\end{pmatrix}
\begin{pmatrix}
x_1(t)\\[4pt]
x_2(t)\\[4pt]
x_3(t)
\end{pmatrix}.
\]

\textbf{אלגוריתם לפתרון}

נחפש ערכים עצמיים (ע׳׳ע) של המטריצה \(A\),  
שכן אלו קובעים את התנהגות הפתרון הדינמי של המערכת.  

נגדיר את הפולינום האופייני (פ׳׳א):
\begin{equation}
P(\lambda) = \det(A - \lambda I),
\end{equation}
כאשר שורשי הפולינום \(\lambda_1, \lambda_2, \dots, \lambda_n\)
מהווים את הערכים העצמיים של \(A\).

לאחר שמצאנו ערך עצמי \(\lambda_0\),  
נמצא את הווקטור העצמי (ו׳׳ע) המתאים לו \(\vec{v}\) על ידי הפתרון של:
\begin{equation}
(A - \lambda_0 I)\vec{v} = \vec{0}.
\end{equation}

אם נסמן את הריבוי האלגברי של \(\lambda_0\) ב-\(k\),
ואת הריבוי הגיאומטרי שלו (מספר הוקטורים העצמיים הבלתי תלויים לינארית המתאימים לו) ב-\(q\),
אז תמיד מתקיים:
\begin{equation}
1 \le q \le k,
\end{equation}

ומספר הפתרונות הבלתי תלויים לינארית (בת׳׳ל) עבור אותו ע׳׳ע יהיה:
\begin{equation}
q = n - \text{rank}(A - \lambda I).
\end{equation}

כאשר \(k = q\) לכל ערך עצמי, נאמר כי המטריצה \textbf{לכסינה},
והפתרון של המערכת מתקבל בצורה סגורה:
\begin{equation}
\boxed{
\vec{x}(t)
= c_1 e^{\lambda_1 t}\vec{v}_1
+ c_2 e^{\lambda_2 t}\vec{v}_2
+ \dots
+ c_n e^{\lambda_n t}\vec{v}_n.
}
\end{equation}

במקרה שבו \(k \neq q\), כלומר המטריצה אינה לכסינה,  
כל ערך עצמי שהריבוי האלגברי שלו לא שווה לריבוי הגיאומטרי שלו, יתרום פתרון מהצורה:
\begin{equation}
\vec{x}(t) = c_1 e^{\lambda_0 t}
\begin{pmatrix}
s_1(t)\\[-3pt]
\vdots\\[-3pt]
s_k(t)
\end{pmatrix},
\end{equation}  כאשר דרגות הפולינומים \(s_i(t)\) הן לכל היותר $k-q$.

\vspace{0.5cm}
בתור כלי יעיל לחישוב דטרמיננטות,
נציג נוסחה ציקלית סגורה לכל דטרמיננטה $3\times3$, הידועה בתור  
\textbf{נוסחת סארוס}:

\begin{equation}
\det
\begin{pmatrix}
a & b & c\\
d & e & f\\
g & h & i
\end{pmatrix}
=
\underbrace{\color{RoyalBlue}{a(ei - fh)}}_{\text{מינור בעמודה 1}}
\;-\;
\underbrace{\color{DarkOrange}{b(di - fg)}}_{\text{מינור בעמודה 2}}
\;+\;
\underbrace{\color{ForestGreen}{c(dh - eg)}}_{\text{מינור בעמודה 3}}.
\end{equation}

כאשר כל ביטוי מהצורה $(ei - fh)$, $(di - fg)$, $(dh - eg)$
מייצג \textbf{מינור מסדר 2} המתאים לעמודה שנבחרה.
הסימנים מתחלפים לפי הכלל:
\[
(+,\;-,\;+).
\]

\example{}

מצאו פתרון כללי למערכת:
\[
\begin{pmatrix}
x_1'(t)\\[4pt]
x_2'(t)\\[4pt]
x_3'(t)
\end{pmatrix}
=
\begin{pmatrix}
4 & -1 & -1\\[4pt]
1 & 2 & -1\\[4pt]
1 & -1 & 2
\end{pmatrix}
\begin{pmatrix}
x_1(t)\\[4pt]
x_2(t)\\[4pt]
x_3(t)
\end{pmatrix}.
\]

\explanation{}

באופן שקול, המערכת נכתבת כך:
\[
\left\{
\begin{aligned}
x_1'(t) &= 4x_1(t) - x_2(t) - x_3(t),\\[4pt]
x_2'(t) &= x_1(t) + 2x_2(t) - x_3(t),\\[4pt]
x_3'(t) &= x_1(t) - x_2(t) + 2x_3(t).
\end{aligned}
\right.
\]

נמצא ראשית את הפולינום האופייני של המערכת:
\[
P(\lambda) = \det(A - \lambda I)
= 
\begin{vmatrix}
4-\lambda & -1 & -1\\
1 & 2-\lambda & -1\\
1 & -1 & 2-\lambda
\end{vmatrix}
= 0
\]

נחשב את הדטרמיננטה לפי חוק סארוס:
\[
\begin{vmatrix}
4-\lambda & -1 & -1\\
1 & 2-\lambda & -1\\
1 & -1 & 2-\lambda
\end{vmatrix}
=
(4-\lambda)\big[(2-\lambda)^2 - 1\big]
- (-1)\big[1\cdot(2-\lambda) - (-1)\big]
+ (-1)\big[1\cdot(-1) - (2-\lambda)(1)\big].
\]


ולכן:
\[
P(\lambda) = -(\lambda-3)^2(\lambda-2)=0.
\]

נמצא את הערכים העצמיים:
\[
\lambda_1 = 3 , \qquad \lambda_2 = 2,
\]
כאשר $\lambda_1$ הוא מריבוי אלגברי 2 ו-$\lambda_2$ מריבוי אלגברי 1.

\textbf{תזכורת}:  
סכום הערכים העצמיים שווה ל-\(\mathrm{trace}(A)\),  
ומכפלתם שווה ל-\(\det(A)\).

\textbf{עבור $\lambda=3$:}
\[
(A - \lambda_0 I)\vec{v} = \vec{0}\rightarrow
\begin{pmatrix}
1 & -1 & -1\\
1 & -1 & -1\\
1 & -1 & -1
\end{pmatrix}
\begin{pmatrix}
a\\ b\\ c
\end{pmatrix}
=
\begin{pmatrix}
0\\0\\0
\end{pmatrix}.
\]

כל השורות תלויות זו בזו. מכאן שדרגת המטריצה היא 1 ולכן הריבוי הגיאומטרי יהיה שווה ל-2, בדומה לריבוי האלגברי של הע׳׳ע. מכאן שהמטריצה לכסינה והפתרונות כולם יהיו אקספוננציאליים (שלא מוכפלים בפולינום).

קיימת רק משוואה אחת 
בלתי תלויה במקרה זה:
\[
a-b-c=0 \quad\Rightarrow\quad a=b+c.
\]

ולכן:
\[
\vec{v}_1 =
\begin{pmatrix}
a\\ b\\ c
\end{pmatrix}
=
\begin{pmatrix}
b+c\\ b\\ c
\end{pmatrix}
= b
\begin{pmatrix}
1\\1\\0
\end{pmatrix}
+ c
\begin{pmatrix}
1\\0\\1
\end{pmatrix}.
\]

\textbf{עבור $\lambda=2$:}
\[
\begin{pmatrix}
2 & -1 & -1\\
1 & 0 & -1\\
1 & -1 & 0
\end{pmatrix}
\begin{pmatrix}
d\\ e\\ f
\end{pmatrix}
=
\begin{pmatrix}
0\\0\\0
\end{pmatrix}.
\]

נבחין כי סכום השורות השנייה והשלישית נותן את הראשונה.  
מכאן אחת מהן תלויה בשאר, ולכן נשתמש בשתיים מהן בלבד (בשנייה ובשלישית שקל לראות כי הן בלתי תלויות):
\[
d-f=0, \qquad d-e=0 \quad\Rightarrow\quad d=e=f.
\]

ולכן:
\[
\vec{v}_2 =
\begin{pmatrix}
d\\ e\\ f
\end{pmatrix}
=
d
\begin{pmatrix}
1\\1\\1
\end{pmatrix}.
\]

\textbf{בסה"כ נקבל בסיס למרחב הפתרונות:
:}
\[
\boxed{
\vec{x}(t)
= c_1 e^{3t}
\begin{pmatrix}
1\\1\\0
\end{pmatrix}
+ c_2 e^{3t}
\begin{pmatrix}
1\\0\\1
\end{pmatrix}
+ c_3 e^{2t}
\begin{pmatrix}
1\\1\\1
\end{pmatrix}
}.
\]

\example{}

מצאו פתרון כללי למערכת המשוואות הבאה:
\[
\dot{\vec{x}} =
\begin{pmatrix}
3 & -2\\[4pt]
4 & -1
\end{pmatrix}
\vec{x}.
\]

\explanation{}

נמצא ראשית את הפי״א של הבעיה:
\[
P(\lambda) =
\begin{vmatrix}
3-\lambda & -2\\
4 & -1-\lambda
\end{vmatrix}
= \lambda^2 - 2\lambda + 5 = 0
\]

כאשר השורשים הם:
\[
\lambda_{1,2} = 1 \pm 2i.
\]

ניקח שרירותית את אחד הצמודים (נניח $\lambda_1 = 1 + 2i$) 
ונמצא ו״ע
מתאים:
\[
\begin{pmatrix}
3 - (1 + 2i) & -2\\[4pt]
4 & -1 - (1 + 2i)
\end{pmatrix}
\begin{pmatrix}
a\\[2pt]
b
\end{pmatrix}
=
\begin{pmatrix}
0\\[2pt]
0
\end{pmatrix}.
\]

נשים לב כי דרגת המטריצה היא 1, לאור הר״א של $\lambda_1$.
נוודא זאת בעזרת ערך הדטרמיננטה:
\[
(2-2i)(-2-2i) - (-2)\cdot4
= -4 - 4i + 4i - 4 + 8 = 0
\]

ומכאן ששורות המטריצה הן אכן ת״יל (מכיוון שהדטרמיננטה שווה לאפס).

נפתור כעת את מערכת המשוואות כאשר למעשה יש לנו משוואה אחת בלבד:
\[
(2 - 2i)a - 2b = 0
\quad\Rightarrow\quad
a - ia - b = 0
\quad\Rightarrow\quad
a(1 - i) = b
\]

ולכן:
\[
\vec{v}_1 =
\begin{pmatrix}
a\\ b
\end{pmatrix}
=
\begin{pmatrix}
a\\ a(1-i)
\end{pmatrix}
=
a
\textcolor{red}{\begin{pmatrix}
1\\ 1-i
\end{pmatrix}}.
\]

\textcolor{red}{באדום – הווקטור המייצג}.

מכיוון שכלל מקדמי המשוואה ממשיים, אנו יודעים להגיד כי גם הצמוד של הווקטור $\vec{v}_1$ הוא ו״ע.

כלומר, עבור $\lambda_2 = 1 - 2i$ הצמוד הוא:
\[
\overline{
\begin{pmatrix}
1\\ 1-i
\end{pmatrix}}
=
\begin{pmatrix}
1\\ 1+i
\end{pmatrix}.
\]

נחפש כעת פתרון ממשי (כמו שעשינו בעבר בהקשר של שורשים קומפלקסיים במד״ר עם מקדמים קבועים). נתחיל עם הע׳׳ע הראשון והו׳׳ע המתאים לו:

\[
e^{(1+2i)t}
\cdot
\begin{pmatrix}
1\\ 1-i
\end{pmatrix}.
\]

נשתמש בזהות אוילר:
\[
e^{a+ib} = e^a \cdot e^{ib} = e^a(\cos b + i\sin b)
\]

ונקבל:
\[
e^t \cdot e^{2it} = e^t(\cos(2t) + i\sin(2t)).
\]

כלומר:
\[
e^{(1+2i)t}
\cdot
\begin{pmatrix}
1\\ 1-i
\end{pmatrix}
=
e^t(\cos(2t) + i\sin(2t))
\cdot
\begin{pmatrix}
1\\ 1-i
\end{pmatrix}.
\]

נפתח את הכפל המטריציוני:
\[
=
\begin{pmatrix}
e^t\cos(2t) + i e^t\sin(2t)\\[4pt]
(1-i)e^t(\cos(2t) + i\sin(2t))
\end{pmatrix}
=
\begin{pmatrix}
e^t\cos(2t) + i e^t\sin(2t)\\[4pt]
e^t(\cos(2t) + i\sin(2t)) - i e^t(\cos(2t) + i\sin(2t))
\end{pmatrix}.
\]

נפשט:
\[
=
\begin{pmatrix}
e^t\cos(2t) + i e^t\sin(2t)\\[4pt]
e^t(\cos(2t) + i\sin(2t)) - i e^t\cos(2t) + e^t\sin(2t)
\end{pmatrix}.
\]

נפצל לחלק ממשי ומדומה ונכליל את התרומה של שני הו׳׳ע המרוכבים בבת אחת:

\[
=
\begin{pmatrix}
e^t\cos(2t)\\[4pt]
e^t\cos(2t) + e^t\sin(2t)
\end{pmatrix}
\pm i
\begin{pmatrix}
e^t\sin(2t)\\[4pt]
e^t\sin(2t) - e^t\cos(2t)
\end{pmatrix}.
\]

ועבור הפתרון הכללי ניקח כמובן את החלקים הממשיים הרלוונטיים ונכפיל בקבועים בהתאם לצורך ולפי מימד הבעיה (2):
\[
\boxed{
\vec{x}(t)
= c_1
\begin{pmatrix}
e^t\cos(2t)\\[4pt]
e^t\cos(2t) + e^t\sin(2t)
\end{pmatrix}
+ c_2
\begin{pmatrix}
e^t\sin(2t)\\[4pt]
e^t\sin(2t) - e^t\cos(2t)
\end{pmatrix},
}
\]
כאשר $c_{1},c_{2}$ האם קבועים ממשיים כידוע.

\textbf{תמיד נחפש פתרון ממשי.} ניתן לחשוב, בדומה למה שעשינו בתת-הפרק מד׳׳ר עם מקדמים קבועים, שניתן לבחור קבוע מדומה טהור אשר יכפיל את האיבר המדומה הטהור הקיים בפתרון, וככה יתן לנו קבוע חדש ממשי טהור.

\example{}

מצאו פתרון כללי למערכת המשוואות הבאה:
\[
\dot{\vec{x}} =
\begin{pmatrix}
1 & 4\\[4pt]
-1 & -3
\end{pmatrix}
\vec{x}.
\]

\explanation{}

נמצא ראשית את הפי״א:
\[
P(\lambda) =
\begin{vmatrix}
1-\lambda & 4\\
-1 & -3-\lambda
\end{vmatrix}
= (\lambda + 1)^2 = 0.
\]

כלומר:
\[
\lambda = -1,
\]
עם ר״א 2.

\textbf{נמצא ו״ע:}
\[
\begin{pmatrix}
2 & 4\\[4pt]
-1 & -2
\end{pmatrix}
\begin{pmatrix}
a\\[2pt]
b
\end{pmatrix}
=
\begin{pmatrix}
0\\[2pt]
0
\end{pmatrix}.
\]

השורות ת״ל ולכן:
\[
1 = q < k=2.
\]
לכן המטריצה אינה לכסינה  
ונשתמש בפתרון הבא למקרה זה:

\[
\vec{x}(t)
= c_1 e^{\lambda_0 t}
\begin{pmatrix}
s_1(t)\\[2pt]
s_2(t)
\end{pmatrix}.
\]

כאשר דרגות הפולינומים הן לכל היותר \(k-q\).  
מציבים לתוך המערכת ופותרים עבור המקדמים של הפולינומים.  
במקרה שלנו:

\[
\vec{x}(t)
= c_1 e^{-t}
\begin{pmatrix}
a_0 + a_1 t\\[2pt]
b_0 + b_1 t
\end{pmatrix}.
\]

נציב למשוואה המקורית (אגף שמאל):
\[
\dot{\vec{x}} =
- e^{-t}
\begin{pmatrix}
a_0 + a_1 t\\[2pt]
b_0 + b_1 t
\end{pmatrix}
+ e^{-t}
\begin{pmatrix}
a_1\\[2pt]
b_1
\end{pmatrix}
= e^{-t}
\begin{pmatrix}
- a_0 + a_1 - a_1 t\\[2pt]
- b_0 + b_1 - b_1 t
\end{pmatrix}.
\]

אגף ימין:
\[
\begin{pmatrix}
1 & 4\\[2pt]
-1 & -3
\end{pmatrix}
e^{-t}
\begin{pmatrix}
a_0 + a_1 t\\[2pt]
b_0 + b_1 t
\end{pmatrix}
=
e^{-t}
\begin{pmatrix}
a_0 + 4b_0 + (a_1 + 4b_1)t\\[4pt]
- a_0 - 3b_0 + (-a_1 - 3b_1)t
\end{pmatrix}.
\]

משווים אגפים (האיבר \(e^{-t}\) מתבטל):
\[
\begin{pmatrix}
- a_0 + a_1 - a_1 t\\[2pt]
- b_0 + b_1 - b_1 t
\end{pmatrix}
=
\begin{pmatrix}
a_0 + 4b_0 + (a_1 + 4b_1)t\\[4pt]
- a_0 - 3b_0 + (-a_1 - 3b_1)t
\end{pmatrix}.
\]

נשווה מקדמים עבור כל קואורדיניטה בהתאמה:

\[
\begin{aligned}
-a_1 &= a_1 + 4b_1,\\[2pt]
- a_0 + a_1 &= a_0 + 4b_0,\\[2pt]
- b_1 &= - a_1 - 3b_1,\\[2pt]
- b_0 + b_1 &= - a_0 - 3b_0.
\end{aligned}
\]

שימו לב שאנו זקוקים בסך הכל להיות מסוגלים לבטא את הפתרון באמצעות שני פרמטרים ולא ארבעה, כיוון שמדובר בפתרון כללי. נוכל מהמשוואת לקבל את הקשרים:
\[
a_1 = 2a_0 + 4b_0,
\qquad
b_1 = -a_0 - 2b_0.
\]

ולכן:
\[
\vec{x}(t)
= e^{-t}
\begin{pmatrix}
a_0 + (2a_0 + 4b_0)t\\[4pt]
b_0 + (-a_0 - 2b_0)t
\end{pmatrix}.
\]

נסדר מעט ונכתוב את הפתרון הכללי לבעיה:
\[
\boxed{
\vec{x}(t)
= a_0 e^{-t}
\begin{pmatrix}
1 + 2t\\[2pt]
- t
\end{pmatrix}
+ b_0 e^{-t}
\begin{pmatrix}
4t\\[2pt]
1 - 2t
\end{pmatrix}.
}
\]

%%%CUT%%%

\newpage
\underline{תרגילים}
%%%%ex.%%%%

\exercise{}

מצאו פתרון כללי למערכת:
\[
\dot{\vec{X}} =
\begin{pmatrix}
1 & 1 & -1\\[4pt]
1 & -1 & 1\\[4pt]
-1 & 1 & 1
\end{pmatrix}
\vec{X}.
\]

\exercise{}

פתרו את המערכת ההומוגנית:
\[
\dot{\vec{X}} =
\begin{pmatrix}
1 & -1 & -1\\[4pt]
1 & 1 & 0\\[4pt]
3 & 0 & 1
\end{pmatrix}\vec{X},
\qquad
\vec{X}(0)=
\begin{pmatrix}
0\\[2pt]
2\\[2pt]
2
\end{pmatrix}.
\]

\newpage
\underline{פתרונות}
\solution{}

\textbf{שלב 1 – מציאת הערכים העצמיים}

נחשב את הפולינום האופייני:
\[
\det(A - \lambda I) =
\begin{vmatrix}
\lambda-1 & -1 & 1\\
-1 & \lambda+1 & -1\\
1 & -1 & \lambda-1
\end{vmatrix}
= (\lambda-2)(\lambda+2)(\lambda-1).
\]

ולכן הערכים העצמיים הם:
\[
\lambda_1 = 1, \qquad \lambda_2 = 2, \qquad \lambda_3 = -2.
\]

\textbf{שלב 2 – מציאת וקטורים עצמיים}

\underline{עבור $\lambda_1 = 1$:}

\[
(A - I)\vec{v} = 0
\quad\Rightarrow\quad
\begin{pmatrix}
0 & -1 & 1\\[4pt]
-1 & 2 & -1\\[4pt]
1 & -1 & 0
\end{pmatrix}
\begin{pmatrix}
a\\ b\\ c
\end{pmatrix}
=
\begin{pmatrix}
0\\0\\0
\end{pmatrix}.
\]

נפתור את מערכת המשוואות:
\[
\left\{
\begin{aligned}
-b + c &= 0,\\[4pt]
- a + 2b - c &= 0,\\[4pt]
a - b &= 0.
\end{aligned}
\right.
\]
הוקטור הוא:
\[
(a,b,c) = (b,b,b) \quad\Rightarrow\quad
\vec{v}_1 =
\begin{pmatrix}
1\\[2pt]
1\\[2pt]
1
\end{pmatrix}.
\]

\underline{עבור $\lambda_2 = 2$:}

\[
(A - 2I)\vec{v} = 0
\quad\Rightarrow\quad
\begin{pmatrix}
1 & -1 & 1\\[4pt]
-1 & 3 & -1\\[4pt]
1 & -1 & 1
\end{pmatrix}
\begin{pmatrix}
a\\ b\\ c
\end{pmatrix}
=
\begin{pmatrix}
0\\0\\0
\end{pmatrix}.
\]

נכתוב את המערכת:
\[
\left\{
\begin{aligned}
a - b + c &= 0,\\[4pt]
- a + 3b - c &= 0.
\end{aligned}
\right.
\]
הוקטור המייצג הוא:
\[
\vec{v}_2 =
\begin{pmatrix}
1\\[2pt]
0\\[2pt]
-1
\end{pmatrix}.
\]

\underline{עבור $\lambda_3 = -2$:}

\[
(A + 2I)\vec{v} = 0
\quad\Rightarrow\quad
\begin{pmatrix}
-3 & -1 & 1\\[4pt]
-1 & -1 & -1\\[4pt]
1 & -1 & -3
\end{pmatrix}
\begin{pmatrix}
a\\ b\\ c
\end{pmatrix}
=
\begin{pmatrix}
0\\0\\0
\end{pmatrix}.
\]

המערכת המתקבלת:
\[
\left\{
\begin{aligned}
-3a - b + c &= 0,\\[4pt]
- a - b - c &= 0,\\[4pt]
a - b - 3c &= 0.
\end{aligned}
\right.
\]
הפתרון המייצג המתקבל הוא:
\[
\vec{v}_3 =
\begin{pmatrix}
1\\[2pt]
-2\\[2pt]
1
\end{pmatrix}.
\]


\textbf{שלב 3 – כתיבת הפתרון ההומוגני}

נשתמש בתבנית הידועה:
\[
\vec{X}_H(t)
= c_1 e^{\lambda_1 t}\vec{v}_1
+ c_2 e^{\lambda_2 t}\vec{v}_2
+ c_3 e^{\lambda_3 t}\vec{v}_3,
\]

ולכן:
\[
\boxed{
\vec{X}_H(t)
= c_1 e^{t}
\begin{pmatrix}
1\\[2pt]
1\\[2pt]
1
\end{pmatrix}
+ c_2 e^{2t}
\begin{pmatrix}
1\\[2pt]
0\\[2pt]
-1
\end{pmatrix}
+ c_3 e^{-2t}
\begin{pmatrix}
1\\[2pt]
-2\\[2pt]
1
\end{pmatrix}.
}
\]



\solution{}

\textbf{שלב 1 – מציאת הערכים העצמיים}

נחשב את הפולינום האופייני:
\[
\det(A - \lambda I) =
\begin{vmatrix}
1-\lambda & -1 & -1\\[4pt]
1 & 1-\lambda & 0\\[4pt]
3 & 0 & 1-\lambda
\end{vmatrix}
= (1-\lambda)^3 + 3(1-\lambda) + (1-\lambda) = 0.
\]

נפשט:
\[
(1-\lambda)\big[(1-\lambda)^2 + 4\big] = 0
\quad\Rightarrow\quad
\lambda_1 = 1, \qquad
\lambda_{2,3} = 1 \pm 2i.
\]

\textbf{שלב 2 – מציאת וקטורים עצמיים}

\underline{עבור $\lambda_1 = 1$:}
\[
(A - I)\vec{v}_1 = 0
\quad\Rightarrow\quad
\begin{pmatrix}
0 & -1 & -1\\[4pt]
1 & 0 & 0\\[4pt]
3 & 0 & 0
\end{pmatrix}
\begin{pmatrix}
v_1\\ v_2\\ v_3
\end{pmatrix}
= \vec{0}.
\]
נקבל:
\[
v_1 = 0, \qquad v_2 = -v_3.
\]
ולכן:
\[
\vec{v}_1 =
\begin{pmatrix}
0\\[2pt]
1\\[2pt]
-1
\end{pmatrix}.
\]

\underline{עבור $\lambda_2 = 1 + 2i$:}
\[
(A - (1+2i)I)\vec{v}_2 = 0
\quad\Rightarrow\quad
\begin{pmatrix}
-2i & -1 & -1\\[4pt]
1 & -2i & 0\\[4pt]
3 & 0 & -2i
\end{pmatrix}
\begin{pmatrix}
v_1\\ v_2\\ v_3
\end{pmatrix}
= \vec{0}.
\]

מן המשוואות נקבל:
\[
\begin{cases}
-2iv_1 -\,v_2-v_3=0,\\[4pt]
-v_1 -2i\,v_2=0,\\[4pt]
3v_1-2iv_3=0.
\end{cases}
\]
 נקבל:
\[
\vec{v}_2 =
\begin{pmatrix}
2i\\[2pt]
1\\[2pt]
3
\end{pmatrix}.
\]

מכאן כי עבור \underline{ $\lambda_2 = 1 + 2i$:}
\[
\vec{v}_3 = \overline{\vec{v}_2} =
\overline{\begin{pmatrix}
2i\\[2pt]
1\\[2pt]
3
\end{pmatrix}}=\begin{pmatrix}
-2i\\[2pt]
1\\[2pt]
3
\end{pmatrix}.
\]


\textbf{שלב 3 – חישוב הפתרון הממשי עבור הערכים העצמיים המרוכבים}

נחפש כעת פתרון ממשי. 
נשתמש בע׳׳ע \(\lambda_2 = 1 + 2i\) ובו׳׳ע המתאים לו:
\[
\vec{v}_2 =
\begin{pmatrix}
2i\\[2pt]
1\\[2pt]
3
\end{pmatrix}.
\]

הפתרון המרוכב המתאים הוא:
\[
e^{(1+2i)t}
\cdot
\begin{pmatrix}
2i\\[2pt]
1\\[2pt]
3
\end{pmatrix}.
\]

נשתמש בזהות אוילר:
\[
e^{(1+2i)t}
\cdot
\begin{pmatrix}
2i\\[2pt]
1\\[2pt]
3
\end{pmatrix}
=
e^t(\cos(2t) + i\sin(2t))
\cdot
\begin{pmatrix}
2i\\[2pt]
1\\[2pt]
3
\end{pmatrix}.
\]

נפתח את הכפל המטריציוני:
\[
=
\begin{pmatrix}
2i e^t(\cos(2t) + i\sin(2t))\\[4pt]
e^t(\cos(2t) + i\sin(2t))\\[4pt]
3e^t(\cos(2t) + i\sin(2t))
\end{pmatrix}.
\]

נפריד לחלק ממשי ולחלק מדומה:

\[
=
\underbrace{
\begin{pmatrix}
-2e^t\sin(2t)\\[4pt]
e^t\cos(2t)\\[4pt]
3e^t\cos(2t)
\end{pmatrix}}_{\text{החלק הממשי}}
+
i
\underbrace{
\begin{pmatrix}
2e^t\cos(2t)\\[4pt]
e^t\sin(2t)\\[4pt]
3e^t\sin(2t)
\end{pmatrix}}_{\text{החלק המדומה}}.
\]

גם הצמוד המרוכב של הפתרון הוא פתרון, ולכן נוכל לקחת את שני הפתרונות הממשיים הבאים:

\[
\begin{aligned}
\vec{X}_2(t) &= 
\begin{pmatrix}
-2e^t\sin(2t)\\[4pt]
e^t\cos(2t)\\[4pt]
3e^t\cos(2t)
\end{pmatrix},
\\[6pt]
\vec{X}_3(t) &=
\begin{pmatrix}
2e^t\cos(2t)\\[4pt]
e^t\sin(2t)\\[4pt]
3e^t\sin(2t)
\end{pmatrix}.
\end{aligned}
\]

ולכן, הפתרון הכללי למערכת הוא:
\[
\vec{X}(t)
=
c_1 e^{t}
\begin{pmatrix}
0\\[2pt]
1\\[2pt]
-1
\end{pmatrix}
+
c_2
\begin{pmatrix}
-2e^t\sin(2t)\\[4pt]
e^t\cos(2t)\\[4pt]
3e^t\cos(2t)
\end{pmatrix}
+
c_3
\begin{pmatrix}
2e^t\cos(2t)\\[4pt]
e^t\sin(2t)\\[4pt]
3e^t\sin(2t)
\end{pmatrix}
=\]\[c_1 e^{t}
\begin{pmatrix}
0\\[2pt]
1\\[2pt]
-1
\end{pmatrix}
+
c_2e^t
\begin{pmatrix}
-2\sin(2t)\\[4pt]
\cos(2t)\\[4pt]
3\cos(2t)
\end{pmatrix}
+
c_3e^t
\begin{pmatrix}
2\cos(2t)\\[4pt]
\sin(2t)\\[4pt]
3\sin(2t)
\end{pmatrix}
\]

\textbf{שלב 4 – הצבת תנאי ההתחלה}

מהנתון:
\[
\vec{X}(0)=
\begin{pmatrix}
0\\[2pt]
2\\[2pt]
2
\end{pmatrix}.
\]

נחשב את \(\vec{X}(0)\) מתוך הפתרון הכללי:
\[
\vec{X}(0)
=
c_1
\begin{pmatrix}
0\\[2pt]
1\\[2pt]
-1
\end{pmatrix}
+
c_2
\begin{pmatrix}
0\\[2pt]
1\\[2pt]
3
\end{pmatrix}
+
c_3
\begin{pmatrix}
2\\[2pt]
0\\[2pt]
0
\end{pmatrix}.
\]

נשווה רכיבים עם תנאי ההתחלה:
\[
\begin{cases}
2c_3 = 0,\\[4pt]
c_1 + c_2 = 2,\\[4pt]
-\,c_1 + 3c_2 = 2.
\end{cases}
\]

נפתור את המערכת:
\[
c_3 = 0, \qquad c_1 = 1, \qquad c_2 = 1.
\]

\textbf{שלב 5 – הפתרון הפרטי}

נציב את הערכים בפתרון הכללי:
\[
\boxed{
\vec{X}(t)
=
e^{t}
\left[
\begin{pmatrix}
0\\[2pt]
1\\[2pt]
-1
\end{pmatrix}
+
\begin{pmatrix}
-2\sin(2t)\\[2pt]
\cos(2t)\\[2pt]
3\cos(2t)
\end{pmatrix}
\right]=e^{t}
\begin{pmatrix}
-2\sin(2t)\\[4pt]
1+\cos(2t)\\[4pt]
-1+3\cos(2t)
\end{pmatrix}
}.
\]

%%%CUT%%%

\newpage
\subsection{מערכת משוואות אי־הומוגנית - פתרון בשיטת וריאציית הפרמטר}

\textbf{\underline{צורה כללית:}}
\begin{equation}\tag{\ref{eq:nonhom_general}}
\dot{\vec{x}}(t) = A_{n \times n}\vec{x}(t) + \vec{b}(t)
\end{equation}

נניח ש־\(\vec{u}_1, \ldots, \vec{u}_n\) הם פתרונות בת״יל של ההומוגנית המתאימה.  
אז יש פתרון פרטי מהצורה:
\[
\vec{X}_p(t) = C_1(t)\vec{u}_1(t) + \cdots + C_n(t)\vec{u}_n(t)
\]
כאשר \(C_1(t), \ldots, C_n(t)\) הן פונקציות ממשיות לא וקטוריות.

\vspace{0.5cm}
\textbf{\underline{האלגוריתם לפתרון:}}
\[
C_1'(t)\vec{u}_1(t) + \cdots + C_n'(t)\vec{u}_n(t) = \vec{b}(t)
\]

 ישנה רק משוואה אחת לפתרון מכיוון שסדר המד"רים שאנחנו פותרים במערכת הוא מסדר 1.

נתחיל כרגיל במציאת הפתרון ההומוגני \(\vec{X}_h(t)\) לבעיה ולאחר מכן נמשיך לחלק הפרטי \(\vec{X}_p(t)\).  
הפתרון הכללי למערכת הוא סכום שניהם:
\[
\boxed{
\vec{X}(t) = \vec{X}_h(t) + \vec{X}_p(t)
}
\]

\example{}

מצאו פתרון פרטי למערכת המשוואות הבאה:
\[
\dot{\vec{x}} =
\begin{pmatrix}
1 & 2\\[4pt]
2 & -2
\end{pmatrix}
\vec{x}
+
\begin{pmatrix}
16te^{t}\\[4pt]
0
\end{pmatrix}.
\]

\explanation{}

נתחיל, כרגיל, מהפ׳׳א בחלק ההומוגני:

\[
P(\lambda) =
\begin{vmatrix}
1-\lambda & 2\\[4pt]
2 & -2-\lambda
\end{vmatrix}
= (\lambda + 3)(\lambda - 2) = 0.
\]

לכן הערכים העצמיים הם:
\[
\lambda_1 = -3, \qquad \lambda_2 = 2.
\]

\textbf{עבור $\lambda_1 = -3$:}
\[
(A - \lambda_1 I)\vec{v}_1 = \vec{0}
\quad\Rightarrow\quad
\begin{pmatrix}
1 - (-3) & 2\\[4pt]
2 & -2 - (-3)
\end{pmatrix}
\begin{pmatrix}
a\\[2pt]
b
\end{pmatrix}
=
\begin{pmatrix}
0\\[2pt]
0
\end{pmatrix}.
\]

נפשט:
\[
\begin{pmatrix}
4 & 2\\[4pt]
2 & 1
\end{pmatrix}
\begin{pmatrix}
a\\[2pt]
b
\end{pmatrix}
=
\begin{pmatrix}
0\\[2pt]
0
\end{pmatrix}.
\]

מהמשוואה הראשונה נקבל \(4a + 2b = 0 \Rightarrow b = -2a.\)

לכן:
\[
\vec{v}_1 =
a\begin{pmatrix}
1\\[2pt]
-2
\end{pmatrix}.
\]

\textbf{עבור $\lambda_2 = 2$:}
\[
(A - \lambda_2 I)\vec{v}_2 = \vec{0}
\quad\Rightarrow\quad
\begin{pmatrix}
1 - 2 & 2\\[4pt]
2 & -2 - 2
\end{pmatrix}
\begin{pmatrix}
a\\[2pt]
b
\end{pmatrix}
=
\begin{pmatrix}
0\\[2pt]
0
\end{pmatrix}.
\]

נפשט:
\[
\begin{pmatrix}
-1 & 2\\[4pt]
2 & -4
\end{pmatrix}
\begin{pmatrix}
a\\[2pt]
b
\end{pmatrix}
=
\begin{pmatrix}
0\\[2pt]
0
\end{pmatrix}.
\]

מהמשוואה הראשונה נקבל \(-a + 2b = 0 \Rightarrow a = 2b.\)

לכן:
\[
\vec{v}_2 =
b\begin{pmatrix}
2\\[2pt]
1
\end{pmatrix}.
\]

לכן הפתרון ההומוגני הוא:
\[
\boxed{
\vec{X}_H(t)
= d_1 e^{-3t}
\begin{pmatrix}
1\\[2pt]
-2
\end{pmatrix}
+ d_2 e^{2t}
\begin{pmatrix}
2\\[2pt]
1
\end{pmatrix}.
}
\]

תה נחפש פתרון פרטי – באמצעות וריאציית הפרמטר.
נניח כי הפתרון הפרטי מהצורה:
\[
\vec{X}_p(t)
= C_1(t)e^{-3t}
\begin{pmatrix}
1\\[2pt]
-2
\end{pmatrix}
+ C_2(t)e^{2t}
\begin{pmatrix}
2\\[2pt]
1
\end{pmatrix}.
\]

מכאן, על פי שיטת וריאציית הפרמטר, נקבל את המשוואה:
\[
C_1'(t)e^{-3t}
\begin{pmatrix}
1\\[2pt]
-2
\end{pmatrix}
+ C_2'(t)e^{2t}
\begin{pmatrix}
2\\[2pt]
1
\end{pmatrix}
=
\begin{pmatrix}
16te^{t}\\[2pt]
0
\end{pmatrix}.
\]

זוהי מערכת של שתי משוואות בשני נעלמים:
\[
\left\{
\begin{aligned}
C_1'(t)e^{-3t} + 2C_2'(t)e^{2t} &= 16te^{t},\\[4pt]
-2C_1'(t)e^{-3t} + C_2'(t)e^{2t} &= 0.
\end{aligned}
\right.
\]
מהמשוואה השנייה:
\[
C_1'(t)e^{-3t} = \tfrac{1}{2}C_2'(t)e^{2t}.
\]

נציב במשוואה הראשונה:
\[
\tfrac{1}{2}C_2'(t)e^{2t} + 2C_2'(t)e^{2t} = 16te^{t}
\quad\Rightarrow\quad
\tfrac{5}{2}C_2'(t)e^{2t} = 16te^{t}.
\]

ולכן:
\[
C_2'(t) = \frac{32}{5}te^{-t}
\quad\Rightarrow\quad
C_2(t) = -\frac{32}{5}e^{-t}(t+1) + d_2.
\]

נציב בביטוי שקיבלנו ל-$C_1'(t)$ ונקבל:
\[
C_1'(t) = \frac{16}{5}te^{4t}
\quad\Rightarrow\quad
C_1(t) = \frac{1}{5}e^{4t}(4t - 1) + d_1.
\]
כפי שהוכחנו בעבר, אין צורך לקחת את קבועי האינטגרציה שכן הם מוכלים בפתרון ההומוגני.
\textbf{ולכן הפתרון הפרטי יהיה:}
\[
\begin{aligned}
\vec{X}_p(t)
&= C_1(t)e^{-3t}
\begin{pmatrix}
1\\[2pt]
-2
\end{pmatrix}
+ C_2(t)e^{2t}
\begin{pmatrix}
2\\[2pt]
1
\end{pmatrix}= \tfrac{1}{5}e^{4t}(4t-1)e^{-3t}
\begin{pmatrix}
1\\[2pt]
-2
\end{pmatrix}
- \tfrac{32}{5}e^{-t}(t+1)e^{2t}
\begin{pmatrix}
2\\[2pt]
1
\end{pmatrix}.
\end{aligned}
\]

נפשט:
\[
\boxed{
\vec{X}_p(t)
= \tfrac{1}{5}e^{t}(4t - 1)
\begin{pmatrix}
1\\[2pt]
-2
\end{pmatrix}
- \tfrac{32}{5}e^{t}(t + 1)
\begin{pmatrix}
2\\[2pt]
1
\end{pmatrix}.
}
\]

\vspace{0.3cm}
\textbf{ולבסוף – הפתרון הכללי:}
\[
\boxed{
\vec{X}(t)
= \vec{X}_H(t) + \vec{X}_p(t)
= c_1 e^{-3t}
\begin{pmatrix}
1\\[2pt]
-2
\end{pmatrix}
+ c_2 e^{2t}
\begin{pmatrix}
2\\[2pt]
1
\end{pmatrix}
+ \tfrac{1}{5}e^{t}(4t - 1)
\begin{pmatrix}
1\\[2pt]
-2
\end{pmatrix}
- \tfrac{32}{5}e^{t}(t + 1)
\begin{pmatrix}
2\\[2pt]
1
\end{pmatrix}.
}
\]

%%%CUT%%%

\newpage
\underline{תרגילים}
\exercise{}

פתרו את המערכת הלא־הומוגנית:
\[
\dot{\vec{X}} =
\begin{pmatrix}
1 & -1\\[4pt]
-1 & 1
\end{pmatrix}
\vec{X}
+
\begin{pmatrix}
0\\[4pt]
1
\end{pmatrix}.
\]

\exercise{}

מצאו עבור איזה \( \beta \) ממשי למערכת הנתונה
\[
X'=
\begin{pmatrix}
-\beta & \beta\\[4pt]
\beta & -\beta
\end{pmatrix}X+
\begin{pmatrix}
\beta+1\\[4pt]
-2\beta
\end{pmatrix}
\]
יש פתרון שהוא וקטור עם רכיבים קבועים ומצאו את הפתרון הכללי של המערכת במקרה זה.

\exercise{}

מצאו פתרון פרטי למערכת:
\[
\dot{\vec{X}} =
\begin{pmatrix}
2 & 3\\[4pt]
3 & 2
\end{pmatrix}\vec{X}
+ e^{2t}
\begin{pmatrix}
3\\[4pt]
0
\end{pmatrix},
\qquad
\vec{X}(0)=
\begin{pmatrix}
0\\[2pt]
1
\end{pmatrix}.
\]

\newpage
\underline{פתרונות}
\solution{}

\textbf{שלב 1 – מציאת הערכים העצמיים}

נחשב את הפולינום האופייני של המערכת ההומוגנית:
\[
\det(A - \lambda I)
=
\begin{vmatrix}
1 - \lambda & -1\\[4pt]
-1 & 1 - \lambda
\end{vmatrix}
= (\lambda - 2)\lambda.
\]
ולכן הערכים העצמיים הם:
\[
\lambda_1 = 0, \qquad \lambda_2 = 2.
\]

\textbf{שלב 2 – מציאת וקטורים עצמיים}

\underline{עבור $\lambda_1 = 0$:}
\[
A\vec{v} = 0
\quad\Rightarrow\quad
\begin{pmatrix}
1 & -1\\[4pt]
-1 & 1
\end{pmatrix}
\begin{pmatrix}
a\\ b
\end{pmatrix}
=
\begin{pmatrix}
0\\0
\end{pmatrix}.
\]
המערכת מתקבלת:
\[
\left\{
\begin{aligned}
a - b &= 0,\\[4pt]
- a + b &= 0.
\end{aligned}
\right.
\]
נמצא \(a = b\), ולכן:
\[
\vec{v}_1 =
\begin{pmatrix}
1\\[2pt]
1
\end{pmatrix}.
\]

\underline{עבור $\lambda_2 = 2$:}
\[
(A - 2I)\vec{v} = 0
\quad\Rightarrow\quad
\begin{pmatrix}
-1 & -1\\[4pt]
-1 & -1
\end{pmatrix}
\begin{pmatrix}
a\\ b
\end{pmatrix}
=
\begin{pmatrix}
0\\0
\end{pmatrix}.
\]
המערכת היא:
\[
- a - b = 0 \quad\Rightarrow\quad a = -b.
\]
ולכן:
\[
\vec{v}_2 =
\begin{pmatrix}
1\\[2pt]
-1
\end{pmatrix}.
\]

\textbf{שלב 3 – כתיבת הפתרון ההומוגני}

נשתמש בתבנית הכללית:
\[
\vec{X}_H(t)
= c_1 e^{\lambda_1 t}\vec{v}_1
+ c_2 e^{\lambda_2 t}\vec{v}_2.
\]
נציב את הערכים שמצאנו:
\[
\boxed{
\vec{X}_H(t)
= c_1
\begin{pmatrix}
1\\[2pt]
1
\end{pmatrix}
+ c_2 e^{2t}
\begin{pmatrix}
1\\[2pt]
-1
\end{pmatrix}.
}
\]

\textbf{שלב 4 – פתרון פרטי בשיטת וריאציית הפרמטר}

נניח כי הפתרון הפרטי הוא מהצורה:
\[
\vec{X}_p(t)
= C_1(t)
\begin{pmatrix}
1\\[2pt]
1
\end{pmatrix}
+ C_2(t)e^{2t}
\begin{pmatrix}
1\\[2pt]
-1
\end{pmatrix}.
\]

מכאן, על פי שיטת וריאציית הפרמטר, נקבל את המשוואה:
\[
C_1'(t)
\begin{pmatrix}
1\\[2pt]
1
\end{pmatrix}
+ C_2'(t)e^{2t}
\begin{pmatrix}
1\\[2pt]
-1
\end{pmatrix}
=
\begin{pmatrix}
0\\[2pt]
1
\end{pmatrix}.
\]

מכאן נקבל מערכת של שתי משוואות בשני נעלמים \(C_1'(t), C_2'(t)\):
\[
\left\{
\begin{aligned}
C_1'(t) + C_2'(t)e^{2t} &= 0,\\[4pt]
C_1'(t) - C_2'(t)e^{2t} &= 1.
\end{aligned}
\right.
\]

נחבר את שתי המשוואות:
\[
2C_1'(t) = 1
\quad\Rightarrow\quad
C_1'(t) = \tfrac{1}{2}.
\]

נחסר את הראשונה מהשנייה:
\[
-2C_2'(t)e^{2t} = 1
\quad\Rightarrow\quad
C_2'(t) = -\tfrac{1}{2}e^{-2t}.
\]

נבצע אינטגרציה:
\[
C_1(t) = \tfrac{1}{2}t, 
\qquad
C_2(t) = \tfrac{1}{4}e^{-2t}.
\]

נחזור להצבה בפתרון הפרטי:
\[
\vec{X}_p(t)
= \tfrac{1}{2}t
\begin{pmatrix}
1\\[2pt]
1
\end{pmatrix}
+ \tfrac{1}{4}e^{-2t}e^{2t}
\begin{pmatrix}
1\\[2pt]
-1
\end{pmatrix}.
\]
מכאן נקבל:
\[
\boxed{
\vec{X}_p(t)
=
\begin{pmatrix}
0.5t + 0.25\\[4pt]
0.5t - 0.25
\end{pmatrix}.
}
\]

\textbf{שלב 5 – כתיבת הפתרון הכללי}

\[
\boxed{
\vec{X}(t)
=
c_1
\begin{pmatrix}
1\\[2pt]
1
\end{pmatrix}
+ c_2 e^{2t}
\begin{pmatrix}
1\\[2pt]
-1
\end{pmatrix}
+
\begin{pmatrix}
0.5t + 0.25\\[4pt]
0.5t - 0.25
\end{pmatrix}.
}
\]



\solution{}

נניח כי \( \vec{X} =
\begin{pmatrix}
x_1\\[2pt]
x_2
\end{pmatrix} \)
הוא וקטור קבוע. נציב במערכת ונקבל:
\[
\begin{pmatrix}
-\beta & \beta\\[4pt]
\beta & -\beta
\end{pmatrix}
\begin{pmatrix}
x_1\\[2pt]
x_2
\end{pmatrix}
+
\begin{pmatrix}
\beta+1\\[4pt]
-2\beta
\end{pmatrix}
=
\begin{pmatrix}
0\\[4pt]
0
\end{pmatrix}.
\]

כעת נבנה את המטריצה המורחבת עם מטריצת היחידה:
\[
\left[
\begin{array}{cc|cc|c}
-\beta & \beta & 1 & 0 & -\beta-1\\[4pt]
\beta & -\beta & 0 & 1 & 2\beta
\end{array}
\right]
\]
נבצע דירוג שורות:

\[
\left[
\begin{array}{cc|cc|c}
-\beta & \beta & 1 & 0 & -\beta-1\\[4pt]
\textcolor{red}{0} & \textcolor{red}{-0} & 1 & 1 & \textcolor{red}{\beta-1}
\end{array}
\right].
\]

נוכל כעת להבחין במשהו חשוב:
הגוש השמאלי כולו בשורה השנייה הוא אפסים,  
כלומר שתי המשוואות המקוריות \textbf{תלויות לינארית} זו בזו.  
מכאן שהמערכת לא תוכל להחזיק פתרון קבוע אלא אם כן גם האיבר החופשי באותה שורה יתאפס.
 
נדרוש:
\[
\textcolor{red}{\beta - 1 = 0}
\quad\Rightarrow\quad
\boxed{\beta = 1}.
\]

נציב כעת \(\beta = 1\) ונפתור את המערכת:
\[
X'=
\begin{pmatrix}
-1 & 1\\[4pt]
1 & -1
\end{pmatrix}X+
\begin{pmatrix}
2\\[4pt]
-2
\end{pmatrix}
\]

\textbf{שלב 1 – מציאת הערכים העצמיים}

נחשב את הפולינום האופייני של המערכת ההומוגנית:
\[
\det(A - \lambda I)
=
\begin{vmatrix}
-1 - \lambda & 1\\[4pt]
1 & -1 - \lambda
\end{vmatrix}
= (\lambda + 1)^2 - 1
= \lambda(\lambda + 2).
\]
ולכן:
\[
\lambda_1 = 0, \qquad \lambda_2 = -2.
\]

\textbf{שלב 2 – מציאת וקטורים עצמיים}

\underline{עבור $\lambda_1 = 0$:}
\[
A\vec{v} = 0
\quad\Rightarrow\quad
\begin{pmatrix}
-1 & 1\\[4pt]
1 & -1
\end{pmatrix}
\begin{pmatrix}
a\\ b
\end{pmatrix}
=
\begin{pmatrix}
0\\0
\end{pmatrix}.
\]
נקבל \(a = b\), ולכן:
\[
\vec{v}_1 =
\begin{pmatrix}
1\\[2pt]
1
\end{pmatrix}.
\]

\underline{עבור $\lambda_2 = -2$:}
\[
(A + 2I)\vec{v} = 0
\quad\Rightarrow\quad
\begin{pmatrix}
1 & 1\\[4pt]
1 & 1
\end{pmatrix}
\begin{pmatrix}
a\\ b
\end{pmatrix}
=
\begin{pmatrix}
0\\0
\end{pmatrix}.
\]
מכאן \(a = -b\), ולכן:
\[
\vec{v}_2 =
\begin{pmatrix}
1\\[2pt]
-1
\end{pmatrix}.
\]

\textbf{שלב 3 – כתיבת הפתרון ההומוגני}

נשתמש בתבנית הכללית:
\[
\vec{X}_H(t)
= c_1 e^{\lambda_1 t}\vec{v}_1
+ c_2 e^{\lambda_2 t}\vec{v}_2.
\]
נציב את הערכים שמצאנו:
\[
\boxed{
\vec{X}_H(t)
=
c_1
\begin{pmatrix}
1\\[2pt]
1
\end{pmatrix}
+
c_2 e^{-2t}
\begin{pmatrix}
1\\[2pt]
-1
\end{pmatrix}.
}
\]

\textbf{שלב 4 – פתרון פרטי בשיטת וריאציית הפרמטר}

נניח כי הפתרון הפרטי הוא מהצורה:
\[
\vec{X}_p(t)
= C_1(t)
\begin{pmatrix}
1\\[2pt]
1
\end{pmatrix}
+ C_2(t)e^{-2t}
\begin{pmatrix}
1\\[2pt]
-1
\end{pmatrix}.
\]

נקבל לפי השיטה מערכת של שתי משוואות בשני נעלמים \(C_1'(t), C_2'(t)\):
\[
\left\{
\begin{aligned}
C_1'(t) + C_2'(t)e^{-2t} &= 2,\\[4pt]
C_1'(t) - C_2'(t)e^{-2t} &= -2.
\end{aligned}
\right.
\]

נחבר את שתי המשוואות:
\[
2C_1'(t) = 0
\quad\Rightarrow\quad
C_1'(t) = 0.
\]

נחסר את הראשונה מהשנייה:
\[
-2C_2'(t)e^{-2t} = -4
\quad\Rightarrow\quad
C_2'(t) = 2e^{2t}.
\]

נבצע אינטגרציה וכאמור לא ניקח קבועים שכן הם מוכלים בפתרון ההומוגני:
\[
C_1(t) = 0, 
\qquad
C_2(t) = e^{2t}.
\]

נציב בחזרה בפתרון הפרטי:
\[
\vec{X}_p(t)
= 0
\begin{pmatrix}
1\\[2pt]
1
\end{pmatrix}
+ e^{2t}e^{-2t}
\begin{pmatrix}
1\\[2pt]
-1
\end{pmatrix}=\begin{pmatrix}
1\\[2pt]
-1
\end{pmatrix}.
\]

\textbf{שלב 5 – כתיבת הפתרון הכללי}

\[
\boxed{
\vec{X}(t)
=
c_1
\begin{pmatrix}
1\\[2pt]
1
\end{pmatrix}
+
c_2 e^{-2t}
\begin{pmatrix}
1\\[2pt]
-1
\end{pmatrix}
+
\begin{pmatrix}
1\\[2pt]
1
\end{pmatrix}
}.
\]



\solution{}

\textbf{שלב 1 – מציאת הערכים העצמיים}

נחשב את הפולינום האופייני של המערכת ההומוגנית:
\[
\det(A - \lambda I)
=
\begin{vmatrix}
2-\lambda & 3\\[4pt]
3 & 2-\lambda
\end{vmatrix}
= (2-\lambda)^2 - 9
= (\lambda-5)(\lambda+1).
\]
ולכן:
\[
\lambda_1 = 5, \qquad \lambda_2 = -1.
\]

\textbf{שלב 2 – מציאת וקטורים עצמיים}

\underline{עבור $\lambda_1 = 5$:}
\[
(A - 5I)\vec{v} = 0
\quad\Rightarrow\quad
\begin{pmatrix}
-3 & 3\\[4pt]
3 & -3
\end{pmatrix}
\begin{pmatrix}
a\\ b
\end{pmatrix}
=
\begin{pmatrix}
0\\0
\end{pmatrix}.
\]
נקבל \(a = b\), ולכן:
\[
\vec{v}_1 =
\begin{pmatrix}
1\\[2pt]
1
\end{pmatrix}.
\]

\underline{עבור $\lambda_2 = -1$:}
\[
(A + I)\vec{v} = 0
\quad\Rightarrow\quad
\begin{pmatrix}
3 & 3\\[4pt]
3 & 3
\end{pmatrix}
\begin{pmatrix}
a\\ b
\end{pmatrix}
=
\begin{pmatrix}
0\\0
\end{pmatrix}.
\]
נקבל \(a = -b\), ולכן:
\[
\vec{v}_2 =
\begin{pmatrix}
1\\[2pt]
-1
\end{pmatrix}.
\]

\textbf{שלב 3 – הפתרון ההומוגני}

נרשום את הפתרון ההומוגני:
\[
\boxed{
\vec{X}_H(t)
=
c_1 e^{5t}
\begin{pmatrix}
1\\[2pt]
1
\end{pmatrix}
+
c_2 e^{-t}
\begin{pmatrix}
1\\[2pt]
-1
\end{pmatrix}.
}
\]

\textbf{שלב 4 – פתרון פרטי בשיטת וריאציית הפרמטר}

נניח פתרון פרטי מהצורה:
\[
\vec{X}_p(t)
= C_1(t)e^{5t}
\begin{pmatrix}
1\\[2pt]
1
\end{pmatrix}
+ C_2(t)e^{-t}
\begin{pmatrix}
1\\[2pt]
-1
\end{pmatrix}.
\]

נקבל מהשיטה כי מתקיים:
\[
C_1'(t)e^{5t}
\begin{pmatrix}
1\\[2pt]
1
\end{pmatrix}
+ C_2'(t)e^{-t}
\begin{pmatrix}
1\\[2pt]
-1
\end{pmatrix}
=
e^{2t}
\begin{pmatrix}
3\\[2pt]
0
\end{pmatrix}.
\]

נרשום את מערכת המשוואות לפי רכיבים:
\[
\left\{
\begin{aligned}
C_1'(t)e^{5t} + C_2'(t)e^{-t} &= 3e^{2t},\\[4pt]
C_1'(t)e^{5t} - C_2'(t)e^{-t} &= 0.
\end{aligned}
\right.
\]

נחבר את שתי המשוואות:
\[
2C_1'(t)e^{5t} = 3e^{2t}
\quad\Rightarrow\quad
C_1'(t) = \tfrac{3}{2}e^{-3t}.
\]

נחסר את השנייה מהראשונה:
\[
2C_2'(t)e^{-t} = 3e^{2t}
\quad\Rightarrow\quad
C_2'(t) = \tfrac{3}{2}e^{3t}.
\]

נבצע אינטגרציה:
\[
C_1(t) = -\tfrac{1}{2}e^{-3t}, 
\qquad
C_2(t) = \tfrac{1}{2}e^{3t}.
\]

נציב לפתרון הפרטי:
\[
\vec{X}_p(t)
= -\tfrac{1}{2}e^{2t}
\begin{pmatrix}
1\\[2pt]
1
\end{pmatrix}
+ \tfrac{1}{2}e^{2t}
\begin{pmatrix}
1\\[2pt]
-1
\end{pmatrix}
=
\boxed{
e^{2t}
\begin{pmatrix}
0\\[2pt]
-1
\end{pmatrix}.
}
\]

\textbf{שלב 5 – הפתרון הכללי ויישום תנאי ההתחלה}

\[
\vec{X}(t)
=
c_1 e^{5t}
\begin{pmatrix}
1\\[2pt]
1
\end{pmatrix}
+
c_2 e^{-t}
\begin{pmatrix}
1\\[2pt]
-1
\end{pmatrix}
+
e^{2t}
\begin{pmatrix}
0\\[2pt]
-1
\end{pmatrix}.
\]

נציב את תנאי ההתחלה \(\vec{X}(0) = \begin{pmatrix}0\\[2pt]1\end{pmatrix}\):
\[
\begin{pmatrix}
0\\[2pt]
1
\end{pmatrix}
=
c_1
\begin{pmatrix}
1\\[2pt]
1
\end{pmatrix}
+
c_2
\begin{pmatrix}
1\\[2pt]
-1
\end{pmatrix}
+
\begin{pmatrix}
0\\[2pt]
-1
\end{pmatrix}.
\]

נפתור:
\[
\begin{cases}
c_1 + c_2 = 0,\\[4pt]
c_1 - c_2 - 1 = 1.
\end{cases}
\quad\Rightarrow\quad
c_1 = 1, \; c_2 = -1.
\]

ולכן:
\[
\boxed{
\vec{X}(t)
=
e^{5t}
\begin{pmatrix}
1\\[2pt]
1
\end{pmatrix}
-
e^{-t}
\begin{pmatrix}
1\\[2pt]
-1
\end{pmatrix}
+
e^{2t}
\begin{pmatrix}
0\\[2pt]
-1
\end{pmatrix}
}.
\]

%%%CUT%%%

\newpage
\section{תורת שטורם-ליוביל - ׳׳טיזר׳׳ למשוואות דיפרנציאליות חלקיות}

תורת שטורם–ליוביל מהווה מעין "טעימה מוקדמת" לקראת הקורס שרובכם תקחו לאחר קורס המד׳׳ר, והוא משוואות דיפרנציאליות \textbf{חלקיות} (מד׳׳ח).  
כאשר אנו מסוגלים לגשת למד׳׳חים באופן אנליטי, בהרבה מן המקרים נשתמש בשיטה פופולרית בשם
\textbf{שיטת הפרדת המשתנים}.  
בשיטה זו אנו ``שוברים'' את המד׳׳ח למערכת של \textbf{מד׳׳רים} פשוטות יותר –  
וחלק מאותן משוואות יתגלו כמשוואות מסוג \textbf{שטורם–ליוביל}.

תורת שטורם-ליוביל מספקת את המסגרת התאורטית להבנת מערכות שבהן מופיעים \textbf{ערכים עצמיים} ו־\textbf{פונקציות עצמיות},  
ומאפשרת לייצג פונקציות כלליות כסכום (או אינטגרל) של פונקציות עצמיות.  
זה בעצם הבסיס המתמטי לרעיון של \textbf{טורי פורייה, לפתרונות של משוואת החום, משוואת הגלים}, ועוד דוגמאות פיזיקליות רבות, אשר תיתקלו בהן במהלך התואר, ואף את רובכם תלוינה כל ימי חייכם.

 משוואת שטורם–ליוביל מוגדרת בצורה הבאה:
\begin{equation}
\frac{d}{dx}\!\left[p(x)\frac{dy}{dx}\right] + \big[\lambda w(x) - q(x)\big]y = 0,
\end{equation}
כאשר:
\[
p(x) > 0, \quad w(x) > 0,
\]
והפונקציות \(p(x), q(x), w(x)\) מוגדרות ורציפות על התחום \([a,b]\). הפונקציה $\omega$ נקראת פונקציית המשקל ולעיתים מסומנת גם ב-$r(x)$.
שימו לב כי המד׳׳׳ר לינארית.

לבעיה מוגדרים תנאי שפה (תנאי גבול) מהצורה:
\begin{equation}
\alpha_1 y(a) + \alpha_2 y'(a) = 0, 
\qquad 
\beta_1 y(b) + \beta_2 y'(b) = 0,
\end{equation}
כאשר כל הפרמטרים כאן ממשיים. אם כל תנאים אלה מתקיימים, מדובר בבעיית שטורם-ליובל \textbf{רגולרית}.

במקרה זה, האופרטור
\[
L[y] = -\frac{d}{dx}\!\left(p(x)\frac{dy}{dx}\right) + q(x)y
\]
הוא \textbf{הרמיט (self–adjoint)}, ולכן כל הערכים העצמיים \(\lambda_n\) הם ממשיים,  
והפונקציות העצמיות \(\{y_n(x)\}\) יוצרות מערכת אורתוגונלית ביחס למשקל \(w(x)\).

במקרה הכללי נרצה למצוא
ערכים עצמיים 
$\lambda_n$ 
ופתרונות מתאימים $y_n(x)$,
הנקראים \textbf{הפונקציות העצמיות}.

\vspace{0.5cm}
\textbf{תכונות עיקריות}

\begin{itemize}
  \item לכל ערך עצמי אמיתי \(\lambda_n\) קיים פתרון לא טריוויאלי \(y_n(x)\).
  \item הערכים העצמיים מסודרים בסדרה עולה:
  \begin{equation}
  \lambda_1 < \lambda_2 < \lambda_3 < \cdots,
  \quad \lim_{n \to \infty}\lambda_n = +\infty.
  \end{equation}
  כמסקנה מכך: יש $\infty$  ע"ע חיוביים, ולכל היותר מספר סופי של ע"ע אי-חיוביים.
  \item הפונקציות העצמיות האורתוגונליות ביחס לפונקציית המשקל \(w(x)\):
  \begin{equation}
  \int_a^b y_m(x)\,y_n(x)\,w(x)\,dx = 0 \quad, m \neq n.
  \end{equation}
  \item כל פונקציה רציפה בתחום \([a,b]\) ניתנת לפירוק כטור של הפונקציות העצמיות:
  \begin{equation}\label{move_base}
  f(x) = \sum_{n=1}^\infty c_n\,y_n(x),
  \quad
  c_n = \frac{\int_a^b f(x)y_n(x)w(x)\,dx}{\int_a^b y_n^2(x)w(x)\,dx},
  \end{equation}
\end{itemize}
כאשר 
הפונקציות העצמיות של בעיית שטורם-ליוביל הרגולרית הן אורתוגונליות עם משקל \(r\):
\begin{equation}
\langle y_n, y_m \rangle_r = \int_a^b y_n y_m r \, dx = \delta_{nm},
\end{equation}
כאשר \(\delta_{nm}\) היא הדלתא של קרונקר.

על מנת לנרמל את הפונקציה העצמית $y_{n}$, יש לחלק אותה במקדם נרמול $k_{n}$, כאשר יש לפתור:
\begin{equation}
\int_a^{b} k_n^2 y_{n}^2\,dx 
 = 1 .
\end{equation}

\example{}
נביט בדוגמה בסיסית - בעיית ערך שפה קלאסית:

\[
y'' + \lambda y = 0, 
\qquad y'(0)=y'(\pi)=0.
\]

\explanation{}
זו בעיה שכבר פתרנו בעבר, ונראה שהיא מקרה פרטי של בעיית שטורם–ליוביל עם:
\[
p(x)=1,\quad q(x)=0,\quad w(x)=1.
\]

נשווה איברים אחד לאחד בין שתי המשוואות.

 נתחיל ממשוואת שטורם-ליוביל:
  \[
  \frac{d}{dx}\!\left[p(x)\frac{dy}{dx}\right]
  = p'(x)\,y' + p(x)\,y''.
  \]

   במקרה הפשוט שלנו, נזהה:
  \[
  p(x) = 1 \quad\Rightarrow\quad p'(x)=0.
  \]
  לכן:
  \[
  \frac{d}{dx}\!\left[p(x)\frac{dy}{dx}\right] = y''.
  \]

   נציב זאת בצורת שטורם-ליוביל הכללית:
  \[
  y'' + \big[\lambda w(x) - q(x)\big]y = 0.
  \]

   נזהה כעת את הפונקציות:
  \[
  w(x) = 1, \qquad q(x) = 0.
  \]

   נקבל אפוא:
  \[
  \boxed{
  \frac{d}{dx}\!\left[\underbrace{1}_{p(x)}\frac{dy}{dx}\right]
  + \big[\lambda \cdot \underbrace{1}_{w(x)} - \underbrace{0}_{q(x)}\big]y = 0
  \quad\Longrightarrow\quad
  y'' + \lambda y = 0.
  }
  \]

  כידוע אנו מחפשים ערכי ע׳׳ע שעבורם לבעיה הנ׳׳ל קיים פתרון לא טריוויאלי.
  
 זו מד"ר עם מקדמים קבועים. הפ"א הוא
\[
r^{2}+\lambda=0.
\] 

כעת נפתור את הבעיה לפי ערכיה העצמיים \(\lambda\).

נניח צורת פתרון כללית:
\[
y(x) = 
\begin{cases}
C_1 e^{\sqrt{-\lambda}x} + C_2 e^{-\sqrt{-\lambda}x}, & \lambda < 0 \\[6pt]
C_1 + C_2 x, & \lambda = 0 \\[6pt]
C_1 \cos(\sqrt{\lambda}x) + C_2 \sin(\sqrt{\lambda}x). & \lambda > 0
\end{cases}
\]

\textbf{מקרה 1 – \(\lambda < 0\):}

נניח \(-\lambda = \mu > 0\), ולכן:
\[
y(x) = C_1 e^{\sqrt{\mu}x} + C_2 e^{-\sqrt{\mu}x}.
\]

נחשב את הנגזרת:
\[
y'(x) = C_1 \sqrt{\mu} e^{\sqrt{\mu}x} - C_2 \sqrt{\mu} e^{-\sqrt{\mu}x}.
\]

נציב תנאי השפה:
\[
\begin{cases}
y'(0) = C_1 \sqrt{\mu} - C_2 \sqrt{\mu} = 0 \Rightarrow C_1 = C_2, \\[4pt]
y'(\pi) = C_1 \sqrt{\mu}\big(e^{\sqrt{\mu}\pi} - e^{-\sqrt{\mu}\pi}\big) = 0.
\end{cases}
\]

אך מאחר ש-\(e^{\sqrt{\mu}\pi} \neq e^{-\sqrt{\mu}\pi}\), נקבל ש-\(C_1 = 0\) ולכן הפתרון הוא טריוויאלי בלבד.

\[
\boxed{ \lambda < 0\,\text{אין ערכים עצמיים עבור }}
\]

\textbf{מקרה 2 – \(\lambda = 0\):}

במקרה זה המשוואה הופכת ל:
\[
y'' = 0 \quad \Rightarrow \quad y(x) = C_1 + C_2 x.
\]

נציב תנאי השפה:
\[
y'(x) = C_2, \quad
y'(0)=y'(\pi)=C_2=0.
\]

נמצא:
\[
C_2 = 0 \Rightarrow y(x) = C_1.
\]

זהו פתרון קבוע, ולכן לא טריוויאלי.

\[
\boxed{\lambda_0 = 0, \quad y_0(x) = 1.}
\]

\textbf{מקרה 3 – \(\lambda > 0\):}

נניח \(\sqrt{\lambda} = k > 0\).  
אזי:
\[
y(x) = C_1 \cos(kx) + C_2 \sin(kx),
\quad
y'(x) = -C_1 k \sin(kx) + C_2 k \cos(kx).
\]

נציב תנאי השפה:
\[
\begin{cases}
y'(0)=0 \Rightarrow C_2 k = 0 \Rightarrow C_2=0,\\[6pt]
y'(\pi)=0 \Rightarrow -C_1 k \sin(k\pi)=0.
\end{cases}
\]

כדי לקבל פתרון לא טריוויאלי נדרוש:
\[
\sin(k\pi) = 0 \Rightarrow k = n, \quad n = 1,2,3,\ldots
\]

ומכאן:
\[
\boxed{
\lambda_n = n^2,
\qquad
y_n(x) = \cos(nx)}.
\]

 אם כל הערכים העצמיים של הבעיה הם
\[
\lambda_0 < \lambda_1 < \lambda_2 < \cdots
\qquad ( \lambda_0 \text{והראשון מסומן כ־})
\]
(כאמור: סדרה מונוטונית עולה ממש אל \(+\infty\)),
אז הפונקציה העצמית \(y_n(x)\), המתאימה לערך העצמי \(\lambda_n\),
מתאפסת בדיוק \(n\) פעמים בקטע הפתוח \((a,b)\).

\textbf{השיטה:}
בודקים האם הפונקציה העצמית המתאימה לערך העצמי הקטן ביותר
מתאפסת בקטע הפתוח.

במקרה שלנו, הפונקציה העצמית הראשונה היא:
\[
y_1(x) = \cos(x),
\qquad
\lambda_1 = 1,
\]
והקטע הפתוח הוא \((0, \pi)\),
מאחר שתנאי השפה הם:
\[
y'(0) = y'(\pi) = 0.
\]

וכידוע, הפונקציה \(\cos(x)\) מתאפסת בדיוק פעם אחת בקטע \((0, \pi)\).

\textbf{מסקנה:}
\[
\boxed{
\text{קיים עוד ערך עצמי אחד בלבד, אי־חיובי (שלילי או אפסי).}
}
\]


\textbf{לסיכום:}

מצאנו כי הפונקציות העצמיות הפותרות את הבעיה הן:
\[
\boxed{
y_0(x)=1
\quad\text{ו־}\quad
 \, y_n(x)=\cos(nx), \; n=1,2,3,\ldots
}
\]

בעצם גילויים אלה מאששים את הפתרון שלנו, שכן הוכחנו כי לא קיים לבעיה ערך עצמי שלילי, אך 0 כן בא בחשבון וזה עולה בקנה אחד עם הנאמר לעיל.
נראה את הפונקציות העצמיות הפותרות גרפית:
\begin{figure}[H]
\centering
\begin{tikzpicture}
\begin{axis}[
    width=16cm, height=9cm,
    domain=0:pi,
    samples=300,
    axis lines=middle,
    axis line style={-stealth, very thick},
    xlabel={$x$},
    ylabel={$y$},
    xlabel style={at={(axis description cs:1,0.5)},anchor=north west, font=\large},
    ylabel style={at={(axis description cs:0,1)},anchor=south east, font=\large},
    xtick=\empty, ytick=\empty,
    xmin=0, xmax=3.2,
    ymin=-1.3, ymax=1.3,
    legend style={
        at={(0.5,1.08)},
        anchor=south,
        legend columns=3,
        draw=none,
        font=\large,
        /tikz/every even column/.append style={column sep=0.4cm},
        row sep=4pt
    },
    every axis plot/.append style={very thick, smooth}
]

% y0(x) = 1
\addplot[black, dashed, ultra thick] {1};
\addlegendentry{$y_0(x)=1$};

% y1(x) = cos(x)
\addplot[blue, thick] {cos(deg(x))};
\addlegendentry{$y_1(x)=\cos(x)$};

% y2(x) = cos(2x)
\addplot[red, thick] {cos(deg(2*x))};
\addlegendentry{$y_2(x)=\cos(2x)$};

% y3(x) = cos(3x)
\addplot[ForestGreen, thick] {cos(deg(3*x))};
\addlegendentry{$y_3(x)=\cos(3x)$};

% y4(x) = cos(4x)
\addplot[orange, thick] {cos(deg(4*x))};
\addlegendentry{$y_4(x)=\cos(4x)$};

% y5(x) = cos(5x)
\addplot[purple, thick] {cos(deg(5*x))};
\addlegendentry{$y_5(x)=\cos(5x)$};

\end{axis}
\end{tikzpicture}

\caption{
הפונקציות העצמיות הראשונות של בעיית הערך השפה 
$y''+\lambda y=0$ עם תנאי שפה $y'(0)=y'(\pi)=0$.  
מוצגות חמש הפונקציות העצמיות הראשונות עבור $\lambda>0$ 
והפתרון הקבוע $y_0(x)=1$ עבור $\lambda=0$.  
כמוסבר, אין פונקציה עצמית עבור $\lambda<0$.
}
\end{figure}

\example{}
נבחן בעיית ערך שפה מסוג אוילר–שטורם–ליוביל:
\[
x^2y'' + 2xy' + \lambda y = 0,
\qquad
y(1) = y(e) = 0.
\]
יש למצוא את הערכים העצמיים והפונקציות העצמיות המנורמלות של המשוואה.

\explanation{}
מדובר במשוואת אוילר:
נניח צורת פתרון \(y = x^r\), ונקבל את הפ"א:
\[
r(r-1) + 2r + \lambda = 0
\quad\Longrightarrow\quad
r^2 + r + \lambda = 0.
\]

נפתור ונקבל:
\[
r_{1,2} = \frac{-1 \pm \sqrt{1-4\lambda}}{2}
= -\frac{1}{2} \pm \sqrt{\frac{1}{4} - \lambda}.
\]

כעת יש להבחין בין המקרים לפי סימן הביטוי \(\tfrac{1}{4}-\lambda\). מאחר שמובטחים $\infty$ ע"ע חיוביים, נתחיל מ  
 \textbf{מקרה 1:}
 \(\lambda > \tfrac{1}{4}\)
השורשים מרוכבים. נסמן $\beta=\sqrt{\lambda - \tfrac{1}{4}}\in\mathbb{C}$. השורשים אם ככה יהיו:
\[
r_{1,2} = -\tfrac{1}{2} \pm i\beta.
\]
לכן:
\[
y(x) = x^{-1/2}\big(C_1\cos(\beta\ln x) + C_2\sin(\beta\ln x)\big).
\]
נשתמש בתנאי השפה:
\[
\begin{cases}
y(1) = 0 \Rightarrow C_1 = 0, \\[4pt]
y(e) = C_2 e^{-1/2}\sin(\beta) = 0.
\end{cases}
\]
כדי לקבל פתרון לא טריוויאלי נדרש \(\sin(\beta)=0 \Rightarrow \beta = n\pi\).

מכאן הערכים העצמיים הם:
\[
\boxed{\lambda_n = \tfrac{1}{4} + n^2\pi^2, \quad n=1,2,3,\ldots}
\]
והפונקציות העצמיות המתאימות:
\[
\boxed{y_n(x) = x^{-1/2}\sin(n\pi\ln x)}.
\]

 בדיקה:
 האם הפ"ע עבור הע"ע הכי קטן \(y_1(x)\) שמצאנו מתאפסת בקטע הפתוח \((1,e)\) ?
נבחן:
\[
y_1(x) = x^{-1/2}\sin(\pi\ln x).
\]
מאחר ש-
\[1<x<e \Rightarrow 0<\ln x<1 \Rightarrow 0<\pi\ln x<\pi,\]
והפונקציה \(\sin(\pi\ln x)\) מתאפסת רק בנקודות הקצה,
הרי שאין אפסים נוספים בתחום הפתוח — מה שמאשש כי \(\lambda_1\)
הוא הערך העצמי הקטן ביותר.
ולכן אין עוד ע"ע עבור $\lambda \leq\frac{1}{4}$.
מכאן, כי אין טעם לבדוק עוד מקרים של ע׳׳ע. נעשה בכל זאת לטובת האימון.

\textbf{מקרה 2 – } \(\lambda = \tfrac{1}{4}\)

במקרה זה נקבל שורש כפול:
\[
r = -\frac{1}{2}.
\]

ולכן הפתרון הכללי של משוואת אוילר הוא:
\[
y(x) = x^{-1/2}(C_1 + C_2 \ln x).
\]

נציב את ערכי הקצה:

\[
\begin{cases}
y(1) = 1^{-1/2}(C_1 + C_2 \ln 1) = C_1 + C_2 \cdot 0 = C_1, \\[6pt]
y(e) = e^{-1/2}(C_1 + C_2 \ln e) = e^{-1/2}(C_1 + C_2 \cdot 1) = e^{-1/2}(C_1 + C_2).
\end{cases}
\]

נציב את תנאי השפה:
\[
\begin{cases}
y(1) = 0 \Rightarrow C_1 = 0, \\[6pt]
y(e) = 0 \Rightarrow e^{-1/2}(C_1 + C_2) = 0.
\end{cases}
\]

נחליף את \(C_1=0\) במשוואה השנייה:
\[
e^{-1/2}(0 + C_2) = 0 \quad \Rightarrow \quad C_2 = 0.
\]

ולכן:
\[
C_1 = C_2 = 0 \quad \Longrightarrow \quad y(x) \equiv 0.
\]

פתרון טריוויאלי בלבד. אין ערכים עצמיים עבור $\lambda = \tfrac{1}{4}$ .

מקרה 3:
 \(\lambda < \tfrac{1}{4}\)
השורשים ממשיים ושונים.
נגדיר את $\alpha$:
\[
r_{1,2} = -\tfrac{1}{2} \pm \alpha, \qquad
\alpha = \sqrt{\tfrac{1}{4}-\lambda}>0.
\]
ולכן הפתרון הכללי הוא:
\[
y(x) = C_1 x^{-1/2 + \alpha} + C_2 x^{-1/2 - \alpha}.
\]
לאחר הצבה בתנאי השפה מתקבל פתרון טריוויאלי בלבד,
ולכן אין ערכים עצמיים עבור \(\lambda < \tfrac{1}{4}\).

\textbf{מקרה 3 – } \(\lambda < \tfrac{1}{4}\)

במקרה זה הביטוי מתחת לשורש חיובי, ולכן השורשים ממשיים ושונים:
\[
r_{1,2} = -\tfrac{1}{2} \pm \alpha,
\qquad
\alpha = \sqrt{\tfrac{1}{4}-\lambda} > 0.
\]

מכאן שהפתרון הכללי הוא:
\[
y(x) = C_1 x^{-1/2 + \alpha} + C_2 x^{-1/2 - \alpha}.
\]

נציב את ערכי הקצה:

\[
\begin{cases}
y(1) = C_1 \cdot 1^{-1/2 + \alpha} + C_2 \cdot 1^{-1/2 - \alpha}
      = C_1 + C_2, \\[6pt]
y(e) = C_1 e^{-1/2 + \alpha} + C_2 e^{-1/2 - \alpha}
      = e^{-1/2}\big(C_1 e^{\alpha} + C_2 e^{-\alpha}\big).
\end{cases}
\]

נציב את תנאי השפה \(y(1)=y(e)=0\):

\[
\begin{cases}
C_1 + C_2 = 0, \\[4pt]
C_1 e^{\alpha} + C_2 e^{-\alpha} = 0.
\end{cases}
\]

נחליף \(C_2 = -C_1\) מהמשוואה הראשונה:
\[
C_1 e^{\alpha} - C_1 e^{-\alpha} = 0
\quad \Longrightarrow \quad
C_1 (e^{\alpha} - e^{-\alpha}) = 0.
\]

מאחר ש-\(e^{\alpha} \neq e^{-\alpha}\) (כי \(\alpha > 0\)),
נקבל בהכרח:
\[
C_1 = 0 \quad \Rightarrow \quad C_2 = 0.
\]

ולכן:
\[
C_1 = C_2 = 0 \quad \Longrightarrow \quad y(x) \equiv 0.
\]

שוב קיבלנו פתרון טריוויאלי בלבד (כמצופה). אין ערכים עצמיים עבור $\lambda < \tfrac{1}{4}$.

 נעבור לנרמול של הפונקציות העצמיות הפותרות את הבעיה.
נזכיר כי:
\[
(p(x)y')' + q(x)y + \lambda r(x)y = 0, \qquad a < x < b
\]

פונקציית המשקל היא \(r(x)\). בנוסף:

\[
\begin{cases}
\alpha y(a) + \beta y'(a) = 0 \\[4pt]
\gamma y(b) + \delta y'(b) = 0
\end{cases}
\]

הפונקציות העצמיות של בעיית שטורם-ליוביל הרגולרית הן אורתוגונליות עם משקל \(r\):
\[
\langle y_n, y_m \rangle_r = \int_a^b y_n y_m r \, dx = \delta_{nm},
\]
כאשר \(\delta_{nm}\) היא הדלתא של קרונקר.

\textbf{מקדמי הנרמול – } הם:
\[
k_n^2 = \int_a^b (y_n(x))^2 r(x) \, dx.
\]
להזכירכם, $r$ ו- $\omega$ הם סימנים מקובלים לפונקציית המשקל.
נמצא את מקדם הנרמול עבור הפונקציות העצמיות שנמצאו:
\[
y_n(x) = x^{-1/2}\sin(n\pi \ln x), \qquad 1 \le x \le e.
\]

כדי להבין מהי פונקציית המשקל, נכתוב את המשוואה בצורה הסטנדרטית:
\[
(p y')' + qy + \lambda ry = 0.
\]
במקרה שלנו:
\[
x^2 y'' + 2x y' + \lambda y = 0
\quad \Longrightarrow \quad
p = x^2, \quad q = 0, \quad r = 1.
\]

נחשב את \(k_n\):
\[
k_n^2 = \int_1^e (y_n)^2 \cdot 1 \, dx
       = \int_1^e \big(x^{-1/2}\sin(n\pi \ln x)\big)^2 dx
       = \int_1^e \sin^2(n\pi \ln x) \cdot \frac{1}{x} dx.
\]

נבצע את ההצבה:
\[
t = n\pi \ln x \quad \Rightarrow \quad dt = n\pi \frac{dx}{x}.
\]

מכאן:
\[
k_n^2 = \frac{1}{n\pi}\int_0^{n\pi} \sin^2 t \, dt
       = \frac{1}{n\pi}\int_0^{n\pi} \frac{1 - \cos(2t)}{2} \, dt
       = \frac{1}{n\pi} \cdot \frac{1}{2} \cdot n\pi
       = \frac{1}{2}.
\]

מכאן שהפונקציות המנורמלות הן:
\[
\boxed{
\phi_n(x) =\frac{y_{n}}{k_{n}}=\frac{x^{-1/2}\sin(n\pi \ln x)}{\sqrt{\frac{1}{2}}}= \sqrt{2}\,x^{-1/2}\sin(n\pi \ln x),
\qquad n = 1,2,\ldots
}
\]
הערה לגבי הנרמול:
במקרה שלנו חישבנו
\[
k_n^2 = \frac{1}{2}.
\]
 $k_n$ מייצג \textbf{גודל נורמה} ,  
 לכן הוא מוגדר להיות \textbf{חיובי בלבד} — בדיוק כפי שאורך וקטור רגיל אינו שלילי.  
לכן נבחר:
\[
k_n = \sqrt{\frac{1}{2}} = \frac{1}{\sqrt{2}}.
\]

%%%CUT%%%

\newpage
\underline{תרגילים}
\exercise{}
מצאו את פונקציית המשקל $r(x)$ כך שהמשוואה הבאה תתאים לצורת שטרום–ליוביל :
\[
x^2 y'' - x y' + \lambda y = 0.
\]

\exercise{}
פתחו את $\varphi(x) = x$ לטור בפונקציות העצמיות של הבעיה
\[
y'' + \lambda y = 0, \qquad y(0)=0, \; y(\pi)=0.
\]

\exercise{}
ניתן כאן מוטביציה לקורס שמגיע לאחר קורס במשוואות דיפרנציאליות רגילות, הלא הוא משוואות דיפרנציאליות חלקיות (מד׳׳ח). בשלב זה תבינו שלמעשה אתם כבר יודעים לפתור מד׳׳חים מסוימות אשר מתארות מערכות דינמיות אמיתיות, מבלי להיחשף עדיין לתכני הקורס. שימו לב שההבדל בין הגדרת מד׳׳ח לעומת מד׳׳ר הוא שבמד׳׳ח, המשתנה התלוי, תלוי בלפחות שני משתנים בלתי תלויים, בניגוד למד׳׳ר, בה המשתנה התלוי תלוי במשתנה בלתי תלוי אחד בלבד.

נבחן תהליך הולכת חום במוט אחיד באורך \(L\), כאשר הקצה השמאלי והימני מוחזקים בטמפרטורה קבועה של 0 (ביחידות מסוימות). המשוואה הדיפרנציאלית החלקית , יחד עם תנאי השפה של המוט ותנאי ההתחלה, מתוארים כאן:

\[
\begin{cases}
u_t = \alpha^2 u_{xx}, & \text{(משוואת החום)} \\[6pt]
u(0,t) = 0, & \text{(תנאי שפה שמאלי – קצה שמוחזק בטמפ׳ קבועה)} \\[6pt]
u(L,t) = 0, & \text{(תנאי שפה ימני – קצה שמוחזק בטמפ׳ קבועה)} \\[6pt]
u(x,0) = f(x), & \text{(תנאי התחלתי – התפלגות טמפרטורה התחלתית)}
\end{cases}
\]

כאשר \(u(x,t)\) מתאר את הטמפרטורה לאורך המוט בזמן \(t\), ו־\(\alpha\) הוא מקדם הדיפוזיה התרמית של החומר. לפניכם סקיצה המתארת את המערכת:

\begin{center}
\begin{tikzpicture}[scale=1.1,>=stealth]
  % rod
  \draw[thick,fill=gray!15,rounded corners] (0,0) rectangle (6,0.6);
  % edges
  \draw[fill=blue!40] (0,0) rectangle (0.3,0.6);
  \draw[fill=blue!40] (5.7,0) rectangle (6,0.6);
  % labels
  \node[below] at (0,0) {$x=0$};
  \node[below] at (6,0) {$x=L$};
  \node[above,blue!70!black] at (0.15,0.6) {$0$};
  \node[above,blue!70!black] at (5.85,0.6) {$0$};
\end{tikzpicture}
\end{center}
קבלו את פרופיל הטמפרטורה עבור המוט במרחב ובזמן $u(x,t)$, בכפוף ל: $\alpha, f(x)$, תוך הנחה כי ניתן לכתוב את הפתרון בצורה $u(x,t)=X(x)T(t)$. למעשה עליכם לגזור קשר זה, להציב אותו חזרה למד׳׳ח, ולקבל שתי מד׳׳רים לפתרון, אותן אתם יודעים לפתור.

הניחו כי מעבר החום מתרחש בצורה חד מימדית בכיוון $x$, מהקצה החם לקצה הקר.

\newpage
\underline{פתרונות}
\solution{}
נרצה להביא את המשוואה לצורה סטנדרטית של שטורם-ליוביל:
\[
(p(x)y')' + \lambda r(x)y = 0.
\]

נחלק את המשוואה המקורית ב־$x^2$ כדי לנרמל את המד׳׳ר (בהנחה כי $x\neq0$:
\[
y'' - \frac{1}{x}y' + \frac{\lambda}{x^2}y = 0.
\]

כעת נרצה שהאיברים $y''$ ו־$y'$ יופיעו בצורה של נגזרת של מכפלה $(\mu y')'$, כלומר:
\[
(\mu y')' = \mu y'' + \textcolor{blue}{\mu' y'} = \textcolor{blue}{\mu} \left( y'' - \textcolor{blue}{\frac{1}{x} y'} \right).
\]
שימו לב שעשינו פעולה זהה בתחילתו של ספר זה בהקשר של גורמי אינטגרציה.
המד׳׳ר ברקע היא
\[
\mu y'' - \frac{1}{x}\mu y' + \frac{1}{x^{2}}\lambda \mu y = 0.
\]

נשווה את \textcolor{blue}{המקדמים} ונקבל תנאי על $\mu$:
\[
\frac{\mu'}{\mu} = -\frac{1}{x} \quad \Longrightarrow \quad \mu' = -\frac{1}{x}\mu.
\]

נפתור:
\[
\frac{d\mu}{\mu} = -\frac{dx}{x} \quad \Longrightarrow \quad \ln \mu = -\ln x + C \quad \Longrightarrow \quad \mu = \frac{1}{x}.
\]

נציב חזרה למשוואה:
\[
\frac{1}{x}y'' - \frac{1}{x^2}y' + \frac{\lambda}{x^3}y = 0.
\]

לפיכך, הצורה כעת מתאימה לצורת שטורם-ליוביל:
\[
\frac{d}{dx}\left(\frac{1}{x}y'\right) + \frac{\lambda}{x^3}y = 0ֿ\rightarrow(p(x)y')' + \lambda r(x)y = 0,
\]
כאשר:
\[
p(x) = \frac{1}{x}, \qquad r(x) = \frac{1}{x^3}.
\]

\textbf{פונקציית המשקל אם כן היא:}
\[
\boxed{r(x) = \frac{1}{x^3}}
\]



\solution{}

נבחין בין מקרים שונים לפי סימן $\lambda$:

\textbf{מקרה 1 – } $\lambda = 0$  
המשוואה הופכת ל־$y'' = 0$, ולכן:
\[
y(x) = C_1 x + C_2.
\]
נציב את תנאי השפה:
\[
\begin{cases}
y(0) = 0 \Rightarrow C_2 = 0, \\[4pt]
y(\pi) = C_1 \pi = 0 \Rightarrow C_1 = 0.
\end{cases}
\]
ולכן נקבל פתרון טריוויאלי בלבד:
\[
y(x) \equiv 0.
\]
אין ערכים עצמיים עבור $\lambda = 0$.

\textbf{מקרה 2 – } $\lambda < 0$  
נסמן $\lambda = -\mu^2$, כאשר $\mu > 0$.  
המשוואה תקבל צורה:
\[
y'' - \mu^2 y = 0,
\]
ולכן הפתרון הכללי הוא:
\[
y(x) = C_1 e^{\mu x} + C_2 e^{-\mu x}.
\]
נציב תנאי שפה:
\[
\begin{cases}
y(0) = C_1 + C_2 = 0, \\[4pt]
y(\pi) = C_1 e^{\mu \pi} + C_2 e^{-\mu \pi} = 0.
\end{cases}
\]
נחליף $C_2 = -C_1$:
\[
C_1(e^{\mu \pi} - e^{-\mu \pi}) = 0.
\]
מאחר ש-$e^{\mu \pi} - e^{-\mu \pi} \neq 0$ לכל $\mu>0$, נובע:
\[
C_1 = C_2 = 0.
\]
ולכן קיבלנו פתרון טריוויאלי בלבד. אין ערכים עצמיים עבור $\lambda < 0$.

\textbf{מקרה 3 – } $\lambda > 0$  
נסמן $\lambda = \omega^2$, $\omega>0$.  
המשוואה תקבל צורה:
\[
y'' + \omega^2 y = 0,
\]
ולכן:
\[
y(x) = C_1 \cos(\omega x) + C_2 \sin(\omega x).
\]
נציב את תנאי השפה:
\[
\begin{cases}
y(0) = 0 \Rightarrow C_1 = 0, \\[4pt]
y(\pi) = C_2 \sin(\omega \pi) = 0.
\end{cases}
\]
כדי לקבל פתרון לא טריוויאלי נדרוש:
\[
\sin(\omega \pi) = 0 \quad \Longrightarrow \quad \omega \pi = n\pi \quad \Longrightarrow \quad \omega = n, \quad n=1,2,3,\ldots
\]
ולכן:
\[
\boxed{\lambda_n = n^2, \qquad n=1,2,3,\ldots}
\]
והפונקציות העצמיות:
\[
\boxed{y_n(x) = C_n \sin(nx)}.
\]

כדי לנרמל את הפונקציות העצמיות נדרוש:
\[
\int_0^{\pi} y_n^2(x)\,dx = 1.
\]
נציב ונשתמש בתוצאה ידועה שכבר קיבלנו בדוגמאות:
\[
\int_0^{\pi} k_n^2 \sin^2(nx)\,dx 
= k_n^2 \frac{\pi}{2} = 1 
\quad \Longrightarrow \quad 
k_n = \sqrt{\frac{2}{\pi}}.
\]

ולכן הפונקציות העצמיות המנורמלות הן:
\[
\boxed{\varphi_n(x) = \sqrt{\frac{2}{\pi}} \sin(nx)}.
\]

ממשוואה (\ref{move_base}) – נרצה לרשום טור פונקציות מן הצורה:
\[
\varphi(x) = \sum_n C_n \varphi_n(x),
\]
כאשר $C_n = \langle \varphi, \varphi_n \rangle_r$ (אם פונקציית המשקל $r=1$). הסימון $\langle \rangle_r$ למעשה מבטא מכפלה פנימית בין שתי פונקציות, הנלקחת על פני משקל בתוך אינטגרל, על פני אינטרוול שמתאים לבעיה.

ניזכר לרגע במה שלמדנו בקורס חדו׳׳א/אינפי 1.
אם הפונקציה $\varphi$ רציפה וגזירה למקוטעין ומקיימת את תנאי השפה, אז מובטחת התכנסות במידה שווה של הטור. אם לא – ייתכן שכן במידה שווה וייתכן שלא, ולכן יש לבדוק כל מקרה לגופו.

שימו לב שבמקרה שלנו פונקציית המשקל שווה ל-1.
נחשב את המקדם ע׳׳י לקיחת המכפלה הפנימית בין הפונקציה אותה אנחנו רוצה לבטא במרחב הפונקציות העצמיות $\sin(nx)$, הלא היא $x$:
\[
C_n = \langle \varphi, \varphi_n \rangle_{r=1} = \int_0^{\pi} x \cdot \sqrt{\frac{2}{\pi}} \sin(nx) \, dx.
\]

נבצע אינטגרציה בחלקים:
\[
C_n = \sqrt{\frac{2}{\pi}} \cdot \frac{1}{n} \big[-x\cos(nx)\big]_0^{\pi} + \sqrt{\frac{2}{\pi}} \cdot \frac{1}{n}\int_0^{\pi} \cos(nx) \, dx.
\]

האיבר השני מתאפס, ולכן נקבל:
\[
C_n = \sqrt{\frac{2}{\pi}} \cdot \frac{1}{n} (-\pi \cos n\pi) = \sqrt{2\pi} \, \frac{(-1)^{n+1}}{n}.
\]

ולכן:
\[
\textcolor{blue}{x} \sim \sum_{n=1}^{\infty} C_n \varphi_n(x)
= \sum_{n=1}^{\infty} \frac{\sqrt{2\pi}}{n} (-1)^{n+1} \sqrt{\frac{2}{\pi}} \sin(nx)
= \textcolor{red}{2\sum_{n=1}^{\infty} \frac{(-1)^{n+1}}{n} \sin(nx)}.
\]

נדון מעט בתכונות הטור שקיבלנו.
הטור מתכנס נקודתית בכל נקודה פנימית בקטע שבה הפונקציה רציפה.
היות ש־$\varphi(x)=x$ רציפה ב־$(0,\pi)$, הטור מתכנס נקודתית לכל $x$ בקטע זה.

 האם ההתכנסות היא במידה שווה ב־$(0,\pi)$?

 לא! כי עבור $x=\pi$ \textcolor{red}{הטור} מתכנס ל־$0$ ולא ל־$\textcolor{blue}{\varphi(\pi)}=\pi$. כלומר פונקציית הסכום אינה רציפה, וזהו טור של פונקציות רציפות. לכן ההתכנסות איננה במידה שווה.

האם מדובר בטור לייבניץ?

זהו לא טור לייבניץ. הסדרה $\frac{\sin(nx)}{n}$ איננה סדרה חיובית ואיננה מונוטונית (ב־$n$, לכל $x$ קבוע).

%%%CUT%%%

\solution{}
נרצה לפתור את משוואת הדיפוזיה בעזרת שיטת \textbf{הפרדת משתנים} תוך שימוש בתורת שטורם–ליוביל. 
הנחת העבודה בשיטת הפרדת המשתנים המפורסמת היא כי הפתרון ניתן להצגה כמכפלה של שתי פונקציות — אחת שתלויה רק במרחב והשנייה רק בזמן:
\[
u(x,t) = X(x)T(t).
\]
אם אכן קיים פתרון כזה, הרי שנקבל אותו כאשר נציב במד׳׳ח.

נציב במשוואת החום תוך לקיחת הנגזרות בכל כיוון בהתאמה (כפי שלמדתם/עדיין לומדים בקורס חדו׳׳א/אינפי 2):
\[
u_t = \alpha^2 u_{xx} 
\quad\Longrightarrow\quad
X(x)T'(t) = \alpha^2 X''(x)T(t).
\]

נחלק ב־\(\alpha^2 X(x)T(t)\) (בהנחה שאף אחד מהם אינו אפס, שכן הפתרון הטריוואלי לא מעניין אותנו):
\[
\frac{1}{\alpha^2}\frac{T'(t)}{ T(t)} = \frac{X''(x)}{X(x)} = -\lambda.
\]
בדומה לעקרון שראינו כבר בתת-הפרק ׳׳משוואות מדויקות׳׳, גם כאן, כששתי פונקציות של משתנה ׳׳טהור׳׳ אחר שוות זו לזו, הדבר אומר בהכרח שכל אחת מהן שווה לקבוע. מטעמי נוחות (תכף תבינו למה), אנו קוראים לו $-\lambda$.
שימו לב שקיבלנו בסופו של דבר שתי משוואות דיפרנציאליות רגילות(!):

\[
\begin{cases}
T'(t) + \alpha^2 \lambda T(t) = 0, \\[4pt]
X''(x) + \lambda X(x) = 0.
\end{cases}
\]
שימו לב כי מדובר בסך הכל בשתי מד׳׳רים עם מקדמים קבועים, הומוגניות, אשר אנו יודעים לפתור. תורת שטורם-ליוביל נכנסת כאן בעקבות הנוכחות של $\lambda$.

\textbf{שלב 1 – פתרון בזמן:}

נפתור את המשוואה בזמן בלבד:
\[
T'(t) + \alpha^2 \lambda T(t) = 0.
\]

שימו לב שזו משוואה דיפרנציאלית \textbf{פרידה מסדר ראשון}, כלומר ניתן לבודד בה את התלות של $T$ ושל $t$ בשני אגפים נפרדים.
ניתן לפתור כמובן גם בדרך של גורם אינטגרציה או וריאציית הפרמטר.
נפעל לפי שלבי האלגוריתם לפתרון משוואות פרידות:

\begin{enumerate}
  \item \textbf{נזהה את הפונקציות:}  
  נוכל לכתוב את המשוואה בצורתה הפרידה:
  \[
  \frac{dT}{dt} = -\alpha^2 \lambda T.
  \]
  מכאן נקבל:
  \[
  \frac{1}{T}\,dT = -\alpha^2 \lambda\,dt.
  \]
  כלומר $f(t) = -\alpha^2\lambda$ ו־$g(T) = T$.

  \item \textbf{נבצע אינטגרציה בשני אגפים:}
  \[
  \int \frac{1}{T}\,dT = \int -\alpha^2 \lambda\,dt.
  \]
  לאחר אינטגרציה נקבל:
  \[
  \ln |T| = -\alpha^2 \lambda t + C.
  \]

  \item \textbf{נחפש את הביטוי המפורש ל־$T(t)$:}
  \[
  |T| = e^{C} e^{-\alpha^2 \lambda t}.
  \]
  נציב $C' = e^{C}$ (קבוע חדש), ונבטל את הערך המוחלט מתוך ההגדרה של $C'$, תוך כדי ׳׳בליעת׳׳ הפתרון הסינגולרי כפי שלמדנו:
  \[
  T(t) = C' e^{-\alpha^2 \lambda t}.
  \]
  מקובל לסמן את הקבוע $C'$ באות $C$:
  \[
  \boxed{T(t) = C e^{-\alpha^2 \lambda t}}.
  \]
\end{enumerate}

\textbf{שלב 2 – פתרון במרחב:}

כעת נפתור את המשוואה המרחבית:
\[
X''(x) + \lambda X(x) = 0.
\]

נרצה כעת להבין מהם \textbf{תנאי השפה} עבור הפונקציה \(X(x)\).  
נחזור לתנאי השפה של המשוואה המקורית עבור \(u(x,t)\):
\[
u(0,t) = 0, \qquad u(L,t) = 0.
\]

נשתמש בעובדה ש־\(u(x,t) = X(x)T(t)\).  
נחשב עבור כל קצה:

\[
\begin{cases}
u(0,t) = X(0)T(t) = 0, \\[4pt]
u(L,t) = X(L)T(t) = 0.
\end{cases}
\]

מכיוון שפתרון טריוויאלי $T(t)\equiv0$ אינו מעניין אותנו (הוא פשוט מאפס את כל הטמפרטורה בכל זמן),
נדרש ש-\(T(t)\neq0\).  
המשמעות היא כי \textbf{החייב להתאפס הוא $X(x)$ בקצוות} — כלומר:

\[
\boxed{X(0)=0, \qquad X(L)=0.}
\]
המשוואה כעת היא בדיוק בעיית ערך שפה מהצורה:
\[
\begin{cases}
X''(x) + \lambda X(x) = 0, \\[4pt]
X(0) = 0, \\[4pt]
X(L) = 0.
\end{cases}
\]
המשוואה השנייה היא משוואת שטרום–ליוביל מהצורה:
\[
(p(x)X')' + \lambda r(x)X = 0
\]
עם $p(x)=r(x)=1$.
כפי שכבר ראינו בבעיות קודמות זהות עם תנאי שפה הומוגניים, ננתח את שלושת התחומים האפשריים של $\lambda$:

\begin{enumerate}
  \item \textbf{מקרה 1 – } $\lambda = 0$  

  נציב במשוואה:
  \[
  X''(x) = 0.
  \]
  נבצע אינטגרציה פעמיים:
  \[
  X(x) = C_1x + C_2.
  \]
  נציב את תנאי השפה:
  \[
  X(0)=0 \Rightarrow C_2=0, \qquad
  X(L)=0 \Rightarrow C_1L=0 \Rightarrow C_1=0.
  \]
  נקבל כי $X(x)\equiv0$, כלומר \textbf{פתרון טריוויאלי בלבד}.  
  לפיכך, $\lambda=0$ איננו ערך עצמי.

  \vspace{8pt}
  \item \textbf{מקרה 2 – } $\lambda < 0$  

  נסמן $\lambda=-\mu^2$ כאשר $\mu>0$, ונקבל:
  \[
  X''(x) - \mu^2 X(x) = 0.
  \]
  הפתרון הכללי הוא:
  \[
  X(x) = C_1 e^{\mu x} + C_2 e^{-\mu x}.
  \]
  נציב את תנאי שפה:
  \[
  \begin{cases}
  X(0)=0 \Rightarrow C_1 + C_2 = 0 \;\Rightarrow\; C_2 = -C_1,\\[4pt]
  X(L)=0 \Rightarrow C_1(e^{\mu L}-e^{-\mu L}) = 0.
  \end{cases}
  \]
  מאחר ש-$e^{\mu L}\neq e^{-\mu L}$ לכל $\mu>0$, נקבל $C_1=0$ ולכן $C_2=0$.  
  שוב, הפתרון טריוויאלי בלבד. \textbf{אין ערכים עצמיים שליליים}.

  \vspace{8pt}
  \item \textbf{מקרה 3 – } $\lambda > 0$  

  נסמן $\lambda=\omega^2$, ונקבל:
  \[
  X''(x) + \omega^2 X(x) = 0.
  \]
  הפתרון הכללי הוא:
  \[
  X(x) = C_1\cos(\omega x) + C_2\sin(\omega x).
  \]
  נציב את תנאי השפה:
  \[
  X(0)=0 \Rightarrow C_1=0, \qquad
  X(L)=0 \Rightarrow C_2\sin(\omega L)=0.
  \]
  כדי שהפתרון לא יהיה טריוויאלי ($C_2\neq0$), חייב להתקיים:
  \[
  \sin(\omega L)=0 \quad\Longrightarrow\quad \omega L = n\pi, \quad n=1,2,3,\ldots
  \]
  ולכן:
  \[
  \boxed{\lambda_n = \left(\frac{n\pi}{L}\right)^2, 
  \qquad 
  X_n(x) = \sin\!\left(\frac{n\pi x}{L}\right)}.
  \]
\end{enumerate}

\textbf{שלב 3 – בניית הפתרון הכולל:}

מצאנו כי למשוואה המרחבית מתקבלות אינסוף פונקציות עצמיות אורתוגונליות:
\[
X_n(x) = \sin\!\left(\frac{n\pi x}{L}\right),
\qquad 
\lambda_n = \left(\frac{n\pi}{L}\right)^2.
\]

עבור כל ערך עצמי $\lambda_n$ תואמת פונקציה בזמן:
\[
T_n(t) = C_n e^{-\alpha^2 \lambda_n t}
       = C_n e^{-\alpha^2 \left(\frac{n\pi}{L}\right)^2 t}.
\]

נוכל כעת לשלב את שתי התרומות לפי עקרון הסופרפוזיציה, 
שהרי משוואת החום ליניארית, וכל סכום של פתרונות הוא פתרון גם כן.  
הפתרון הכללי של בעיית הערך ההתחלתי מתקבל כסכום אינסופי של פונקציות עצמיות:

\[
\boxed{
u(x,t) = \sum_{n=1}^{\infty} A_n 
e^{-\alpha^2 \left(\frac{n\pi}{L}\right)^2 t}
\sin\!\left(\frac{n\pi x}{L}\right)
}.
\]

\textbf{קביעת המקדמים $A_n$}

נשתמש בתנאי ההתחלה ונרצה לפתח אותו כטור המכילות את הפונקציות העצמיות של הבעיה:
\[
u(x,0) = f(x) = \sum_{n=1}^{\infty} A_n \sin\!\left(\frac{n\pi x}{L}\right).
\]
נזהה כאן כי מדובר בייצוג של $f(x)$ כטור פורייה סינוסי בתחום $[0,L]$.  
נשתמש בתכונת האורתוגונליות של הפונקציות $\sin\!\left(\frac{n\pi x}{L}\right)$:

\[
\int_0^L 
\sin\!\left(\frac{m\pi x}{L}\right)
\sin\!\left(\frac{n\pi x}{L}\right)
\,dx =
\begin{cases}
0, & m\neq n,\\[4pt]
\dfrac{L}{2}, & m=n.
\end{cases}
\]

נכפיל את המשוואה עבור $f(x)$ ב־$\sin\!\left(\frac{m\pi x}{L}\right)$ וניקח את האינטגרל מ־$0$ עד $L$:
\[
\int_0^L f(x)\sin\!\left(\frac{m\pi x}{L}\right)dx 
= \sum_{n=1}^{\infty} A_n 
\int_0^L 
\sin\!\left(\frac{m\pi x}{L}\right)
\sin\!\left(\frac{n\pi x}{L}\right)dx.
\]
כל האיברים באגף ימין נעלמים פרט לזה עם $n=m$, ולכן:
\[
\int_0^L f(x)\sin\!\left(\frac{m\pi x}{L}\right)dx
= A_m \frac{L}{2}.
\]
ולכן:
\[
\boxed{
A_n = \frac{2}{L} 
\int_0^L f(x)\sin\!\left(\frac{n\pi x}{L}\right)dx.
}
\]
 
הפתרון הסופי של בעיית הולכת החום במוט באורך \(L\) הוא:

\[
\boxed{
u(x,t) =
\sum_{n=1}^{\infty} 
\left[
\frac{2}{L}\int_0^L f(s)\sin\!\left(\frac{n\pi s}{L}\right)ds
\right]
e^{-\alpha^2 \left(\frac{n\pi}{L}\right)^2 t}
\sin\!\left(\frac{n\pi x}{L}\right)
}.
\]

נבחן כעת מקרה ספציפי שבו התנאי ההתחלתי הוא:
\[
f(x) = x.
\]

במקרה זה נחשב את המקדמים $A_n$ על פי הנוסחה הכללית:
\[
A_n = \frac{2}{L}\int_0^L x \sin\!\left(\frac{n\pi x}{L}\right)dx.
\]

נבצע אינטגרציה בחלקים:
\[
\begin{aligned}
A_n &= \frac{2}{L} 
\left[
-\frac{xL}{n\pi}\cos\!\left(\frac{n\pi x}{L}\right)
\Big|_0^L
+ \frac{L}{n\pi}\int_0^L \cos\!\left(\frac{n\pi x}{L}\right)dx
\right].
\end{aligned}
\]

האיבר השני יתאפס כי $\sin(n\pi)=0$, ונקבל:
\[
A_n = \frac{2}{L}
\left[
-\frac{L^2}{n\pi}\cos(n\pi) + 0
\right]
= \frac{2L}{n\pi}(-1)^{n+1}.
\]

לפיכך:
\[
\boxed{A_n = \frac{2L}{n\pi}(-1)^{n+1}.}
\]

נחזיר את התוצאה לביטוי הכללי של הפתרון:

\[
\boxed{
u(x,t) =
\sum_{n=1}^{\infty} 
\frac{2L}{n\pi}(-1)^{n+1}
e^{-\alpha^2 \left(\frac{n\pi}{L}\right)^2 t}
\sin\!\left(\frac{n\pi x}{L}\right)
}.
\]

\textbf{פרשנות פיזיקלית:}  
התפלגות הטמפרטורה ההתחלתית $f(x)=x$ מתארת מצב שבו הקצה הימני $(x=L)$ חם יותר מן הקצה השמאלי $(x=0)$.  
עם הזמן, הפרשי הטמפרטורה הולכים ומצטמצמים כתוצאה מהדיפוזיה, וכל רכיב של טור פורייה דועך באופן מעריכי בזמן לפי הגורם
\[
e^{-\alpha^2 \left(\frac{n\pi}{L}\right)^2 t}.
\]
המוד הראשון ($n=1$) הוא האיטי ביותר בדעיכה ולכן שולט בהתנהגות לזמנים גדולים.
נראה את הפתרון בצורה גרפית:

\begin{figure}[H]
    \centering
    \includegraphics[width=0.7\textwidth]{SL.png}
    \caption{%
    התפלגות הטמפרטורה $u(x,t)$ לאורך המוט עבור תנאי התחלה $f(x)=x$. נלקחו בחשבון 100 איברים בטור ו- $L=1$.  
    מוצגים מספר פרופילים בזמנים שונים $t$, 
    כאשר ניתן לראות כי עם הזמן הפתרון מתקרב בהדרגה למצב היציב $u(x,t)\equiv0$. 
    העקומות הצבעוניות מייצגות את הפתרונות עבור ערכי זמן שונים, 
    והקו המקווקו בצבע תכלת מתאר את הפתרון היציב.}
    \label{fig:heat_fx_x}
\end{figure}

%%%CUT%%%

\newpage
\section{מבחנים}
כל מבחן מורכב מ-5 שאלות אשר נוגעות בנושאים שונים בסילבוס קורס משוואות דיפרנציאליות רגילות. זמן מומלץ לפתרון כל מבחן הוא כ-2.5 שעות.
\subsection{מבחן 1}

\question[exam1_ex1]
התבוננו במשוואה הדיפרנציאלית הבאה:
\[
y' = \frac{y\sin y - \sqrt{3}\,y\cos y}{x^2 + y^2 + 2}, 
\qquad y(0)=\frac{1}{4}.
\]
הראו כי הפתרון $y(x)$ מקיים $0<y(x)<\frac{\pi}{3}$ בתחום הגדרתו.

\question[exam1_ex2]
מצאו פתרון של המשוואה
\[
x y y' - y^2 - x^2 e^{-y/x} = 0
\]
העובר דרך הנקודה \((e, e)\).
אין צורך לקבל את $y$ באופן מפורש.

\question[exam1_ex3]

נתונה המשוואה הדיפרנציאלית:
\[
y'' + 2x^3 y' - 3x^2 y = 0,
\]
עם תנאי ההתחלה:
\[
y(0)=1, \qquad y'(0)=1.
\]
ויהי
\[
y(x) = \sum_{n=0}^{\infty} a_n x^n
\]
פתרון של המשוואה.

(א) מצאו את חמשת המקדמים הראשונים של הטור.

(ב)
האם קיימים ערכים של תנאי ההתחלה עבורם הפתרון הוא פולינום (סופי) לא טריוויאלי? אם כן כתבו את הפולינום ואם לא נמקו מדוע לא.

\question[exam1_ex4]}
נתונה המשוואה הדיפרנציאלית הבאה עבור \(t \ge 0\):
\[
y'' - 4y' + 3y = f(t), 
\qquad
y(0)=y'(0)=0,
\]
כאשר:
\[
f(t) =
\begin{cases}
\cos t, & 2\pi \le t < 4\pi, \\[3pt]
0, & \text{אחרת}.
\end{cases}
\]

\begin{enumerate}[label=\textbf{(\alph*)}, leftmargin=2em]

\item[(א׳) \; ]
מצאו פתרון פרטי \(y_p(t)\) של המשוואה.

\item[(ב׳) \; ]
מצאו את הפתרון הכללי של המשוואה \(y(t)\).

\end{enumerate}

\question[exam1_ex5]
ידוע כי המשוואה הדיפרנציאלית הרגילה המתארת תנועה של מתנד הרמוני עם גורם מרסן ניתנת לתיאור ע״י:
\[
mx'' + \gamma x' + kx = 0,
\]
כאשר \(x\) הוא המיקום היחסי של הגוף, \(\gamma\) הוא מקדם החיכוך (המשפיע על עוצמת ההתנגדות לתנועה),
ו-\(k\) הוא הקבוע האלסטי של המתנד. המשתנה הבלתי תלוי בבעיה הוא הזמן $t$.

במערכת היחידות \(\text{mks}\) נתונים:
\[
m = 0.5, \qquad \gamma = 1, \qquad k = 1.
\]

\begin{enumerate}[label=\textbf{(\hebrew*)}, leftmargin=2em, itemsep=1.2em]

\item[(א׳)] רשמו את המשוואה הדיפרנציאלית לאחר הצבת המקדמים, וזהו את סוג המד''ר (סדר, לינאריות, מקדמים קבועים או לא, הומוגניות ונרמול).

\item[(ב׳)] קבלו פתרון כללי מפורש \(x = f(t)\) של המשוואה, והסבירו לאורך הדרך באיזו שיטה החלטתם לפתור ומדוע היא מתאימה לפתרון מד''ר זו. יש לנמק כל שלב ולרשום את תחום ההגדרה של הפתרון כולל פתרונות סינגולריים (באם יש כאלה).

\item[(ג׳)] קבלו פתרון פרטי מפורש \(x = f(t)\) עבור תנאי ההתחלה:
\[
x(0)=1, \qquad x'(0)=0.
\]

\item[(ד׳)] כעת נניח כי על הגוף פועל כוח חיצוני מחזורי, והמשוואה מקבלת את הצורה:
\[
mx'' + \gamma x' + kx = F(t),
\]
כאשר \(F(t) = F_0 \cos(\omega t)\),  
עבור \(F_0 = 5, \, \omega = 2\).  
קבלו את הפתרון הפרטי של המשוואה, עם אותם תנאי התחלה מקוריים.

\end{enumerate}

\newpage
\underline{פתרון מלא למבחן}

\answer{exam1_ex1}
נבחן את המשוואה:
\[
y' = f(x,y) = \frac{y\sin y - \sqrt{3}\,y\cos y}{x^2 + y^2 + 2}.
\]
כיוון שמדובר בפונקציה אלמנטרית שהמכנה שלה לעולם אינו מתאפס, נקבל כי
\[
f(x,y), \quad f_y(x,y)
\]
רציפות לכל $(x,y)\in\mathbb{R}^2$, ולכן מתקיימים תנאי \textbf{משפט הקיום והיחידות} בכל המישור.

\textbf{שלב 1 – חיפוש פתרונות קבועים}

נניח פתרון קבוע $y(x)\equiv c$. נציב במשוואה:
\[
0 = f(x,c) = \frac{c\sin c - \sqrt{3}\,c\cos c}{x^2 + c^2 + 2}.
\]
המכנה לעולם אינו אפס, ולכן נדרוש שיתקיים:
\[
c\big(\sin c - \sqrt{3}\cos c\big)=0.
\]
כלומר:
\[
c=0,\qquad\sin c = \sqrt{3}\cos c \quad\Longrightarrow\quad \tan c = \sqrt{3}.
\]
מכאן:
\[
c=0, \qquad c=\frac{\pi}{3}.
\]
אלו הם הפתרונות הקבועים של המשוואה.

\textbf{שלב 2 – ייחודיות הפתרון עם תנאי ההתחלה}

בהינתן $y(0)=\frac{1}{4}$, הפתרון עובר דרך נקודה שבין שני הפתרונות הקבועים $y\equiv0$ ו־$y\equiv\frac{\pi}{3}$.  
בהתאם ל\textbf{משפט קיום ויחידות}, שני פתרונות שונים לעולם לא יכולים להיחתך.  
לכן הפתרון $y(x)$ לא יכול לעבור את $y\equiv0$ ולא את $y\equiv\frac{\pi}{3}$.

\[
\boxed{
0 < y(x) < \frac{\pi}{3}, \qquad \forall x\in\mathbb{R}.
}
\]

\textbf{מסקנה:}
הפתרון היחיד העובר דרך $y(0)=\frac{1}{4}$ חסום בין שני הפתרונות הקבועים שהתקבלו.  
כל ניסיון לחצות את אחד מהם היה סותר את משפט הקיום והיחידות.  

%%%%%%%

\answer{exam1_ex2}
\textbf{שלב 1 – זיהוי סוג המשוואה}

נרשום את המשוואה:
\[
x y y' - y^2 - x^2 e^{-y/x} = 0.
\]

נבודד את הנגזרת $y'$:
\[
y' = \frac{y^2 + x^2 e^{-y/x}}{x y}.
\]

מדובר במשוואה \textbf{דיפרנציאלית מסדר ראשון},  
\textbf{לא לינארית}, אך נבחין כי כל האיברים יכולים להיכתב כביטויים של $\frac{y}{x}$.  
לכן מדובר ב־\textbf{משוואה מטיפוס הומוגני} מסדר ראשון.

נחלק מונה ומכנה ב-$x^{2}$:
\[
y' = \frac{\left(\frac{y}{x}\right)^{2} + e^{-y/x}}{\frac{y}{x}}.
\]
 נשתמש בהצבה:
\[
v = \frac{y}{x}.
\]

\textbf{שלב 2 – נבצע את ההצבה ונעבור למשתנה חדש}

מאחר ש־$y = v x$, מתקיים גם:
\[
y' = v'x + v.
\]

נציב במשוואה המנורמלת:
\[
v'x + v = \frac{v^2 + e^{-v}}{v} = v + \frac{e^{-v}}{v}.
\]

נחסר $v$ משני אגפי המשוואה ונקבל:
\[
v'x = \frac{e^{-v}}{v}\rightarrow v' = \frac{1}{x} \frac{e^{-v}}{v} 
.
\]
קיבלנו מד׳׳ר פרידה ב-$v(x)$, ללא פתרונות סינגולריים. נכתוב:
\[
\int \frac{1}{g(v)}\,dv = \int f(x)\,dx + C.
\]
 נבצע אינטגרציה בהתאם:
\[
\int v e^{v}\,dv = \int \frac{dx}{x}.
\]

נחשב את האיבר השמאלי לפי אינטגרציה בחלקים:
\[
(v - 1)e^{v} = \ln|x| + C.
\]

\textbf{שלב 3 – חזרה למשתנים המקוריים}

נחזיר את $v = \frac{y}{x}$:
\[
\left(\frac{y}{x} - 1\right)e^{y/x} = \ln|x| + C.
\]

בהינתן $(x_0, y_0) = (e, e)$:
\[
\left(\frac{e}{e} - 1\right)e^{e/e} = \ln|e| + C 
\quad \Longrightarrow \quad 0 = 1 + C 
\quad \Longrightarrow \quad C = -1.
\]

ולכן הפתרון הפרטי הוא:
\[
\boxed{
\left(\frac{y}{x} - 1\right)e^{y/x} = \ln|x| - 1,\qquad x\neq0.
}
\]


\answer{exam1_ex3}

(א)
נחשב את הנגזרות של $y(x)$:
\[
y'(x) = \sum_{n=1}^{\infty} n a_n x^{n-1}, 
\qquad
y''(x) = \sum_{n=2}^{\infty} n(n-1)a_n x^{n-2}.
\]

נציב במשוואה:
\[
\sum_{n=2}^{\infty} n(n-1)a_n x^{n-2}
+ 2x^3 \sum_{n=1}^{\infty} n a_n x^{n-1}
- 3x^2 \sum_{n=0}^{\infty} a_n x^n = 0.
\]

נפשט:
\[
\sum_{n=2}^{\infty} n(n-1)a_n x^{n-2}
+ 2\sum_{n=1}^{\infty} n a_n x^{n+2}
- 3\sum_{n=0}^{\infty} a_n x^{n+2} = 0.
\]
נכתוב את כל הסכומים כך שכל איבר יכיל את $x^{n+2}$.
באיבר הראשון נבצע הזחת אינדקסים:  
נגדיר $k = n - 2 \Rightarrow n = k + 2$, ונחזיר את הסימון $n$ במקום $k$:

\[
\sum_{n=2}^{\infty} n(n-1)a_n x^{n-2} 
= \sum_{n=-2}^{\infty} (n+4)(n+3)a_{n+4}x^{n+2}.
\]

כדי להתחיל את הסכום מ־$n=0$, נוציא את שני האיברים הראשונים ($n=-2,-1$):

\[
\sum_{n=-2}^{\infty} (n+4)(n+3)a_{n+4}x^{n+2}
= 2a_2 + 6a_3x + \sum_{n=0}^{\infty}(n+4)(n+3)a_{n+4}x^{n+2}.
\]

נציב זאת יחד עם שאר האיברים מהמשוואה:
\[
2a_2 + 6a_3x +
\sum_{n=0}^{\infty}\Big[(n+4)(n+3)a_{n+4} + 2n a_n - 3a_n\Big]x^{n+2} = 0.
\]

נפשט את האיבר בסוגריים:
\[
2a_2 + 6a_3x +
\sum_{n=0}^{\infty}\Big[(n+4)(n+3)a_{n+4} + (2n - 3)a_n\Big]x^{n+2} = 0.
\]

מהשוואת מקדמים נקבל:
\[
x^0 : \quad 2a_2 = 0 \quad \Longrightarrow \quad a_2 = 0,
\]
\[
x^1 : \quad 6a_3 = 0 \quad \Longrightarrow \quad a_3 = 0,
\]
ולחזקות הגבוהות יותר:
\[
x^{n+2}: \quad (n+4)(n+3)a_{n+4} + (2n - 3)a_n = 0.
\]

ולכן נוסחת הנסיגה היא:
\[
\boxed{
a_{n+4} = \frac{3 - 2n}{(n+4)(n+3)}\,a_n,\qquad n\geq0.
}
\]

תנאי ההתחלה שלנו הם: \(a_0=y(0)=1, \ a_1=y'(0)=1\).
נחשב את איברי הרקורסיה:
\[
\begin{aligned}
a_4 &= \frac{3 - 2(0)}{(4)(3)}a_0 = \frac{3}{12} = \tfrac{1}{4},\\[4pt]
a_5 &= \frac{3 - 2(1)}{(5)(4)}a_1 = \frac{1}{20},\\[4pt]
a_6 &= \frac{3 - 2(2)}{(6)(5)}a_2 = 0.
\end{aligned}
\]

ולכן חמשת המקדמים הראשונים הם:
\[
\boxed{
a_0 = 1, \quad a_1 = 1, \quad a_2 = 0, \quad a_3 = 0, \quad a_4 = \tfrac{1}{4}.
}
\]

(ב)
נבחן האם קיימים תנאי התחלה עבורם הפתרון הוא \textbf{פולינום סופי ולא טריוויאלי}.

נוסחת הנסיגה:
\[
a_{n+4} = \frac{3 - 2n}{(n+4)(n+3)}\,a_n.
\]

כדי שפתרון יהיה פולינום סופי, צריך להיות לפחות תנאי עצירה רקורסיבי, כלומר הסדרה צריכה לעצור מתישהו:
\[
a_{n+4} = 0 \quad \Longleftrightarrow \quad 3 - 2n = 0 \ \Rightarrow\ n = \tfrac{3}{2}.
\]

מכיוון ש־$n$ חייב להיות מספר שלם, לא ייתכן ערך כזה.  
ולכן אם $a_0$ או $a_1$ שונים מאפס, יתקבל טור אינסופי;  
ואם שניהם אפס — נקבל את הפתרון הטריוויאלי $y(x)\equiv0$.

\[
\boxed{\text{אין פתרון פולינומי סופי שאינו טריוויאלי.}}
\]

%%%CUT%%%

\answer{exam1_ex4}
(א׳)
\textbf{שלב 1 – נכתוב את } \( f(t) \) \textbf{בעזרת פונקציית, הביסייד ושימוש במחזורית של קוסינוס}
\[
f(t) = \cos t\,[u_{2\pi}(t) - u_{4\pi}(t)]
= u_{2\pi}(t)\cos(t) - u_{4\pi}(t)\cos(t)= u_{2\pi}(t)\cos(t-2\pi) - u_{4\pi}(t)\cos(t-4\pi).
\]

\textbf{שלב 2 – הפעלת התמרת לפלס על שני אגפי המשוואה}

נשתמש בהתמרת לפלס של הנגזרות לפי \textbf{שורה~3} בטבלה~\ref{lap_table},  
ובנוסף ניעזר בהתמרת לפלס של פונקציית הקוסינוס לפי \textbf{שורה~15}  
וכן בזהות ההזזה לפי \textbf{שורה~7}:

\[
L[\cos t] = \frac{s}{s^2 + 1}, 
\qquad 
L[u_a(t)f(t - a)] = e^{-as}F(s).
\]

נפעיל התמרת לפלס על שני אגפי המשוואה לאחר הצבת תנאי ההתחלה:
\[
s^2L[y] - 4sL[y] + 3L[y]
= e^{-2\pi s}\frac{s}{s^2 + 1} - e^{-4\pi s}\frac{s}{s^2 + 1}.
\]

נבודד את \(L[y]\):
\[
L[y]
= \frac{e^{-2\pi s} - e^{-4\pi s}}{(s^2 - 4s + 3)(s^2 + 1)}\,s.
\]

\textbf{שלב 3 – פירוק לשברים חלקיים}

נרצה לפרק את הביטוי:
\[
\frac{s}{(s^2 - 4s + 3)(s^2 + 1)}.
\]

נפשט תחילה את הגורמים:
\[
s^2 - 4s + 3 = (s - 1)(s - 3),
\]
ולכן נניח:
\[
\frac{s}{(s - 1)(s - 3)(s^2 + 1)} 
= \frac{A}{s - 1} + \frac{B}{s - 3} + \frac{Cs + D}{s^2 + 1}.
\]

נכפיל את שני אגפי המשוואה במכנה המשותף \((s - 1)(s - 3)(s^2 + 1)\):
\[
s = A(s - 3)(s^2 + 1) + B(s - 1)(s^2 + 1) + (Cs + D)(s - 1)(s - 3).
\]

נפתח ונקבץ לפי חזקות של \(s\):

\[
\begin{aligned}
A(s - 3)(s^2 + 1)
&= A(s^3 - 3s^2 + s - 3), \\[4pt]
B(s - 1)(s^2 + 1)
&= B(s^3 - s^2 + s - 1), \\[4pt]
(Cs + D)(s - 1)(s - 3)
&= (Cs + D)(s^2 - 4s + 3) \\[2pt]
&= C(s^3 - 4s^2 + 3s) + D(s^2 - 4s + 3) \\[2pt]
&= C s^3 - 4C s^2 + 3C s + D s^2 - 4D s + 3D.
\end{aligned}
\]

כעת נחבר את כל האיברים:

\[
\begin{aligned}
s
&= (A + B + C)s^3
+ (-3A - B - 4C + D)s^2
+ (A + B + 3C - 4D)s
+ (-3A - B + 3D).
\end{aligned}
\]

נשווה מקדמים של חזקות זהות של \(s\):

\[
\begin{cases}
s^3: & A + B + C = 0, \\[4pt]
s^2: & -3A - B - 4C + D = 0, \\[4pt]
s^1: & A + B + 3C - 4D = 1, \\[4pt]
s^0: & -3A - B + 3D = 0.
\end{cases}
\]

מן המשוואה הראשונה:
\[
B = -A - C.
\]

נציב בשאר המשוואות.

\underline{משוואת \(s^0\):}
\[
-3A - (-A - C) + 3D = 0 
\quad\Longrightarrow\quad
-2A + C + 3D = 0 
\quad\Longrightarrow\quad
C = 2A - 3D.
\]

\underline{משוואת \(s^2\):}
\[
-3A - (-A - C) - 4C + D = 0
\quad\Longrightarrow\quad
-3A + A + C - 4C + D = 0
\quad\Longrightarrow\quad
-2A - 3C + D = 0
\quad\Longrightarrow\quad
D = 2A + 3C.
\]

נציב כעת את \(C = 2A - 3D\) לתוך \(D = 2A + 3C\):
\[
D = 2A + 3(2A - 3D)
\quad\Longrightarrow\quad
D = 2A + 6A - 9D
\quad\Longrightarrow\quad
10D = 8A
\quad\Longrightarrow\quad
D = \tfrac{4}{5}A.
\]

נחזיר ל־\(C = 2A - 3D\):
\[
C = 2A - 3(\tfrac{4}{5}A) = 2A - \tfrac{12}{5}A = -\tfrac{2}{5}A.
\]

ולכן:
\[
B = -A - C = -A + \tfrac{2}{5}A = -\tfrac{3}{5}A.
\]

\underline{משוואת \(s^1\):}
\[
A + B + 3C - 4D = 1.
\]
נציב את \(B, C, D\):
\[
A - \tfrac{3}{5}A + 3(-\tfrac{2}{5}A) - 4(\tfrac{4}{5}A) = 1.
\]
נפשט:
\[
A\!\left(1 - \tfrac{3}{5} - \tfrac{6}{5} - \tfrac{16}{5}\right) = 1
\quad\Longrightarrow\quad
A\!\left(\tfrac{5 - 25}{5}\right) = 1
\quad\Longrightarrow\quad
A(-4) = 1
\quad\Longrightarrow\quad
A = -\tfrac{1}{4}.
\]

נחשב את השאר:

\[
\begin{aligned}
B &= -\tfrac{3}{5}A = -\tfrac{3}{5}\!\left(-\tfrac{1}{4}\right) = \tfrac{3}{20},\\[4pt]
C &= -\tfrac{2}{5}A = -\tfrac{2}{5}\!\left(-\tfrac{1}{4}\right) = \tfrac{1}{10},\\[4pt]
D &= \tfrac{4}{5}A = \tfrac{4}{5}\!\left(-\tfrac{1}{4}\right) = -\tfrac{1}{5}.
\end{aligned}
\]

נקבל:

\[
\frac{s}{(s - 1)(s - 3)(s^2 + 1)}
= -\frac{1}{4}\frac{1}{s - 1}
+ \frac{3}{20}\frac{1}{s - 3}
+ \frac{1}{10}\frac{s - 2}{s^2 + 1}.
\]

נפשט מעט:

\[
\boxed{
\frac{s}{(s - 1)(s - 3)(s^2 + 1)}
= -\frac{1}{4}\frac{1}{s - 1}
+ \frac{3}{20}\frac{1}{s - 3}
+ \frac{1}{10}\frac{s}{s^2 + 1}
- \frac{1}{5}\frac{1}{s^2 + 1}.
}
\]

ולכן:
\[
L[y]
= (e^{-2\pi s} - e^{-4\pi s})
\!\left(
-\frac{1}{4}\frac{1}{s - 1}
+ \frac{3}{20}\frac{1}{s - 3}
+ \frac{1}{10}\frac{s}{s^2 + 1}
- \frac{1}{5}\frac{1}{s^2 + 1}
\right).
\]

\textbf{שלב 4 – ביצוע ההתמרה ההפוכה}

ניעזר בהתמרות ההפוכות מתוך \textbf{שורות~11, 14, 15, 7}  
בטבלה~\ref{lap_table}:

\[
\begin{aligned}
L^{-1}\!\left[\frac{1}{s - a}\right] &= e^{at} &(\text{שורה 11}) \\[4pt]
L^{-1}\!\left[\frac{s}{s^2 + 1}\right] &= \cos t &(\text{שורה 15}) \\[4pt]
L^{-1}\!\left[\frac{1}{s^2 + 1}\right] &= \sin t &(\text{שורה 14}) \\[4pt]
L^{-1}\!\left[e^{-as}F(s)\right] &= u_a(t)f(t - a) &(\text{שורה 7})
\end{aligned}
\]

ניישם את ההתמרה ההפוכה לכל איבר ונקבל:

\[
\begin{aligned}
y_p(t)
&= u_{2\pi}(t)\!\left[
-\tfrac{1}{4}e^{(t - 2\pi)}
+ \tfrac{3}{20}e^{3(t - 2\pi)}
+ \tfrac{1}{10}\cos(t - 2\pi)
- \tfrac{1}{5}\sin(t - 2\pi)
\right] \\[6pt]
&\quad - u_{4\pi}(t)\!\left[
-\tfrac{1}{4}e^{(t - 4\pi)}
+ \tfrac{3}{20}e^{3(t - 4\pi)}
+ \tfrac{1}{10}\cos(t - 4\pi)
- \tfrac{1}{5}\sin(t - 4\pi)
\right].
\end{aligned}
\]

\textbf{שלב 5 – הפתרון הפרטי} 

\[
\boxed{
\begin{aligned}
y_p(t)
&= -\tfrac{1}{4}u_{2\pi}(t)e^{(t - 2\pi)}
+ \tfrac{3}{20}u_{2\pi}(t)e^{3(t - 2\pi)}
+ \tfrac{1}{10}u_{2\pi}(t)\cos(t - 2\pi)
- \tfrac{1}{5}u_{2\pi}(t)\sin(t - 2\pi) \\[3pt]
&\quad + \tfrac{1}{4}u_{4\pi}(t)e^{(t - 4\pi)}
- \tfrac{3}{20}u_{4\pi}(t)e^{3(t - 4\pi)}
- \tfrac{1}{10}u_{4\pi}(t)\cos(t - 4\pi)
+ \tfrac{1}{5}u_{4\pi}(t)\sin(t - 4\pi).
\end{aligned}
}
\]

נכתוב את המשוואה ההומוגנית המתאימה:
\[
y'' - 4y' + 3y = 0.
\]

נרשום את המשוואה האופיינית, שכן מדובר במד׳׳ר עם מקדמים קבועים:
\[
L(r) = r^2 - 4r + 3 = 0.
\]

נפתור ונקבל:
\[
(r - 1)(r - 3) = 0 
\quad \Longrightarrow \quad 
r_1 = 1, \quad r_2 = 3.
\]

מאחר שמדובר בשני שורשים ממשיים שונים,  
הפתרון ההומוגני יהיה סכום של שני איברים מעריכיים בלתי תלויים:
\[
y_h(t) = C_1 e^{t} + C_2 e^{3t}.
\]
נזכור כי פתרון כללי למד׳׳ר לינארית הוא סופרפוזיציה של החלק ההומוגני עם החלק הפרטי (שמצאנו בסעיף א׳).
לכן הפתרון הכללי של המשוואה הוא:
\[
\boxed{y(t) = C_1 e^{t} + C_2 e^{3t} + y_p(t)},
\]
כאשר $y_{p}$ נתון ע׳׳י:
\[
\begin{aligned}
y_p(t)
&= -\tfrac{1}{4}u_{2\pi}(t)e^{(t - 2\pi)}
+ \tfrac{3}{20}u_{2\pi}(t)e^{3(t - 2\pi)}
+ \tfrac{1}{10}u_{2\pi}(t)\cos(t - 2\pi)
- \tfrac{1}{5}u_{2\pi}(t)\sin(t - 2\pi) \\[3pt]
&\quad + \tfrac{1}{4}u_{4\pi}(t)e^{(t - 4\pi)}
- \tfrac{3}{20}u_{4\pi}(t)e^{3(t - 4\pi)}
- \tfrac{1}{10}u_{4\pi}(t)\cos(t - 4\pi)
+ \tfrac{1}{5}u_{4\pi}(t)\sin(t - 4\pi).
\end{aligned}
\]



\answer{exam1_ex5}

(א׳) נציב את הנתונים \(m = 0.5,\; \gamma = 1,\; k = 1\):
\[
\boxed{
0.5x'' + x' + x = 0}.
\]

מדובר במשוואה דיפרנציאלית מסדר שני, ליניארית, עם מקדמים קבועים, הומוגנית, מנורמלת.

\vspace{0.5cm}
(ב׳)
נחלק ב־0.5 ונקבל:
\[
x'' + 2x' + 2x = 0.
\]

 נפתור בעזרת שיטת השורשים האופייניים — מתאימה כיוון שמדובר במד״ר ליניארית מסדר שני עם מקדמים קבועים.

נכתוב את המשוואה האופיינית:
\[
\lambda^2 + 2\lambda + 2 = 0.
\]
נפתור ונקבל:
\[
\lambda_{1,2} = -1 \pm i.
\]
הפתרון הוא:
\[
\boxed{
x(t) = e^{-t}\big(A\cos t + B\sin t\big)}.
\]
תחום ההגדרה של הפתרון: \(t \ge 0\).
אין פתרונות סינגולריים.

\vspace{0.5cm}
(ג׳) 
נגזור ונציב תנאי התחלה:
\[
\begin{aligned}
x(t) &= e^{-t}\big(A\cos t + B\sin t\big), \\[4pt]
x'(t) &= -e^{-t}\big(A\cos t + B\sin t\big)
       + e^{-t}\big(-A\sin t + B\cos t\big) \\[4pt]
      &= e^{-t}\big[(B - A)\cos t + (-A - B)\sin t\big].
\end{aligned}
\]

נציב \(t=0\):
\[
x(0) = A = 1, \qquad x'(0) = (B - A) = 0.
\]
מכאן נקבל:
\[
B = A = 1.
\]

ולכן:
\[
\boxed{x(t) = e^{-t}\big(\cos t + \sin t\big),\qquad t\geq0}.
\]

\vspace{0.5cm}
(ד׳) נניח כעת כי על הגוף פועל כוח חיצוני מחזורי:
\[
mx'' + \gamma x' + kx = F(t),
\qquad
F(t) = F_0\cos(\omega t),
\]
כאשר \(F_0 = 5, \, \omega = 2.\)

נציב את הנתונים:
\[
0.5x'' + x' + x = 5\cos(2t),
\]
ונחלק שוב ב־0.5:
\[
x'' + 2x' + 2x = 10\cos(2t).
\]
את הפתרון ההומוגני כבר יש לנו.
נניח פתרון פרטי מהצורה:

אגף ימין מתאים לצורה הכללית:
\[
G(y) = P_m(x)e^{ax}\cos bx + Q_m(x)e^{ax}\sin bx.
\]
במקרה שלנו:
\[
a = 0, \qquad b = 2, \qquad m = 0, \qquad k = 0.
\]
לכן נבחר פתרון פרטי מהצורה:
\[
x_p(t) = C\cos(2t) + D\sin(2t).
\]

נחשב נגזרות:
\[
x_p' = -2C\sin(2t) + 2D\cos(2t),
\qquad
x_p'' = -4C\cos(2t) - 4D\sin(2t).
\]

נציב במשוואה:
\[
(-4C\cos(2t) - 4D\sin(2t))
+ 2(-2C\sin(2t) + 2D\cos(2t))
+ 2(C\cos(2t) + D\sin(2t))
= 10\cos(2t).
\]

נאגד איברים דומים:
\[
(-4C + 4D + 2C)\cos(2t) + (-4D - 4C + 2D)\sin(2t) = 10\cos(2t).
\]

נפשט:
\[
(-2C + 4D)\cos(2t) + (-2D - 4C)\sin(2t) = 10\cos(2t).
\]

נשווה מקדמים:
\[
\begin{cases}
-2C + 4D = 10, \\[3pt]
-2D - 4C = 0.
\end{cases}
\]

מן המשוואה השנייה \(D = -2C.\)  
נציב בראשונה:
\[
-2C + 4(-2C) = 10 \;\Rightarrow\; -10C = 10 \;\Rightarrow\; C = -1.
\]
ולכן \(D = 2\)
.

 הפתרון הפרטי אם כך הוא:
\[
x_p(t) = -\cos(2t) + 2\sin(2t).
\]

ולכן הפתרון הפרטי של הבעיה כולה הוא הסכום של הפתרון ההומוגני והפתרון הפרטי:
\[
x(t) = x_h(t) + x_p(t)
= e^{-t}(A\cos t + B\sin t) - \cos(2t) + 2\sin(2t).
\]

נשתמש כעת בתנאי ההתחלה כדי למצוא את \(A,B\):
\[
x(0)=1, \qquad x'(0)=0.
\]

נחשב תחילה את הנגזרת:
\[
\begin{aligned}
x'(t)
&= \frac{d}{dt}\Big[e^{-t}(A\cos t + B\sin t)\Big]
    - \frac{d}{dt}\big[\cos(2t)\big]
    + \frac{d}{dt}\big[2\sin(2t)\big] \\[4pt]
&= e^{-t}\big[(B - A)\cos t + (-A - B)\sin t\big]
   + 2\sin(2t) + 4\cos(2t).
\end{aligned}
\]

נציב את תנאי ההתחלה ב- \(t=0\):

\[
\begin{cases}
x(0) = e^{0}(A\cos 0 + B\sin 0) - \cos(0) + 2\sin(0) = A - 1 = 1, \\[6pt]
x'(0) = e^{0}\big[(B - A)\cos 0 + (-A - B)\sin 0\big] + 2\sin(0) + 4\cos(0)
       = (B - A) + 4 = 0.
\end{cases}
\]

נפתור את המערכת:

\[
A - 1 = 1 \;\Rightarrow\; A = 2,
\qquad
B - A + 4 = 0 \;\Rightarrow\; B = A - 4 = -2.
\]

מכאן כי קיבלנו את הפתרון הפרטי:
\[
\boxed{
x(t) = e^{-t}(2\cos t - 2\sin t) - \cos(2t) + 2\sin(2t),\qquad t\geq0.
}
\]

%%%CUT%%%

\newpage
\subsection{מבחן 2}

\question[exam2_ex1]
נתונה המשוואה הדיפרנציאלית:
\[
2x^2 y'' + 2x y' + 2y = \frac{\ln(x)}{x},\qquad x>0.
\]

(א)
זהו את סוג המשוואה (סדר, לינאריות, מקדמים קבועים או לא, הומוגניות ונרמול).

(ב)
	קבלו פתרון כללי מפורש לבעיה ($y=f(x)$). הסבירו לאורך הדרך באיזו שיטה החלטתם לפתור ומדוע היא מתאימה לפתרון מד''ר זו. יש לנמק כל שלב ולרשום את תחום ההגדרה של הפתרון.

\question[exam2_ex2]
נתונה המשוואה:
\[
(x - 1)y'' - x y' + y = (x - 1)^2.
\]
מצאו את הפתרון הכללי של המשוואה, בהינתן כי \(x, e^x\) הם פתרונות של המשוואה ההומוגנית המתאימה.  
מצאו פתרון כללי מפורש \(y = f(x)\) של משוואה זו, והסבירו לאורך הדרך באיזו שיטה החלטתם לפתור את המשוואה ומדוע היא מתאימה לפתרון המד״ר הנתונה.  
יש לנמק כל שלב ולרשום את תחום ההגדרה של הפתרון כולל פתרונות סינגולריים (באם יש כאלה).

\question[exam2_ex3]
נתונה המשוואה:
\[
y' - \tfrac{3}{2} = \frac{(3x - 2y)^2 + 1}{3x - 2y}.
\]
מצאו פתרון כללי מפורש \(y = f(x)\) של המשוואה,  
והסבירו לאורך הדרך באיזו שיטה החלטתם לפתור ומדוע היא מתאימה למד״ר הנתונה.  
יש לנמק כל שלב ולרשום את תחום ההגדרה של הפתרון כולל פתרונות סינגולריים (אם קיימים).

\question[exam2_ex4]
ידוע כי המשוואה הדיפרנציאלית הרגילה המתארת את הזרם החשמלי במעגל RLC עם מתח קבוע ניתנת לתיאור ע״י:
\[
L I''(t) + R I'(t) + C^{-1} I(t) = 0,
\]
כאשר \(I\) הוא הזרם במעגל, \(t\) הוא הזמן, ו־\(L, R, C\) הם קבועים המתארים את ההשראה המגנטית של הסליל, ההתנגדות החשמלית של הנגד ויכולת הקבל לאגור אנרגיה חשמלית (קיבול), בהתאמה.

\begin{hebrewenum}
\item[(א)]
הציבו \(L = 1,\; R = 1,\; C^{-1} = 0.5\) (כל היחידות ב־mks),  
וזהו את סוג המד״ר (סדר, לינאריות, מקדמים קבועים או לא, הומוגניות ונרמול).

\item[(ב)]
קבלו פתרון כללי מפורש \(I = f(t)\) לבעיה.  
הסבירו באיזו שיטה החלטתם להשתמש ומדוע היא מתאימה לפתרון מד״ר זו.  
יש לנמק כל שלב ולציין את תחום ההגדרה של הפתרון כולל פתרונות סינגולריים (אם קיימים).

\item[(ג)]
מצאו פתרון פרטי מפורש \(I = f(t)\) (הזרם החשמלי במעגל כתלות בזמן)  עבור תנאי ההתחלה:
\[
I(0)=0, \qquad I'(0)=1.
\]

\item[(ד)]
בהינתן מקור מתח חיצוני סינוסואידלי, המשוואה מקבלת את הצורה:
\[
L I''(t) + R I'(t) + C^{-1} I(t)
= \frac{C^{-1}}{R} V_0 \sin(\omega t),
\]
כאשר $ V_0$ הוא המתח המקסימלי של מקור מתח זה ו $\omega$ מתאר את תדירות מקור המתח. 
עבור:

$L=1, C^{-1}=0.5, R=1, V_0=1, \omega=1$,
מצאו פתרון פרטי לבעיה.  
\end{hebrewenum}

\question[exam2_ex5]
נתונה המשוואה:
\[
x^2 = y^2 - 2xy\,y'.
\]

\begin{hebrewenum}

\item[א]
מצאו פתרון כללי מפורש \(y=f(x)\) בשתי דרכים שונות. הסבירו למה כל שיטה מתאימה לפתרון המשוואה.

\item[ב]
בהינתן תנאי ההתחלה \(y(1)=1\), מצאו פתרון פרטי מפורש.
\end{hebrewenum}

\newpage
\underline{פתרון מלא למבחן}

\answer{exam2_ex1}

נשים לב שהמשוואה מתאימה לצורת \textbf{משוואת אוילר}:
\[
a_2 x^2 y'' + a_1 x y' + a_0 y = b(x),
\]
כאשר:
\[
a_2 = 2, \quad a_1 = 2, \quad a_0 = 2, \quad b(x) = \frac{\ln(x)}{x}.
\]

זוהי משוואה \textbf{ליניארית לא הומוגנית מסדר שני עם מקדמים לא קבועים}.
היא מתאימה לצורת משוואת אוילר ואינה מנורמלת.

\vspace{0.4cm}
(ב׳)
קבלו פתרון כללי מפורש לבעיה \(y = f(x)\).  
נשתמש בשיטת אוילר באמצעות ההצבה:
\[
x = e^{t}\rightarrow t=\ln(x), \qquad Y(t) = y(e^{t}).
\]

נחשב את הנגזרות:
\[
\frac{dy}{dx} = \frac{1}{x}\frac{dY}{dt}, \qquad
\frac{d^2y}{dx^2} = \frac{1}{x^2}\left(\frac{d^2Y}{dt^2} - \frac{dY}{dt}\right).
\]

נציב במשוואה המקורית:
\[
2x^2\left(\frac{Y'' - Y'}{x^2}\right) + 2x\left(\frac{Y'}{x}\right) + 2Y = \frac{\ln(x)}{x}.
\]
לאחר פישוט:
\[
2Y'' + (2 - 2)Y' + 2Y = \frac{t}{e^{t}}.
\]

נפשט:
\[
2Y'' + 2Y = t e^{-t}.
\]

\vspace{0.3cm}
נפתור תחילה את \textbf{המשוואה ההומוגנית} המתאימה:
\[
2Y'' + 2Y = 0
\quad\Longrightarrow\quad
Y'' + Y = 0.
\]

המשוואה האופיינית:
\[
\lambda^2 + 1 = 0
\quad\Longrightarrow\quad
\lambda = \pm i.
\]

ולכן הפתרון הכללי של ההומוגנית הוא:
\[
Y_h(t) = A\cos t + B\sin t.
\]

\vspace{0.3cm}
כעת נמצא פתרון פרטי למשוואה הלא הומוגנית בעזרת \textbf{שיטת השוואת מקדמים}.

אגף ימין מתאים לצורה:
\[
G(y) = e^{a t} P_m(t),
\]
כאשר \(a = -1\), ו-\(P_m(t) = t\).  
לכן נניח פתרון פרטי מהצורה:
\[
Y_p(t) = (C t + D)e^{-t}.
\]

נחשב נגזרות:
\[
\begin{aligned}
Y_p'(t) &= C e^{-t} - (C t + D)e^{-t}
       = (-C t + C - D)e^{-t}, \\[4pt]
Y_p''(t) &= -C e^{-t} - (-C t + C - D)e^{-t}
       = (C t - 2C + D)e^{-t}.
\end{aligned}
\]

נציב במשוואה \(2Y'' + 2Y = t e^{-t}\):
\[
2(C t - 2C + D)e^{-t} + 2(C t + D)e^{-t} = t e^{-t}.
\]

נחלק ב־\(e^{-t}\) ונפשט:
\[
4C t - 4C + 4D = t.
\]

נשווה מקדמים:
\[
\begin{cases}
4C = 1, \\[3pt]
-4C + 4D = 0.
\end{cases}
\]

נפתור:
\[
C = \tfrac{1}{4}, \qquad D = \tfrac{1}{4}.
\]

ולכן:
\[
Y_p(t) = \tfrac{1}{4}(t + 1)e^{-t}.
\]

נחבר את הפתרון ההומוגני והפרטי:
\[
Y(t) = A\cos t + B\sin t + \tfrac{1}{4}(t + 1)e^{-t}.
\]

\vspace{0.3cm}
נחזור למשתנים המקוריים:
\[
\boxed{
y(x) = A\cos(\ln x) + B\sin(\ln x) + \tfrac{1}{4}\big(\ln x + 1\big)x^{-1},\qquad x>0.
}
\]



\answer{exam2_ex2}
מדובר במשוואה ליניארית לא־הומוגנית מסדר שני עם מקדמים לא קבועים, שאינה מנורמלת.

ננרמל את המשוואה ע״י חלוקה ב־\((x-1)\) (בהנחה ש-\(x\neq1\)):
\[
y'' - \frac{x}{x-1}y' + \frac{1}{x-1}y = x - 1, 
\qquad x \neq 1.
\]

מאחר שידועים שני פתרונות בלתי־תלויים של המשוואה ההומוגנית המתאימה  
(\(y_1 = x, \; y_2 = e^x\)), נשתמש בשיטת \textbf{וריאציית הפרמטרים} כדי למצוא פתרון פרטי.
נניח פתרון מהצורה:
\[
y = C(x)x + D(x)e^x.
\]

נדרוש שמתקיים:
\[
\begin{cases}
C'(x)x + D'(x)e^x = 0, \\[4pt]
C'(x) + D'(x)e^x = x - 1.
\end{cases}
\]

נחסר את המשוואות ונקבל:
\[
C'(x)(x - 1) = -(x - 1)
\quad \Longrightarrow \quad
C'(x) = -1
\quad \Longrightarrow \quad
C(x) = -x.
\]

נחשב את \(D'(x)\):
\[
D'(x)e^x = x - 1 - C'(x) = x - 1 + 1 = x
\quad \Longrightarrow \quad
D'(x) = x e^{-x}.
\]
נבצע אינטגרציה:
\[
D(x) = \int x e^{-x}\,dx = -(x + 1)e^{-x}.
\]

נחשב את הפתרון הפרטי:
\[
y_p = C(x)x + D(x)e^x = (-x)x + [-(x + 1)e^{-x}]e^x = -x^2 - x - 1.
\]

ולכן הפתרון הכללי של המשוואה הוא:
\[
\boxed{
y = A x + B e^x - x^2 - x - 1,\qquad x\in\mathbb{R}.
}
\]

תחום ההגדרה: על פניו עושה רושם כי \(x \in \mathbb{R} \setminus \{1\}\),  
אך ניתן לוודא שהפתרון תקף גם בנקודה \(x=1\) ע״י הצבה ישירה במשוואה המקורית. כלומר פעולת הנרמול שביצענו הכניסה ׳׳זמנית׳׳ אי ודאות לגבי $x=1$, אך המד׳׳ר המקורית והפתרון קיימים בנקודה זו, ועל כן תחום ההגדרה של הבעיה הוא כל $x$ ממשי. 
אין פתרונות סינגולריים.

%%%CUT%%%

\answer{exam2_ex3}
מדובר במשוואה לא־לינארית מסדר ראשון, מנורמלת. 
ניתן לזהות שזו משוואה עם \textbf{הצבה ליניארית} מן הצורה:
\[
y' = f(3x - 2y).
\] 
נבחין גם כי במשוואה חייב להתקיים \(y \neq \tfrac{3}{2}x\)
.

נפתור ע״י הצבה:
\[
t = 3x - 2y.
\]
נחשב נגזרת:
\[
t' = 3 - 2y'
\quad \Longrightarrow \quad
y' = \tfrac{3}{2} - \tfrac{t'}{2}.
\]

נציב במשוואה הנתונה:
\[
\tfrac{3}{2} - \tfrac{t'}{2} - \tfrac{3}{2} = \frac{t^2 + 1}{t},
\]
ונקבל:
\[
-\tfrac{t'}{2} = \frac{t^2 + 1}{t}\rightarrow t'=-2\frac{t^2 + 1}{t}, \qquad t \ne 0.
\]

זוהי משוואה פרידה אוטונומית (אין בה מופע של \(x\)). שימו לב שאין פתרונות סינגולריים. נכתוב את הפתרון בצורה ׳׳סתומה׳׳ ע׳׳י האינטגרלים:

\[
\int \frac{t}{t^2 + 1}\,dt = -2 \int dx,
\]
ונקבל:
\[
\tfrac{1}{2}\ln|t^2 + 1| = -2x + C
\quad \Longrightarrow \quad
\ln|t^2 + 1| = -4x + \tilde{C}.
\]

נעלה בחזקת \(e\) ונשחרר את הערך המוחלט שכן הארגומנט גדול מ-1:
\[
t^2 + 1 = \tilde{C} e^{-4x}, \qquad \tilde{C} > 0.
\]

נבודד את \(t\):
\[
t = \pm\sqrt{\tilde{C} e^{-4x} - 1}.
\]

נחזיר את ההצבה \(t = 3x - 2y\):
\[
3x - 2y = \pm\sqrt{\tilde{C} e^{-4x} - 1}.
\]

ולכן נקבל:
\[
\boxed{
y = \tfrac{3}{2}x \mp \tfrac{1}{2}\sqrt{\tilde{C} e^{-4x} - 1}, \qquad \tilde{C} > 0,\qquad x\in\mathbb{R}.
}
\]



\answer{exam2_ex4}
(א) 
נכתוב את המשוואה לאחר הצבת הקבועים:
\[
I''(t) + I'(t) + 0.5 I(t) = 0.
\]

מדובר במשוואה דיפרנציאלית \textbf{מסדר שני, ליניארית, עם מקדמים קבועים, הומוגנית, מנורמלת}.

(ב) 
נפתור בעזרת \textbf{שיטת הפולינום האופייני}:

\[
L(r) = r^2 + r + 0.5 = 0.
\]

נחשב את השורשים:
\[
r_{1,2} = \frac{-1 \pm \sqrt{1 - 4 \cdot 1 \cdot 0.5}}{2}
= -\frac{1}{2} \pm \frac{i}{2}.
\]

ולכן הפתרון הכללי:
\[
\boxed{
I(t) = e^{-\frac{t}{2}}\!\big[C_1 \cos\!\tfrac{t}{2} + C_2 \sin\!\tfrac{t}{2}\big], \quad t \ge 0.
}
\]

תחום ההגדרה הפיזיקלי: \(t \ge 0\).  
אין פתרונות סינגולריים, כיוון שהמשוואה ליניארית.

(ג) נציב את תנאי ההתחלה: \(I(0)=0, \; I'(0)=1\).

מן הפתרון הכללי:
\[
I(t) = e^{-\frac{t}{2}}\!\big[C_1 \cos\!\tfrac{t}{2} + C_2 \sin\!\tfrac{t}{2}\big].
\]

נחשב נגזרת:
\[
I'(t) = e^{-\frac{t}{2}}\!\left[-\tfrac{1}{2}(C_1 \cos\tfrac{t}{2} + C_2 \sin\tfrac{t}{2})
+ \tfrac{1}{2}(-C_1 \sin\tfrac{t}{2} + C_2 \cos\tfrac{t}{2})\right].
\]

נציב \(t=0\):
\[
I(0) = C_1 = 0, \qquad
I'(0) = \tfrac{1}{2}C_2 = 1 \;\Longrightarrow\; C_2 = 2.
\]

ולכן הפתרון הפרטי יהיה:
\[
\boxed{
I(t) = 2e^{-\frac{t}{2}}\sin\!\tfrac{t}{2}, \quad t \ge 0.
}
\]

(ד) נוסיף מקור מתח חיצוני סינוסואידלי:

\[
I''(t) + I'(t) + 0.5 I(t) = 0.5 \sin t.
\]

יש לנו כבר את הפתרון ההומוגני:
\[
I_H(t) = e^{-\frac{t}{2}}\!\big[C_1 \cos\!\tfrac{t}{2} + C_2 \sin\!\tfrac{t}{2}\big].
\]

אגף ימין של המשוואה הוא \(0.5\sin(t)\) ולכן מתאים בשיטת השוואת מקדמים לצורה:
\[
G(y) = P_m(x)e^{ax}\cos bx + Q_m(x)e^{ax}\sin bx,
\]

במקרה שלנו מתקיים:
\[
a = 0, \qquad b = 1, \qquad m = 0, \qquad k = 0.
\]
$k=0$ שכן $a+ib=i$ אינו פותר את הפ׳׳א.
 
לפיכך נניח פתרון פרטי מהצורה:
\[
I_p(t) = A\cos t + B\sin t.
\]

נחשב נגזרות:
\[
I_p'(t) = -A \sin t + B \cos t, 
\qquad
I_p''(t) = -A \cos t - B \sin t.
\]

נציב במשוואה:
\[
(-A \cos t - B \sin t) + (-A \sin t + B \cos t) + 0.5(A \cos t + B \sin t) = 0.5 \sin t.
\]

נאגד איברים:
\[
(-A + B + 0.5A)\cos t + (-B - A + 0.5B)\sin t = 0.5\sin t.
\]

נקבל מערכת:
\[
\begin{cases}
- A + B + 0.5A = 0, \\[4pt]
- B - A + 0.5B = 0.5.
\end{cases}
\]

נפתור:
\[
\begin{aligned}
&(-0.5A + B = 0) \;\Rightarrow\; B = 0.5A,\\[4pt]
&(-0.5B - A = 0.5) \;\Rightarrow\; A = -\tfrac{2}{5}, \; B = -\tfrac{1}{5}.
\end{aligned}
\]

ולכן:
\[
I_p(t) = -\tfrac{2}{5}\cos t - \tfrac{1}{5}\sin t.
\]

הפתרון הכללי יהיה:
\[
I(t) = I_{H}(t)+I_{p}(t) =  e^{-\frac{t}{2}}\!\big[C_1 \cos\!\tfrac{t}{2} + C_2 \sin\!\tfrac{t}{2}\big]
- \tfrac{2}{5}\cos t - \tfrac{1}{5}\sin t, \quad t \ge 0.
\]

נשתמש כעת באותם תנאי ההתחלה כדי לקבל פתרון פרטי מפורש:
\[
I(0) = 0, \qquad I'(0) = 1.
\]

נציב תחילה \(t=0\) בביטוי של \(I(t)\):

\[
I(0)
= e^{0}\!\big[C_1 \cos 0 + C_2 \sin 0\big]
- \tfrac{2}{5}\cos 0 - \tfrac{1}{5}\sin 0
= C_1 - \tfrac{2}{5} = 0.
\]
מכאן נקבל:
\[
C_1 = \tfrac{2}{5}.
\]

כעת נגזור את \(I(t)\):
\[
\begin{aligned}
I'(t)
&= \frac{d}{dt}\!\left[e^{-\frac{t}{2}}\!\big(C_1\cos\tfrac{t}{2} + C_2\sin\tfrac{t}{2}\big)\right]
- \frac{d}{dt}\!\big(\tfrac{2}{5}\cos t + \tfrac{1}{5}\sin t\big) \\[6pt]
&= e^{-\frac{t}{2}}\!\Big[
-\tfrac{1}{2}(C_1\cos\tfrac{t}{2} + C_2\sin\tfrac{t}{2})
+ \tfrac{1}{2}(-C_1\sin\tfrac{t}{2} + C_2\cos\tfrac{t}{2})
\Big]
+ \tfrac{2}{5}\sin t - \tfrac{1}{5}\cos t.
\end{aligned}
\]

נפשט מעט:
\[
I'(t)
= \tfrac{1}{2} e^{-\frac{t}{2}}\!\big[
(-C_1 - C_2)\sin\tfrac{t}{2} + (-C_1 + C_2)\cos\tfrac{t}{2}
\big]
+ \tfrac{2}{5}\sin t - \tfrac{1}{5}\cos t.
\]

נציב \(t=0\):
\[
I'(0)
= \tfrac{1}{2}\!\big[(-C_1 + C_2)\big] - \tfrac{1}{5} = 1.
\]
נפשט:
\[
-\tfrac{1}{2}C_1 + \tfrac{1}{2}C_2 - \tfrac{1}{5} = 1
\quad\Longrightarrow\quad
-C_1 + C_2 = \tfrac{12}{5}.
\]

נשתמש ב־\(C_1 = \tfrac{2}{5}\):
\[
C_2 = C_1 + \tfrac{12}{5} = \tfrac{14}{5}.
\]

נציב את הערכים חזרה בביטוי ונקבל את הפתרון הפרטי:
\[
\boxed{
I(t) = e^{-\frac{t}{2}}\!\left[\tfrac{2}{5}\cos\tfrac{t}{2} + \tfrac{14}{5}\sin\tfrac{t}{2}\right]
- \tfrac{2}{5}\cos t - \tfrac{1}{5}\sin t, \quad t \ge 0.
}
\]



\answer{exam2_ex5}

(א.)
ננרמל תחילה את המשוואה:
\[
y' = \frac{y^2 - x^2}{2xy}
= \frac{1}{2}\!\left(\frac{y}{x} - \frac{x}{y}\right),
\qquad x \ne 0.
\]

\textbf{שיטה 1 – זיהוי המשוואה כמשוואה מטיפוס הומוגני}  

נגדיר \(v = \tfrac{y}{x} \Rightarrow y = xv,\; y' = v + xv'\).  
נציב ונקבל:
\[
v + xv' = \tfrac{1}{2}\!\left(v - \tfrac{1}{v}\right)
\;\Rightarrow\;
xv' = -\tfrac{1}{2}\!\left(v + \tfrac{1}{v}\right)
\;\Rightarrow\;
v' = -\tfrac{1}{2x}\!\left(\tfrac{v^2 + 1}{v}\right).
\]

קיבלנו משוואה פרידה במישור \(x,v(x)\):
\[
\frac{v}{v^2 + 1}\,dv = -\frac{1}{2x}\,dx.
\]

נבצע אינטגרציה לשני האגפים:
\[
\int \frac{v}{v^2 + 1}\,dv = -\frac{1}{2}\int \frac{1}{x}\,dx
\;\Longrightarrow\;
\frac{1}{2}\ln|v^2 + 1| = -\frac{1}{2}\ln|x| + C.
\]

נכפיל ב־2 ונחזיר למישור \(x,y\) בעזרת \(v = \tfrac{y}{x}\):
\[
\ln\!\left|\left(\frac{y}{x}\right)^2 + 1\right|
= -\ln|x| + \tilde{C}.
\]

נמשיך לפתח:
\[
\left|\frac{y^2}{x^2} + 1\right|
= e^{\tilde{C}} \cdot \frac{1}{x}
\;\Rightarrow\;
\frac{y^2 + x^2}{x^2} = \frac{c_1}{x}
\;\Rightarrow\;
y^2 + x^2 = c_1 x,
\qquad (c_1 = e^{\tilde{C}},\; c_1 \ne 0).
\]

ולכן הפתרון הכללי בצורה מפורשת הוא:
\[
\boxed{y = \pm\sqrt{c_1x - x^2}}, \qquad x \in \mathbb{R},\; c_1 \ne 0.
\]

נחזיר למישור \(x,y\):
\[
\left(\frac{y}{x}\right)^2 + 1 = \frac{C_1}{x}
\;\Longrightarrow\;
y^2 = C_1x - x^2
\;\Longrightarrow\;
\boxed{y = \pm\sqrt{C_1x - x^2}}, \qquad x\in\mathbb{R},\; C_1\ne0.
\]

\textbf{שיטה 2 – משוואת ברנולי
}  
נרשום את המשוואה מחדש:
\[
y' - \frac{1}{2x}y = -\frac{x}{2y},
\qquad x \ne 0.
\]
נזהה משוואת ברנולי עם \(
\alpha=-1\) (אין פתרונות סינגולריים) ונבצע את ההצבה \(z(x)=y^2\).  
נגזור: \(z' = 2yy'\).  
נציב חזרה:
\[
\frac{1}{2}z' - \frac{1}{2x}z = -\frac{x}{2}
\;\Longrightarrow\;
z' - \frac{1}{x}z = -x.
\]
וקיבלנו מצופה מד׳׳ר לינארית ב-$z(x)$.
נפתור בעזרת גורם אינטגרציה:
\[
\mu(x)=e^{\int -\frac{1}{x}dx}=x^{-1}.
\]
נכפיל ונקבל:
\[
(zx^{-1})' = -1
\;\Longrightarrow\;
zx^{-1} = -x + C
\;\Longrightarrow\;
z = -x^2 + Cx.
\]
נחזור ל־\(y\) ע׳׳י הקשר $z=y^{2}$ ונקבל:
\[
\boxed{y = \pm\sqrt{Cx - x^2}}, \qquad x\in\mathbb{R}.
\]

(ב.)
\textbf{פתרון פרטי:}  
בהינתן \(y(1)=1\)

\[
1 = \sqrt{C(1) - 1^2} \;\Longrightarrow\; C-1=1 \;\Longrightarrow\; C=2.
\]
ולכן:
\[
\boxed{y = \sqrt{2x - x^2}}, \qquad x>0.
\]
שימו לב לנקודה מאוד חשובה! בחרנו בענף החיובי של הפתרון. למה? כיוון שתנאי ההתחלה נתון ברביע הראשון של מישור $x,y(x)$, בו ערך הפונקציה חיובי. על כן, בחרנו בענף החיובי של הפתרון.

%%%CUT%%%

\newpage
\subsection{מבחן 3}

\question[exam3_ex1]
נתונה המשוואה:
\[
(x - 1)y'' - x y' + y = 0,
\]
ויודע כי אחד הפתרונות של המשוואה הוא \(y_1 = e^x\).  
מצאו פתרון כללי מפורש (מהצורה \(y = f(x)\)) למשוואה זו. 

\question[exam3_ex2]
נתונה המשוואה:
\[
x(x+1)y'' - 2y' - 2y = 0.
\]

\begin{enumerate}

\item[א.] הראו כי הפונקציה \(y_1 = \tfrac{1}{x+1}\) היא אחד הפתרונות למשוואה.

\item[ב.] מצאו פתרון כללי למשוואה הדיפרנציאלית.  

\item[ג.] מצאו פתרון פרטי המקיים את תנאי ההתחלה:
\[
y(1)=0, \qquad y'(1)=\tfrac{3}{2}.
\]

\end{enumerate}

\question[exam3_ex3]
נתונה המערכת הדיפרנציאלית:
\[
\dot{\vec{X}}(t) = A\vec{X}(t),
\qquad
\vec{X}(t) =
\begin{pmatrix}
x_1(t) \\[4pt]
x_2(t) \\[4pt]
x_3(t)
\end{pmatrix},
\qquad
A =
\begin{pmatrix}
2 & 0 & 3 \\[4pt]
0 & 5 & 0 \\[4pt]
4 & 1 & 3
\end{pmatrix}.
\]

נתונים תנאי ההתחלה:
\[
\vec{X}(0) =
\begin{pmatrix}
3 \\[4pt]
0 \\[4pt]
4
\end{pmatrix}.
\]

מצאו את הערך:
\(
\boxed{x_3(1) - x_1(1)}
\)

\question[exam3_ex4]

קבלו פתרון כללי למשוואה הבאה:
\[
y^{4} - y'' = 3e^{\alpha x}, \qquad \alpha > 0.
\]

\question[exam3_ex5]
נתון כי הפונקציה $\mu(x,y)=x^{\alpha}y^{\beta}$ היא גורם האינטגרציה
של המשוואה הדיפרנציאלית:
\[
(2 + \frac{1}{y})\,dx + \left(\tfrac{x}{y} - \tfrac{1}{x}\right)dy = 0.
\]

\begin{hebrewenum}

\item[א.] מצאו את ערכי \(\alpha\) ו־\(\beta\).

\item[ב.] עבור אותם ערכי 
\(\alpha\) ו־\(\beta\)
, מצאו את הפתרון הכללי (ניתן להשאירו בצורה סתומה).
\end{hebrewenum}

\newpage
\underline{פתרון מלא למבחן}

\answer{exam3_ex1}

מדובר במשוואה דיפרנציאלית לינארית מסדר 2 עם מקדמים לא קבועים, הומוגנית, לא מנורמלת.  
ננרמל את המשוואה כדי להשתמש בנוסחת Abel למציאת הפתרון השני, בהינתן פתרון אחד ידוע.

נחלק ב־\((x-1)\):
\[
y'' - \frac{x}{x-1}y' + \frac{1}{x-1}y = 0,
\qquad x \ne 1.
\]

נוסחת Abel למציאת הפתרון השני היא:
\[
y_2(x)
= y_1(x)
  \int \frac{e^{-\int p(x)\,dx}}{y_1^2(x)}\,dx,
\]
כאשר \(p(x)\) הוא המקדם של \(y'\) במשוואה המנורמלת.  
כאן \(p(x) = -\tfrac{x}{x-1}\).

נחשב:

\[
\begin{aligned}
y_2(x)
&= e^x \int
   \frac{e^{\int x/(x-1)\,dx}}{e^{2x}}\,dx
 = e^x \int
   e^{\int \frac{x-1+1}{x-1}\,dx - 2x}\,dx \\[6pt]
&= e^x \int
   e^{\int (1 + \tfrac{1}{x-1})\,dx - 2x}\,dx
 = e^x \int
   e^{x + \ln|x-1| - 2x}\,dx
 = e^x \int
   \frac{|x-1|}{e^x}\,dx.
\end{aligned}
\]

נניח \(x>1\) ולכן \(|x-1| = x-1\):
\[
y_2(x)
= e^x \int \frac{x-1}{e^x}\,dx
= e^x \int e^{-x}(x-1)\,dx.
\]

נפתור לפי שיטת אינטגרציה בחלקים:  
נבחר \(u' = e^{-x},\; v = x-1\), ולכן \(u = -e^{-x},\; v' = 1.\)

\[
\int e^{-x}(x-1)\,dx\rightarrow \int u'v\,dx
= uv - \int uv'\,dx
= (-e^{-x})(x-1) - \int (-e^{-x})\,dx
= -e^{-x}(x-1) - e^{-x}.
\]

נציב בחזרה:
\[
y_2(x)
= e^x[-e^{-x}(x-1) - e^{-x}]
= e^x[-e^{-x}x + e^{-x} - e^{-x}]
= -x.
\]

לכן הפתרון הכללי הוא:
\[
\boxed{
y(x) = c_1 e^x + c_2 x, \qquad x>1.
}
\]

שימו לב:  
אם נניח \(x<1\), החישוב ייתן את אותה התוצאה מוכפלת במינוס 1, אך אנו יודעים שכל קבוע כפול פתרון הוא גם פתרון למד׳׳ר הומוגנית. בנוסף, $x=1$ הוא לא נקודת אי הגדרה, לא של המד׳׳ר המקורית ולא של הפתרון. האילוץ הזמני הזה נכנס בזמן שנרמלנו את המד׳׳ר כדי להשתמש בנוסחת אבל, אך הוא מתברר כ׳׳זמני׳׳. לכן הפתרון הכללי התקף לכל $x$ ממשי הוא:
\[
\boxed{
y(x) = c_1 e^x + c_2 x, \qquad x\in\mathbb{R}.
}
\]



\answer{exam3_ex2}

\textbf{(א)}  
נציב את \(y = \tfrac{1}{x+1}\) במשוואה ונחשב:

\[
y' = -\tfrac{1}{(x+1)^2}, \qquad
y'' = \tfrac{2}{(x+1)^3}.
\]

נציב במשוואה:
\[
x(x+1)\cdot \tfrac{2}{(x+1)^3}
- 2\left(-\tfrac{1}{(x+1)^2}\right)
- 2\left(\tfrac{1}{x+1}\right)
= \tfrac{2x}{(x+1)^2} + \tfrac{2}{(x+1)^2} - \tfrac{2(x+1)}{(x+1)^2} = 0.
\]

ולכן \(y_1 = \tfrac{1}{x+1}\) אכן פותרת את המשוואה ההומוגנית הנתונה.

\textbf{(ב)}  
כעת נמצא פתרון בלתי תלוי ליניארית בעזרת נוסחת אבל, שכן המד׳׳ר הומוגנית מסדר 2 ואנו יודעים פתרון אחד.  
ננרמל תחילה את המשוואה (תנאי הכרחי לשימוש באבל):

\[
y'' - \frac{2}{x(x+1)}y' - \frac{2}{x(x+1)}y = 0,
\qquad x \ne 0, -1.
\]

נוסחת אבל עבור הפתרון השני:
\[
y_2(x)
= y_1(x) \int \frac{e^{-\int p(x)\,dx}}{y_1^2(x)}\,dx,
\]
כאשר \(p(x) = -\tfrac{2}{x(x+1)}\).

נחשב את האיברים בנוסחה:

\[
\begin{aligned}
y_2(x)
&= \frac{1}{x+1} \int
  \frac{e^{\int 2/[x(x+1)]\,dx}}{(1/(x+1))^2}\,dx
= \frac{1}{x+1}\int (x+1)^2 e^{\int \frac{2}{x(x+1)}\,dx}\,dx.
\end{aligned}
\]

נחשב את האינטגרל הפנימי:
\[
\int \frac{2}{x(x+1)}\,dx
= 2\!\int\!\left(\frac{1}{x} - \frac{1}{x+1}\right)\!dx
= 2\ln\!\left|\frac{x}{x+1}\right|.
\]

נציב חזרה:
\[
y_2(x)
= \frac{1}{x+1}\int (x+1)^2 e^{2\ln|x/(x+1)|}\,dx
= \frac{1}{x+1}\int (x+1)^2 \left(\frac{x}{x+1}\right)^2 dx
= \frac{1}{x+1}\int x^2\,dx.
\]

נחשב:
\[
y_2(x) = \frac{1}{x+1} \cdot \frac{x^3}{3}
= \frac{x^3}{3(x+1)}.
\]

ולכן בסיס למרחב הפתרונות הוא \(\{y_1, y_2\}\),  
ומתקבל הפתרון הכללי:
\[
\boxed{
y(x) = c_1\frac{1}{x+1} + c_2\frac{x^3}{(x+1)},
\qquad x \ne -1.
}
\]
שימו לב לנקודה מאוד מעניינת. המד׳׳ר המקורית הייתה מוגדרת על כל הישר הממשי. במהלך הנרמול, קיבלנו אילוץ על שני איקסים ׳׳בעייתיים׳׳, אך רק אחד מהם ׳׳שרד׳׳, בזמן שהשני נשר במהלך הפתרון.

\textbf{(ג)}  
נמצא פתרון פרטי המקיים את תנאי ההתחלה:
\[
y(1)=0, \qquad y'(1)=\tfrac{3}{2}.
\]

מתוך הפתרון הכללי:
\[
y(x) = c_1\frac{1}{x+1} + c_2\frac{x^3}{x+1},
\]
נחשב נגזרת:
\[
\begin{aligned}
y'(x)
&= c_1\!\left(-\frac{1}{(x+1)^2}\right)
  + c_2\!\left(\frac{3x^2(x+1) - x^3}{(x+1)^2}\right) \\[4pt]
&= -\frac{c_1}{(x+1)^2}
  + c_2\,\frac{x^2(2x+3)}{(x+1)^2}.
\end{aligned}
\]

נחיל את תנאי ההתחלה \(x=1\):

\[
\begin{cases}
y(1) = \dfrac{c_1}{2} + \dfrac{c_2}{2} = 0, \\[6pt]
y'(1) = -\dfrac{c_1}{4} + \dfrac{5c_2}{4} = \dfrac{3}{2}.
\end{cases}
\]

מהמשוואה הראשונה:
\[
c_1 = -c_2.
\]

נציב בשנייה:
\[
-\frac{(-c_2)}{4} + \frac{5c_2}{4} = \frac{3}{2}
\;\Rightarrow\;
\frac{6c_2}{4} = \frac{3}{2}
\;\Rightarrow\;
c_2 = \frac{1}{2}, \qquad c_1 = -\frac{1}{2}.
\]

ולכן הפתרון הפרטי הוא:
\[
\boxed{
y(x) = \frac{1}{2}\left(\frac{x^3 - 1}{x+1}\right),
\qquad x \ne -1.
}
\]



\answer{exam3_ex3}

\textbf{שלב 1 – מציאת הערכים העצמיים של $A$}

נחשב את הפולינום האופייני:
\[
\det(A - \lambda I) =
\begin{vmatrix}
2-\lambda & 0 & 3\\[4pt]
0 & 5-\lambda & 0\\[4pt]
4 & 1 & 3-\lambda
\end{vmatrix}=(5-\lambda)(\lambda^2 - 5\lambda - 6) = 0.
\]

לכן:
\[
\boxed{\lambda_1 = 5, \quad \lambda_2 = 6, \quad \lambda_3 = -1.}
\]

\textbf{שלב 2 – מציאת וקטורים עצמיים}

\underline{עבור $\lambda_1 = 5$:}

\[
(A - 5I)\vec{v}_1 = 0
\quad\Rightarrow\quad
\begin{pmatrix}
-3 & 0 & 3\\
0 & 0 & 0\\
4 & 1 & -2
\end{pmatrix}
\begin{pmatrix}a\\b\\c\end{pmatrix}
=
\begin{pmatrix}0\\0\\0\end{pmatrix}.
\]

מהשורה הראשונה: \(-3a + 3c = 0 \Rightarrow a = c.\)  
מהשורה השלישית: \(4a + b - 2c = 0 \Rightarrow b = -2a.\)

נבחר:
\[
\vec{v}_1 =
\begin{pmatrix}
1\\[2pt]
-2\\[2pt]
1
\end{pmatrix}.
\]

\underline{עבור $\lambda_2 = 6$:}

\[
(A - 6I)\vec{v}_2 = 0
\quad\Rightarrow\quad
\begin{pmatrix}
-4 & 0 & 3\\
0 & -1 & 0\\
4 & 1 & -3
\end{pmatrix}
\begin{pmatrix}a\\b\\c\end{pmatrix}
=
\begin{pmatrix}0\\0\\0\end{pmatrix}.
\]

מן השורה השנייה: \(-b = 0 \Rightarrow b = 0.\)  
מן הראשונה: \(-4a + 3c = 0 \Rightarrow c = \tfrac{4}{3}a.\)  
נבדוק בשלישית: \(4a - 3c = 0\) — מתקיים אוטומטית.

נבחר:
\[
\vec{v}_2 =
\begin{pmatrix}
3\\[2pt]
0\\[2pt]
4
\end{pmatrix}.
\]

\underline{עבור $\lambda_3 = -1$:}

\[
(A + I)\vec{v}_3 = 0
\quad\Rightarrow\quad
\begin{pmatrix}
3 & 0 & 3\\
0 & 6 & 0\\
4 & 1 & 4
\end{pmatrix}
\begin{pmatrix}a\\b\\c\end{pmatrix}
=
\begin{pmatrix}0\\0\\0\end{pmatrix}.
\]

מהשורה השנייה: \(6b = 0 \Rightarrow b = 0.\)  
מן הראשונה: \(3a + 3c = 0 \Rightarrow c = -a.\)  
נבדוק בשלישית: \(4a + 4c = 0\) — מתקיים.

נבחר:
\[
\vec{v}_3 =
\begin{pmatrix}
1\\[2pt]
0\\[2pt]
-1
\end{pmatrix}.
\]

\textbf{שלב 3 – כתיבת הפתרון הכללי}

\[
\vec{X}(t)
= c_1 e^{5t}\vec{v}_1
+ c_2 e^{6t}\vec{v}_2
+ c_3 e^{-t}\vec{v}_3.
\]

נציב את הווקטורים העצמיים שמצאנו:
\[
\vec{X}(t)
= c_1 e^{5t}
\begin{pmatrix}
1\\[2pt]
-2\\[2pt]
1
\end{pmatrix}
+ c_2 e^{6t}
\begin{pmatrix}
3\\[2pt]
0\\[2pt]
4
\end{pmatrix}
+ c_3 e^{-t}
\begin{pmatrix}
1\\[2pt]
0\\[2pt]
-1
\end{pmatrix}.
\]

\textbf{שלב 4 – מציאת הקבועים לפי תנאי ההתחלה}

בהינתן:
\[
\vec{X}(0) =
\begin{pmatrix}
3\\[2pt]
0\\[2pt]
4
\end{pmatrix}.
\]

נרשום את מערכת המשוואות:
\[
\begin{cases}
c_1 + 3c_2 + c_3 = 3,\\[4pt]
-2c_1 + 0 + 0 = 0,\\[4pt]
c_1 + 4c_2 - c_3 = 4.
\end{cases}
\]

מהשורה השנייה: \(c_1 = 0\).
נציב בשתיים האחרות:
\[
\begin{cases}
3c_2 + c_3 = 3,\\[4pt]
4c_2 - c_3 = 4.
\end{cases}
\]

נחבר:
\[
7c_2 = 7 \;\Rightarrow\; c_2 = 1, \qquad c_3 = 0.
\]

נחזיר לקומבינציה הכללית:
\[
\vec{X}(t)
= e^{6t}
\begin{pmatrix}
3\\[2pt]
0\\[2pt]
4
\end{pmatrix}.
\]

ומכאן:
\[
x_1(t) = 3e^{6t}, \qquad
x_2(t) = 0, \qquad
x_3(t) = 4e^{6t}.
\]

\textbf{שלב 5 – מציאת הערך המבוקש}

\[
x_3(1) - x_1(1)
= 4e^{6} - 3e^{6}
= e^{6}.
\]

\[
\boxed{x_3(1) - x_1(1) = e^{6}}
\]

%%%CUT%%%

\answer{exam3_ex4}

\textbf{שלב 1 – זיהוי סוג המשוואה}

מדובר במשוואה לינארית לא הומוגנית מסדר רביעי, עם מקדמים קבועים.  
נפתור תחילה את החלק ההומוגני:
\[
y^{(4)} - y'' = 0.
\]

\textbf{שלב 2 – פתרון המשוואה ההומוגנית}

נכתוב את הפולינום האופייני:
\[
r^4 - r^2 = 0
\quad\Rightarrow\quad
r^2(r^2 - 1) = 0.
\]

ולכן:
\[
r_1 = 0, \quad r_2 = 0, \quad r_3 = 1, \quad r_4 = -1.
\]

מכאן הפתרון ההומוגני:
\[
y_h(x) = C_1 + C_2 x + C_3 e^{x} + C_4 e^{-x}.
\]

\textbf{שלב 3 – בחירת צורת הפתרון הפרטי}

אגף ימין הוא \(3e^{\alpha x}\), ולכן לפי שיטת השוואת מקדמים נשתמש בתבנית:
\[
G(y) = P_m(x)e^{ax}\cos bx + Q_m(x)e^{ax}\sin bx.
\]

נזהה את הפרמטרים:
\[
a = \alpha, \qquad b = 0, \qquad m = 0.
\]

כעת נבדוק האם \(a+ib = \alpha\) הוא שורש של הפולינום האופייני \(r^4 - r^2 = 0\).  
השורשים הם: \(r = 0,0,1,-1\), ולכן:
\[
k =
\begin{cases}
1, &  \alpha = 0 \text{ או } \alpha = \pm1,\\[4pt]
0, & \alpha \neq 0, \qquad \pm1.
\end{cases}
\]

נתון כי $\alpha>0$ ועל כן המקרים השונים נהיה פשוטים יותר וניתן לרשום אותם כך:
\[
\begin{cases}
 \alpha \neq 1: & k = 0, \\[4pt]
 \alpha = 1: & k = 1.
\end{cases}
\]
בהתאם לערכי \(k\), נניח פתרון פרטי מהצורה:
\[
\boxed{
\begin{cases}
y_p = A e^{\alpha x}, & \alpha \neq 1, \\[6pt]
y_p = A x e^{x}, & \alpha = 1.
\end{cases}
}
\]

\textbf{שלב 4 – חישוב הפתרון הפרטי בכל אחד מהמקרים}


\begin{cases}
\alpha \neq 1: &
\begin{aligned}[t]
&y_p'' = A\alpha^2 e^{\alpha x}, \qquad
y_p^{(4)} = A\alpha^4 e^{\alpha x}, \\[3pt]
&\Rightarrow A(\alpha^4 - \alpha^2)e^{\alpha x} = 3e^{\alpha x}
\Rightarrow
A = \dfrac{3}{\alpha^2(\alpha^2 - 1)},\\[6pt]
&\therefore\quad
y_p(x) = \dfrac{3}{\alpha^2(\alpha^2 - 1)} e^{\alpha x}.
\end{aligned}
\\[10pt]
\alpha = 1: &
\begin{aligned}[t]
&y_p = A x e^{x}, \quad
y_p'' = A e^{x}(x+2), \quad
y_p^{(4)} = A e^{x}(x+4),\\[3pt]
&\Rightarrow A e^{x}(x+4) - A e^{x}(x+2) = 3e^{x}
\Rightarrow 2A e^{x} = 3e^{x}
\Rightarrow A = \tfrac{3}{2},\\[6pt]
&\therefore\quad
y_p(x) = \tfrac{3}{2}x e^{x}.
\end{aligned}
\end{cases}
\]

\textbf{שלב 5 – הפתרון הכללי לפי המקרים השונים}

\[
\boxed{
y(x) =
\begin{cases}
C_1 + C_2 x + C_3 e^{x} + C_4 e^{-x}
+ \dfrac{3}{\alpha^2(\alpha^2 - 1)} e^{\alpha x},
& \alpha > 0, \; \alpha \neq 1, \\[10pt]
C_1 + C_2 x + C_3 e^{x} + C_4 e^{-x}
+ \tfrac{3}{2}x e^{x},
& \alpha = 1.
\end{cases}
}
\]



\answer{exam3_ex5}

נרצה למצוא ערכים של \(\alpha, \beta \in \mathbb{R}\) כך שהפונקציה 
\[
\mu(x,y) = x^{\alpha}y^{\beta}
\]
תהפוך את המשוואה
\[
(2 + \tfrac{1}{y})\,dx + \left(\tfrac{x}{y} - \tfrac{1}{x}\right)dy = 0
\]
למדויקת.

נסמן:
\[
P(x,y) = 2 + \tfrac{1}{y}, 
\qquad 
Q(x,y) = \tfrac{x}{y} - \tfrac{1}{x}.
\]

\textbf{שלב 1 – בדיקה כללית של תנאי הדיוק}

נכפיל את המשוואה בגורם האינטגרציה \(\mu(x,y)=x^{\alpha}y^{\beta}\), ונקבל:

\[
\mu P = x^{\alpha}y^{\beta}\!\left(2 + \tfrac{1}{y}\right)
= 2x^{\alpha}y^{\beta} + x^{\alpha}y^{\beta-1},
\]
\[
\mu Q = x^{\alpha}y^{\beta}\!\left(\tfrac{x}{y} - \tfrac{1}{x}\right)
= x^{\alpha+1}y^{\beta-1} - x^{\alpha-1}y^{\beta}.
\]

נדרוש את תנאי הדיוק:
\[
\frac{\partial(\mu P)}{\partial y}
= \frac{\partial(\mu Q)}{\partial x}.
\]

נחשב כל צד בנפרד.

\textbf{שלב 2 – גזירות חלקיות}

\[
\frac{\partial(\mu P)}{\partial y}
= \frac{\partial}{\partial y}\left(2x^{\alpha}y^{\beta} + x^{\alpha}y^{\beta-1}\right)
= 2\beta x^{\alpha}y^{\beta-1} + (\beta - 1)x^{\alpha}y^{\beta-2}.
\]

\[
\frac{\partial(\mu Q)}{\partial x}
= \frac{\partial}{\partial x}\left(x^{\alpha+1}y^{\beta-1} - x^{\alpha-1}y^{\beta}\right)
= (\alpha+1)x^{\alpha}y^{\beta-1} - (\alpha-1)x^{\alpha-2}y^{\beta}.
\]

נשווה:

\[
2\beta x^{\alpha}y^{\beta-1} + (\beta - 1)x^{\alpha}y^{\beta-2}
= (\alpha+1)x^{\alpha}y^{\beta-1} - (\alpha-1)x^{\alpha-2}y^{\beta}.
\]

נבחין כי ניתן להשוות בין איברים בעלי חזקות זהות של \(x\) ושל \(y\).  
באגף שמאל ובאגף ימין מופיעים שלושה סוגים של חזקות:  
\(x^{\alpha}y^{\beta-1}\), \(x^{\alpha}y^{\beta-2}\), ו־\(x^{\alpha-2}y^{\beta}\).  
כדי שהשוויון יתקיים לכל \(x,y\), יש להשוות את המקדמים של כל חזקת־צמד זהה.

נרשום שוב את המשוואה:

\[
\textcolor{blue}{2\beta x^{\alpha}y^{\beta-1}} 
+ (\beta - 1)x^{\alpha}y^{\beta-2}
= \textcolor{blue}{(\alpha + 1)x^{\alpha}y^{\beta-1}} 
- (\alpha - 1)x^{\alpha-2}y^{\beta}.
\]

\textbf{כעת נבצע השוואת איברים:}

\[
\textcolor{blue}{x^{\alpha}y^{\beta-1}}:
\quad 2\beta = \alpha + 1
\]

\[
\textcolor{teal}{x^{\alpha}y^{\beta-2}}:
\quad \beta - 1 = 0
\]

\[
\textcolor{red}{x^{\alpha-2}y^{\beta}}:
\quad -(\alpha - 1) = 0
\]

\textbf{שלב 3 – פתרון המערכת עבור $\alpha,\beta$}

מן המשוואות נקבל:

\[
\begin{cases}
2\beta = \alpha + 1,\\[4pt]
\beta - 1 = 0,\\[4pt]
\alpha - 1 = 0.
\end{cases}
\]

נפתור:

\[
\boxed{
\alpha = 1, \qquad \beta = 1.
}
\]

\textbf{שלב 4 – כתיבת גורם האינטגרציה}

נמצא כי:
\[
\boxed{
\mu(x,y) = x^{1}y^{1} = xy.
}
\]

\textbf{שלב 5 – הכפלה בגורם האינטגרציה}

נכפיל את המשוואה המקורית ב־\(\mu(x,y)=xy\):

\[
xy\!\left(2 + \tfrac{1}{y}\right)dx + xy\!\left(\tfrac{x}{y} - \tfrac{1}{x}\right)dy = 0.
\]

נפשט כל איבר:

\[
(2xy + x)\,dx + (x^2 - y)\,dy = 0.
\]

כעת המשוואה מדויקת, עם:
\[
P = 2xy + x, 
\qquad 
Q = x^2 - y\rightarrow P_y=Q_x=2x.
\]

\textbf{שלב 6 – מציאת פונקציית הפוטנציאל \(F(x,y)\)}

נרצה למצוא פונקציה \(F(x,y)\) כך שיתקיים:
\[
F_x = P = 2xy + x,
\qquad
F_y = Q = x^2 - y.
\]

נחשב תחילה את האינטגרל של \(P\) לפי \(x\):
\[
\int P\,dx = \int (2xy + x)\,dx = x^2y + \tfrac{1}{2}x^2 + g(y),
\]
כאשר \(g(y)\) הוא פונקציה שתלויה רק ב־\(y\).

נחשב כעת את האינטגרל של \(Q\) לפי \(y\):
\[
\int Q\,dy = \int (x^2 - y)\,dy = x^2y - \tfrac{1}{2}y^2 + h(x),
\]
כאשר \(h(x)\) הוא פונקציה שתלויה רק ב־\(x\).

נשווה זהותית את שני הביטויים שקיבלנו:

\[
x^2y + \tfrac{1}{2}x^2 + g(y)
\;\equiv\;
x^2y - \tfrac{1}{2}y^2 + h(x).
\]

נשווה איברים בעלי תלות זהה במשתנים:

\[
x:\quad h(x) = \tfrac{1}{2}x^2 + C_1,
\qquad
y:\quad g(y) = -\tfrac{1}{2}y^2 + C_2.
\]

נבחר ייצוג יחיד של \(F(x,y)\) ונקבל:

\[
F(x,y) = x^2y + \tfrac{1}{2}x^2 - \tfrac{1}{2}y^2 + C.
\]

\textbf{שלב 7 – כתיבת הפתרון הכללי}

נכתוב את קווי הרמה של הפונקציה הפוטנציאלית:

\[
F(x,y) = C
\quad\Longrightarrow\quad
x^2y + \tfrac{1}{2}x^2 - \tfrac{1}{2}y^2 = C.
\]

ולכן הפתרון הכללי של המשוואה הוא:

\[
\boxed{
x^2y + \tfrac{1}{2}x^2 - \tfrac{1}{2}y^2 = C,
\qquad x,y \neq 0.
}
\]
שימו לב כי ניתן לפרש את הפתרון $y(x)$, אך הורשינו להשאיר אותו בצורה ׳׳סתומה׳׳.

\includepdf[pages={4}]{ode_cover.pdf}

\end{document}





